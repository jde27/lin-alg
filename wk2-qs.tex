\documentclass{article}
\title{Linear Algebra Worksheet 2}
\author{Jonny Evans}
\date{}
\usepackage[utf8]{inputenc}
\usepackage[T1]{fontenc}
\usepackage[a4paper,margin=1in]{geometry}
\usepackage{longtable}
\usepackage{fixltx2e}
\usepackage{graphicx}
\usepackage{longtable}
\usepackage{float}
\usepackage{parallel}
\usepackage{parcolumns}
\usepackage{wrapfig}
\usepackage{rotating}
\usepackage{amsmath}
\usepackage{textcomp}
\usepackage{marvosym}
\usepackage{wasysym}
\usepackage{amssymb}
\usepackage{amsthm,amsmath,amsfonts,amscd,setspace}
\renewcommand{\partname}{Week}
\newcommand{\mks}[1]{\marginpar{\colorbox{white}{\parbox{\marginparwidth}{%
  \setstretch{0.5}{\bf\scriptsize{#1 marks}}}}}}
\usepackage{tikz}
\usetikzlibrary{decorations.markings,decorations.pathmorphing,shapes}
\usepackage{parskip}
\newcommand{\FF}{\mathfrak{f}}
\newcommand{\GG}{\mathfrak{g}}
\newcommand{\CC}{\mathbb{C}}
\newcommand{\QQ}{\mathbb{Q}}
\newcommand{\RR}{\mathbb{R}}
\newcommand{\UU}{\mathbb{U}}
\newcommand{\XX}{\mathbb{X}}
\newcommand{\YY}{\mathbb{Y}}
\newcommand{\ZZ}{\mathbb{Z}}
\newcommand{\Link}{\operatorname{Link}}
\newcommand{\Cone}{\operatorname{Cone}}
\newcommand{\colim}{\operatorname{colim}}
%\newcommand{\amal}{\operatorname{amal}}
\newcommand{\adj}{\operatorname{adj}}
\newcommand{\cp}[1]{\mathbf{CP}^{#1}}
\newcommand{\rp}[1]{\mathbf{RP}^{#1}}
\newcommand{\OP}[1]{\mathrm{#1}}
\newcommand{\ma}{\begin{pmatrix}}
\newcommand{\mz}{\end{pmatrix}}
\newcommand{\tka}{\begin{center}\begin{tikzpicture}}
\newcommand{\tkz}{\end{tikzpicture}\end{center}}
\newcommand{\matr}[4]{\left(\begin{array}{cc}#1 & #2\\ #3 & #4\end{array}\right)}
\newcommand{\mthrthr}[9]{\left(\begin{array}{ccc}#1 & #2 & #3\\ #4 & #5 & #6\\ #7 & #8 & #9\end{array}\right)}
\newcommand{\vect}[2]{\left(\begin{array}{c}#1\\#2\end{array}\right)}
\newcommand{\vthr}[3]{\left(\begin{array}{c}#1\\#2\\#3\end{array}\right)}
\newcommand{\TO}[3]{#1\stackrel{#2}{\longrightarrow}#3}
\newenvironment{Proof}{\begin{proof}}{\end{proof}\ignorespacesafterend}
\makeatletter
\renewcommand*\env@matrix[1][*\c@MaxMatrixCols c]{%
  \hskip -\arraycolsep
  \let\@ifnextchar\new@ifnextchar
  \array{#1}}
\makeatother

\begingroup
\makeatletter
\@for\theoremstyle:=definition,remark,plain\do{%
\expandafter\g@addto@macro\csname th@\theoremstyle\endcsname{%
\addtolength\thm@preskip\parskip
}%
}
\endgroup
\usepackage{graphicx}
\usepackage[capitalise]{cleveref}
\newtheorem{Theorem}{Theorem}[section]
\newtheorem{Lemma}[Theorem]{Lemma}
\newtheorem{Corollary}[Theorem]{Corollary}
\newtheorem{Proposition}[Theorem]{Proposition}
\theoremstyle{remark}
\newtheorem{Remark}[Theorem]{Remark}
\theoremstyle{definition}
\newtheorem{Definition}[Theorem]{Definition}
\newtheorem{Example}[Theorem]{Example}
\newtheorem{Exercise}[Theorem]{Exercise}
\newtheorem{Question}[Theorem]{Question}
\newtheorem{Solution}[Theorem]{Solution}
\newtheorem{Answer}[Theorem]{Answer}
\crefname{Theorem}{Theorem}{Theorems}
\Crefname{Theorem}{Theorem}{Theorems}
\crefname{Lemma}{Lemma}{Lemmas}
\Crefname{Lemma}{Lemma}{Lemmas}
\crefname{Corollary}{Corollary}{Corollaries}
\Crefname{Corollary}{Corollary}{Corollaries}
\crefname{Claim}{Claim}{Claims}
\Crefname{Claim}{Claim}{Claims}
\crefname{Proposition}{Proposition}{Propositions}
\Crefname{Proposition}{Proposition}{Propositions}
\crefname{Remark}{Remark}{Remarks}
\Crefname{Remark}{Remark}{Remarks}
\crefname{Definition}{Definition}{Definitions}
\Crefname{Definition}{Definition}{Definitions}
\crefname{Example}{Example}{Examples}
\Crefname{Example}{Example}{Examples}
\crefname{Exercise}{Exercise}{Exercises}
\Crefname{Exercise}{Exercise}{Exercises}
\crefname{Question}{Question}{Questions}
\Crefname{Question}{Question}{Questions}
\crefname{Solution}{Solution}{Solutions}
\Crefname{Solution}{Solution}{Solutions}
\crefname{Answer}{Answer}{Answers}
\Crefname{Answer}{Answer}{Answers}
\crefformat{footnote}{#2\footnotemark[#1]#3}

\begin{document}
\maketitle
\setcounter{section}{2}


Here is a list \(\mathcal{V}\) of vectors \[u=\ma 1 \\ 1 \mz,\qquad
v=\ma 1 \\ 2\mz,\qquad w=\ma -3 \\ 1 \\ 2\mz,\qquad \xi=\ma 0 \\ 1
\\ -1\mz.\]


\begin{Exercise}\label{exr:orth}
For every vector in \(\mathcal{V}\), find its length and write down
a vector orthogonal to it.


\end{Exercise}
\begin{Exercise}\label{ex:dotprods}
Find the angle between \(u\) and \(v\). Find the angle between \(w\)
and \(\xi\).


\end{Exercise}
Here is a list \(\mathcal{M}\) of matrices. \[A=\ma \frac{1}{2} & 0 &
\frac{1}{\sqrt{2}} \\ 0 & 1 & 0 \\ \frac{1}{2} & 0 &
-\frac{1}{\sqrt{2}}\mz,\quad B=\ma \frac{3}{4} & \frac{1}{4} &
\frac{1}{2}\sqrt{\frac{3}{2}} \\ \frac{1}{4} & \frac{3}{4} &
-\frac{1}{2}\sqrt{\frac{3}{2}} \\ -\frac{1}{2}\sqrt{\frac{3}{2}} &
\frac{1}{2}\sqrt{\frac{3}{2}} & \frac{1}{2} \mz,\quad C=\ma
\frac{1}{3} & \frac{1}{3}+\frac{1}{\sqrt{3}} &
-\frac{1}{3}+\frac{1}{\sqrt{3}} \\ \frac{1}{3}-\frac{1}{\sqrt{3}} &
\frac{1}{3} &
-\frac{1}{3}-\frac{1}{\sqrt{3}}\\ -\frac{1}{3}-\frac{1}{\sqrt{3}} &
-\frac{1}{3}+\frac{1}{\sqrt{3}} & \frac{1}{3} \mz.\]


\begin{Exercise}\label{exr:orthogmat}
Which matrices \(M\in\mathcal{M}\) are orthogonal matrices? (Hint:
There should be two!)


\end{Exercise}
\begin{Exercise}\label{exr:3drot}
The orthogonal matrices from \(\mathcal{M}\) are actually rotation
matrices. In each case, find the axis and angle of rotation.


\end{Exercise}
Here is a list \(\mathcal{N}\) of matrices
\[D=\ma 1 & 2 & 0 & 1 & 3 \\ 0 & 0 & 1 & 2 & 1\mz,\quad E=\ma 0 & 1 & 0
& 2 \\ 1 & 0 & 0 & 0 \\ 0 & 0 & 0 & 0\mz,\qquad F=\ma 2 & 2 & 3 \\ 0 & 1
& 1 \\ 0 & 0 & 1\mz,\qquad G=\ma 1 & 0 & 8 \\ 0 & 1 & 2 \\ 0 & 0 & 0\mz.\]


\begin{Exercise}\label{exr:echform}
Which of the matrices \(N\in\mathcal{N}\) are in echelon form? Which
are in reduced echelon form?


\end{Exercise}
\begin{Exercise}\label{exr:simeqech}
For each \(N\in\mathcal{N}\) which is in reduced echelon form, state
(a) for which vectors \(b\) the equation \(Nv=b\) has a solution and
(b) the dimension of the space of solutions to \(Nv=b\), assuming
that \(b\) is chosen so that there is a solution.


\end{Exercise}
\begin{Exercise}\label{exr:simeq}
For each system of simultaneous equations below, write it in matrix
form, put the augmented matrix into reduced echelon form using row
operations. Determine if the system has a solution and, if it does,
give the general solution.


\begin{center}
\begin{tabular}{p{3cm}|p{3cm}|p{3cm}}
{\begin{align*}x+y+2z+3w&=0\\ y+4z-w&=1\end{align*}} &
{\begin{align*}x&=y-3\\ 2x+y&=6\\ y-3x&=1\end{align*}} &
{\begin{align*} 4x-w&=0\\ 3y-2z+w&=4\\ 4x-2y+4z-3w&=0\\ 3x+y-z&=2 \end{align*}}
\end{tabular}
\end{center}


\end{Exercise}
\begin{Exercise}
Put the following matrices into reduced echelon form using row
operations. In each case, what is the number of free indices?
\[X=\ma 1 & 1 & 0 & 0 & 0 \\ 1 & 1 & 1 & 0 & 0 \\ 1 & 1 & 1 & 1 & 0
\\ 1 & 1 & 1 & 1 & 1 \mz,\quad Y=\ma 2 & 0 & 1 \\ 0 & 1 & 0 \\ 1 & 3
& -1\mz,\quad Z=\ma 1 & 2 & 3 & 4 \\ 5 & 6 & 7 & 8 \\ 9 & 10 & 11 &
12\mz.\]


\end{Exercise}
\newpage


\begin{Exercise}\label{exr:assoc}
Let \(A,B,C\) be \(m\)-by-\(n\), \(n\)-by-\(p\) and \(p\)-by-\(q\)
matrices respectively. Write out the matrix products \(A(BC)\) and
\((AB)C\) in index notation and check that they give the same answer
(this shows that matrix multiplication is associative).


\end{Exercise}
\begin{Exercise}\label{exr:orthogmat}
Suppose that \(A\) is an \(n\)-by-\(n\) matrix whose columns are the
vectors \(v_1,\ldots,v_n\). Show that \(A\) is an orthogonal matrix
(i.e. \(A^TA=I\)) if and only if \[v_i\cdot v_j=\begin{cases}1\mbox{
if }i=j\\0\mbox{ if }i\neq j\end{cases}\mbox{ for all }i,j.\] In
other words, the columns of \(A\) are orthogonal to one another
(this is where the name ``orthogonal matrix'' comes from).


\end{Exercise}
\begin{Exercise}\label{exr:symmantisymm}
We say that a matrix \(M\) is {\em symmetric} if \(M^T=M\) and {\em
antisymmetric} if \(M^T=-M\).
\begin{enumerate}
\item Show that if \(N\) is an \(m\)-by-\(n\) matrix then \(MM^T\) is a
symmetric \(m\)-by-\(m\) matrix and \(M^TM\) is a symmetric
\(n\)-by-\(n\) matrix.
\item Show that, given any \(n\)-by-\(n\)
matrix \(C\), the matrix \(A=C+C^T\) is symmetric and the matrix
\(B=C-C^T\) is antisymmetric. Deduce that \(C\) can be written as
the sum of a symmetric and an antisymmetric matrix (called the {\em
symmetric} and {\em antisymmetric} parts of \(C\) respectively).


\end{enumerate}
\end{Exercise}
\begin{Exercise}\label{exr:determined}
A system of \(m\) equations in \(n\) unknowns is called {\em
underdetermined} if \(m<n\) and overdetermined if \(m>n\). As rules
of thumb, underdetermined equations tend to have general solutions
with \(m-n\) free parameters, and overdetermined equations tend to
have no solutions. Give counterexamples to these rules of thumb
(e.g. an underdetermined system with no solutions and an
overdetermined system with a solution).


\end{Exercise}
\newpage


\section{Assessed problems for Week 2}


Total marks available: 20.


\begin{Question}
Find the angle between the vectors \(v=\ma
\frac{1}{2\sqrt{6}}(3+\sqrt{2}) \\ \frac{1}{2\sqrt{6}}(-3+\sqrt{2})
\\ \frac{1}{2\sqrt{3}} \mz\) and \(w=\ma \frac{1}{\sqrt{3}}
\\ \frac{1}{\sqrt{3}} \\ \frac{1}{\sqrt{3}}\mz\). Find a vector
\(u\) which is orthogonal to both of them. \mks{4}


\end{Question}
\begin{Question}
The matrix \(A=\ma \frac{1}{2} & \frac{1}{2} & \frac{1}{\sqrt{2}}
\\ \frac{1}{2} & \frac{1}{2} & -\frac{1}{\sqrt{2}}
\\ -\frac{1}{\sqrt{2}} & \frac{1}{\sqrt{2}} & 0\mz\) is a 3-d
rotation matrix. Check that \(A\) is orthogonal. Find the axis and
angle of rotation. \mks{8}


\end{Question}
\begin{Question}
Write the system of simultaneous equations
\begin{align*}3x+2y+z&=0\\ 3x+y+2z&=3\\ -x-y&=1\end{align*}
as a matrix equation. Using row operations, put the augmented matrix
\mks{8} in reduced echelon form, and hence find the general solution
of the system of equations.


\end{Question}
\rule{\textwidth}{0.4pt}


\mks{Total/20}




\end{document}
