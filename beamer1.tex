\documentclass{beamer}
\title{Linear Algebra, Week 1}
\author{Jonny Evans}
\date{MATH105}
\usenavigationsymbolstemplate{}
\usepackage[utf8]{inputenc}
\usepackage[T1]{fontenc}
\usepackage{fixltx2e}
\usepackage{graphicx}
\usepackage{longtable}
\usepackage{float}
\usepackage{parallel}
\usepackage{parcolumns}
\usepackage{wrapfig}
\usepackage{rotating}
\usepackage{amsmath}
\usepackage{textcomp}
\usepackage{marvosym}
\usepackage{wasysym}
\usepackage{amssymb}
\usepackage{amsthm,amsmath,amsfonts,amscd,setspace}
\renewcommand{\partname}{Week}
\usepackage{tikz}
\usetikzlibrary{decorations.markings,decorations.pathmorphing,shapes}
\usepackage{parskip}
\newcommand{\FF}{\mathfrak{f}}
\newcommand{\GG}{\mathfrak{g}}
\newcommand{\CC}{\mathbb{C}}
\newcommand{\QQ}{\mathbb{Q}}
\newcommand{\RR}{\mathbb{R}}
\newcommand{\UU}{\mathbb{U}}
\newcommand{\XX}{\mathbb{X}}
\newcommand{\YY}{\mathbb{Y}}
\newcommand{\ZZ}{\mathbb{Z}}
\newcommand{\Link}{\operatorname{Link}}
\newcommand{\Cone}{\operatorname{Cone}}
\newcommand{\colim}{\operatorname{colim}}
%\newcommand{\amal}{\operatorname{amal}}
\newcommand{\adj}{\operatorname{adj}}
\newcommand{\cp}[1]{\mathbf{CP}^{#1}}
\newcommand{\rp}[1]{\mathbf{RP}^{#1}}
\newcommand{\OP}[1]{\mathrm{#1}}
\newcommand{\ma}{\begin{pmatrix}}
\newcommand{\mz}{\end{pmatrix}}
\newcommand{\tka}{\begin{center}\begin{tikzpicture}}
\newcommand{\tkz}{\end{tikzpicture}\end{center}}
\newcommand{\matr}[4]{\left(\begin{array}{cc}#1 & #2\\ #3 & #4\end{array}\right)}
\newcommand{\mthrthr}[9]{\left(\begin{array}{ccc}#1 & #2 & #3\\ #4 & #5 & #6\\ #7 & #8 & #9\end{array}\right)}
\newcommand{\vect}[2]{\left(\begin{array}{c}#1\\#2\end{array}\right)}
\newcommand{\vthr}[3]{\left(\begin{array}{c}#1\\#2\\#3\end{array}\right)}
\newcommand{\TO}[3]{#1\stackrel{#2}{\longrightarrow}#3}
\makeatletter
\renewcommand*\env@matrix[1][*\c@MaxMatrixCols c]{%
  \hskip -\arraycolsep
  \let\@ifnextchar\new@ifnextchar
  \array{#1}}
\makeatother

\begingroup
\makeatletter
\@for\theoremstyle:=definition,remark,plain\do{%
\expandafter\g@addto@macro\csname th@\theoremstyle\endcsname{%
\addtolength\thm@preskip\parskip
}%
}
\endgroup
\usepackage{graphicx}
\usepackage[capitalise]{cleveref}
%\newtheorem{Theorem}{Theorem}[section]
%\newtheorem{Lemma}[Theorem]{Lemma}
%\newtheorem{Corollary}[Theorem]{Corollary}
%\newtheorem{Proposition}[Theorem]{Proposition}
%\theoremstyle{remark}
%\newtheorem{Remark}[Theorem]{Remark}
%\theoremstyle{definition}
%\newtheorem{Definition}[Theorem]{Definition}
%\newtheorem{Example}[Theorem]{Example}
%\newtheorem{Exercise}[Theorem]{Exercise}
%\newtheorem{Question}[Theorem]{Question}
%\newtheorem{Solution}[Theorem]{Solution}
%\newtheorem{Answer}[Theorem]{Answer}
\crefname{Theorem}{Theorem}{Theorems}
\Crefname{Theorem}{Theorem}{Theorems}
\crefname{Lemma}{Lemma}{Lemmas}
\Crefname{Lemma}{Lemma}{Lemmas}
\crefname{Corollary}{Corollary}{Corollaries}
\Crefname{Corollary}{Corollary}{Corollaries}
\crefname{Claim}{Claim}{Claims}
\Crefname{Claim}{Claim}{Claims}
\crefname{Proposition}{Proposition}{Propositions}
\Crefname{Proposition}{Proposition}{Propositions}
\crefname{Remark}{Remark}{Remarks}
\Crefname{Remark}{Remark}{Remarks}
\crefname{Definition}{Definition}{Definitions}
\Crefname{Definition}{Definition}{Definitions}
\crefname{Example}{Example}{Examples}
\Crefname{Example}{Example}{Examples}
\crefname{Exercise}{Exercise}{Exercises}
\Crefname{Exercise}{Exercise}{Exercises}
\crefname{Question}{Question}{Questions}
\Crefname{Question}{Question}{Questions}
\crefname{Solution}{Solution}{Solutions}
\Crefname{Solution}{Solution}{Solutions}
\crefname{Answer}{Answer}{Answers}
\Crefname{Answer}{Answer}{Answers}
\crefformat{footnote}{#2\footnotemark[#1]#3}

\begin{document}
\maketitle
\begin{frame}
\includegraphics[width=300px]{tree}


\end{frame}
\begin{frame}
\begin{center}
{\huge Matrices and vectors}
\end{center}


\end{frame}
\begin{frame}
A vector \(v=\ma x \\ y\mz\) in \(\RR^2\) represents an arrow:


\onslide<1->\tka
\onslide<1->{\draw[->,thick] (0,0) -- (3,2);
\draw[dotted] (0,0) -- (3,0) node [midway,below] {\(x\)};
\draw[dotted] (3,0) -- (3,2) node [midway,right] {\(y\)};}
\onslide<2->{\node at (1.5,1) [above left] {\(|v|\)};}
\onslide<4->{\node at (0.6,0.2) {\(\theta\)};
\draw (0.9,0) arc [radius=0.9,start angle=0,end angle=35];}
\tkz


\begin{columns}[t]
\begin{column}{.3\linewidth}<1->
\onslide<2->Length of \(v\)?
\vspace{0.4cm}


\onslide<4->Angle \(\theta\)?
\end{column}
\begin{column}{.3\linewidth}<1->
\onslide<3-> \(|v|=\sqrt{x^2+y^2}\)
\vspace{0.4cm}


\onslide<5-> \(\theta=\arctan(y/x)\)
\end{column}
\end{columns}
\vspace{0.2cm}


\onslide<6->\[v=\ma |v|\cos\theta \\ |v|\sin\theta\mz.\]


\end{frame}
\begin{frame}
\onslide<1->
\begin{Theorem}
If \(w\) is obtained by rotating \(v=\ma x \\ y\mz\) by an angle
\(\phi\) anticlockwise around its basepoint then
\[w=\ma x\cos\phi-y\sin\phi \\ x\sin\phi+y\cos\phi\mz.\]


\end{Theorem}
\onslide<2->
\begin{Definition}
A {\em 2-by-2 matrix} is a grid of numbers \(\ma a & b \\ c &
d\mz\).


Given \(M=\ma a & b \\ c & d\mz\) and \(v=\ma x \\ y\mz\), define
the {\em linear map} \[\RR^2\to\RR^2,\qquad v\mapsto Mv:=\ma a & b
\\ c & d\mz\ma x \\ y\mz=\ma ax+by \\ cx+dy\mz.\]


\end{Definition}
\end{frame}
\begin{frame}
\onslide<1->
Similarly, a vector \(\ma x \\ y \\ z\mz\) represents an arrow in
\(\RR^3\).


\onslide<2->
A 3-by-3 matrix \(\ma a & b & c \\ d & e & f \\ g & h & i\mz\)
defines a linear map \(\RR^3\to\RR^3\),


\onslide<3->
\[\ma a & b & c \\ d & e & f \\ g & h & i\mz\ma x \\ y \\ z\mz=\ma ax+by+cz \\ dx+ey+fz \\ gx+hy+iz\mz.\]


\end{frame}
\begin{frame}
\onslide<1->
\begin{Definition}
An \(m\)-by-\(n\) matrix is a grid with \(m\) rows and \(n\)
columns
\tka
\node at (0,0) {\(\ma A_{11} & A_{12} & \cdots & \cdots & A_{1n} \\ A_{21} & A_{22} & \cdots & \cdots & A_{2n} \\ \vdots & \vdots & & & \vdots \\ A_{m1} & A_{m2} & \cdots & \cdots & A_{mn} \mz\)};
\draw[<->,thick] (-2,1.4) -- (2,1.4) node [midway,above] {\(n\)};
\draw[<->,thick] (-3,1) -- (-3,-1) node [midway,left] {\(m\)};
\tkz


\end{Definition}
\onslide<2-> This defines a linear map \(\RR^n\to\RR^m\) \[\ma
A_{11} & A_{12} & \cdots & \cdots & A_{1n} \\ A_{21} & A_{22} &
\cdots & \cdots & A_{2n} \\ \vdots & \vdots & & & \vdots
\\ A_{m1} & A_{m2} & \cdots & \cdots & A_{mn} \\ \mz\ma x_1
\\ x_2 \\ \vdots \\ \vdots \\ x_n\mz=\ma A_{11}x_1+\cdots+A_{1n}x_n
\\ A_{21}x_1+\cdots+A_{2n}x_n \\ \vdots
\\ A_{m1}x_1+\cdots+A_{mn}x_n \mz\]


\end{frame}
\begin{frame}
\begin{center}
{\huge Matrix multiplication}
\end{center}


\end{frame}
\begin{frame}
{Back to 2-by-2}


\onslide<1->Given \(A=\ma A_{11} & A_{12} \\ A_{21} & A_{22}\mz\), and \(B=\ma
B_{11} & B_{12} \\ B_{21} & B_{22}\mz\) we get
\onslide<1->\begin{align*}
\onslide<2->{A(B(v))&=\ma A_{11} & A_{12} \\ A_{21} & A_{22}\mz\ma B_{11} & B_{12} \\ B_{21} & B_{22}\mz\ma x \\ y\mz\\}
\onslide<3->{&\vdots\\}
\onslide<4->{&=\ma A_{11}B_{11}+A_{12}B_{21} & A_{11}B_{12}+A_{12}B_{22}\\ A_{21}B_{11}+A_{22}B_{21} & A_{21}B_{12}+A_{22}B_{22}\mz\ma x \\ y\mz}
\end{align*}


\end{frame}
\begin{frame}
\begin{Definition}[Matrix multiplication: 2-by-2]\label{dfn:matmul2}
Define \[AB=\ma A_{11}B_{11}+A_{12}B_{21} &
A_{11}B_{12}+A_{12}B_{22}\\ A_{21}B_{11}+A_{22}B_{21} &
A_{21}B_{12}+A_{22}B_{22}\mz.\]


\onslide<2-> Equivalently, \[(AB)_{ij}=\sum_{k=1}^2A_{ik}B_{kj}.\]


\onslide<3->
\begin{Definition}[Matrix multiplication]\label{dfn:matmul}
Given an \(m\)-by-\(n\) matrix \(A\) and an \(n\)-by-\(p\) matrix
\(B\), define \(AB\) to be the \(m\)-by-\(p\) matrix with entries
\[(AB)_{ij}=\sum_{k=1}^nA_{ik}B_{kj}.\]


\end{Definition}
\end{Definition}
\end{frame}
\begin{frame}
\onslide<1->
\begin{Definition}[Matrix powers]\label{dfn:matrixpowers}
If \(A\) is a square matrix and \(k\) is a positive integer then
\(A^k\) denotes the product \(AA\cdots A\) (\(k\) times, where
\(k=0\) means \(A=I\)).


\end{Definition}
\onslide<2->
\begin{Definition}[Matrix sum]\label{dfn:matrixsum}
\[(A+B)_{ij}=A_{ij}+B_{ij}.\]


\end{Definition}
\onslide<3->
\begin{Definition}[Matrix scaling]\label{dfn:matrixscaling}
\[(\lambda A)_{ij}=\lambda A_{ij}.\]


\end{Definition}
\end{frame}
\begin{frame}
{Matrix exponential}


\onslide<2->
\begin{Definition}[Matrix exponential]\label{dfn:matexp}
Given an \(n\)-by-\(n\) matrix \(A\), define
\[\exp(A)=I+A+\frac{1}{2!}A^2+\frac{1}{3!}A^3+\cdots=\sum_{n\geq 0}\frac{1}{n!}A^n.\]


\end{Definition}
\end{frame}
\begin{frame}
\begin{center}
{\huge Dot products \&\\ orthogonal matrices}
\end{center}


\end{frame}
\begin{frame}
\onslide<1->
\begin{Definition}
Given vectors \(v=\ma v_1 \\ \vdots \\ v_n\mz\) and \(w=\ma w_1
\\ \vdots \\ w_n\mz\) define the {\em dot product} \[v\cdot
w:=v_1w_1+\cdots+v_nw_n=\sum_{i=1}^nv_iw_i.\]


\end{Definition}
\onslide<2->
\begin{Theorem}[Proof later!]\label{thm:angle}
If \(v\) and \(w\) make an angle \(\phi\) then
\[v\cdot w=|v||w|\cos\phi.\]


\end{Theorem}
\end{frame}
\begin{frame}
\onslide<1->
Note that \[\ma v_1\\ \vdots\\v_n\mz\cdot\ma
w_1\\\vdots\\ w_n\mz=\ma v_1 & \cdots & v_n\mz\ma
w_1\\\vdots\\ w_n\mz.\]


\onslide<2->
\begin{Definition}
Define the {\em transpose} \(A^T\) of an \(m\)-by-\(n\) matrix
\(A\) to be the \(n\)-by-\(m\) matrix with entries
\((A^T)_{ij}=A_{ji}\).


\end{Definition}
So \(v\cdot w=v^Tw\).


\end{frame}
\begin{frame}
\onslide<1->
\begin{Lemma}
\[(AB)^T=B^TA^T.\]


\end{Lemma}
\onslide<2->
\begin{Definition}[Orthogonal matrix]\label{dfn:orthomat}
A square matrix \(A\) is {\em orthogonal} if \(A^TA=I\).


\end{Definition}
\onslide<3->
\begin{Lemma}
If \(A\) is orthogonal then \((Av)\cdot(Aw)=v\cdot w\).
\vspace{0.4cm}


\onslide<3-> i.e. orthogonal matrices preserve lengths and angles.


\end{Lemma}
\end{frame}
\begin{frame}
\begin{center}
{\huge 3-d rotations}
\end{center}


\end{frame}
\begin{frame}
\begin{Example}
\[A=\ma \cos\phi & -\sin\phi & 0 \\ \sin\phi & \cos\phi & 0 \\ 0 &
0 & 1\mz\]


\[B=\ma 1 & 0 & 0 \\ 0 & 0 & -1 \\ 0 & 1 & 0 \mz\]


\[C=\ma 0 & 0 & 1 \\ 0 & 1 & 0 \\ -1 & 0 & 0\mz\]


\end{Example}
\end{frame}
\begin{frame}
\begin{Example}
\[D=\ma 0 & 0 & 1 \\ 1 & 0 & 0 \\ 0 & 1 & 0\mz.\]


\end{Example}
\begin{itemize}
\item Find axis \(u\): solve \(Du=u\).
\item Find angle: pick \(v\perp u\) and compute angle between \(v\) and
\(Dv\).


\end{itemize}
\end{frame}
\begin{frame}
\begin{Example}
\[E=\ma \frac{1}{3} & \frac{1}{3}+\frac{1}{\sqrt{3}} &
-\frac{1}{3}+\frac{1}{\sqrt{3}} \\ \frac{1}{3}-\frac{1}{\sqrt{3}}
& \frac{1}{3} &
-\frac{1}{3}-\frac{1}{\sqrt{3}}\\ -\frac{1}{3}-\frac{1}{\sqrt{3}}
& -\frac{1}{3}+\frac{1}{\sqrt{3}} & \frac{1}{3} \mz.\]


\end{Example}
\begin{itemize}
\item Find axis \(u\): solve \(Eu=u\).
\item Find angle: pick \(v\perp u\) and compute angle between \(v\) and
\(Ev\).


\end{itemize}
\end{frame}
\begin{frame}
\onslide<1->We saw earlier that \(\exp\ma 0 & -\theta \\ \theta & 0
\mz=\ma \cos\theta & -\sin\theta \\ \sin\theta & \cos\theta\mz\). In
fact...


\onslide<2->
\begin{Theorem}
\(\exp(tA)\) is orthogonal for all \(t\in\RR\) if and only if
\(A\) is {\em antisymmetric}, i.e. \(A^T=-A\).
\end{Theorem}
\onslide<3->
\begin{proof}
If \(A^T=-A\) then
\[(\exp(tA))^T=\exp(tA^T)=\exp(-tA).\]
\[\exp(-tA)\exp(tA)=I,\]
so \(\exp(tA)\) is orthogonal. Conversely...


\end{proof}
\end{frame}
\begin{frame}
\onslide<1->
\begin{Theorem}
\(\exp(tA)\) is orthogonal for all \(t\in\RR\) if and only if
\(A\) is {\em antisymmetric}, i.e. \(A^T=-A\).
\end{Theorem}
\onslide<2->
\begin{proof}
If \(\exp(tA)\) is orthogonal, then \(\exp(tA)^T\exp(tA)=I\) for
all \(t\). Differentiate with respect to \(t\):
\[A^T\exp(tA^T)\exp(tA)+\exp(tA^T)A\exp(tA)=0,\]
and set \(t=0\):
\[A^T+A=0,\qquad\mbox{ as }\exp(0A)=I.\]


\end{proof}
\end{frame}
\begin{frame}
Need to show:
\begin{align*}
\exp(B)^T&=\exp(B^T)\\
\exp(-B)\exp(B)&=I\\
\frac{d}{dt}\exp(tA)&=A\exp(tA)\\
\frac{d}{dt}(A(t)B(t))&=\frac{dA(t)}{dt}B(t)+A(t)\frac{dB(t)}{dt}.
\end{align*}


\end{frame}
\end{document}
