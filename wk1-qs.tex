\documentclass{article}
\title{Linear Algebra Worksheet 1}
\author{Jonny Evans}
\date{\vspace{-1cm}}
\usepackage[utf8]{inputenc}
\usepackage[T1]{fontenc}
\usepackage[a4paper,margin=1in]{geometry}
\usepackage{longtable}
\usepackage{fixltx2e}
\usepackage{graphicx}
\usepackage{longtable}
\usepackage{float}
\usepackage{parallel}
\usepackage{parcolumns}
\usepackage{wrapfig}
\usepackage{rotating}
\usepackage{amsmath}
\usepackage{textcomp}
\usepackage{marvosym}
\usepackage{wasysym}
\usepackage{amssymb}
\usepackage{amsthm,amsmath,amsfonts,amscd,setspace}
\renewcommand{\partname}{Week}
\newcommand{\mks}[1]{\marginpar{\colorbox{white}{\parbox{\marginparwidth}{%
  \setstretch{0.5}{\bf\scriptsize{#1 marks}}}}}}
\usepackage{tikz}
\usetikzlibrary{decorations.markings,decorations.pathmorphing,shapes}
\usepackage{parskip}
\newcommand{\FF}{\mathfrak{f}}
\newcommand{\GG}{\mathfrak{g}}
\newcommand{\CC}{\mathbb{C}}
\newcommand{\QQ}{\mathbb{Q}}
\newcommand{\RR}{\mathbb{R}}
\newcommand{\UU}{\mathbb{U}}
\newcommand{\XX}{\mathbb{X}}
\newcommand{\YY}{\mathbb{Y}}
\newcommand{\ZZ}{\mathbb{Z}}
\newcommand{\Link}{\operatorname{Link}}
\newcommand{\Cone}{\operatorname{Cone}}
\newcommand{\colim}{\operatorname{colim}}
%\newcommand{\amal}{\operatorname{amal}}
\newcommand{\adj}{\operatorname{adj}}
\newcommand{\cp}[1]{\mathbf{CP}^{#1}}
\newcommand{\rp}[1]{\mathbf{RP}^{#1}}
\newcommand{\OP}[1]{\mathrm{#1}}
\newcommand{\ma}{\begin{pmatrix}}
\newcommand{\mz}{\end{pmatrix}}
\newcommand{\tka}{\begin{center}\begin{tikzpicture}}
\newcommand{\tkz}{\end{tikzpicture}\end{center}}
\newcommand{\matr}[4]{\left(\begin{array}{cc}#1 & #2\\ #3 & #4\end{array}\right)}
\newcommand{\mthrthr}[9]{\left(\begin{array}{ccc}#1 & #2 & #3\\ #4 & #5 & #6\\ #7 & #8 & #9\end{array}\right)}
\newcommand{\vect}[2]{\left(\begin{array}{c}#1\\#2\end{array}\right)}
\newcommand{\vthr}[3]{\left(\begin{array}{c}#1\\#2\\#3\end{array}\right)}
\newcommand{\TO}[3]{#1\stackrel{#2}{\longrightarrow}#3}
\newenvironment{Proof}{\begin{proof}}{\end{proof}\ignorespacesafterend}
\makeatletter
\renewcommand*\env@matrix[1][*\c@MaxMatrixCols c]{%
  \hskip -\arraycolsep
  \let\@ifnextchar\new@ifnextchar
  \array{#1}}
\makeatother

\begingroup
\makeatletter
\@for\theoremstyle:=definition,remark,plain\do{%
\expandafter\g@addto@macro\csname th@\theoremstyle\endcsname{%
\addtolength\thm@preskip\parskip
}%
}
\endgroup
\usepackage{graphicx}
\usepackage[capitalise]{cleveref}
\newtheorem{Theorem}{Theorem}[section]
\newtheorem{Lemma}[Theorem]{Lemma}
\newtheorem{Corollary}[Theorem]{Corollary}
\newtheorem{Proposition}[Theorem]{Proposition}
\theoremstyle{remark}
\newtheorem{Remark}[Theorem]{Remark}
\theoremstyle{definition}
\newtheorem{Definition}[Theorem]{Definition}
\newtheorem{Example}[Theorem]{Example}
\newtheorem{Exercise}[Theorem]{Exercise}
\newtheorem{Question}[Theorem]{Question}
\newtheorem{Solution}[Theorem]{Solution}
\newtheorem{Answer}[Theorem]{Answer}
\crefname{Theorem}{Theorem}{Theorems}
\Crefname{Theorem}{Theorem}{Theorems}
\crefname{Lemma}{Lemma}{Lemmas}
\Crefname{Lemma}{Lemma}{Lemmas}
\crefname{Corollary}{Corollary}{Corollaries}
\Crefname{Corollary}{Corollary}{Corollaries}
\crefname{Claim}{Claim}{Claims}
\Crefname{Claim}{Claim}{Claims}
\crefname{Proposition}{Proposition}{Propositions}
\Crefname{Proposition}{Proposition}{Propositions}
\crefname{Remark}{Remark}{Remarks}
\Crefname{Remark}{Remark}{Remarks}
\crefname{Definition}{Definition}{Definitions}
\Crefname{Definition}{Definition}{Definitions}
\crefname{Example}{Example}{Examples}
\Crefname{Example}{Example}{Examples}
\crefname{Exercise}{Exercise}{Exercises}
\Crefname{Exercise}{Exercise}{Exercises}
\crefname{Question}{Question}{Questions}
\Crefname{Question}{Question}{Questions}
\crefname{Solution}{Solution}{Solutions}
\Crefname{Solution}{Solution}{Solutions}
\crefname{Answer}{Answer}{Answers}
\Crefname{Answer}{Answer}{Answers}
\crefformat{footnote}{#2\footnotemark[#1]#3}

\begin{document}
\maketitle
\setcounter{section}{1}


Here is a list \(\mathcal{V}\) of vectors
\begin{align*}
p&=\ma 1 \\ 0\mz& q&=\ma 0 \\ 1\mz & r&=\ma 1\\ 1\mz\\
s&=\ma 1 \\ -2\mz& t&=\ma -1/2 \\ 7 \\ i\mz& u&=\ma 1 \\ 2 \\ 3\mz\\
v&=\ma -1 \\ 0 \\ 3 \\ 0\mz& w&=\ma 1 \\ -2 \\ 0 \\ 0\mz &\xi&=\ma b\\ b\\ b\\ -b\mz
\end{align*}
Here is a list \(\mathcal{M}\) of matrices
\begin{align*}
A&=\ma 0 & 1 \\ 1 & 1\mz&B&=\ma 2 & 3 \\ 1 & 1\mz&C&=\ma 1 & 1 & 1\\ 1 & 0 & -2\mz\\
D&=\ma 1 & 1 \\ -2 & 8 \\ 1/2 & 3\mz&E&=\ma -1 & 1 & 0\mz& F&=\ma 1 & 0 & 2 \\ 1 & -1 & 1 \\ 3 & 0 & 0\mz\\
G&=\ma 1 & 3 & -1\\ 0 & 1 & 0 \\ 1 & 1 & 0\mz&H&=\ma 1 & 0 & -1 & 0 \\ -2 & 1 & 0 & 4 \\ -17 & 2 & 3 & 5 \\ 1 & -2 & 0 & 0\mz&J&=\ma 0 & 1 & -2 & 1 \\ 1 & 1 & 1 & 1 \mz
\end{align*}


\begin{Exercise}\label{exr:matvec}
For each \(V\in\mathcal{V}\) and each \(M\in\mathcal{M}\), state
whether the vector \(MV\) is defined and, if it is defined, compute
it.


\end{Exercise}
\begin{Exercise}\label{exr:matmul}
For \(N=\ma 1 & -1\\ 2 & 0 \\ 4 & -2\mz\) and for each
\(M\in\mathcal{M}\) state whether \(NM\) and/or \(MN\) is defined
and calculate any products which are defined.


\end{Exercise}
\begin{Exercise}\label{exr:exp}
Find the exponentials of the following matrices (\(\lambda\) is just
some number, \(i\) is the square root of \(-1\)): \[A=\ma 1 & 1 \\ 0
& 0\mz,\quad B=\ma 0 & 1 & 0 \\ 0 & 0 & 1 \\ 0 & 0 & 0\mz,\quad
C=\ma \lambda & 0 \\ 0 & \lambda\mz,\quad D=\ma 0 & i \\ i & 0\mz.\]




\end{Exercise}
\newpage


\begin{Exercise}\label{exr:cols}
Show that if \[A=\ma A_{11} & \cdots & A_{1n} \\ \vdots & & \vdots
\\ A_{m1} &\cdots & A_{mn}\mz\quad\mbox{and}\quad e_1=\ma 1 \\ 0
\\ \vdots \\ 0\mz\] then \(Ae_1\) is the first column of
\(A\). Which vectors \(e_2,\ldots,e_n\) will give the second,
third,..., \(n\)th columns?


\end{Exercise}
\begin{Exercise}\label{exr:truefalse}
Let \(X\) and \(Y\) denote \(2\)-by-\(2\) matrices. Are the
following statements true or false? In each case, give a proof or a
counterexample to support your claim.
\begin{itemize}
\item If \(X^2=I\) then \(X=\pm I\).
\item If \(XY=0\) then \(X=0\) or \(Y=0\).
\item If \(X\) has real entries then \(X^2\neq -I\).
\item If \(Xe_1=Xe_2=0\) then \(X=0\) (\(e_1,e_2\) are from
\cref{exr:cols}).


\end{itemize}
\end{Exercise}
\begin{Exercise}\label{exr:fib}
Take the vector \(\ma 1 \\ 1 \mz\) and the matrix \(A=\ma 0 & 1 \\ 1
& 1 \mz\). Apply \(A\) to \(v\). Then apply \(A\) again. Then apply
\(A\) again. Continue until you spot a pattern. Can you express the
pattern as a formula? Can you prove that this pattern is going to
continue? (Hint: You may write \(F_n\) for the \(n\)th term in a
certain famous sequence of numbers).


\end{Exercise}
\begin{Exercise}\label{exr:ref}
Check that the matrix \[H_\phi:=\ma \cos\phi & \sin\phi \\ \sin\phi
& -\cos\phi\mz\] fixes the vector \(v=\ma \cos(\phi/2)
\\ \sin(\phi/2)\mz\) and sends the vector \(w=\ma -\sin(\phi/2)
\\ \cos(\phi/2)\mz\) to \(-w\). (Hint: Remember your trigonometric
identities...)


{\em This means that \(H_\phi\) represents a reflection in the line
containing \(v\).}


\end{Exercise}
\begin{Exercise}[Special relativity velocity addition]\label{exr:lorentz}
Given a number \(v\), define the matrix \(\Lambda(v)=\ma
\frac{1}{\sqrt{1-v^2}} & \frac{-v}{\sqrt{1-v^2}}
\\ \frac{-v}{\sqrt{1-v^2}} & \frac{1}{\sqrt{1-v^2}}\mz\). Check that
\[\frac{1}{\sqrt{1-\left(\frac{u+v}{1+uv}\right)^2}}=\frac{1+uv}{\sqrt{(1-u^2)(1-v^2)}}\]
for all \(u,v\). Deduce that
\[\Lambda(u)\Lambda(v)=\Lambda\left(\frac{u+v}{1+uv}\right).\]


\end{Exercise}
%# Exercise exr:lorentzexp
% In the notation of \cref{exr:lorentz}, find a matrix \(M(v)\) such
% that \(\exp(M(v))=\Lambda(v)\). (Hint: Use Assessed
% \cref{exr:exphyp}.)


\newpage


\section{Assessed questions for week 1}


\setcounter{section}{1}


\begin{Question}\label{exr:pq}
Let \(A=\ma 1-pq & p^2 \\ -q^2 & 1+pq\mz\) and let \(v=\ma
p\\ q\mz\). Find \(Av\). \mks{2}


\end{Question}
\begin{Question}\label{exr:nc}
Let \(A=\ma 1 & 1 & 0 \\ 0 & 1 & 0 \\ 0 & 0 & 1\mz\) and \(B=\ma 1 &
0 & 0 \\ 0 & 1 & 1 \\ 0 & 0 & 1\mz\). Prove \(AB\neq BA\). \mks{4}


\end{Question}
\begin{Question}\label{exr:matmult}
For each pair of matrices \(M,N\) from the list below, state whether
the products \(MN\) and \(NM\) are well-defined and compute
any\mks{6} which are well-defined. \[A=\ma 2 & 0 & 1 \\ 0 & 1 & 0
\\ 1 & 0 & 2\mz,\quad B=\ma 1 & 1 & 1 \\ -1 & 0 & 3\mz,\quad C=\ma 1
& 1 \\ 1 & 1 \\ 0 & 0\mz,\quad D=\ma 1 \\ 2 \\ 3 \\ 4\mz.\]


\end{Question}
\begin{Question}\label{exr:trace}
Let \(n>0\). Define the {\em trace} \(\mathrm{Tr}(A)\) of an
\(n\)-by-\(n\) matrix \(A\) to be the sum of its diagonal entries
i.e. \(\mathrm{Tr}(A)=\sum_{i=1}^nA_{ii}\). Show that \mks{3}
\(\mathrm{Tr}(AB)=\mathrm{Tr}(BA)\). (Hint: What are the \(ii\)
entries of \(AB\) and of \(BA\)?) Find \(\mathrm{Tr}(I)\). Prove
that there are no matrices \(A,B\) such that \(AB-BA=I\) (Hint: Take
the trace of this formula!)


\end{Question}
\begin{Question}\label{exr:exphyp}
Show that if \(X=\ma 0 & t \\ t & 0 \mz\) then \(\exp(X)=\ma \cosh t
& \sinh t \\ \sinh t & \cosh t\mz\).\mks{4}


{\em If you've never met them, \(\cosh\) and \(\sinh\) are the
hyperbolic trigonometric functions defined by their Taylor series:}
\[\cosh t=1+\frac{t^2}{2!}+\frac{t^4}{4!}+\cdots\qquad
\sinh(t)=t+\frac{t^3}{3!}+\frac{t^5}{5!}+\cdots.\]


\end{Question}
\begin{Question}\label{exr:pell}
The {\em Pell numbers} \(P_1=0\), \(P_2=1\), \(P_3=2\), \(P_4=5\),
\(P_5=12\), \(P_6=29\), etc are obtained by the recursion\mks{1}
\(P_{n+2}=P_n+2P_{n+1}\). Write down a \(2\)-by-\(2\) matrix \(A\)
such that \(A\ma P_n \\ P_{n+1}\mz=\ma P_{n+1}
\\ P_{n+2}\mz\). (Hint: Try \cref{exr:fib} first.)


\end{Question}
\rule{\textwidth}{0.4pt}


\mks{Total/20}


\end{document}
