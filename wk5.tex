\documentclass{article}
\title{Linear Algebra Worksheet 5}
\author{Jonny Evans}
\date{Workshop 5}
\usepackage[utf8]{inputenc}
\usepackage[T1]{fontenc}
\usepackage[a4paper,margin=1in]{geometry}
\usepackage{longtable}
\usepackage{fixltx2e}
\usepackage{graphicx}
\usepackage{longtable}
\usepackage{float}
\usepackage{parallel}
\usepackage{parcolumns}
\usepackage{wrapfig}
\usepackage{rotating}
\usepackage{amsmath}
\usepackage{textcomp}
\usepackage{marvosym}
\usepackage{wasysym}
\usepackage{amssymb}
\usepackage{amsthm,amsmath,amsfonts,amscd,setspace}
\renewcommand{\partname}{Week}
\newcommand{\mks}[1]{\marginpar{\colorbox{white}{\parbox{\marginparwidth}{%
  \setstretch{0.5}{\bf\scriptsize{#1 marks}}}}}}
\usepackage{tikz}
\usetikzlibrary{decorations.markings,decorations.pathmorphing,shapes}
\usepackage{parskip}
\newcommand{\FF}{\mathfrak{f}}
\newcommand{\GG}{\mathfrak{g}}
\newcommand{\CC}{\mathbb{C}}
\newcommand{\QQ}{\mathbb{Q}}
\newcommand{\RR}{\mathbb{R}}
\newcommand{\UU}{\mathbb{U}}
\newcommand{\XX}{\mathbb{X}}
\newcommand{\YY}{\mathbb{Y}}
\newcommand{\ZZ}{\mathbb{Z}}
\newcommand{\Link}{\operatorname{Link}}
\newcommand{\Cone}{\operatorname{Cone}}
\newcommand{\colim}{\operatorname{colim}}
%\newcommand{\amal}{\operatorname{amal}}
\newcommand{\adj}{\operatorname{adj}}
\newcommand{\cp}[1]{\mathbf{CP}^{#1}}
\newcommand{\rp}[1]{\mathbf{RP}^{#1}}
\newcommand{\OP}[1]{\mathrm{#1}}
\newcommand{\ma}{\begin{pmatrix}}
\newcommand{\mz}{\end{pmatrix}}
\newcommand{\tka}{\begin{center}\begin{tikzpicture}}
\newcommand{\tkz}{\end{tikzpicture}\end{center}}
\newcommand{\matr}[4]{\left(\begin{array}{cc}#1 & #2\\ #3 & #4\end{array}\right)}
\newcommand{\mthrthr}[9]{\left(\begin{array}{ccc}#1 & #2 & #3\\ #4 & #5 & #6\\ #7 & #8 & #9\end{array}\right)}
\newcommand{\vect}[2]{\left(\begin{array}{c}#1\\#2\end{array}\right)}
\newcommand{\vthr}[3]{\left(\begin{array}{c}#1\\#2\\#3\end{array}\right)}
\newcommand{\TO}[3]{#1\stackrel{#2}{\longrightarrow}#3}
\newenvironment{Proof}{\begin{proof}}{\end{proof}\ignorespacesafterend}
\makeatletter
\renewcommand*\env@matrix[1][*\c@MaxMatrixCols c]{%
  \hskip -\arraycolsep
  \let\@ifnextchar\new@ifnextchar
  \array{#1}}
\makeatother

\begingroup
\makeatletter
\@for\theoremstyle:=definition,remark,plain\do{%
\expandafter\g@addto@macro\csname th@\theoremstyle\endcsname{%
\addtolength\thm@preskip\parskip
}%
}
\endgroup
\usepackage{graphicx}
\usepackage[capitalise]{cleveref}
\newtheorem{Theorem}{Theorem}[section]
\newtheorem{Lemma}[Theorem]{Lemma}
\newtheorem{Corollary}[Theorem]{Corollary}
\newtheorem{Proposition}[Theorem]{Proposition}
\theoremstyle{remark}
\newtheorem{Remark}[Theorem]{Remark}
\theoremstyle{definition}
\newtheorem{Definition}[Theorem]{Definition}
\newtheorem{Example}[Theorem]{Example}
\newtheorem{Exercise}[Theorem]{Exercise}
\newtheorem{Question}[Theorem]{Question}
\newtheorem{Solution}[Theorem]{Solution}
\newtheorem{Answer}[Theorem]{Answer}
\crefname{Theorem}{Theorem}{Theorems}
\Crefname{Theorem}{Theorem}{Theorems}
\crefname{Lemma}{Lemma}{Lemmas}
\Crefname{Lemma}{Lemma}{Lemmas}
\crefname{Corollary}{Corollary}{Corollaries}
\Crefname{Corollary}{Corollary}{Corollaries}
\crefname{Claim}{Claim}{Claims}
\Crefname{Claim}{Claim}{Claims}
\crefname{Proposition}{Proposition}{Propositions}
\Crefname{Proposition}{Proposition}{Propositions}
\crefname{Remark}{Remark}{Remarks}
\Crefname{Remark}{Remark}{Remarks}
\crefname{Definition}{Definition}{Definitions}
\Crefname{Definition}{Definition}{Definitions}
\crefname{Example}{Example}{Examples}
\Crefname{Example}{Example}{Examples}
\crefname{Exercise}{Exercise}{Exercises}
\Crefname{Exercise}{Exercise}{Exercises}
\crefname{Question}{Question}{Questions}
\Crefname{Question}{Question}{Questions}
\crefname{Solution}{Solution}{Solutions}
\Crefname{Solution}{Solution}{Solutions}
\crefname{Answer}{Answer}{Answers}
\Crefname{Answer}{Answer}{Answers}
\crefformat{footnote}{#2\footnotemark[#1]#3}

\begin{document}
\maketitle
\setcounter{section}{5}


\begin{Exercise}\label{exr:ellipsoid}
Consider the ellipsoid defined by the equation
\[2(x^2+y^2+z^2-xy-yz)=1.\] Write this equation in the form
\(v^TAv=1\) for \(v=\ma x\\y\\z\mz\) and some matrix \(A\). Find the
eigenvalues of \(A\) and check they are all positive. Find the
principal directions and principal radii of this ellipsoid. (Hint:
Recall from lectures that if the ellipsoid is cut out by the
equation \(v^TAv=1\) then the principal directions are the
eigendirections for \(A\) and the principal radii are
\(1/\sqrt{\lambda}\) for the corresponding eigenvalues \(\lambda\).)


\end{Exercise}
\begin{Exercise}
Let \(a,b\) be two numbers and consider the matrix \(M=\ma a & b
\\ b & a \mz\). What are the eigenvalues and eigenvectors of this
matrix? Suppose that \(a+b>1\) and \(1>a-b>0\). What happens to the
vectors \(M^n\ma 1 \\ 0\mz\) and \(M^n\ma 0 \\ 1\mz\) as
\(n\to\infty\)?


\end{Exercise}
\begin{Exercise}
Find the kernel and nullity for each of the following matrices
\[A=\ma 1 & 0 & 7 \\ 4 & 2 & 1\\ 3 & 2 & -6\mz,\quad B=\ma 2 & 3 & 0
& 1 \\ 1 & 0 & 2 & 0\mz,\quad C=\ma 1 & 0 & -1\mz.\]
What is the rank in each case?


\end{Exercise}
\begin{Exercise}\label{exr:hermitian}
We say that a complex matrix \(A\) is {\em Hermitian} if
\(\bar{A}^T=A\) (here the bar means complex conjugation of each
matrix entry). Show that any eigenvalue of a Hermitian matrix is
real and that if \(v\) and \(w\) are eigenvectors for distinct
eigenvalues \(\lambda\neq\mu\) then \(\bar{v}^Tw=0\). {\em These
observations are important in quantum mechanics, because eigenvalues
of Hermitian operators are what we measure as observable quantities
like energy.}


\end{Exercise}
\begin{Exercise}
Find the limit \(\lim_{n\to\infty}\frac{P_{n+1}}{P_n}\) where
\(P_n\) is the sequence of Pell numbers (defined by the recurrence
\(P_{n+2}=2P_{n+1}+P_n\), \(P_1=0\), \(P_2=1\)).




\end{Exercise}
\begin{Exercise}\label{exr:companion}
Find the characteristic polynomials of the matrices: \[\ma 0 & -c_0
\\ 1 & -c_1\mz,\quad\ma 0 & 0 & -c_0 \\ 1 & 0 & -c_1 \\ 0 & 1 &
-c_2\mz,\quad\ma 0 & 0 & 0 & -c_0 \\ 1 & 0 & 0 & -c_1 \\ 0 & 1 & 0 &
-c_2 \\ 0 & 0 & 1 & -c_3\mz.\] Have a guess at the characteristic
polynomial of
\[\ma 0 & 0 & \cdots & 0 & -c_0 \\ 1 & \ddots & \ddots & \vdots &
\vdots\\ 0 & \ddots & \ddots & 0 & -c_{n-3} \\ \vdots & \ddots & 1 &
0 & -c_{n-2} \\ 0 & \cdots & 0 & 1 & -c_{n-1}\mz.\] Prove your
guess by induction.


\end{Exercise}
\begin{Exercise}\label{exr:companionrevisited}
Solve \cref{exr:companion} using as many different methods to
compute determinants as you can.




\end{Exercise}
\newpage


\section{Challenge problems}


\begin{Exercise}\label{exr:vandermonde}
Given numbers \(\lambda_1,\ldots,\lambda_n\), the matrix \[V=\ma 1 &
\lambda_1 & \lambda_1^2 & \cdots & \lambda_1^{n-1} \\ 1 & \lambda_2
& \lambda_2^2 & \cdots & \lambda_2^{n-1} \\ \vdots & \vdots & &
\vdots \\ 1 & \lambda_n & \lambda_n^2 & \cdots & \lambda_n^{n-1}
\mz\] is called the {\em Vandermonde matrix}. Consider the quantity
\[Q=\prod_{1\leq i<j\leq n}(\lambda_j-\lambda_i)\] (for example, if
\(n=3\),
\(Q=(\lambda_3-\lambda_2)(\lambda_3-\lambda_1)(\lambda_2-\lambda_1)\)). Verify
that \(\det(V)=Q\) for \(n=2\) and \(n=3\). Show that both
\(\det(V)\) and \(Q\) vanish if \(\lambda_k=\lambda_\ell\) for some
\(k\neq \ell\).


{\em In fact, \(\det(V)=Q\) for all \(n\).}
\end{Exercise}
\begin{Exercise}\label{exr:companionrerevisited}
Let \(C=\ma 0 & 0 & \cdots & 0 & -c_0 \\ 1 & \ddots & \ddots &
\vdots & \vdots\\ 0 & \ddots & \ddots & 0 & -c_{n-3} \\ \vdots &
\ddots & 1 & 0 & -c_{n-2} \\ 0 & \cdots & 0 & 1 & -c_{n-1}\mz\),
(see \cref{exr:companion} for its characteristic polynomial!). Show
that if \(\lambda\) is an eigenvalue of \(C\) then the row vector
\(w=\ma 1 & \lambda & \lambda^2 & \cdots & \lambda^{n-1}\mz\)
satisfies \(wC=\lambda w\). Deduce that if \(C\) has distinct
eigenvalues \(\lambda_1,\ldots,\lambda_n\) and \(V\) is the
Vandermonde matrix from \cref{exr:vandermonde} whose \(i\)th row is
\(\ma 1 & \lambda_i & \lambda_i^2 & \cdots & \lambda_i^{n-1}\mz\)
then \(VCV^{-1}\) is the diagonal matrix \(\ma \lambda_1 & & \\ &
\ddots & \\ & & \lambda_n\mz\). Hence prove that
\(Tr(C^m)=\sum_{i=1}^n\lambda_i^m\) for all \(m\).


\end{Exercise}
\begin{Exercise}\label{exr:coeffs}
If \(t^n+c_{n-1}t^{n-1}+\cdots+c_0\) is a polynomial of degree \(n\)
with leading coefficient \(1\) and distinct roots, show that the sum
of squares of its roots is \(c_{n-1}^2-2c_{n-2}\) and find a similar
expression for the sum of cubes of the roots. (Hint: Use
\cref{exr:companionrerevisited})


\end{Exercise}
\begin{Exercise}
Given a polynomial \(p(z)=p_mz^m+\cdots+p_0\) of degree \(m\) and a
polynomial \(q(z)=q_nz^n+\cdots+q_0\) of degree \(n\), the {\em
resultant} of \(p\) and \(q\) is defined to be the determinant of
the \((m+n)\)-by-\((m+n)\) matrix \[\ma p_m & p_{m-1} & \cdots & p_1
& p_0 & 0 & 0 & \cdots & 0 \\ 0 & p_m & p_{m-1} & \cdots & p_1 & p_0
& 0 & \ddots & \vdots \\ 0 & 0 & p_m & p_{m-1} & \cdots & p_1 & p_0
& \ddots & 0 \\ \vdots & \vdots & \ddots & \ddots & \ddots & & &
\ddots & 0 \\ 0 & \cdots & \cdots & 0 & p_m & p_{m-1} & \cdots & p_1
& p_0 \\ q_n & q_{n-1} & \cdots & q_0 & 0 & 0 & 0 & \cdots & 0 \\ 0
& q_m & q_{m-1} & \cdots & q_0 & 0 & 0 & \ddots & \vdots \\ 0 & 0 &
q_m & q_{m-1} & \cdots & q_0 & 0 & \ddots & 0 \\ \vdots & \vdots &
\ddots & \ddots & \ddots & & & \ddots & \vdots
\\ \vdots&&&&&\ddots&&\ddots&0\\ 0 & \cdots & \cdots & 0 & 0 & q_m &
q_{m-1} & \cdots & q_0 \mz\] It is a theorem of Sylvester that the
resultant of \(p\) and \(q\) vanishes if and only if \(p\) and \(q\)
have a common factor.


Write this matrix out in the case when \(m=2\), \(n=1\) and verify
that the resultant vanishes if and only if there exists a polynomial
\(a+bz\) such that \(p(z)=q(z)(a+bz)\)).


\end{Exercise}
\newpage


\section{Solutions}


\begin{Solution}
The ellipsoid equation can be written as \(v^TAv=1\) for \[A=\ma 2 &
-1 & 0 \\ -1 & 2 & -1 \\ 0 & -1 & 2\mz.\] The eigenvalues of \(A\)
are the solutions to \(\det(A-tI)=0\), i.e. \(2,2\pm\sqrt{2}\). The
eigenvectors are \(\ma 1 \\ 0 \\ -1\mz\), \(\ma 1 \\ \mp\sqrt{2}
\\ 1\mz\): these are the principal directions. The principal radii
are \(\frac{1}{\sqrt{2}},\frac{1}{\sqrt{2\pm\sqrt{2}}}\).


\end{Solution}
\begin{Solution}
The eigenvalues of \(\ma a & b \\ b & a\mz\) are the solutions of
\((a-t)^2-b^2=0\), i.e. \(t^2-2at+a^2-b^2=0\). These solutions are
\(\frac{2a\pm\sqrt{4a^2-4a^2+4b^2}}{2}=a\pm b\). The corresponding
eigenvectors are the solutions of
\begin{align*}
ax+by&=(a\pm b)x\\
bx+ay&=(a\pm b)y,
\end{align*}
which means \(y=\pm x\). Therefore the eigenvectors are multiples of
\(\ma 1 \\ \pm 1\mz\).


We have \(\ma 1 \\ 0 \mz=\frac{1}{2}\ma 1 \\ 1\mz+\frac{1}{2}\ma 1
\\ -1\mz\), so \[M^n\ma 1 \\ 0\mz=\frac{1}{2}M^n\ma 1 \\ 1
\mz+\frac{1}{2}M^n\ma 1 \\ -1\mz=\frac{1}{2}(a+b)^n\ma 1
\\ 1\mz+\frac{1}{2}(a-b)^n\ma 1 \\ -1\mz,\] so \(M^n\ma 1 \\ 0 \mz\)
gets closer and closer to the line \(y=x\) as \(n\to\infty\)
(because \((a+b)^n\to\infty\) as \(a+b>1\) and \((a-b)^n\to 0\) as
\(0<a-b<1\); see the red vectors in the figure below). Similarly
\(M^n\ma 0 \\ 1\mz\) gets closer to this line (blue vectors in the
figure below).


\tka
\draw[->,thick] (0,0) -- (2,2);
\draw[->,thick] (0,0) -- (2,-2);
\draw[->,red] (0,0) -- (2,0);
\draw[->,red] (0,0) -- (4,2);
\draw[->,red] (0,0) -- (10/2,8/2);
\draw[dotted,red] (10/2,8/2) -- (10/1.7,8/1.7);
\draw[->,red] (0,0) -- (14/3,13/3);
\draw[dotted,red] (14/3,13/3) -- (14/2.5,13/2.5);
\draw[->,blue] (0,0) -- (0,2);
\draw[->,blue] (0,0) -- (2,4);
\draw[->,blue] (0,0) -- (8/2,10/2);
\draw[dotted,blue] (8/2,10/2) -- (8/1.7,10/1.7);
\draw[->,blue] (0,0) -- (13/3,14/3);
\draw[dotted,blue] (13/3,14/3) -- (13/2.5,14/2.5);
\tkz


\end{Solution}
\begin{Solution}
The equation \(Av=0\) is
\begin{align*}
x+7z&=0\\
4x+2y+z&=0\\
3x+2y-6z&=0
\end{align*}
which has solution \(x=-7z\), \(y=27z/2\), so the kernel consists of
vectors of the form \(\ma -7z \\ 27z/2 \\ z\mz\). The nullity is 1
in this instance. By the rank-nullity theorem, the rank is 2.


The equation \(Bv=0\) is
\begin{align*}
2x+3y+w&=0\\
x+2z&=0,
\end{align*}
which has solution \(x=-2z\), \(w=4z-3y\), so the kernel consists of
vectors of the form \(\ma -2z \\ y \\ z \\ 4z-3y\mz\). The nullity
is 2 in this instance. By the rank-nullity theorem, the rank is 2.


The equation \(Cv=0\) is
\[x-z=0,\]
which has solution \(z=x\), so the kernel consists of vectors of the
form \(\ma x \\ y \\ x\mz\). The nullity in this instance is 2. By
the rank-nullity theorem, the rank is 1.


\end{Solution}
\begin{Solution}[Solution to \cref{exr:hermitian}]\label{sol:exr:hermitian}
Suppose that \(Av=\lambda v\). Consider the expression
\(\bar{v}^TAv\), where \(\bar{v}\) denotes complex
conjugation. Then, because \(A=\bar{A}^T\), we have
\[\bar{\lambda}\bar{v}^Tv=(\overline{Av})^{T}v=\bar{v}^TAv=\lambda\bar{v}^Tv.\]
Note that if \(v=\ma x_1\\ \vdots\\ x_n\mz\) then
\(\bar{v}^Tv=\sum |x_1|^2+\cdots+|x_n|^2>0\) if \(v\neq 0\), so
dividing through by \(\bar{v}^Tv\) we get \(\bar{\lambda}=\lambda\)
and deduce that \(\lambda\) is real.


If \(v\) and \(w\) are two eigenvectors for distinct eigenvalues
\(\lambda,\mu\) then
\begin{align*}
\lambda \bar{w}^Tv&=\bar{w}^T(Av)\\
&=\overline{(Aw)}^Tv\\
&=\mu \bar{w}^Tv\\
\end{align*}
(since \(\bar{\mu}=\mu\) by the first part) so, since
\(\lambda\neq\mu\), we must have \(\bar{w}^Tv=0\). \qedhere


\end{Solution}
\begin{Solution}
The Pell numbers satisfy \[\ma P_{n+2} \\ P_{n+1}\mz=\ma 0 & 1 \\ 1
& 2\mz\ma P_{n+1} \\ P_n\mz.\] (Let's write \(M\) for the matrix in
this expression). The eigenvalues of \(M\) are the solutions of
\(0=\det(M-tI)\), i.e. \[0=-t(2-t)-1=t^2-2t-1,\] which are
\(\frac{2\pm\sqrt{4+4}}{2}=1\pm\sqrt{2}\). The eigenvectors are
\(\ma 1 \\ 1\pm\sqrt{2}\mz\). This means that
\(\lim_{n\to\infty}\frac{P_{n+1}}{P_n}=1+\sqrt{2}\) (by the same
argument that we used for the Fibonacci numbers in lectures).


\end{Solution}
\begin{Solution}
We have
\begin{align*}
\det\ma -t & -c_0 \\ 1 & -c_1-t\mz=t^2+c_1t+c_0\\
\det\ma -t & 0 & -c_0 \\ 1 & -t & -c_1 \\ 0 & 1 & -c_2-t\mz&=-(t^3+c_2t^2+c_1t+c_0)\\
\det\ma -t & 0 & 0 & -c_0 \\ 1 & -t & 0 & -c_1 \\ 0 & 1 & -t & -c_2 \\ 0 & 0 & 1 & -c_3-t\mz&=t^4+c_3t^3+c_2t^2+c_1t+c_0.
\end{align*}
More generally, we have \[\det\ma -t & 0 & \cdots & 0 & -c_0 \\ 1 &
\ddots & \ddots & \vdots & \vdots\\ 0 & \ddots & \ddots & 0 &
-c_{n-3} \\ \vdots & \ddots & 1 & -t & -c_{n-2} \\ 0 & \cdots & 0 &
1 & -c_{n-1}-t\mz=(-1)^n(t^n+c_{n-1}t^{n-1}+\cdots+c_0).\] To prove
this by induction, assume it's true for \(n-1\) and let's evaluate
the determinant \[D_n:=\det\ma -t & 0 & \cdots & 0 & -c_0 \\ 1 &
\ddots & \ddots & \vdots & \vdots\\ 0 & \ddots & \ddots & 0 &
-c_{n-3} \\ \vdots & \ddots & 1 & -t & -c_{n-2} \\ 0 & \cdots & 0 &
1 & -c_{n-1}-t\mz.\] Expanding along the first column, we get
\[-tD_{n-1}-\det\ma 0 & \cdots & \cdots & 0 & -c_0 \\ 1 & -t & & 0 &
-c_1 \\ 0 & 1 & \ddots & & \vdots \\ \vdots & \ddots & \ddots & -t &
-c_{n-2} \\ 0 & \cdots & 0 & 1 & -c_{n-1}-t\mz\] We evaluate this
final determinant as follows: pick \(c_0\) from the top row
(everything else is zero); pick either \(1\) or \(-t\) from the
second row; if you picked \(1\) you can pick either \(1\) or \(-t\)
from the third row whereas if you picked \(-t\) you have to pick
\(-t\) from the third row; continue in this manner and you either
pick all the \(1\)s or else you pick \(-t\) until you reach the last
row and you're forced to pick a zero. Therefore the only term we get
is \(c_0\) (with a sign). The sign is \((-1)^{n+1}\) as we can see
because this permutation is \((123\cdots n)\). Therefore
\[D_n=-tD_{n-1}-(-1)^{n+1}c_0=(-1)^n(t^n+c_{n-1}t^{n-1}+\cdots+c_0)\]
by inductive hypothesis (i.e. substituting in what we assumed was
the formula for \(D_{n-1}\)). We already checked the base case
\(D_2\), so we're done.


\end{Solution}
\section{Solutions to challenge problems}


\begin{Solution}
We have \[\det\ma 1 & \lambda_1 \\ 1 &
\lambda_2\mz=\lambda_2-\lambda_2.\] Next: \[\det\ma 1 & \lambda_1 &
\lambda_1^2 \\ 1 & \lambda_2 & \lambda_2^2 \\ 1 & \lambda_3 &
\lambda_3^2\mz=\lambda_2\lambda_3^2-\lambda_2^2\lambda_3-\lambda_1(\lambda_3^2-\lambda_2^2)+\lambda_1^2(\lambda_3-\lambda_2)\]
i.e. \(\lambda_2\lambda_3(\lambda_3-\lambda_2)-\lambda_1(\lambda_3-\lambda_2)(\lambda_3+\lambda_2)+\lambda_1^2(\lambda_3-\lambda_2)\),
so we can pull out a factor of \(\lambda_3-\lambda_2\), leaving
\[\lambda_2\lambda_3-\lambda_1\lambda_3-\lambda_1\lambda_2+\lambda_1^2=(\lambda_3-\lambda_1)(\lambda_2-\lambda_1)\]
and again we get \(\det(V)=Q\) as required.


If \(\lambda_k=\lambda_{\ell}\) then the Vandermonde determinant
vanishes (two rows coincide) and the expression \(Q\) vanishes
(because \(\lambda_k-\lambda_{\ell}\) is a factor.


\end{Solution}
\begin{Solution}
We saw that the characteristic polynomial of \(C\) is
\(\pm(t^n+c_{n-1}t^{n-1}+\cdots+c_1t+c_0\), so the eigenvalues of
\(C\) are the roots \(\lambda_1,\ldots,\lambda_n\) of this
polynomial. If we form the corresponding Vandermonde matrix whose
\(i\)th row is given by powers of \(\lambda_i\) then we see that
\begin{align*}\ma 1 & \lambda_i & \lambda_i^2 &\cdots
&\lambda_i^{n-1}\mz C&=\ma \lambda_i & \lambda_i^2 & \cdots &
\lambda_i^{n-1} &
-c_0-c_1\lambda_i-\cdots-c_{n-1}\lambda_i^{n-1}\mz\\
&=\ma \lambda_i & \lambda_i^2 & \cdots & \lambda_i^{n-1} &
\lambda_i^n\mz\\
&=\lambda\ma 1 & \lambda & \cdots &\lambda^{n-1}\mz.\end{align*}
Therefore the \(i\)th row of \(VC\) is \(\lambda_i\) times the
\(i\)th row of \(V\). Let \(R_i\) be the \(i\)th row of \(V\) and
\(C_i\) be the \(i\)th columns of \(V^{-1}\) (so
\(R_iC_j=\delta_{ij}\)). Then \(VC\) has rows \(\lambda_i R_i\), so
the \(ij\) entry of \(VCV^{-1}\) is
\(\lambda_iR_iC_j=\lambda_i\delta_{ij}\), and we see that
\(VCV^{-1}\) is the diagonal matrix with
\(\lambda_1,\ldots,\lambda_n\) down the diagonal.


We know that
\(\mathrm{Tr}((VCV^{-1})^m)=\mathrm{Tr}(VC^mV^{-1})=\mathrm{Tr}(V^{-1}VC^m)=\mathrm{Tr}(C^m)\)
because \(\mathrm{Tr}(AB)=\mathrm{Tr}(BA)\) from sheet 1. Therefore
\(\mathrm{Tr}(C^m)=\mathrm{Tr}(diag(\lambda_1^m,\ldots,\lambda_n^m))=\sum\lambda_i^m\).


\end{Solution}
\begin{Solution}
By the previous question, we know that the sum of squares of the
roots of this polynomial is \(\mathrm{Tr}(C^2)\). The only nonzero
diagonal entries of \(C^2\) are the final two, which are
\(-c_{n-2}\) and \(c_{n-1}^2-c_{n-2}\), so we get
\(c_{n-1}^2-2c_{n-2}\) as required. For the sum of cubes, we compute
\(\mathrm{Tr}(C^3)\) and we get
\(-c_{n-1}^3+3c_{n-2}c_{n-1}-3c_{n-3}\).


\end{Solution}
\begin{Solution}
In the case \(m=2,n=1\) the Sylvester matrix is
\[\ma p_2 & p_1 & p_0 \\ q_1 & q_0 & 0 \\ 0 & q_1 & q_0\mz,\]
whose determinant is \(p_2q_0^2-p_1q_1q_0+p_0q_1^2\).


If there exist \(a,b\) such that \(p=(a+bz)q\) then \[p_2=bq_1,\quad
p_1=aq_1+bq_0,\quad p_0=aq_0,\] so the determinant is
\[bq_1q_0^2-(aq_1+bq_0)q_1q_0+aq_0q_1^2=0.\] Conversely, suppose
that \(p_2q_0^2-p_1q_1q_0+p_0q_1^2=0\). Note that \(q_1\neq 0\) or
else \(q\) has degree less than \(1\).
\begin{itemize}
\item If \(q_0=0\) then \(p_0q_1^2=0\), so \(p_0=0\) and
\(z|p(z)\). Since \(q_0=0\), \(q(z)=q_1z\), so \(z|q(z)\) and we
have found a common factor.
\item if \(q_0\neq 0\) then consider the polynomial
\((p_2/q_1)z+(p_0/q_0)\). We have
\[(p_2/q_1)z+(p_0/q_0)(q_1z+q_0)=p_2z^2+\left(\frac{p_2q_0}{q_1}+\frac{p_0q_1}{q_0}\right)+p_0\]
and the term in brackets is equal to \(p_1\) because
\(p_2q_0^2+p_0q_1^2=p_1q_0q_1\). Therefore \(q\) is a common
factor of \(p\) and \(q\).
\end{itemize}
\end{Solution}
\end{document}
