\documentclass{article}
\title{Linear Algebra Worksheet 2}
\author{Jonny Evans}
\date{Workshop 2}
\usepackage[utf8]{inputenc}
\usepackage[T1]{fontenc}
\usepackage[a4paper,margin=1in]{geometry}
\usepackage{longtable}
\usepackage{fixltx2e}
\usepackage{graphicx}
\usepackage{longtable}
\usepackage{float}
\usepackage{parallel}
\usepackage{parcolumns}
\usepackage{wrapfig}
\usepackage{rotating}
\usepackage{amsmath}
\usepackage{textcomp}
\usepackage{marvosym}
\usepackage{wasysym}
\usepackage{amssymb}
\usepackage{amsthm,amsmath,amsfonts,amscd,setspace}
\renewcommand{\partname}{Week}
\newcommand{\mks}[1]{\marginpar{\colorbox{white}{\parbox{\marginparwidth}{%
  \setstretch{0.5}{\bf\scriptsize{#1 marks}}}}}}
\usepackage{tikz}
\usetikzlibrary{decorations.markings,decorations.pathmorphing,shapes}
\usepackage{parskip}
\newcommand{\FF}{\mathfrak{f}}
\newcommand{\GG}{\mathfrak{g}}
\newcommand{\CC}{\mathbb{C}}
\newcommand{\QQ}{\mathbb{Q}}
\newcommand{\RR}{\mathbb{R}}
\newcommand{\UU}{\mathbb{U}}
\newcommand{\XX}{\mathbb{X}}
\newcommand{\YY}{\mathbb{Y}}
\newcommand{\ZZ}{\mathbb{Z}}
\newcommand{\Link}{\operatorname{Link}}
\newcommand{\Cone}{\operatorname{Cone}}
\newcommand{\colim}{\operatorname{colim}}
%\newcommand{\amal}{\operatorname{amal}}
\newcommand{\adj}{\operatorname{adj}}
\newcommand{\cp}[1]{\mathbf{CP}^{#1}}
\newcommand{\rp}[1]{\mathbf{RP}^{#1}}
\newcommand{\OP}[1]{\mathrm{#1}}
\newcommand{\ma}{\begin{pmatrix}}
\newcommand{\mz}{\end{pmatrix}}
\newcommand{\tka}{\begin{center}\begin{tikzpicture}}
\newcommand{\tkz}{\end{tikzpicture}\end{center}}
\newcommand{\matr}[4]{\left(\begin{array}{cc}#1 & #2\\ #3 & #4\end{array}\right)}
\newcommand{\mthrthr}[9]{\left(\begin{array}{ccc}#1 & #2 & #3\\ #4 & #5 & #6\\ #7 & #8 & #9\end{array}\right)}
\newcommand{\vect}[2]{\left(\begin{array}{c}#1\\#2\end{array}\right)}
\newcommand{\vthr}[3]{\left(\begin{array}{c}#1\\#2\\#3\end{array}\right)}
\newcommand{\TO}[3]{#1\stackrel{#2}{\longrightarrow}#3}
\newenvironment{Proof}{\begin{proof}}{\end{proof}\ignorespacesafterend}
\makeatletter
\renewcommand*\env@matrix[1][*\c@MaxMatrixCols c]{%
  \hskip -\arraycolsep
  \let\@ifnextchar\new@ifnextchar
  \array{#1}}
\makeatother

\begingroup
\makeatletter
\@for\theoremstyle:=definition,remark,plain\do{%
\expandafter\g@addto@macro\csname th@\theoremstyle\endcsname{%
\addtolength\thm@preskip\parskip
}%
}
\endgroup
\usepackage{graphicx}
\usepackage[capitalise]{cleveref}
\newtheorem{Theorem}{Theorem}[section]
\newtheorem{Lemma}[Theorem]{Lemma}
\newtheorem{Corollary}[Theorem]{Corollary}
\newtheorem{Proposition}[Theorem]{Proposition}
\theoremstyle{remark}
\newtheorem{Remark}[Theorem]{Remark}
\theoremstyle{definition}
\newtheorem{Definition}[Theorem]{Definition}
\newtheorem{Example}[Theorem]{Example}
\newtheorem{Exercise}[Theorem]{Exercise}
\newtheorem{Question}[Theorem]{Question}
\newtheorem{Solution}[Theorem]{Solution}
\newtheorem{Answer}[Theorem]{Answer}
\crefname{Theorem}{Theorem}{Theorems}
\Crefname{Theorem}{Theorem}{Theorems}
\crefname{Lemma}{Lemma}{Lemmas}
\Crefname{Lemma}{Lemma}{Lemmas}
\crefname{Corollary}{Corollary}{Corollaries}
\Crefname{Corollary}{Corollary}{Corollaries}
\crefname{Claim}{Claim}{Claims}
\Crefname{Claim}{Claim}{Claims}
\crefname{Proposition}{Proposition}{Propositions}
\Crefname{Proposition}{Proposition}{Propositions}
\crefname{Remark}{Remark}{Remarks}
\Crefname{Remark}{Remark}{Remarks}
\crefname{Definition}{Definition}{Definitions}
\Crefname{Definition}{Definition}{Definitions}
\crefname{Example}{Example}{Examples}
\Crefname{Example}{Example}{Examples}
\crefname{Exercise}{Exercise}{Exercises}
\Crefname{Exercise}{Exercise}{Exercises}
\crefname{Question}{Question}{Questions}
\Crefname{Question}{Question}{Questions}
\crefname{Solution}{Solution}{Solutions}
\Crefname{Solution}{Solution}{Solutions}
\crefname{Answer}{Answer}{Answers}
\Crefname{Answer}{Answer}{Answers}
\crefformat{footnote}{#2\footnotemark[#1]#3}

\begin{document}
\maketitle
\setcounter{section}{2}


Here is a list \(\mathcal{V}\) of vectors \[u=\ma 1 \\ 1 \mz,\qquad
v=\ma 1 \\ 2\mz,\qquad w=\ma -3 \\ 1 \\ 2\mz,\qquad \xi=\ma 0 \\ 1
\\ -1\mz.\]


\begin{Exercise}\label{exr:orth}
For every vector in \(\mathcal{V}\), find its length and write down
a vector orthogonal to it.


\end{Exercise}
\begin{Exercise}\label{ex:dotprods}
Find the angle between \(u\) and \(v\). Find the angle between \(w\)
and \(\xi\).


\end{Exercise}
Here is a list \(\mathcal{M}\) of matrices. \[A=\ma \frac{1}{2} & 0 &
\frac{1}{\sqrt{2}} \\ 0 & 1 & 0 \\ \frac{1}{2} & 0 &
-\frac{1}{\sqrt{2}}\mz,\quad B=\ma \frac{3}{4} & \frac{1}{4} &
\frac{1}{2}\sqrt{\frac{3}{2}} \\ \frac{1}{4} & \frac{3}{4} &
-\frac{1}{2}\sqrt{\frac{3}{2}} \\ -\frac{1}{2}\sqrt{\frac{3}{2}} &
\frac{1}{2}\sqrt{\frac{3}{2}} & \frac{1}{2} \mz,\quad C=\ma
\frac{1}{3} & \frac{1}{3}+\frac{1}{\sqrt{3}} &
-\frac{1}{3}+\frac{1}{\sqrt{3}} \\ \frac{1}{3}-\frac{1}{\sqrt{3}} &
\frac{1}{3} &
-\frac{1}{3}-\frac{1}{\sqrt{3}}\\ -\frac{1}{3}-\frac{1}{\sqrt{3}} &
-\frac{1}{3}+\frac{1}{\sqrt{3}} & \frac{1}{3} \mz.\]


\begin{Exercise}\label{exr:orthogmat}
Which matrices \(M\in\mathcal{M}\) are orthogonal matrices? (Hint:
There should be two!)


\end{Exercise}
\begin{Exercise}\label{exr:3drot}
The orthogonal matrices from \(\mathcal{M}\) are actually rotation
matrices. In each case, find the axis and angle of rotation.


\end{Exercise}
Here is a list \(\mathcal{N}\) of matrices
\[D=\ma 1 & 2 & 0 & 1 & 3 \\ 0 & 0 & 1 & 2 & 1\mz,\quad E=\ma 0 & 1 & 0
& 2 \\ 1 & 0 & 0 & 0 \\ 0 & 0 & 0 & 0\mz,\qquad F=\ma 2 & 2 & 3 \\ 0 & 1
& 1 \\ 0 & 0 & 1\mz,\qquad G=\ma 1 & 0 & 8 \\ 0 & 1 & 2 \\ 0 & 0 & 0\mz.\]


\begin{Exercise}\label{exr:echform}
Which of the matrices \(N\in\mathcal{N}\) are in echelon form? Which
are in reduced echelon form?


\end{Exercise}
\begin{Exercise}\label{exr:simeqech}
For each \(N\in\mathcal{N}\) which is in reduced echelon form, state
(a) for which vectors \(b\) the equation \(Nv=b\) has a solution and
(b) the dimension of the space of solutions to \(Nv=b\), assuming
that \(b\) is chosen so that there is a solution.


\end{Exercise}
\begin{Exercise}\label{exr:simeq}
For each system of simultaneous equations below, write it in matrix
form, put the augmented matrix into reduced echelon form using row
operations. Determine if the system has a solution and, if it does,
give the general solution.


\begin{center}
\begin{tabular}{p{3cm}|p{3cm}|p{3cm}}
{\begin{align*}x+y+2z+3w&=0\\ y+4z-w&=1\end{align*}} &
{\begin{align*}x&=y-3\\ 2x+y&=6\\ y-3x&=1\end{align*}} &
{\begin{align*} 4x-w&=0\\ 3y-2z+w&=4\\ 4x-2y+4z-3w&=0\\ 3x+y-z&=2 \end{align*}}
\end{tabular}
\end{center}


\end{Exercise}
\begin{Exercise}
Put the following matrices into reduced echelon form using row
operations. In each case, what is the number of free indices?
\[X=\ma 1 & 1 & 0 & 0 & 0 \\ 1 & 1 & 1 & 0 & 0 \\ 1 & 1 & 1 & 1 & 0
\\ 1 & 1 & 1 & 1 & 1 \mz,\quad Y=\ma 2 & 0 & 1 \\ 0 & 1 & 0 \\ 1 & 3
& -1\mz,\quad Z=\ma 1 & 2 & 3 & 4 \\ 5 & 6 & 7 & 8 \\ 9 & 10 & 11 &
12\mz.\]


\end{Exercise}
\newpage


\begin{Exercise}\label{exr:assoc}
Let \(A,B,C\) be \(m\)-by-\(n\), \(n\)-by-\(p\) and \(p\)-by-\(q\)
matrices respectively. Write out the matrix products \(A(BC)\) and
\((AB)C\) in index notation and check that they give the same answer
(this shows that matrix multiplication is associative).


\end{Exercise}
\begin{Exercise}\label{exr:orthogmat}
Suppose that \(A\) is an \(n\)-by-\(n\) matrix whose columns are the
vectors \(v_1,\ldots,v_n\). Show that \(A\) is an orthogonal matrix
(i.e. \(A^TA=I\)) if and only if \[v_i\cdot v_j=\begin{cases}1\mbox{
if }i=j\\0\mbox{ if }i\neq j\end{cases}\mbox{ for all }i,j.\] In
other words, the columns of \(A\) are orthogonal to one another
(this is where the name ``orthogonal matrix'' comes from).


\end{Exercise}
\begin{Exercise}\label{exr:symmantisymm}
We say that a matrix \(M\) is {\em symmetric} if \(M^T=M\) and {\em
antisymmetric} if \(M^T=-M\).
\begin{enumerate}
\item Show that if \(N\) is an \(m\)-by-\(n\) matrix then \(MM^T\) is a
symmetric \(m\)-by-\(m\) matrix and \(M^TM\) is a symmetric
\(n\)-by-\(n\) matrix.
\item Show that, given any \(n\)-by-\(n\)
matrix \(C\), the matrix \(A=C+C^T\) is symmetric and the matrix
\(B=C-C^T\) is antisymmetric. Deduce that \(C\) can be written as
the sum of a symmetric and an antisymmetric matrix (called the {\em
symmetric} and {\em antisymmetric} parts of \(C\) respectively).


\end{enumerate}
\end{Exercise}
\begin{Exercise}\label{exr:determined}
A system of \(m\) equations in \(n\) unknowns is called {\em
underdetermined} if \(m<n\) and overdetermined if \(m>n\). As rules
of thumb, underdetermined equations tend to have general solutions
with \(m-n\) free parameters, and overdetermined equations tend to
have no solutions. Give counterexamples to these rules of thumb
(e.g. an underdetermined system with no solutions and an
overdetermined system with a solution).


\end{Exercise}
\newpage


\section{Solutions}


\begin{Solution}\label{sol:exr:orth}
Any of the following are correct, but there are many possible
answers (just check that the dot product with the original vector is
zero): \[\ma 1 \\ -1\mz,\quad \ma -2 \\ 1\mz,\quad \ma -1 \\ -3
\\ 0\mz, \quad \ma 1 \\ 0 \\ 0\mz.\]
We have \(|u|=\sqrt{2}\), \(|v|=\sqrt{5}\), \(|w|=\sqrt{14}\),
\(|\xi|=\sqrt{2}\).


\end{Solution}
\begin{Solution}
We have \(u\cdot v=3\), \(|u|=\sqrt{2}\), \(|v|=\sqrt{5}\), so the
angle between \(u\) and \(v\) is \(\cos^{-1}(3/\sqrt{10})\approx
0.321750554\) radians.


We have \(w\cdot\xi=1\), \(|w|=\sqrt{14}\) and \(|\xi|=\sqrt{2}\),
so the angle between \(w\) and \(\xi\) is
\(\cos^{-1}(1/\sqrt{28})\approx 1.38067072\) radians.


\end{Solution}
\begin{Solution}
The matrices \(B,C\) are orthogonal; \(A\) is not. You can check
this explicitly: \(A^TA=\ma \frac{1}{2} & 0 & 0 \\ 0 & 1 & 0 \\ 0 &
0 & 1\mz\), is not too hard; more of a pain
is: \begin{gather*}B^TB=\ma \frac{3}{4} & \frac{1}{4} &
-\frac{1}{2}\sqrt{\frac{3}{2}} \\ \frac{1}{4} & \frac{3}{4} &
\frac{1}{2}\sqrt{\frac{3}{2}} \\ \frac{1}{2}\sqrt{\frac{3}{2}} &
-\frac{1}{2}\sqrt{\frac{3}{2}} & \frac{1}{2} \mz\ma \frac{3}{4} &
\frac{1}{4} & \frac{1}{2}\sqrt{\frac{3}{2}} \\ \frac{1}{4} &
\frac{3}{4} & -\frac{1}{2}\sqrt{\frac{3}{2}}
\\ -\frac{1}{2}\sqrt{\frac{3}{2}} & \frac{1}{2}\sqrt{\frac{3}{2}} &
\frac{1}{2} \mz\\
=\ma\frac{9}{16}+\frac{1}{16}+\frac{3}{8} &
\frac{3}{16}+\frac{3}{16}-\frac{3}{8} &
\left(\frac{3}{8}-\frac{1}{8}+\frac{1}{4}\right)\sqrt{\frac{3}{2}}
\\ \frac{3}{16}+\frac{3}{16}-\frac{3}{8} &
\frac{1}{16}+\frac{9}{16}+\frac{3}{8} &
\left(\frac{1}{8}-\frac{3}{8}+\frac{1}{4}\right)\sqrt{\frac{3}{2}}
\\ \left(\frac{3}{8}-\frac{1}{8}+\frac{1}{4}\right)\sqrt{\frac{3}{2}}
& \left(\frac{1}{8}-\frac{3}{8}+\frac{1}{4}\right)\sqrt{\frac{3}{2}}
& \frac{3}{8}+\frac{3}{8}+\frac{1}{4}\mz =\ma 1 & 0 & 0 \\ 0 & 1 & 0
\\ 0 & 0 & 1\mz,\end{gather*} and similarly \(C^TC=I\).


\end{Solution}
\begin{Solution}
For \(B\): A vector \(v=\ma x \\ y\\ z\mz\) pointing along the axis
solves \(Bv=v\). This means
\begin{align*}
\frac{3x}{4}+\frac{y}{4}+\frac{1}{2}\sqrt{\frac{3}{2}}z&=x\\
\frac{x}{4}+\frac{3y}{4}-\frac{1}{2}\sqrt{\frac{3}{2}}z&=y\\
\frac{1}{2}\left(z-\sqrt{\frac{3}{2}}x+\sqrt{\frac{3}{2}}y\right)&=z,
\end{align*}
so \(x-y=2z\sqrt{\frac{3}{2}}\) and
\(x-y=z\sqrt{\frac{2}{3}}\). This implies that \(z=0\) and
\(x=y\). Therefore the axis is \(\ma 1 \\ 1 \\ 0\mz\). If we pick
\(v=\ma 0 \\ 0 \\ 1\mz\) (orthogonal to the axis), it goes to
\(Bv=\ma \frac{1}{2}\sqrt{\frac{3}{2}}
\\ -\frac{1}{2}\sqrt{\frac{3}{2}} \\ \frac{1}{2}\mz\), and \(v\cdot
Bv=\frac{1}{2}\), so the angle of rotation is
\(\pi/3=\cos^{-1}(1/2)\).


For \(C\): Similar arguments give axis \(u=\ma 1 \\ 1 \\ -1\mz\) and
angle \(90\) degrees (e.g. if we pick \(v=\ma 1 \\ -1 \\ 0\mz\)
orthogonal to the axis \(u\), it goes to \(w=\ma -\frac{1}{\sqrt{3}}
\\ -\frac{1}{\sqrt{3}} \\ -\frac{2}{\sqrt{3}}\mz\), and \(v\cdot
w=0\)). \qedhere


\end{Solution}
\begin{Solution}
\(D,F,G\) are in echelon form, \(D,G\) are in reduced echelon form,
\(E\) is in neither.


\end{Solution}
\begin{Solution}
For \(D\): there are no zero-rows, so there are no constraints on
\(b\) for a solution to \(Dv=b\) to exist; are three free indices,
so the space of solutions is \(3\)-dimensional.


For \(G\): the last row is zero, so we need \(b_3=0\). There is one
free index, so the space of solutions (assuming there are some
solutions) is \(1\)-dimensional.


\end{Solution}
\begin{Solution}
The first system is \[\begin{pmatrix}[cccc|c] 1 & 1 & 2 & 3 & 0 \\ 0
& 1 & 4 & -1 & 1\end{pmatrix},\] which is almost in reduced echelon
form already. The row operation \(R_1\mapsto R_1-R_2\) gives
\[\begin{pmatrix}[cccc|c] 1 & 0 & -2 & 4 & -1 \\ 0 & 1 & 4 & -1 &
1\end{pmatrix}.\] The general solution is therefore \(x=-1+2z-4w\),
\(y=1+w-4z\), i.e. \(\ma -1+2z-4w \\ 1+w-4z \\ z \\ w\mz\).


The second system is \[\begin{pmatrix}[cc|c] 1 & -1 & -3 \\ 2 & 1 &
6 \\ -3 & 1 & 1\end{pmatrix}\] (be careful because I mixed up the
order of the letters a little to trick you; systems of equations in
real life rarely come in the nice form we've been studying them
without any rearrangement). We perform the row operations
\(R_2\mapsto R_2-2R_1\), \(R_3\mapsto R_3+3R_1\) to clear the first
column: \[\begin{pmatrix}[cc|c] 1 & -1 & -3 \\ 0 & 3 & 12 \\ 0 & -2
& -8\end{pmatrix}.\] Now \(R_2\mapsto\frac{1}{3}R_2\),
\(R_3\mapsto-\frac{1}{2}R_3\) and \(R_3\mapsto R_3-R_2\) gives
\[\begin{pmatrix}[cc|c] 1 & -1 & -3 \\ 0 & 1 & 4 \\ 0 & 0 &
0\end{pmatrix}.\] Finally, \(R_1\mapsto R_1+R_2\) gives
\[\begin{pmatrix}[cc|c] 1 & 0 & 1 \\ 0 & 1 & 4 \\ 0 & 0 &
0\end{pmatrix},\] so the unique solution is \(x=1\), \(y=4\).


The third system is \[\ma 4 & 0 & 0 & -1 \\ 0 & 3 & -2 & 1 \\ 4 & -2
& 4 & -3 \\ 3 & 1 & -1 & 0 \mz\ma x \\y \\z \\w\mz=\ma 0 \\ 4 \\ 0
\\ 2\mz.\] The augmented matrix is \[\begin{pmatrix}[cccc|c] 4 & 0 &
0 & -1 & 0 \\ 0 & 3 & -2 & 1 & 4 \\ 4 & -2 & 4 & -3 & 0 \\ 3 & 1 &
-1 & 0 &2 \end{pmatrix}.\] We use the row operations \(R_3\mapsto
R_3-R_1\), \(R_4\mapsto R_4-\frac{3}{4}R_1\) to get
\[\begin{pmatrix}[cccc|c] 4 & 0 & 0 & -1 & 0 \\ 0 & 3 & -2 & 1 & 4
\\ 0 & -2 & 4 & -2 & 0 \\ 0 & 1 & -1 & 3/4 & 2 \end{pmatrix}.\] Now
we use \(R_3\mapsto R_3+\frac{2}{3}R_2\) and \(R_4\mapsto
R_4-\frac{1}{3}R_2\) to get \[\begin{pmatrix}[cccc|c] 4 & 0 & 0 & -1
& 0 \\ 0 & 3 & -2 & 1 & 4 \\ 0 & 0 & 8/3 & -4/3 & 8/3 \\ 0 & 0 &
-1/3 & 5/12 & 2/3\end{pmatrix}.\] Now \(R_4\mapsto
R_4+\frac{1}{8}R_3\) gives \[\begin{pmatrix}[cccc|c] 4 & 0 & 0 & -1
& 0 \\ 0 & 3 & -2 & 1 & 4 \\ 0 & 0 & 8/3 & -4/3 & 8/3 \\ 0 & 0 & 0 &
1/4 & 1\end{pmatrix}.\] Let's tidy up a bit with \(R_4\mapsto
4R_4\), \(R_3\mapsto\frac{3}{8}R_3\), which gives
\[\begin{pmatrix}[cccc|c] 4 & 0 & 0 & -1 & 0 \\ 0 & 3 & -2 & 1 & 4
\\ 0 & 0 & 1 & -1/2 & 1 \\ 0 & 0 & 0 & 1 & 4\end{pmatrix}.\] Now do
\(R_3\mapsto R_3+\frac{1}{2}R_4\), \(R_2\mapsto R_2+2R_3\)
\(R_2\mapsto R_2-R_4\), \(R_1\mapsto R_1+R_4\) to get
\[\begin{pmatrix}[cccc|c] 4 & 0 & 0 & 0 & 4 \\ 0 & 3 & 0 & 0 & 6
\\ 0 & 0 & 1 & 0 & 3 \\ 0 & 0 & 0 & 1 & 4\end{pmatrix}\] We finish
with \(R_1\mapsto \frac{1}{4}R_1\) and \(R_2\mapsto
\frac{1}{3}R_2\): \[\begin{pmatrix}[cccc|c] 1 & 0 & 0 & 0 & 1 \\ 0 &
1 & 0 & 0 & 2 \\ 0 & 0 & 1 & 0 & 3 \\ 0 & 0 & 0 & 1 &
4\end{pmatrix},\] so \(\ma 1\\ 2 \\ 3 \\4\mz\) is the unique
solution.


\end{Solution}
\begin{Solution}
First, take \[X=\ma 1 & 1 & 0 & 0 & 0 \\ 1 & 1 & 1 & 0 & 0 \\ 1 & 1
& 1 & 1 & 0 \\ 1 & 1 & 1 & 1 & 1 \mz.\] We can clear the first
column just by subtracting \(R_1\) from all the other rows: \[\ma 1
& 1 & 0 & 0 & 0 \\ 0 & 0 & 1 & 0 & 0 \\ 0 & 0 & 1 & 1 & 0 \\ 0 & 0 &
1 & 1 & 1 \mz.\] Now clear column three by subtracting \(R_2\) from
\(R_3\) and \(R_4\), and column four by further subtracting \(R_3\)
from \(R_4\): \[\ma 1 & 1 & 0 & 0 & 0 \\ 0 & 0 & 1 & 0 & 0 \\ 0 & 0
& 0 & 1 & 0 \\ 0 & 0 & 0 & 0 & 1 \mz.\] This has one free index (the
index 2).


Second, take \[Y=\ma 2 & 0 & 1 \\ 0 & 1 & 0 \\ 1 & 3
& -1\mz.\] Subtract \(2R_1\) from \(R_3\) and \(3R_2\) from
\(R_3\). This gives
\[\ma 2 & 0 & 1 \\ 0 & 1 & 0 \\ 0 & 0 & -3\mz.\]
Now \(R_1\mapsto\frac{1}{2}R_1\), \(R_3\mapsto-\frac{1}{3}R_3\) and
\(R_1\mapsto R_1-\frac{1}{2}R_3\) gives the identity matrix
\[\ma 1 & 0 & 0 \\ 0 & 1 & 0 \\ 0 & 0 & 1\mz,\]
as the echelon form, so there are no free indices.


Finally, take \[Z=\ma 1 & 2 & 3 & 4 \\ 5 & 6 & 7 & 8 \\ 9 & 10 & 11
& 12\mz.\] \(R_2\mapsto R_2-5R_1\), \(R_3\mapsto R_3-9R_1\) gives
\[\ma 1 & 2 & 3 & 4 \\ 0 & -4 & -8 & -12 \\ 0 & -8 & -16 & -24\mz.\]
\(R_2\mapsto-\frac{1}{4}R_2\), \(R_3\mapsto-\frac{1}{8}R_3\) gives
\[\ma 1 & 2 & 3 & 4 \\ 0 & 1 & 2 & 3 \\ 0 & 1 & 2 & 3\mz.\]
\(R_3\mapsto R_3-R_2\), \(R_1\mapsto R_1-2R_2\) gives
\[\ma 1 & 0 & -1 & -2 \\ 0 & 1 & 2 & 3 \\ 0 & 0 & 0 & 0\mz.\] There
are therefore two free indices (3 and 4).


\end{Solution}
\begin{Solution}\label{sol:exr:assoc}
The two expressions are:
\[\sum_{j=1}^nA_{ij}\sum_{k=1}^pB_{jk}C_{k\ell}\qquad
\sum_{k=1}^p\left(\sum_{j=1}^nA_{ij}B_{jk}\right)C_{k\ell},\] and
these are both equal to \(\sum_{j=1}^n\sum_{k=1}^p
A_{ij}B_{jk}C_{k\ell}\).


\end{Solution}
\begin{Solution}\label{sol:exr:orthogmat}
If the columns of \(A\) are \(v_1,\ldots,v_n\) then the rows of
\(A^T\) are \(v_1^T,\ldots,v_n^T\). The product \(A^TA\) has \(ij\)
entry equal to \(v_i^Tv_j\) (multiplying the \(i\)th row of \(A^T\)
into the \(j\)th column of \(A\)) which is precisely \(v_i\cdot
v_j\). Since \(A^TA=I\), this means \(v_i\cdot v_j=\delta_{ij}\), as
required. \qedhere


\end{Solution}
\begin{Solution}\label{sol:exr:symmantisymm}
\begin{enumerate}
\item We have \((M^TM)^T=M^T(M^T)^T=M^TM\), so \(M^TM\) is
symmetric. Similarly \((MM^T)^T=(M^T)^TM^T=MM^T\).
\item We have \((C+C^T)^T=C^T+(C^T)^T=C^T+C\), so \(C+C^T\) is
symmetric. Similarly, \((C-C^T)^T=C^T-(C^T)^T=C^T-C\), so
\(C-C^T\) is antisymmetric. Since
\[C=\frac{1}{2}(C+C^T)+\frac{1}{2}(C-C^T),\]
we see that \(C\) can be written as the sum of a symmetric and an
antisymmetric matrix.


\end{enumerate}
\end{Solution}
\begin{Solution}
\begin{align*}
x+y+z&=0\\
x+y+z&=1
\end{align*}
is an underdetermined system (three variables, two equations) with
no solutions.


\begin{align*}
x&=1\\
2x&=2
\end{align*}
is an overdetermined system (one variable, two equations) with a
solution.


\end{Solution}
\newpage


\end{document}
