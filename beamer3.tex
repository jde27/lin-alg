\documentclass{beamer}
\title{Linear Algebra, Week 3}
\author{Jonny Evans}
\date{MATH105}
\usenavigationsymbolstemplate{}
\usepackage[utf8]{inputenc}
\usepackage[T1]{fontenc}
\usepackage{fixltx2e}
\usepackage{graphicx}
\usepackage{longtable}
\usepackage{float}
\usepackage{parallel}
\usepackage{parcolumns}
\usepackage{wrapfig}
\usepackage{rotating}
\usepackage{amsmath}
\usepackage{textcomp}
\usepackage{marvosym}
\usepackage{wasysym}
\usepackage{amssymb}
\usepackage{amsthm,amsmath,amsfonts,amscd,setspace}
\renewcommand{\partname}{Week}
\usepackage{tikz}
\usetikzlibrary{decorations.markings,decorations.pathmorphing,shapes}
\usepackage{parskip}
\newcommand{\FF}{\mathfrak{f}}
\newcommand{\GG}{\mathfrak{g}}
\newcommand{\CC}{\mathbb{C}}
\newcommand{\QQ}{\mathbb{Q}}
\newcommand{\RR}{\mathbb{R}}
\newcommand{\UU}{\mathbb{U}}
\newcommand{\XX}{\mathbb{X}}
\newcommand{\YY}{\mathbb{Y}}
\newcommand{\ZZ}{\mathbb{Z}}
\newcommand{\Link}{\operatorname{Link}}
\newcommand{\Cone}{\operatorname{Cone}}
\newcommand{\colim}{\operatorname{colim}}
%\newcommand{\amal}{\operatorname{amal}}
\newcommand{\adj}{\operatorname{adj}}
\newcommand{\cp}[1]{\mathbf{CP}^{#1}}
\newcommand{\rp}[1]{\mathbf{RP}^{#1}}
\newcommand{\OP}[1]{\mathrm{#1}}
\newcommand{\ma}{\begin{pmatrix}}
\newcommand{\mz}{\end{pmatrix}}
\newcommand{\tka}{\begin{center}\begin{tikzpicture}}
\newcommand{\tkz}{\end{tikzpicture}\end{center}}
\newcommand{\matr}[4]{\left(\begin{array}{cc}#1 & #2\\ #3 & #4\end{array}\right)}
\newcommand{\mthrthr}[9]{\left(\begin{array}{ccc}#1 & #2 & #3\\ #4 & #5 & #6\\ #7 & #8 & #9\end{array}\right)}
\newcommand{\vect}[2]{\left(\begin{array}{c}#1\\#2\end{array}\right)}
\newcommand{\vthr}[3]{\left(\begin{array}{c}#1\\#2\\#3\end{array}\right)}
\newcommand{\TO}[3]{#1\stackrel{#2}{\longrightarrow}#3}
\makeatletter
\renewcommand*\env@matrix[1][*\c@MaxMatrixCols c]{%
  \hskip -\arraycolsep
  \let\@ifnextchar\new@ifnextchar
  \array{#1}}
\makeatother

\begingroup
\makeatletter
\@for\theoremstyle:=definition,remark,plain\do{%
\expandafter\g@addto@macro\csname th@\theoremstyle\endcsname{%
\addtolength\thm@preskip\parskip
}%
}
\endgroup
\usepackage{graphicx}
\usepackage[capitalise]{cleveref}
%\newtheorem{Theorem}{Theorem}[section]
%\newtheorem{Lemma}[Theorem]{Lemma}
%\newtheorem{Corollary}[Theorem]{Corollary}
%\newtheorem{Proposition}[Theorem]{Proposition}
%\theoremstyle{remark}
%\newtheorem{Remark}[Theorem]{Remark}
%\theoremstyle{definition}
%\newtheorem{Definition}[Theorem]{Definition}
%\newtheorem{Example}[Theorem]{Example}
%\newtheorem{Exercise}[Theorem]{Exercise}
%\newtheorem{Question}[Theorem]{Question}
%\newtheorem{Solution}[Theorem]{Solution}
%\newtheorem{Answer}[Theorem]{Answer}
\crefname{Theorem}{Theorem}{Theorems}
\Crefname{Theorem}{Theorem}{Theorems}
\crefname{Lemma}{Lemma}{Lemmas}
\Crefname{Lemma}{Lemma}{Lemmas}
\crefname{Corollary}{Corollary}{Corollaries}
\Crefname{Corollary}{Corollary}{Corollaries}
\crefname{Claim}{Claim}{Claims}
\Crefname{Claim}{Claim}{Claims}
\crefname{Proposition}{Proposition}{Propositions}
\Crefname{Proposition}{Proposition}{Propositions}
\crefname{Remark}{Remark}{Remarks}
\Crefname{Remark}{Remark}{Remarks}
\crefname{Definition}{Definition}{Definitions}
\Crefname{Definition}{Definition}{Definitions}
\crefname{Example}{Example}{Examples}
\Crefname{Example}{Example}{Examples}
\crefname{Exercise}{Exercise}{Exercises}
\Crefname{Exercise}{Exercise}{Exercises}
\crefname{Question}{Question}{Questions}
\Crefname{Question}{Question}{Questions}
\crefname{Solution}{Solution}{Solutions}
\Crefname{Solution}{Solution}{Solutions}
\crefname{Answer}{Answer}{Answers}
\Crefname{Answer}{Answer}{Answers}
\crefformat{footnote}{#2\footnotemark[#1]#3}

\begin{document}
\maketitle
\begin{frame}
\begin{center}
{\huge Determinants}
\end{center}


\end{frame}
\begin{frame}
\onslide<1->
\(A=\ma a & b \\ c & d \mz\) is invertible if and only if
\(ad-bc\neq 0\).


What about \(n\)-by-\(n\) matrices?


\onslide<2->
\begin{Theorem}
An \(n\)-by-\(n\) matrix \(A\) is invertible if and only if its
{\em determinant} \(\det(A)\) is nonzero.


\end{Theorem}
\end{frame}
\begin{frame}
{Definition of determinant}


\tka
\node at (0,0) {\(\ma a & b \\ c & d\mz\)};
\onslide<2-3> {\draw[thick,blue] (-0.3,0.26) circle [radius=0.25];
\draw[thick,blue] (0.3,-0.26) circle [radius=0.25];}
\onslide<3-> {\node at (-1,-1.5) {\(ad\)};}
\onslide<4-5> {\draw[thick,blue] (0.3,0.26) circle [radius=0.25];
\draw[thick,blue] (-0.3,-0.26) circle [radius=0.25];}
\onslide<5-> {\node at (1,-1.5) {\(bc\)};}
\onslide<6-> {\node at (0,-1.5) {\(-\)};}
\tkz


\end{frame}
\begin{frame}
{Definition of determinant}


\tka
\node at (0,0) {\(\ma A_{11} & A_{12} & A_{13} \\ A_{21} & A_{22} & A_{23} \\ A_{31} & A_{32} & A_{33}\mz\)};
\onslide<2-3> {\draw[thick,blue] (-0.95,0.5) circle [radius=0.35];
\draw[thick,blue] (0,0) circle [radius=0.35];
\draw[thick,blue] (0.95,-0.5) circle [radius=0.35];}
\onslide<4-5> {\draw[thick,blue] (-0.95,0.5) circle [radius=0.35];
\draw[thick,blue] (0.95,0) circle [radius=0.35];
\draw[thick,blue] (0,-0.5) circle [radius=0.35];}
\onslide<6-7> {\draw[thick,blue] (0,0.5) circle [radius=0.35];
\draw[thick,blue] (0.95,0) circle [radius=0.35];
\draw[thick,blue] (-0.95,-0.5) circle [radius=0.35];}
\onslide<8-9> {\draw[thick,blue] (0,0.5) circle [radius=0.35];
\draw[thick,blue] (-0.95,0) circle [radius=0.35];
\draw[thick,blue] (0.95,-0.5) circle [radius=0.35];}
\onslide<10-11> {\draw[thick,blue] (0.95,0.5) circle [radius=0.35];
\draw[thick,blue] (-0.95,0) circle [radius=0.35];
\draw[thick,blue] (0,-0.5) circle [radius=0.35];}
\onslide<12-13> {\draw[thick,blue] (0.95,0.5) circle [radius=0.35];
\draw[thick,blue] (0,0) circle [radius=0.35];
\draw[thick,blue] (-0.95,-0.5) circle [radius=0.35];}
\tkz


\tka
\onslide<3->{\node at (-2.3,0) {\(A_{11}A_{22}A_{33}\)};}
\onslide<5->{\node at (0,0) {\(-A_{11}A_{23}A_{32}\)};}
\onslide<7->{\node at (2.3,0) {\(+A_{12}A_{23}A_{31}\)};}
\onslide<9->{\node at (-2.3,-0.5) {\(-A_{12}A_{21}A_{33}\)};}
\onslide<11->{\node at (0,-0.5) {\(+A_{13}A_{21}A_{32}\)};}
\onslide<13->{\node at (2.3,-0.5) {\(-A_{13}A_{22}A_{31}\)};}
\onslide<14->{\begin{scope}[shift={(-2.45,0)}]\draw[thick,red] (-0.22,-0.09) circle [radius=0.13];
\draw[thick,red] (0.35,-0.09) circle [radius=0.13];
\draw[thick,red] (0.92,-0.09) circle [radius=0.13];\end{scope}}
\onslide<15->{\node[red] at (-2.3,0.5) {\(123\)};}
\onslide<16->{\draw[thick,red] (-0.22,-0.09) circle [radius=0.13];
\draw[thick,red] (0.35,-0.09) circle [radius=0.13];
\draw[thick,red] (0.92,-0.09) circle [radius=0.13];}
\onslide<17->{\node[red] at (0,0.5) {\(132\)};}
\onslide<18->{\begin{scope}[shift={(2.3,0)}]\draw[thick,red] (-0.22,-0.09) circle [radius=0.13];
\draw[thick,red] (0.35,-0.09) circle [radius=0.13];
\draw[thick,red] (0.92,-0.09) circle [radius=0.13];\end{scope}}
\onslide<19->{\node[red] at (2.3,0.5) {\(231\)};}
\onslide<20->{\begin{scope}[shift={(-2.3,-0.5)}]\draw[thick,red] (-0.22,-0.09) circle [radius=0.13];
\draw[thick,red] (0.35,-0.09) circle [radius=0.13];
\draw[thick,red] (0.92,-0.09) circle [radius=0.13];\end{scope}}
\onslide<21->{\node[red] at (-2.3,-1.1) {\(213\)};}
\onslide<22->{\begin{scope}[shift={(0,-0.5)}]\draw[thick,red] (-0.22,-0.09) circle [radius=0.13];
\draw[thick,red] (0.35,-0.09) circle [radius=0.13];
\draw[thick,red] (0.92,-0.09) circle [radius=0.13];\end{scope}}
\onslide<23->{\node[red] at (0,-1.1) {\(312\)};}
\onslide<24->{\begin{scope}[shift={(2.3,-0.5)}]\draw[thick,red] (-0.22,-0.09) circle [radius=0.13];
\draw[thick,red] (0.35,-0.09) circle [radius=0.13];
\draw[thick,red] (0.92,-0.09) circle [radius=0.13];\end{scope}}
\onslide<25->{\node[red] at (2.3,-1.1) {\(321\)};}
\tkz


\onslide<26->{\begin{center}\color{red} Permutations of 123.\end{center}}




\end{frame}
\begin{frame}
{Definition of determinant}


\begin{Definition}[Determinant]\label{dfn:det}
\end{Definition}
\begin{itemize}
\item <1-> For each permutation \(\sigma\) of \(\{1,\ldots,n\}\)
\item <2-> form the product \(A_{1\sigma(1)}\cdots A_{n\sigma(n)}\)
\item <3-> add a minus sign if \(\sigma\) involves an odd number of
swaps
\item <4-> and sum the resulting \(n!\) terms.


\end{itemize}
\onslide<5->
\[\det(A)=\sum_{\sigma}\mathrm{sign}(\sigma)A_{1\sigma(1)}\cdots
A_{n\sigma(n)}.\]


\end{frame}
\begin{frame}
{Examples: diagonal matrices}


\[\ma \lambda_1 & 0 & \cdots & 0\\ 0 & \lambda_2 & & \vdots
\\ \vdots & & \ddots & 0 \\ 0 & \cdots & 0 &\lambda_n\mz\]


\end{frame}
\begin{frame}
{Examples: upper triangular matrices}


\[\ma A_{11} & A_{12} & \cdots & A_{1n}\\ 0 & A_{22} & & A_{2n}
\\ \vdots & & \ddots & \vdots \\ 0 & \cdots & 0 & A_{nn}\mz\]


\end{frame}
\begin{frame}
{Examples: elementary matrices}


\[E_{ij}(\lambda)=\ma 1 & & & & \mbox{col }j & & \\ & \ddots & & &
\downarrow & & \\ \mbox{row }i&\rightarrow & 1 & & \lambda & &\\ & &
& \ddots & & & \\ & & & & 1 & & \\ & & & & & \ddots &\\ & & & & & &
1 \mz\]


\end{frame}
\begin{frame}
{Examples: elementary matrices}


\[E_{i}(\lambda)=\ma 1 & & \mbox{col }i & & & & \\ &
\ddots & \downarrow & & & & \\ \mbox{row }i&\rightarrow & \lambda &
& & &\\ & & & \ddots & & & \\ & & & & 1 & & \\ & & & & & \ddots
&\\ & & & & & & 1 \mz\]


\end{frame}
\begin{frame}
\begin{center}
{\huge Computing determinants}
\end{center}


\end{frame}
\begin{frame}
{Properties of the determinant}


\[\det(A)=\sum_{\sigma}\mathrm{sign}(\sigma)A_{1\sigma(1)}\cdots A_{n\sigma(n)}.\]
\onslide<2->
\begin{Lemma}
If two rows of \(A\) coincide then \(\det(A)=0\).


\end{Lemma}
\onslide<3->
\begin{Lemma}
If \(A'\) is obtained from \(A\) by a row operation of type I
(\(R_i\mapsto R_i+\lambda R_j\)) then \(\det(A')=\det(A)\).


\end{Lemma}
\onslide<4->
\begin{Corollary}
Put \(A\) into echelon form using row operations of type I. Then
\(\det(A)\) is the product of the diagonal entries.


\end{Corollary}
\end{frame}
\begin{frame}
\onslide<1->
\begin{Example}
\[A=\begin{pmatrix} 1 & 4 & -4 \\ -2 & -2 & -4 \\ 3 & -3 & 3
\end{pmatrix}\]


\end{Example}
\onslide<2->
\begin{Example}
\[B=\begin{pmatrix} 2 & -3 & -1 & 4 \\ 2 & -3 & 2 & 4 \\ 2 & -1 &
-4 & -3 \\ 2 & -3 & 4 & 2 \end{pmatrix}\]


\end{Example}
\end{frame}
\begin{frame}
{A useful trick}
\begin{Lemma}
If \(A'\) is obtained from \(A\) by swapping two rows then
\(\det(A')=-\det(A)\).


\end{Lemma}
\onslide<2->
\begin{Example}
\[\det\ma 0 & 0 & 0 & 1 \\ 0 & 1 & 0 & 0 \\ 0 & 0 & 1 & 0 \\ 1 & 0
& 0 & 0\mz=-1\]


\end{Example}
\end{frame}
\begin{frame}
\begin{center}
{\huge Formulas for determinants \\ and inverses}
\end{center}


\end{frame}
\begin{frame}
{``Expanding along first row.''}


\tka[decoration=snake]
\node at (0,0) {\(\ma A_{11} & A_{12} & A_{13} & A_{14} \\ A_{21} & A_{22} & A_{23} & A_{24} \\ A_{31} & A_{32} & A_{33} & A_{34} \\ A_{41} & A_{42} & A_{43} & A_{44} \mz\)};
\onslide<2-6>{\draw[thick,blue] (-1.42,0.68) circle [radius=0.3];}
\onslide<3-6>{\draw[thick] (-1.8,0.65) -- (1.8,0.65);
\draw[thick] (-1.42,0.8) -- (-1.42,-0.8);}
\onslide<4-6>{\draw[red] (-0.8,-0.9) -- (1.7,-0.9) -- (1.7,0.45) -- (-0.8,0.45) -- cycle;}
\onslide<5-6>{\draw[->,thick,decorate] (2,0) -- (2.9,0);
\node at (5,0) {\(C_{11}=\ma A_{22} & A_{23} & A_{24} \\ A_{32} & A_{33} & A_{34} \\ A_{42} & A_{43} & A_{44}\mz\)};
\draw[red] (4.3,-0.7) -- (6.8,-0.7) -- (6.8,0.7) -- (4.3,0.7) -- cycle;}
\onslide<7-11>{\draw[thick,blue] (-0.48,0.68) circle [radius=0.3];}
\onslide<8-11>{\draw[thick] (-1.8,0.65) -- (1.8,0.65);
\draw[thick] (-0.48,0.8) -- (-0.48,-0.8);}
\onslide<9-11>{\draw[red] (-1.7,-0.9) -- (-1,-0.9) -- (-1,0.45) -- (-1.7,0.45) -- cycle;
\draw[red] (0.2,-0.9) -- (1.7,-0.9) -- (1.7,0.45) -- (0.2,0.45) -- cycle;}
\onslide<10-11>{\draw[->,thick,decorate] (2,0) -- (2.9,0);
\node at (5,0) {\(C_{12}=\ma A_{21} & A_{23} & A_{24} \\ A_{31} & A_{33} & A_{34} \\ A_{41} & A_{43} & A_{44}\mz\)};
\draw[red] (4.3,-0.7) -- (6.8,-0.7) -- (6.8,0.7) -- (4.3,0.7) -- cycle;}
\onslide<12-16>{\draw[thick,blue] (0.48,0.68) circle [radius=0.3];}
\onslide<13-16>{\draw[thick] (-1.8,0.65) -- (1.8,0.65);
\draw[thick] (0.48,0.8) -- (0.48,-0.8);}
\onslide<14-16>{\draw[red] (-1.7,-0.9) -- (-0.1,-0.9) -- (-0.1,0.45) -- (-1.7,0.45) -- cycle;
\draw[red] (1,-0.9) -- (1.7,-0.9) -- (1.7,0.45) -- (1,0.45) -- cycle;}
\onslide<15-16>{\draw[->,thick,decorate] (2,0) -- (2.9,0);
\node at (5,0) {\(C_{13}=\ma A_{21} & A_{22} & A_{24} \\ A_{31} & A_{32} & A_{34} \\ A_{41} & A_{42} & A_{44}\mz\)};
\draw[red] (4.3,-0.7) -- (6.8,-0.7) -- (6.8,0.7) -- (4.3,0.7) -- cycle;}
\onslide<17-22>{\draw[thick,blue] (1.42,0.68) circle [radius=0.3];}
\onslide<18-22>{\draw[thick] (-1.8,0.65) -- (1.8,0.65);
\draw[thick] (1.42,0.8) -- (1.42,-0.8);}
\onslide<19-22>{\draw[red] (-1.7,-0.9) -- (0.8,-0.9) -- (0.8,0.45) -- (-1.7,0.45) -- cycle;}
\onslide<20-22>{\draw[->,thick,decorate] (2,0) -- (2.9,0);
\node at (5,0) {\(C_{14}=\ma A_{21} & A_{22} & A_{23} \\ A_{31} & A_{32} & A_{33} \\ A_{41} & A_{42} & A_{43}\mz\)};
\draw[red] (4.3,-0.7) -- (6.8,-0.7) -- (6.8,0.7) -- (4.3,0.7) -- cycle;}
\tkz
\[\onslide<6->A_{11}\det(C_{11})\onslide<11->-A_{12}\det(C_{12})\onslide<16->+A_{13}\det(C_{13})\onslide<21->-A_{14}\det(C_{14}).\]


\onslide<22->
\begin{Example}
\[A=\ma 1 & 2 & 3 \\ 4 & 5 & 6 \\ 7 & 8 & 9\mz\]


\end{Example}
\end{frame}
\begin{frame}
{``Expanding along second column.''}


\tka[decoration=snake]
\node at (0,0) {\(\ma A_{11} & A_{12} & A_{13} & A_{14} \\ A_{21} & A_{22} & A_{23} & A_{24} \\ A_{31} & A_{32} & A_{33} & A_{34} \\ A_{41} & A_{42} & A_{43} & A_{44} \mz\)};
\onslide<2-6>{\draw[thick,blue] (-0.48,0.68) circle [radius=0.3];}
\onslide<3-6>{\draw[thick] (-1.8,0.65) -- (1.8,0.65);
\draw[thick] (-0.48,0.8) -- (-0.48,-0.8);}
\onslide<4-6>{\draw[red] (-1.7,-0.9) -- (-1,-0.9) -- (-1,0.45) -- (-1.7,0.45) -- cycle;
\draw[red] (0.2,-0.9) -- (1.7,-0.9) -- (1.7,0.45) -- (0.2,0.45) -- cycle;}
\onslide<5-6>{\draw[->,thick,decorate] (2,0) -- (2.9,0);
\node at (5,0) {\(C_{12}=\ma A_{21} & A_{23} & A_{24} \\ A_{31} & A_{33} & A_{34} \\ A_{41} & A_{43} & A_{44}\mz\)};
\draw[red] (4.3,-0.7) -- (6.8,-0.7) -- (6.8,0.7) -- (4.3,0.7) -- cycle;}
\onslide<7-11>{\draw[thick,blue] (-0.48,0.28) circle [radius=0.3];}
\onslide<8-11>{\draw[thick] (-1.8,0.25) -- (1.8,0.25);
\draw[thick] (-0.48,0.8) -- (-0.48,-0.8);}
\onslide<9-11>{\draw[red] (-1.7,-0.9) -- (-1,-0.9) -- (-1,0) -- (-1.7,0) -- cycle;
\draw[red] (0.2,-0.9) -- (1.7,-0.9) -- (1.7,0) -- (0.2,0) -- cycle;
\draw[red] (0.2,0.45) -- (1.7,0.45) -- (1.7,0.9) -- (0.2,0.9) -- cycle;
\draw[red] (-1.7,0.45) -- (-1,0.45) -- (-1,0.9) -- (-1.7,0.9) -- cycle;}
\onslide<10-11>{\draw[->,thick,decorate] (2,0) -- (2.9,0);
\node at (5,0) {\(C_{22}=\ma A_{11} & A_{13} & A_{14} \\ A_{31} & A_{33} & A_{34} \\ A_{41} & A_{43} & A_{44}\mz\)};
\draw[red] (4.3,-0.7) -- (6.8,-0.7) -- (6.8,0.7) -- (4.3,0.7) -- cycle;}
\onslide<12-16>{\draw[thick,blue] (-0.48,-0.28) circle [radius=0.3];}
\onslide<13-16>{\draw[thick] (-1.8,-0.28) -- (1.8,-0.28);
\draw[thick] (-0.48,0.8) -- (-0.48,-0.8);}
\onslide<14-16>{\draw[red] (-1.7,-0.9) -- (-1,-0.9) -- (-1,-0.45) -- (-1.7,-0.45) -- cycle;
\draw[red] (0.2,-0.9) -- (1.7,-0.9) -- (1.7,-0.45) -- (0.2,-0.45) -- cycle;
\draw[red] (0.2,0) -- (1.7,0) -- (1.7,0.9) -- (0.2,0.9) -- cycle;
\draw[red] (-1.7,0) -- (-1,0) -- (-1,0.9) -- (-1.7,0.9) -- cycle;}
\onslide<15-16>{\draw[->,thick,decorate] (2,0) -- (2.9,0);
\node at (5,0) {\(C_{32}=\ma A_{11} & A_{13} & A_{14} \\ A_{21} & A_{23} & A_{24} \\ A_{41} & A_{43} & A_{44}\mz\)};
\draw[red] (4.3,-0.7) -- (6.8,-0.7) -- (6.8,0.7) -- (4.3,0.7) -- cycle;}
\onslide<17-22>{\draw[thick,blue] (-0.48,-0.68) circle [radius=0.3];}
\onslide<18-22>{\draw[thick] (-1.8,-0.65) -- (1.8,-0.65);
\draw[thick] (-0.48,0.8) -- (-0.48,-0.8);}
\onslide<19-22>{\draw[red] (-1.7,-0.45) -- (-1,-0.45) -- (-1,0.9) -- (-1.7,0.9) -- cycle;
\draw[red] (0.1,-0.45) -- (1.7,-0.45) -- (1.7,0.9) -- (0.1,0.9) -- cycle;}
\onslide<20-22>{\draw[->,thick,decorate] (2,0) -- (2.9,0);
\node at (5,0) {\(C_{42}=\ma A_{11} & A_{13} & A_{14} \\ A_{21} & A_{23} & A_{24} \\ A_{31} & A_{33} & A_{34}\mz\)};
\draw[red] (4.3,-0.7) -- (6.8,-0.7) -- (6.8,0.7) -- (4.3,0.7) -- cycle;}
\tkz
\[\onslide<6->-A_{12}\det(C_{12})\onslide<11->+A_{22}\det(C_{22})\onslide<16->-A_{32}\det(C_{32})\onslide<21->+A_{42}\det(C_{42}).\]


\onslide<22->
\begin{Example}
\[B=\ma 1 & 1 & 2 & 3 \\ 0 & 0 & 4 & 5 \\ -1 & 2 & 1 & 1 \\ 0 & 0
& 2 & 3\mz\]


\end{Example}
\end{frame}
\begin{frame}
{Signs}


\[\ma + & - & + & \cdots \\ - & + & - & \cdots \\+ & - & + & \cdots
\\ \vdots & \vdots & \vdots & \mz\]


\[A_{11}\det(C_{11})-A_{12}\det(C_{12})+A_{13}\det(C_{13})-A_{14}\det(C_{14}).\]


\end{frame}
\begin{frame}
{Signs}


\[\ma + & - & + & \cdots \\ - & + & - & \cdots \\+ & - & + & \cdots
\\ \vdots & \vdots & \vdots & \mz\]


\[-A_{12}\det(C_{12})+A_{22}\det(C_{22})-A_{32}\det(C_{32})+A_{42}\det(C_{42}).\]


\end{frame}
\begin{frame}
{Formula for inverses (Cramer's rule)}


\begin{Definition}
The {\em adjugate matrix} of \(A\) is the matrix \[\adj(A):=\ma
+\det(C_{11}) & -\det(C_{12}) & +\det(C_{13}) & \cdots
\\ \det(C_{21}) & +\det(C_{22}) & -\det(C_{23}) & \cdots
\\+\det(C_{31}) & -\det(C_{32}) & +\det(C_{33}) & \cdots \\ \vdots
& \vdots & \vdots & \mz^T.\]


\end{Definition}
\begin{Theorem}
If \(\det(A)\neq 0\) then \(A^{-1}=\frac{1}{\det(A)}\adj(A)\).


\end{Theorem}
\end{frame}
\begin{frame}
\begin{center}
{\huge More about determinants}
\end{center}


\end{frame}
\begin{frame}
\begin{Lemma}
If \(A'\) is obtained from \(A\) by \(R_i\mapsto\lambda R_i\) then
\(\det(A')=\lambda\det(A)\).


\end{Lemma}
\begin{Corollary}
An \(n\)-by-\(n\) matrix is invertible if and only if its
determinant is nonzero.


\end{Corollary}
\end{frame}
\begin{frame}
\begin{Theorem}
\[\det(AB)=\det(A)\det(B).\]


\end{Theorem}
\end{frame}
\begin{frame}
{Geometric interpretation of determinants}


\begin{Theorem}
\tka
\filldraw[draw=none,fill=gray,opacity=0.5] (0,0) -- (2,0) -- (2,2) -- (0,2) -- cycle;
\draw[thick,->] (0,0) -- (2,0) node [midway,below] {\(\ma 1 \\ 0\mz\)};
\draw[thick,->] (0,0) -- (0,2) node [midway,left] {\(\ma 0 \\ 1\mz\)};
\onslide<2->{\draw[thick,->] (3,1) -- (5,1) node [midway,above] {\(A=\ma a & b \\ c & d\mz\)};}
\onslide<3->{\filldraw[draw=none,fill=red,opacity=0.5] (6,0) -- (8,1) -- (9,3) -- (7,2) -- cycle;
\draw[thick,->] (6,0) -- (8,1) node [midway,below] {\(\ma a \\ c\mz\)};
\draw[thick,->] (6,0) -- (7,2) node [midway,left] {\(\ma b \\ d\mz\)};}
\tkz


\onslide<4-> The area of the red parallelogram is \(|\det(A)|\).


\end{Theorem}
\end{frame}
\begin{frame}
\begin{Theorem}
\(|\det(A)|\) is the volume of \(A(C)\), where \(C\) is the unit
\(n\)-cube.


\tka
\node (A) at (0,0) {\(\bullet\)};
\node (B) at (1,-0.5) {\(\bullet\)};
\node (C) at (3,0) {\(\bullet\)};
\node (D) at (2,0.5) {\(\bullet\)};
\node (E) at (2.5,2) {\(\bullet\)};
\node (F) at (4.5,2.5) {\(\bullet\)};
\node (G) at (3.5,1.5) {\(\bullet\)};
\node (H) at (5.5,2) {\(\bullet\)};
\filldraw[fill=gray,opacity=0.25] (D.center) -- (C.center) -- (H.center) -- (F.center) -- cycle;
\filldraw[fill=gray,opacity=0.25] (A.center) -- (E.center) -- (F.center) -- (D.center) -- cycle;
\filldraw[fill=gray,opacity=0.25] (A.center) -- (B.center) -- (C.center) -- (D.center) -- cycle;
\filldraw[fill=gray,opacity=0.25] (E.center) -- (G.center) -- (H.center) -- (F.center) -- cycle;
\filldraw[fill=gray,opacity=0.25] (A.center) -- (B.center) -- (G.center) -- (E.center) -- cycle;
\filldraw[fill=gray,opacity=0.25] (B.center) -- (C.center) -- (H.center) -- (G.center) -- cycle;
\node at (4,-1) {``Parallelopiped''};
\tkz


\end{Theorem}
\end{frame}
\begin{frame}
\begin{Theorem}
Let \(a_1,\ldots,a_n\) be column vectors of height \(n\).


Let \(A\) be the matrix whose columns are \(a_1,\ldots,a_n\).


Let \(\Delta\) be the simplex whose vertices are at
\(0,a_1,\ldots,a_n\).


Then \(\mathrm{Vol}(\Delta)=\frac{1}{n!}\det(A)\).
\vspace{0.5cm}


\tka
\draw (0,-1) -- (0,2);
\draw (-2,1) -- (2,-1);
\draw (-2,-1) -- (2,1);
\node (A) at (0,0) {\(\bullet\)};
\node (B) at (-2,-1) {\(\bullet\)};
\node (C) at (-0.3,0.5) {\(\bullet\)};
\node (D) at (-2.3,1.5) {\(\bullet\)};
\filldraw[fill=gray,opacity=0.6] (A.center) -- (B.center) -- (D.center) -- cycle;
\filldraw[fill=gray,opacity=0.5] (C.center) -- (B.center) -- (D.center) -- cycle;
\filldraw[fill=gray,opacity=0.4] (C.center) -- (A.center) -- (D.center) -- cycle;
\filldraw[fill=gray,opacity=0.3] (A.center) -- (B.center) -- (C.center) -- cycle;
\node at (4,0.5) {\(A=\ma 1 & \frac{1}{2} & \frac{1}{2} \\ 0 & \frac{1}{2} & -\frac{1}{2} \\ 0 & \sqrt{2} & \sqrt{2}\mz\)};
\tkz
\end{Theorem}
\end{frame}
\end{document}
