\documentclass{beamer}
\title{Linear Algebra, Week 4}
\author{Jonny Evans}
\date{MATH105}
\usenavigationsymbolstemplate{}
\usepackage[utf8]{inputenc}
\usepackage[T1]{fontenc}
\usepackage{fixltx2e}
\usepackage{graphicx}
\usepackage{longtable}
\usepackage{float}
\usepackage{parallel}
\usepackage{parcolumns}
\usepackage{wrapfig}
\usepackage{rotating}
\usepackage{amsmath}
\usepackage{textcomp}
\usepackage{marvosym}
\usepackage{wasysym}
\usepackage{amssymb}
\usepackage{amsthm,amsmath,amsfonts,amscd,setspace}
\renewcommand{\partname}{Week}
\usepackage{tikz}
\usetikzlibrary{decorations.markings,decorations.pathmorphing,shapes}
\usepackage{parskip}
\newcommand{\FF}{\mathfrak{f}}
\newcommand{\GG}{\mathfrak{g}}
\newcommand{\CC}{\mathbb{C}}
\newcommand{\QQ}{\mathbb{Q}}
\newcommand{\RR}{\mathbb{R}}
\newcommand{\UU}{\mathbb{U}}
\newcommand{\XX}{\mathbb{X}}
\newcommand{\YY}{\mathbb{Y}}
\newcommand{\ZZ}{\mathbb{Z}}
\newcommand{\Link}{\operatorname{Link}}
\newcommand{\Cone}{\operatorname{Cone}}
\newcommand{\colim}{\operatorname{colim}}
%\newcommand{\amal}{\operatorname{amal}}
\newcommand{\adj}{\operatorname{adj}}
\newcommand{\cp}[1]{\mathbf{CP}^{#1}}
\newcommand{\rp}[1]{\mathbf{RP}^{#1}}
\newcommand{\OP}[1]{\mathrm{#1}}
\newcommand{\ma}{\begin{pmatrix}}
\newcommand{\mz}{\end{pmatrix}}
\newcommand{\tka}{\begin{center}\begin{tikzpicture}}
\newcommand{\tkz}{\end{tikzpicture}\end{center}}
\newcommand{\matr}[4]{\left(\begin{array}{cc}#1 & #2\\ #3 & #4\end{array}\right)}
\newcommand{\mthrthr}[9]{\left(\begin{array}{ccc}#1 & #2 & #3\\ #4 & #5 & #6\\ #7 & #8 & #9\end{array}\right)}
\newcommand{\vect}[2]{\left(\begin{array}{c}#1\\#2\end{array}\right)}
\newcommand{\vthr}[3]{\left(\begin{array}{c}#1\\#2\\#3\end{array}\right)}
\newcommand{\TO}[3]{#1\stackrel{#2}{\longrightarrow}#3}
\makeatletter
\renewcommand*\env@matrix[1][*\c@MaxMatrixCols c]{%
  \hskip -\arraycolsep
  \let\@ifnextchar\new@ifnextchar
  \array{#1}}
\makeatother

\begingroup
\makeatletter
\@for\theoremstyle:=definition,remark,plain\do{%
\expandafter\g@addto@macro\csname th@\theoremstyle\endcsname{%
\addtolength\thm@preskip\parskip
}%
}
\endgroup
\usepackage{graphicx}
\usepackage[capitalise]{cleveref}
%\newtheorem{Theorem}{Theorem}[section]
%\newtheorem{Lemma}[Theorem]{Lemma}
%\newtheorem{Corollary}[Theorem]{Corollary}
%\newtheorem{Proposition}[Theorem]{Proposition}
%\theoremstyle{remark}
%\newtheorem{Remark}[Theorem]{Remark}
%\theoremstyle{definition}
%\newtheorem{Definition}[Theorem]{Definition}
%\newtheorem{Example}[Theorem]{Example}
%\newtheorem{Exercise}[Theorem]{Exercise}
%\newtheorem{Question}[Theorem]{Question}
%\newtheorem{Solution}[Theorem]{Solution}
%\newtheorem{Answer}[Theorem]{Answer}
\crefname{Theorem}{Theorem}{Theorems}
\Crefname{Theorem}{Theorem}{Theorems}
\crefname{Lemma}{Lemma}{Lemmas}
\Crefname{Lemma}{Lemma}{Lemmas}
\crefname{Corollary}{Corollary}{Corollaries}
\Crefname{Corollary}{Corollary}{Corollaries}
\crefname{Claim}{Claim}{Claims}
\Crefname{Claim}{Claim}{Claims}
\crefname{Proposition}{Proposition}{Propositions}
\Crefname{Proposition}{Proposition}{Propositions}
\crefname{Remark}{Remark}{Remarks}
\Crefname{Remark}{Remark}{Remarks}
\crefname{Definition}{Definition}{Definitions}
\Crefname{Definition}{Definition}{Definitions}
\crefname{Example}{Example}{Examples}
\Crefname{Example}{Example}{Examples}
\crefname{Exercise}{Exercise}{Exercises}
\Crefname{Exercise}{Exercise}{Exercises}
\crefname{Question}{Question}{Questions}
\Crefname{Question}{Question}{Questions}
\crefname{Solution}{Solution}{Solutions}
\Crefname{Solution}{Solution}{Solutions}
\crefname{Answer}{Answer}{Answers}
\Crefname{Answer}{Answer}{Answers}
\crefformat{footnote}{#2\footnotemark[#1]#3}

\begin{document}
\maketitle
\begin{frame}
\begin{center}
{\huge Eigenvalues, eigenvectors}
\end{center}


\end{frame}
\begin{frame}
\onslide<1->
\begin{Definition}
Let \(A\) be an \(n\)-by-\(n\) matrix. A vector \(v\in\RR^n\) is
called an {\em eigenvector with eigenvalue \(\lambda\)} if
\[Av=\lambda v.\]


\end{Definition}
\onslide<2->
\begin{Example}
\begin{align*}
\onslide<2->{\ma 2 & -1 \\ 1 & 0\mz\ma 1 \\ 1\mz&=\ma 1 \\ 1\mz,\\}
\onslide<3->{\ma -3 & 2 \\ 0 & 1\mz\ma 1 \\ 0\mz&=-3\ma 1\\ 0\mz}
\end{align*}


\end{Example}
\end{frame}
\begin{frame}
\begin{Theorem}
Define the {\em characteristic polynomial} of an \(n\)-by-\(n\)
matrix \(A\) to be the polynomial \[\chi_A(t):=\det(A-tI).\] Then
the eigenvalues of \(A\) are the roots of \(\chi_A\).


\end{Theorem}
\onslide<2->
\begin{Example}
\begin{align*}
A&=\ma 2 & -1 \\ 1 & 0\mz\\
B&=\ma 2 & 1 \\ 1 & 1\mz\\
C&=\ma \frac{3}{2} & \frac{5}{2} & 3 \\ -\frac{1}{2} & -\frac{3}{2} & -3 \\ 1 & 1 & 2 \mz
\end{align*}


\end{Example}
\end{frame}
\begin{frame}
\begin{center}
{\huge Applications, I: Differential equations}
\end{center}


\end{frame}
\begin{frame}
{Differential equations}


\begin{columns}[t]
\begin{column}{0.4\textwidth}
\begin{align*}
\dot{x}&=2x+y\\
\dot{y}&=x+y
\end{align*}
\vspace{0.5cm}
\onslide<3->Eigenvalues \(\lambda_{\pm}=\frac{3\pm\sqrt{5}}{2}\)
\end{column}
\begin{column}{0.45\textwidth}
\onslide<2->\[\frac{d}{dt}\ma x \\ y \mz=\ma 2 & 1\\ 1 &
1\mz\ma x \\ y \mz\]
\onslide<4->Eigenvectors \(v_{\pm}=\ma 1 \\ \frac{1\pm\sqrt{5}}{2}\mz\)
\end{column}
\end{columns}


\onslide<5-> General solution
\[\ma x \\ y\mz=Ce^{\lambda_-t}v_-+De^{\lambda_+t}v_+.\]


\end{frame}
\begin{frame}
{Differential equations}


\begin{columns}[t]
\begin{column}{0.4\textwidth}
\begin{align*}
\dot{x}&=2x+y\\
\dot{y}&=2y-x
\end{align*}
\vspace{0.5cm}
\onslide<3->Eigenvalues \(\lambda_{\pm}=2\pm i\)
\end{column}
\begin{column}{0.45\textwidth}
\onslide<2->\[\frac{d}{dt}\ma x \\ y \mz=\ma 2 & 1\\ -1 &
2\mz\ma x \\ y \mz\]
\onslide<4->Eigenvectors \(v_{\pm}=\ma 1 \\ \pm i\mz\)
\end{column}
\end{columns}


\onslide<5-> General solution
\[\ma x \\ y\mz=Ce^{\lambda_-t}v_-+De^{\lambda_+t}v_+.\]


\end{frame}
\begin{frame}
{Differential equations}


\begin{columns}[t]
\begin{column}{0.4\textwidth}
\begin{align*}
\dot{x}&=x+y\\
\dot{y}&=y
\end{align*}
\vspace{0.5cm}
\onslide<3->Eigenvalues \(\lambda=1\)
\end{column}
\begin{column}{0.45\textwidth}
\onslide<2->\[\frac{d}{dt}\ma x \\ y \mz=\ma 1 & 1\\ 0 &
1\mz\ma x \\ y \mz\]
\onslide<4->Eigenvectors \(v=\ma 1 \\ 0\mz\)
\end{column}
\end{columns}


\onslide<5-> General solution?
\onslide<6->
\[y=Ce^t,\quad x=(Ct+D)e^t.\]


\end{frame}
\begin{frame}
\begin{center}
{\huge Applications, II: Ellipsoids}
\end{center}


\end{frame}
\begin{frame}
\begin{Definition}
We say that an \(n\)-by-\(n\) matrix \(A\) is {\em positive
definite} if \(v^TAv>0\) for any nonzero vector \(v\in\RR^n\).


\end{Definition}
\begin{Definition}
An {\em ellipsoid} is a subset in \(\RR^n\) of the form
\[\left\{v\in\RR^n\ :\ v^TAv=c\right\}\] for some positive
definite symmetric matrix \(A\) and constant \(c>0\).


\end{Definition}
\end{frame}
\begin{frame}
\begin{Lemma}
Suppose that \(A\) is a {\em real} symmetric matrix.
\begin{itemize}
\item <2-> The eigenvalues of \(A\) are real.
\item <3-> If \(Au=\lambda u\) and \(Av=\mu v\) with
\(\lambda\neq\mu\) then \(u\cdot v=0\).


\end{itemize}
\end{Lemma}
\end{frame}
\begin{frame}
\begin{Theorem}[Spectral theorem]\label{thm:spectral}
Suppose that \(A\) is a symmetric matrix.
\begin{itemize}
\item <2-> \(A\) has \(n\) orthogonal eigenvectors \(u_1,\ldots,u_n\)
with eigenvalues \(\lambda_1,\ldots,\lambda_n\).
\item <3-> If \(A\) is positive definite, the ellipsoid \[\{v\in\RR^n\ :\
v^TAv=c\}\] is the result of rotating the ellipsoid
\[\{(x_1,\ldots,x_n)\in\RR^n\ :\ \sum \lambda_ix_i^2=c\}\] so that
the \(x_i\)-direction points along \(u_i\).


\end{itemize}
\end{Theorem}
\onslide<4->
\begin{Example}
\(A=\ma \frac{3}{2} & -\frac{1}{2} \\ -\frac{1}{2} & \frac{3}{2}\mz\)


\end{Example}
\end{frame}
\begin{frame}
\begin{center}
{\huge Applications, III: Dynamics}
\end{center}


\end{frame}
\begin{frame}
\begin{Example}[Fibonacci dynamics]\label{exm:fib}


\begin{columns}[t]
\begin{column}{0.3\textwidth}
\[\ma 0 & 1 \\ 1 & 1\mz\]
\end{column}
\begin{column}{0.7\textwidth}
\onslide<2->Eigenvalues \(\lambda_{\pm}=\frac{1\pm\sqrt{5}}{2}=-0.618,\quad 1.618\)
\vspace{0.5cm}


\onslide<3->Eigenvectors \(v_{\pm}=\ma 1 \\ \frac{1\pm\sqrt{5}}{2}\mz\)
\end{column}
\end{columns}
\vspace{-3.5cm}
\onslide<4->
\tka
\draw[thick,->] (0,0) -- (1,1.618) node [above right] {\(v_+\)};
\draw[thick,->] (0,0) -- (1,-0.618) node [below left] {\(v_-\)};
\onslide<5>{\draw[thick,red] (1,1/2) -- (0,2.236/2) -- (-1,-1/2) -- (0,-2.236/2) -- cycle;}
\onslide<6>{\draw[thick,red] (1/2,3/2) -- (2.236/2,2.236/2) -- (-1/2,-3/2) -- (-2.236/2,-2.236/2) -- cycle;}
\onslide<7>{\draw[thick,red] (3/2,2) -- (2.236/2,2.236) -- (-3/2,-2) -- (-2.236/2,-2.236) -- cycle;}
\onslide<8>{\draw[thick,red] (2,7/2) -- (2.236,3*2.236/2) -- (-2,-7/2) -- (-2.236,-3*2.236/2) -- cycle;}
\onslide<9>{\draw[thick,red] (7/2,11/2) -- (3*2.236/2,5*2.236/2) -- (-7/2,-11/2) -- (-3*2.236/2,-5*2.236/2) -- cycle;}
\onslide<10->{\draw[->,blue] (0,0) -- (1,1) node [below] {\(\ma 1 \\ 1\mz\)};}
\onslide<11->{\draw[->,blue] (0,0) -- (1,2) node [left] {\(\ma 1 \\ 2\mz\)};}
\onslide<12->{\draw[->,blue] (0,0) -- (2,3) node [below] {\(\ma 2 \\ 3\mz\)};}
\onslide<13->{\draw[->,blue] (0,0) -- (3,5) node [midway,above] {\(\ma 3 \\ 5\mz\)};}
\onslide<14->{\draw[->,blue] (0,0) -- (5,8) node [midway,below] {\(\ma 5 \\ 8\mz\)};}
\onslide<15>{\node at (5,0) {\(\lim_{n\to\infty}\frac{F_{n+1}}{F_n}=1.618\cdots\)};}
\tkz


\end{Example}
\end{frame}
\end{document}
