\documentclass{article}
\title{Linear Algebra Worksheet 1}
\author{Jonny Evans}
\date{\vspace{-1cm}}
\usepackage[utf8]{inputenc}
\usepackage[T1]{fontenc}
\usepackage[a4paper,margin=1in]{geometry}
\usepackage{longtable}
\usepackage{fixltx2e}
\usepackage{graphicx}
\usepackage{longtable}
\usepackage{float}
\usepackage{parallel}
\usepackage{parcolumns}
\usepackage{wrapfig}
\usepackage{rotating}
\usepackage{amsmath}
\usepackage{textcomp}
\usepackage{marvosym}
\usepackage{wasysym}
\usepackage{amssymb}
\usepackage{amsthm,amsmath,amsfonts,amscd,setspace}
\renewcommand{\partname}{Week}
\newcommand{\mks}[1]{\marginpar{\colorbox{white}{\parbox{\marginparwidth}{%
  \setstretch{0.5}{\bf\scriptsize{#1 marks}}}}}}
\usepackage{tikz}
\usetikzlibrary{decorations.markings,decorations.pathmorphing,shapes}
\usepackage{parskip}
\newcommand{\FF}{\mathfrak{f}}
\newcommand{\GG}{\mathfrak{g}}
\newcommand{\CC}{\mathbb{C}}
\newcommand{\QQ}{\mathbb{Q}}
\newcommand{\RR}{\mathbb{R}}
\newcommand{\UU}{\mathbb{U}}
\newcommand{\XX}{\mathbb{X}}
\newcommand{\YY}{\mathbb{Y}}
\newcommand{\ZZ}{\mathbb{Z}}
\newcommand{\Link}{\operatorname{Link}}
\newcommand{\Cone}{\operatorname{Cone}}
\newcommand{\colim}{\operatorname{colim}}
%\newcommand{\amal}{\operatorname{amal}}
\newcommand{\adj}{\operatorname{adj}}
\newcommand{\cp}[1]{\mathbf{CP}^{#1}}
\newcommand{\rp}[1]{\mathbf{RP}^{#1}}
\newcommand{\OP}[1]{\mathrm{#1}}
\newcommand{\ma}{\begin{pmatrix}}
\newcommand{\mz}{\end{pmatrix}}
\newcommand{\tka}{\begin{center}\begin{tikzpicture}}
\newcommand{\tkz}{\end{tikzpicture}\end{center}}
\newcommand{\matr}[4]{\left(\begin{array}{cc}#1 & #2\\ #3 & #4\end{array}\right)}
\newcommand{\mthrthr}[9]{\left(\begin{array}{ccc}#1 & #2 & #3\\ #4 & #5 & #6\\ #7 & #8 & #9\end{array}\right)}
\newcommand{\vect}[2]{\left(\begin{array}{c}#1\\#2\end{array}\right)}
\newcommand{\vthr}[3]{\left(\begin{array}{c}#1\\#2\\#3\end{array}\right)}
\newcommand{\TO}[3]{#1\stackrel{#2}{\longrightarrow}#3}
\newenvironment{Proof}{\begin{proof}}{\end{proof}\ignorespacesafterend}
\makeatletter
\renewcommand*\env@matrix[1][*\c@MaxMatrixCols c]{%
  \hskip -\arraycolsep
  \let\@ifnextchar\new@ifnextchar
  \array{#1}}
\makeatother

\begingroup
\makeatletter
\@for\theoremstyle:=definition,remark,plain\do{%
\expandafter\g@addto@macro\csname th@\theoremstyle\endcsname{%
\addtolength\thm@preskip\parskip
}%
}
\endgroup
\usepackage{graphicx}
\usepackage[capitalise]{cleveref}
\newtheorem{Theorem}{Theorem}[section]
\newtheorem{Lemma}[Theorem]{Lemma}
\newtheorem{Corollary}[Theorem]{Corollary}
\newtheorem{Proposition}[Theorem]{Proposition}
\theoremstyle{remark}
\newtheorem{Remark}[Theorem]{Remark}
\theoremstyle{definition}
\newtheorem{Definition}[Theorem]{Definition}
\newtheorem{Example}[Theorem]{Example}
\newtheorem{Exercise}[Theorem]{Exercise}
\newtheorem{Question}[Theorem]{Question}
\newtheorem{Solution}[Theorem]{Solution}
\newtheorem{Answer}[Theorem]{Answer}
\crefname{Theorem}{Theorem}{Theorems}
\Crefname{Theorem}{Theorem}{Theorems}
\crefname{Lemma}{Lemma}{Lemmas}
\Crefname{Lemma}{Lemma}{Lemmas}
\crefname{Corollary}{Corollary}{Corollaries}
\Crefname{Corollary}{Corollary}{Corollaries}
\crefname{Claim}{Claim}{Claims}
\Crefname{Claim}{Claim}{Claims}
\crefname{Proposition}{Proposition}{Propositions}
\Crefname{Proposition}{Proposition}{Propositions}
\crefname{Remark}{Remark}{Remarks}
\Crefname{Remark}{Remark}{Remarks}
\crefname{Definition}{Definition}{Definitions}
\Crefname{Definition}{Definition}{Definitions}
\crefname{Example}{Example}{Examples}
\Crefname{Example}{Example}{Examples}
\crefname{Exercise}{Exercise}{Exercises}
\Crefname{Exercise}{Exercise}{Exercises}
\crefname{Question}{Question}{Questions}
\Crefname{Question}{Question}{Questions}
\crefname{Solution}{Solution}{Solutions}
\Crefname{Solution}{Solution}{Solutions}
\crefname{Answer}{Answer}{Answers}
\Crefname{Answer}{Answer}{Answers}
\crefformat{footnote}{#2\footnotemark[#1]#3}

\begin{document}
\maketitle
\setcounter{section}{1}


Here is a list \(\mathcal{V}\) of vectors
\begin{align*}
p&=\ma 1 \\ 0\mz& q&=\ma 0 \\ 1\mz & r&=\ma 1\\ 1\mz\\
s&=\ma 1 \\ -2\mz& t&=\ma -1/2 \\ 7 \\ i\mz& u&=\ma 1 \\ 2 \\ 3\mz\\
v&=\ma -1 \\ 0 \\ 3 \\ 0\mz& w&=\ma 1 \\ -2 \\ 0 \\ 0\mz &\xi&=\ma b\\ b\\ b\\ -b\mz
\end{align*}
Here is a list \(\mathcal{M}\) of matrices
\begin{align*}
A&=\ma 0 & 1 \\ 1 & 1\mz&B&=\ma 2 & 3 \\ 1 & 1\mz&C&=\ma 1 & 1 & 1\\ 1 & 0 & -2\mz\\
D&=\ma 1 & 1 \\ -2 & 8 \\ 1/2 & 3\mz&E&=\ma -1 & 1 & 0\mz& F&=\ma 1 & 0 & 2 \\ 1 & -1 & 1 \\ 3 & 0 & 0\mz\\
G&=\ma 1 & 3 & -1\\ 0 & 1 & 0 \\ 1 & 1 & 0\mz&H&=\ma 1 & 0 & -1 & 0 \\ -2 & 1 & 0 & 4 \\ -17 & 2 & 3 & 5 \\ 1 & -2 & 0 & 0\mz&J&=\ma 0 & 1 & -2 & 1 \\ 1 & 1 & 1 & 1 \mz
\end{align*}


\begin{Exercise}\label{exr:matvec}
For each \(V\in\mathcal{V}\) and each \(M\in\mathcal{M}\), state
whether the vector \(MV\) is defined and, if it is defined, compute
it.


\end{Exercise}
\begin{Exercise}\label{exr:matmul}
For \(N=\ma 1 & -1\\ 2 & 0 \\ 4 & -2\mz\) and for each
\(M\in\mathcal{M}\) state whether \(NM\) and/or \(MN\) is defined
and calculate any products which are defined.


\end{Exercise}
\begin{Exercise}\label{exr:exp}
Find the exponentials of the following matrices (\(\lambda\) is just
some number, \(i\) is the square root of \(-1\)): \[A=\ma 1 & 1 \\ 0
& 0\mz,\quad B=\ma 0 & 1 & 0 \\ 0 & 0 & 1 \\ 0 & 0 & 0\mz,\quad
C=\ma \lambda & 0 \\ 0 & \lambda\mz,\quad D=\ma 0 & i \\ i & 0\mz.\]




\end{Exercise}
\newpage


\begin{Exercise}\label{exr:cols}
Show that if \[A=\ma A_{11} & \cdots & A_{1n} \\ \vdots & & \vdots
\\ A_{m1} &\cdots & A_{mn}\mz\quad\mbox{and}\quad e_1=\ma 1 \\ 0
\\ \vdots \\ 0\mz\] then \(Ae_1\) is the first column of
\(A\). Which vectors \(e_2,\ldots,e_n\) will give the second,
third,..., \(n\)th columns?


\end{Exercise}
\begin{Exercise}\label{exr:truefalse}
Let \(X\) and \(Y\) denote \(2\)-by-\(2\) matrices. Are the
following statements true or false? In each case, give a proof or a
counterexample to support your claim.
\begin{itemize}
\item If \(X^2=I\) then \(X=\pm I\).
\item If \(XY=0\) then \(X=0\) or \(Y=0\).
\item If \(X\) has real entries then \(X^2\neq -I\).
\item If \(Xe_1=Xe_2=0\) then \(X=0\) (\(e_1,e_2\) are from
\cref{exr:cols}).


\end{itemize}
\end{Exercise}
\begin{Exercise}\label{exr:fib}
Take the vector \(\ma 1 \\ 1 \mz\) and the matrix \(A=\ma 0 & 1 \\ 1
& 1 \mz\). Apply \(A\) to \(v\). Then apply \(A\) again. Then apply
\(A\) again. Continue until you spot a pattern. Can you express the
pattern as a formula? Can you prove that this pattern is going to
continue? (Hint: You may write \(F_n\) for the \(n\)th term in a
certain famous sequence of numbers).


\end{Exercise}
\begin{Exercise}\label{exr:ref}
Check that the matrix \[H_\phi:=\ma \cos\phi & \sin\phi \\ \sin\phi
& -\cos\phi\mz\] fixes the vector \(v=\ma \cos(\phi/2)
\\ \sin(\phi/2)\mz\) and sends the vector \(w=\ma -\sin(\phi/2)
\\ \cos(\phi/2)\mz\) to \(-w\). (Hint: Remember your trigonometric
identities...)


{\em This means that \(H_\phi\) represents a reflection in the line
containing \(v\).}


\end{Exercise}
\begin{Exercise}[Special relativity velocity addition]\label{exr:lorentz}
Given a number \(v\), define the matrix \(\Lambda(v)=\ma
\frac{1}{\sqrt{1-v^2}} & \frac{-v}{\sqrt{1-v^2}}
\\ \frac{-v}{\sqrt{1-v^2}} & \frac{1}{\sqrt{1-v^2}}\mz\). Check that
\[\frac{1}{\sqrt{1-\left(\frac{u+v}{1+uv}\right)^2}}=\frac{1+uv}{\sqrt{(1-u^2)(1-v^2)}}\]
for all \(u,v\). Deduce that
\[\Lambda(u)\Lambda(v)=\Lambda\left(\frac{u+v}{1+uv}\right).\]


\end{Exercise}
%# Exercise exr:lorentzexp
% In the notation of \cref{exr:lorentz}, find a matrix \(M(v)\) such
% that \(\exp(M(v))=\Lambda(v)\). (Hint: Use Assessed
% \cref{exr:exphyp}.)


\newpage


\setcounter{section}{1}
\section{Solutions}
\setcounter{section}{1}


\begin{Solution}\label{sol:exr:matvec}
Here are tables of solutions for matrix/vector multiplications which
are well-defined:


\begin{tabular}{c|ccc}
& \(A\) & \(B\) & \(D\) \\
\hline
\(p\) & \(\ma 0 \\ 1 \mz\) & \(\ma 2 \\ 1 \mz\) & \(\ma 1 \\ -2 \\ 1/2\mz\) \\
\(q\) & \(\ma 1 \\ 1 \mz\) & \(\ma 3 \\ 1 \mz\) & \(\ma 1 \\ 8 \\ 3 \mz\) \\
\(r\) & \(\ma 1 \\ 2 \mz\) & \(\ma 5 \\ 2 \mz\) & \(\ma 2 \\ 6 \\ 7/2\mz\) \\
\(s\) & \(\ma -2 \\ -1\mz\) & \(\ma -4 \\ -1\mz\) & \(\ma -1 \\ -18 \\ -11/2\mz\)
\end{tabular}
\qquad
\begin{tabular}{c|cc}
& \(H\) & \(J\) \\
\hline
\(v\) & \(\ma -4 \\ 2 \\ 26 \\ -1\mz\) & \(\ma -6 \\ 2\mz\)\\
\(w\) & \(\ma 1 \\ -4 \\ -21 \\ 5\mz\) & \(\ma -2 \\ -1\mz\)\\
\(\xi\) & \(\ma 0 \\ -5b \\ -17b \\ -b\mz\) & \(\ma -2b \\ 2b\mz\)
\end{tabular}


\begin{tabular}{c|cccc}
& \(C\) & \(E\) & \(F\) & \(G\)\\
\hline
\(t\) & \(\ma i+13/2 \\ -2i-1/2\mz\) & \(\ma 13/2\mz\) & \(\ma 2i-1/2 \\ i-13/2 \\ -3/2\mz\) & \(\ma -i+41/2\\ 7 \\ 13/2\mz\)\\
\(u\) & \(\ma 6 \\ -5 \mz\) & \(\ma 1 \mz\) & \(\ma 7 \\ 2 \\ 3 \mz\) & \(\ma 4 \\ 2 \\ 3\mz\).
\end{tabular}


\end{Solution}
\begin{Solution}\label{sol:exr:matmul}
Here are the matrix multiplications which are well-defined:
\begin{align*}
\ma 1 & -1 \\ 2 & 0 \\ 4 & -2 \mz\ma 0 & 1 \\ 1 & 1 \mz&=\ma -1 & 0 \\ 0 & 2 \\ -2 & 2\mz\\
\ma 1 & -1 \\ 2 & 0 \\ 4 & -2 \mz\ma 2 & 3 \\ 1 & 1 \mz&=\ma 1 & 2 \\ 4 & 6 \\ 6 & 10 \mz\\
\ma 1 & -1 \\ 2 & 0 \\ 4 & -2 \mz\ma 1 & 1 & 1 \\ 1 & 0 & 2 \mz&=\ma 0 & 1 & -1 \\ 2 & 2 & 2 \\ 2 & 4 & 0\mz\\
\ma 1 & -1 \\ 2 & 0 \\ 4 & -2 \mz\ma 0 & 1 & -2 & 1 \\ 1 & 1 & 1 & 1 \mz&=\ma -1 & 0 & -3 & 0 \\ 0 & 2 & -4 & 2 \\ -2 & 2 & -10 & 2\mz\\
\ma 1 & 1 & 1 \\ 1 & 0 & 2 \mz\ma 1 & -1 \\ 2 & 0 \\ 4 & -2 \mz&=\ma 7 & -3 \\ 9 & -5\mz\\
\ma -1 & 1 & 0\mz\ma 1 & -1 \\ 2 & 0 \\ 4 & -2 \mz&=\ma 1 & 1 \mz\\
\ma 1 & 0 & 2 \\ 1 & -1 & 1 \\ 3 & 0 & 0\mz\ma 1 & -1 \\ 2 & 0 \\ 4 & -2 \mz&=\ma 9 & -5 \\ 3 & -3 \\ 3 & -3\mz\\
\ma 1 & 3 & -1 \\ 0 & 1 & 0 \\ 1 & 1 & 0\mz\ma 1 & -1 \\ 2 & 0 \\ 4 & -2 \mz&=\ma 3 & 1 \\ 2 & 0 \\ 3 & -1\mz
\end{align*}


\end{Solution}
\begin{Solution}\label{sol:exr:exp}
\begin{enumerate}
\item If \(A=\ma 1 & 1 \\ 0 & 0 \mz\) then
\begin{align*}
A^2&=\ma 1 & 1 \\ 0 & 0 \mz\\
A^3&=\ma 1 & 1 \\ 0 & 0 \mz\\
\cdots
\end{align*}
so \(\exp(A)=I+A+\frac{1}{2}A^2+\cdots=
I+A\left(1+\frac{1}{2}+\frac{1}{3!}+\cdots\right)\). Therefore,
since \(e-1=1+\frac{1}{2}+\frac{1}{3!}+\cdots\), we have
\(\exp(A)=\ma e & e-1 \\ 0 & 1\mz\).
\item If \(B=\ma 0 & 1 & 0 \\0 & 0 & 1 \\ 0 & 0 & 0\mz\) then
\begin{align*}
B^2&=\ma 0 & 0 & 1 \\ 0 & 0 & 0 \\ 0 & 0 & 0 \mz\\
B^3&=0=B^4=\cdots,
\end{align*}
so \(\exp(B)=I+B+\frac{1}{2}B^2=\ma 1 & 1 & 1/2 \\ 0 & 1 & 1 \\ 0
& 0 & 1\mz\).
\item If \(C=\ma \lambda & 0 \\ 0 & \lambda\mz\) then \(C^n=\ma
\lambda^n & 0 \\ 0 & \lambda^n\mz\), so \(\exp(C)=\ma 1 & 0 \\ 0 &
1 \mz+\ma \lambda & 0 \\ 0 & \lambda\mz+\frac{1}{2}\ma \lambda^2 &
0 \\ 0 & \lambda^2\mz+\cdots=\ma e^{\lambda} & 0 \\ 0 &
e^{\lambda}\mz\).
\item If \(D=\ma 0 & i \\ i & 0 \mz\) then \(D^2=-I\), \(D^3=-D\),
\(D^4=I\) and we get
\[\exp(D)=I\left(1-\frac{1}{2!}+\frac{1}{4!}-\cdots\right)+
D\left(1-\frac{1}{3!}+\frac{1}{5!}-\cdots\right)\]
i.e. \(\exp(D)=\ma \cos(1) & i\sin(1) \\ i\sin(1) &
\cos(1)\mz\). \qedhere


\end{enumerate}
\end{Solution}
\begin{Solution}\label{sol:exr:cols}
When we multiply the \(i\)th row of \(A\) into \(e_1\), we just pick
up the first entry of \(A\) because only the first entry of \(e_1\)
is nonzero and it is equal to one. Therefore \(Ae_1\) is the first
column of \(A\). To get all the columns we use vectors \(e_i\),
\(i=1,\ldots,n\) where \(e_i\) is the vector with zeros everywhere
except in the \(i\)th row, where it has a \(1\).


\end{Solution}
\begin{Solution}\label{sol:exr:truefalse}
\begin{itemize}
\item If \(X^2=I\) then \(X=\pm I\). This is false, for example \(X=\ma
0 & 1 \\ 1 & 0\mz\) satisfies \(X^2=I\).
\item If \(XY=0\) then \(X=0\) or \(Y=0\). This is false, for example
\(X=\ma 1 & 0 \\ 0 & 0 \mz\) and \(Y=\ma 0 & 0 \\ 0 & 1\mz\)
satisfy \(XY=0\).
\item If \(X\) has real entries then \(X^2\neq -I\). This is false, for
example \(X=\ma 0 & -1 \\ 1 & 0 \mz\) satisfies \(X^2=-I\).
\item If \(Xe_1=Xe_2=0\) then \(X=0\) (\(e_1,e_2\) are from
\cref{exr:cols}). This is true, because if \(Xe_1=Xe_2=0\) then
the columns of \(X\) are zero.


\end{itemize}
\end{Solution}
\begin{Solution}\label{sol:exr:fib}
We have \[v=\ma 1\\ 1\mz,\quad Av=\ma 1\\ 2\mz,\quad A^2v=\ma
2\\ 3\mz,\quad A^3v=\ma 3\\ 5\mz,\quad A^4v=\ma 5\\8\mz,\ldots\] so
it looks like the formula \(A^nv=\ma F_n\\ F_{n+1}\mz\) should hold,
where \[F_1,F_2,F_3,F_4,F_5,F_6\ldots=1,1,2,3,5,8,\ldots\] is the
Fibonacci sequence. Indeed, by definition \(F_{n+2}=F_{n+1}+F_n\),
and our matrix gives \[A\ma F_n\\ F_{n+1}\mz=\ma
F_{n+1}\\ F_n+F_{n+1}\mz=\ma F_{n+1}\\ F_{n+2}\mz,\] so since
\(v=\ma F_1\\ F_2\mz\), the result follows by induction.


\end{Solution}
\begin{Solution}\label{sol:exr:ref}
We have \[\ma \cos\phi & \sin\phi\\ \sin\phi &
-\cos\phi\mz\ma\cos(\phi/2)\\ \sin(\phi/2)\mz=\ma
\cos\phi\cos(\phi/2)+\sin\phi\sin(\phi/2)\\ \sin\phi\cos(\phi/2)-\cos\phi\sin(\phi/2)\mz,\]
and the trigonometric identities \(\cos(A+B)=\cos A\cos B-\sin A\sin
B\) and \(\sin(A+B)=\sin A\cos B+\sin B\cos A\) imply that this is
equal to \(\ma \cos(\phi-(\phi/2))\\\sin(\phi-(\phi/2))\mz=\ma
\cos(\phi/2)\\ \sin(\phi/2)\mz\). Similarly, we get \(H_\phi w=-w\)
where \(w=\ma -\sin(\phi/2)\\ \cos(\phi/2)\mz\). We note that
\(v\cdot w=-\cos(\phi/2)\sin(\phi/2)+\sin(\phi/2)\cos(\phi/2)=0\),
so \(v\) and \(w\) are orthogonal. Therefore \(H_\phi\) represents
the transformation which reflects in the line spanned by \(v\).


\end{Solution}
\begin{Solution}\label{sol:exr:lorentz}
We have
\begin{align*}
\frac{1}{\sqrt{1-\left(\frac{u+v}{1+uv}\right)^2}}&=\frac{1}{\sqrt{\frac{(1+uv)^2-(u+v)^2}{(1+uv)^2}}}\\
&=\frac{1+uv}{\sqrt{(1+uv)^2-(u+v)^2}}\\
&=\frac{1+uv}{\sqrt{1+u^2v^2+2uv-u^2-v^2-2uv}}\\
&=\frac{1+uv}{\sqrt{1+u^2v^2-u^2-v^2}}\\
&=\frac{1+uv}{\sqrt{(1-u^2)(1-v^2)}}.
\end{align*}
Now
\begin{align*}
\Lambda(u)\Lambda(v)&=\frac{1}{\sqrt{(1-u^2)(1-v^2)}}\ma 1 & -u \\ -u & 1 \mz\ma 1 & -v \\ -v & 1\mz\\
&=\frac{1}{\sqrt{(1-u^2)(1-v^2)}}\ma 1+uv & -(u+v) \\ -(u+v) & 1+uv\mz\\
\Lambda\left(\frac{u+v}{1+uv}\right)&=\frac{1}{\sqrt{1-\left(\frac{u+v}{1+uv}\right)^2}}\ma 1 & \frac{-(u+v)}{1+uv} \\ \frac{-(u+v)}{1+uv} & 1 \mz\\
&=\frac{1+uv}{\sqrt{(1-u^2)(1-v^2)}}\ma 1 & \frac{-(u+v)}{1+uv} \\ \frac{-(u+v)}{1+uv} & 1 \mz\\
&=\frac{1}{\sqrt{(1-u^2)(1-v^2)}}\ma 1+uv & -(u+v) \\ -(u+v) & 1+uv \mz.
\end{align*}
This shows
\(\Lambda(u)\Lambda(v)=\Lambda\left(\frac{u+v}{1+uv}\right)\).


{\em This matrix is used in special relativity to transform from one
reference frame to another which is moving with relative velocity
\(v\) (working in units where the speed of light is 1, and with the
simplifying assumption that space is 1-dimensional!). Naively, you
would expect that if you increase the relative velocity by \(v\) and
then by \(u\), you would end up increasing it overall by \(u+v\),
but this formula shows that velocity addition is more subtle.}


\end{Solution}
%# Solution sol:exr:lorentzexp Solution to \cref{exr:lorentzexp}
% \Cref{exr:exphyp} shows that \(\exp\ma 0 & t \\ t & 0 \mz=\ma \cosh
% t & \sinh t \\ \sinh t & \cosh t\mz\). Any numbers \(x,y\) with
% \(x^2-y^2=1\) can be realised as \(x=\cosh t\) and \(y=\sinh t\) for
% some unique \(t\). If we take \(x=\frac{1}{\sqrt{1-v^2}}\) and
% \(y=\frac{-v}{\sqrt{1-v^2}}\) then these satisfy
% \(x^2-y^2=\frac{1}{1-v^2}-\frac{v^2}{1-v^2}=1\), so there exists a
% \(t\) such that \(\cosh t=\frac{1}{\sqrt{1-v^2}}\) and \(\sinh
% t=\frac{-v}{\sqrt{1-v^2}}\), and we see that \(\exp\ma 0 & t \\ t &
% 0 \mz=\Lambda(v)\).


\setcounter{section}{3}
\end{document}
