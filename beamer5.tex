\documentclass{beamer}
\title{Linear Algebra, Week 5}
\author{Jonny Evans}
\date{MATH105}
\usenavigationsymbolstemplate{}
\usepackage[utf8]{inputenc}
\usepackage[T1]{fontenc}
\usepackage{fixltx2e}
\usepackage{graphicx}
\usepackage{longtable}
\usepackage{float}
\usepackage{parallel}
\usepackage{parcolumns}
\usepackage{wrapfig}
\usepackage{rotating}
\usepackage{amsmath}
\usepackage{textcomp}
\usepackage{marvosym}
\usepackage{wasysym}
\usepackage{amssymb}
\usepackage{amsthm,amsmath,amsfonts,amscd,setspace}
\renewcommand{\partname}{Week}
\usepackage{tikz}
\usetikzlibrary{decorations.markings,decorations.pathmorphing,shapes}
\usepackage{parskip}
\newcommand{\FF}{\mathfrak{f}}
\newcommand{\GG}{\mathfrak{g}}
\newcommand{\CC}{\mathbb{C}}
\newcommand{\QQ}{\mathbb{Q}}
\newcommand{\RR}{\mathbb{R}}
\newcommand{\UU}{\mathbb{U}}
\newcommand{\XX}{\mathbb{X}}
\newcommand{\YY}{\mathbb{Y}}
\newcommand{\ZZ}{\mathbb{Z}}
\newcommand{\Link}{\operatorname{Link}}
\newcommand{\Cone}{\operatorname{Cone}}
\newcommand{\colim}{\operatorname{colim}}
%\newcommand{\amal}{\operatorname{amal}}
\newcommand{\adj}{\operatorname{adj}}
\newcommand{\cp}[1]{\mathbf{CP}^{#1}}
\newcommand{\rp}[1]{\mathbf{RP}^{#1}}
\newcommand{\OP}[1]{\mathrm{#1}}
\newcommand{\ma}{\begin{pmatrix}}
\newcommand{\mz}{\end{pmatrix}}
\newcommand{\tka}{\begin{center}\begin{tikzpicture}}
\newcommand{\tkz}{\end{tikzpicture}\end{center}}
\newcommand{\matr}[4]{\left(\begin{array}{cc}#1 & #2\\ #3 & #4\end{array}\right)}
\newcommand{\mthrthr}[9]{\left(\begin{array}{ccc}#1 & #2 & #3\\ #4 & #5 & #6\\ #7 & #8 & #9\end{array}\right)}
\newcommand{\vect}[2]{\left(\begin{array}{c}#1\\#2\end{array}\right)}
\newcommand{\vthr}[3]{\left(\begin{array}{c}#1\\#2\\#3\end{array}\right)}
\newcommand{\TO}[3]{#1\stackrel{#2}{\longrightarrow}#3}
\makeatletter
\renewcommand*\env@matrix[1][*\c@MaxMatrixCols c]{%
  \hskip -\arraycolsep
  \let\@ifnextchar\new@ifnextchar
  \array{#1}}
\makeatother

\begingroup
\makeatletter
\@for\theoremstyle:=definition,remark,plain\do{%
\expandafter\g@addto@macro\csname th@\theoremstyle\endcsname{%
\addtolength\thm@preskip\parskip
}%
}
\endgroup
\usepackage{graphicx}
\usepackage[capitalise]{cleveref}
%\newtheorem{Theorem}{Theorem}[section]
%\newtheorem{Lemma}[Theorem]{Lemma}
%\newtheorem{Corollary}[Theorem]{Corollary}
%\newtheorem{Proposition}[Theorem]{Proposition}
%\theoremstyle{remark}
%\newtheorem{Remark}[Theorem]{Remark}
%\theoremstyle{definition}
%\newtheorem{Definition}[Theorem]{Definition}
%\newtheorem{Example}[Theorem]{Example}
%\newtheorem{Exercise}[Theorem]{Exercise}
%\newtheorem{Question}[Theorem]{Question}
%\newtheorem{Solution}[Theorem]{Solution}
%\newtheorem{Answer}[Theorem]{Answer}
\crefname{Theorem}{Theorem}{Theorems}
\Crefname{Theorem}{Theorem}{Theorems}
\crefname{Lemma}{Lemma}{Lemmas}
\Crefname{Lemma}{Lemma}{Lemmas}
\crefname{Corollary}{Corollary}{Corollaries}
\Crefname{Corollary}{Corollary}{Corollaries}
\crefname{Claim}{Claim}{Claims}
\Crefname{Claim}{Claim}{Claims}
\crefname{Proposition}{Proposition}{Propositions}
\Crefname{Proposition}{Proposition}{Propositions}
\crefname{Remark}{Remark}{Remarks}
\Crefname{Remark}{Remark}{Remarks}
\crefname{Definition}{Definition}{Definitions}
\Crefname{Definition}{Definition}{Definitions}
\crefname{Example}{Example}{Examples}
\Crefname{Example}{Example}{Examples}
\crefname{Exercise}{Exercise}{Exercises}
\Crefname{Exercise}{Exercise}{Exercises}
\crefname{Question}{Question}{Questions}
\Crefname{Question}{Question}{Questions}
\crefname{Solution}{Solution}{Solutions}
\Crefname{Solution}{Solution}{Solutions}
\crefname{Answer}{Answer}{Answers}
\Crefname{Answer}{Answer}{Answers}
\crefformat{footnote}{#2\footnotemark[#1]#3}

\begin{document}
\maketitle
\begin{frame}
\begin{center}
{\huge Subspaces, I}
\end{center}


\end{frame}
\begin{frame}
\begin{Definition}
A subset \(V\subset\RR^n\) is a {\em linear subspace} if
\begin{itemize}
\item \(v,w\in V\) implies \(v+w\in V\).
\item \(v\in V\) and \(\lambda\in\RR\) implies \(\lambda v\in V\).


\end{itemize}
\end{Definition}
\tka
\draw[dotted] (0,-0.8) -- (0,0);
\draw (0,-1.2) -- (0,-0.8);
\draw (0,2) -- (0,2.2);
\draw (-2,-1) -- (-2*0.5,-1*0.5);
\draw[dotted] (2*0.5,1*0.5) -- (0,0);
\draw (-2,1) -- (-2*0.64,1*0.64);
\draw[dotted] (-2*0.64,1*0.64) -- (0,0);
\draw[thick,blue] (-1,-2.5) -- (-1*0.35,-2.5*0.35);
\draw[thick,blue,dotted] (-1*0.35,-2.5*0.35) -- (0,0);
\filldraw[thick,red,opacity=0.5] (-1,-1) -- (2,-1/2) -- (0.8,2.2) -- (-1.5,2) -- cycle;
\draw[thick,blue] (0,0) -- (1,2.5);
\draw (0,0) -- (2*0.4,-1*0.4);
\draw (0,0) -- (-2*0.54,-1*0.54);
\draw (2*0.4,-1*0.4) -- (2,-1);
\draw (0,0) -- (0,2);
\node at (0,0) {\(\bullet\)};
\tkz


\end{frame}
\begin{frame}
\begin{Definition}
A subset \(V\subset\RR^n\) is an {\em affine subspace} if there
exists a vector \(w\) and a linear subspace \(V'\) such that
\[V=\{v\in\RR^n\ :\ v=w+v',\ v'\in V'\}.\]


\end{Definition}
\tka
\draw (0,-0.8) -- (0,1.12);
\draw[dotted] (0,1.12) -- (0,2.2);
\draw (0,2) -- (0,2.5);
\draw (-2,-1) -- (-2*0.5,-1*0.5);
\draw (2*0.5,1*0.5) -- (0,0);
\draw (-2,1) -- (-2*0.64,1*0.64);
\draw (-2*0.64,1*0.64) -- (0,0);
\draw[thick,blue] (-1+0.5,-2.5) -- (1*0.5+0.5,2.5*0.5);
\draw[thick,blue,dotted] (1*0.5+0.5,2.5*0.5) -- (1*0.65+0.5,2.5*0.65);
\filldraw[thick,red,opacity=0.5] (-1,1) -- (2,1.4) -- (0.8,2.2) -- (-1.5,2) -- cycle;
\draw (0,0) -- (2*0.4,-1*0.4);
\draw (0,0) -- (-2*0.54,-1*0.54);
\draw (2*0.4,-1*0.4) -- (2,-1);
\draw[thick,blue] (1*0.65+0.5,2.5*0.65) -- (1+0.5,2.5);
\node at (0,0) {\(\bullet\)};
\draw[thick,purple,->] (0,0) -- (0.2*2,0.2*-1);
\draw[thick,purple,->] (0,0) -- (0.3,1.4);
\tkz


\end{frame}
\begin{frame}
\begin{center}
\begin{tabular}{ccc}
\onslide<1->Subspace & dimension & codimension \\
\hline\onslide<14->Point & \onslide<15->\(0\) & \onslide<16->\(n\)\\
\onslide<2->Line & \onslide<3->\(1\) & \onslide<4-> \onslide<12->\(n-1\)\\
\onslide<4->Plane & \onslide<5->\(2\) & \onslide<5-> \onslide<13->\(n-2\)\\
\onslide<6->\(\vdots\) & \onslide<6->\(\vdots\) & \onslide<13->\(\vdots\)\\
\onslide<7->Hyperplane & \onslide<8->\(n-1\) & \onslide<10->\(1\)\\
\onslide<9->Ambient space \(\RR^n\) & \onslide<9->\(n\) & \onslide<11->\(0\)
\end{tabular}
\end{center}




\end{frame}
\begin{frame}
{Hyperplanes}


\onslide<1->A linear hyperplane is cut out by a single linear equation
\[r_1x_1+\cdots+r_nx_n=0.\]


\onslide<2->i.e. fix a row vector \(r=\ma r_1 & \cdots & r_n\mz\);
we get the hyperplane \[r^{\perp}:=\left\{x\in\RR^n\ :\
rx=0\right\}.\] \onslide<3->\(r^{\perp}\) is the {\em orthogonal
complement} to \(r^T=\ma r_1 \\ \vdots \\ r_n\mz\).


\end{frame}
\begin{frame}
{Intersections of hyperplanes}


An \(m\)-by-\(n\) matrix \(A\) defines \(m\) linear hyperplanes:
\begin{align*}
A_{11}x_1+\cdots+A_{1n}x_n&=0\\
\vdots\qquad\qquad\vdots\qquad\qquad&\vdots\\
A_{m1}x_1+\cdots+A_{mn}x_n&=0.
\end{align*}
i.e. \(Ax=0\). A solution \(x\) to this system of equations
represents a point in the intersection of these hyperplanes.


\begin{Definition}
The kernel \(\ker(A)\) is defined to be this intersection, i.e.
\[\ker(A)=\{x\in\RR^n\ :\ Ax=0\}.\]


\end{Definition}
\end{frame}
\begin{frame}
\begin{Example}
Let \(A=\ma 1 & 1 \\ 1 & -1 \\ 0 & 1\mz\) and \(b=\ma \temporal<4>{1}{1}{\color{red}3\color{black}} \\ 1
\\ 1\mz\).


\(Ax=b\) defines three {\em lines} (hyperplanes in \(\RR^2\)):


\end{Example}
\onslide<1->
\begin{flushright}
\tka
\draw[->] (-2,0) -- (3,0) node [right] {\(x\)};
\draw[->] (0,-2) -- (0,2) node [above] {\(y\)};
\onslide<2-4>{\draw[red,thick] (-1,2) -- (2,-1) node [right] {\(x+y=1\)};}
\onslide<5->{\draw[red,thick] (1,2) -- (4,-1) node [right] {\(x+y=3\)};}
\onslide<3->{\draw[purple,thick] (-1,-2) -- (3,2) node [above] {\(x-y=1\)};}
\onslide<4->{\draw[blue,thick] (-2,1) -- (3,1) node [right] {\(y=1\)};}
\tkz
\end{flushright}
\color{black}


\end{frame}
\begin{frame}
\begin{Definition}
{\em Nullity} of \(A\) \(=\dim\ker(A)\).


\end{Definition}
\begin{Theorem}
Let \(A\) be an \(m\)-by-\(n\) matrix and \(b\in\RR^m\) be a
vector. The space of solutions to \(Ax=b\) is either empty or a
translate of \(\ker(A)\).


\end{Theorem}
\begin{Corollary}
\(\dim\ker(A)=\) number of free indices of reduced echelon form of
\(A\).


\end{Corollary}
\end{frame}
\begin{frame}
\begin{center}
{\huge Subspaces, II}
\end{center}


\end{frame}
\begin{frame}
\begin{Definition}
A {\em linear combination} of vectors \(v_1,\ldots,v_k\in\RR^n\)
is any vector of the form
\[\lambda_1v_1+\cdots+\lambda_kv_k,\qquad\lambda_i\in\RR.\]


\end{Definition}
\begin{Definition}
The {\em subspace spanned by \(v_1,\ldots,v_k\)} is the set of all their
linear combinations:
\[\mathrm{span}(v_1,\ldots,v_k)=\{\lambda_1v_1+\cdots+\lambda_kv_k\
:\ \lambda_i\in\RR\}.\]


\end{Definition}
\begin{Lemma}
\(\mathrm{span}(v_1,\ldots,v_k)\) is a linear subspace of \(\RR^n\).


\end{Lemma}
\end{frame}
\begin{frame}
\begin{Definition}
If \(V=\mathrm{span}(v_1,\ldots,v_k)\), we call
\(\{v_1,\ldots,v_k\}\) a {\em spanning set} for \(V\). A minimal
spanning set is called a {\em basis}.


\end{Definition}
\begin{Theorem}
Two bases for the same subspace \(V\) have the same size\\ (this
size is called the {\em dimension} of \(V\)).


\end{Theorem}
\end{frame}
\begin{frame}
{Image}


\begin{Definition}
The image \(\mathrm{im}(A)\) of an \(m\)-by-\(n\) matrix \(A\) is
the subspace of \(\RR^m\) consisting of all \(b\in\RR^m\) such
that \(Ax=b\) has a solution \(x\in\RR^n\).


\end{Definition}
\begin{Lemma}
\(\mathrm{im}(A)\) is a subspace of \(\RR^n\). It is spanned by
the columns of \(A\).


\end{Lemma}
\end{frame}
\begin{frame}
\begin{Definition}
The {\em rank} of \(A\) is the dimension of \(\mathrm{im}(A)\).


\end{Definition}
\begin{Theorem}
The rank of \(A\) is the number of leading indices of the reduced
echelon form of \(A\).


\end{Theorem}
\begin{Corollary}[Rank-nullity theorem]\label{cor:ranknullity}
If \(A\) is \(m\)-by-\(n\) then
\(\mathrm{rank(A)}+\mathrm{null}(A)=n\).


\end{Corollary}
\end{frame}
\begin{frame}
{Summary}


\begin{Theorem}
Let \(A\) be an \(m\)-by-\(n\) matrix and \(b\in\RR^m\) be a
vector.
\begin{itemize}
\item \(Ax=b\) has a solution if and only if \(b\in\mathrm{im}(A)\),
\item the space of solutions is a translate of \(\ker(A)\).


\end{itemize}
\end{Theorem}
\tka
\node (A) at (0,0) {\(\RR^n\)};
\node (B) at (2,0) {\(\RR^m\)};
\node at (0,-1) {\(\ker(A)\)};
\node at (2,-1) {\(\mathrm{im}(A)\)};
\draw[thick,->] (A) -- (B) node [midway,above] {\(A\)};
\node[rotate=90] at (0,-1/2) {\(\subseteq\)};
\node[rotate=90] at (2,-1/2) {\(\subseteq\)};
\tkz


\end{frame}
\begin{frame}
\begin{center}
{\huge Linear maps}
\end{center}


\end{frame}
\begin{frame}
{Vector space}
\begin{Definition}
A (\temporal<15>{\(\RR\)}{\(\CC\)}{\(k\)}-)vector space is a set \(v\) together with:
\begin{itemize}
\item <2-> a map \(V\times V\to V\), written \((v,w)\mapsto v+w\),
\item <3-> a map \(\temporal<15>{\RR}{\CC}{k}\times V\to V\), written \((\lambda,v)\mapsto
\lambda v\),
\item <4-> an element \(0\in V\),
\end{itemize}
\onslide<5->such that:
\begin{align*}
\onslide<6->{u+(v+w)&=(u+v)+w&}\onslide<7->{ v+w&=w+v\\}
\onslide<8->{v&=0+v=v+0&}\onslide<9->{v+(-v)&=0\\}
\onslide<10->{1v&=v&}\onslide<11->{\lambda(\mu v)&=(\lambda\mu)v\\}
\onslide<12->{(\lambda+\mu)v&=\lambda v+\mu v&}\onslide<13->{\lambda(v+w)&=\lambda v+\lambda w}
\end{align*}
\onslide<14->for all \(u,v,w\in V\) and
\(\lambda,\mu\in\temporal<15>{\RR}{\CC}{k}\).


\onslide<16>\(k\) can be any {\em field}: \(\RR\), \(\CC\),
\(\QQ\), \(\ZZ/p\), \(\QQ_p\),...


\end{Definition}
\end{frame}
\begin{frame}
\begin{Example}
\begin{itemize}
\item <1-> \(V=\RR^n\) (usual addition and scaling of vectors)
\item <2-> \(V=\mathcal{C}^0(\RR)\) (continuous functions
\(\RR\to\RR\)):
\begin{itemize}
\item <3-> \((f+g)(x)=f(x)+g(x)\),
\item <4-> \((\lambda f)(x)=\lambda f(x)\).
\end{itemize}
\item <5-> \(V=\mathcal{C}^k(\RR)\) (\(k\)-times-ctsly-differentiable
functions \(\RR\to\RR\)).
\item <6-> \(V=\mathcal{C}^\omega(\RR)\) (analytic functions).
\end{itemize}
\end{Example}
\onslide<7->
\[\mathcal{C}^0(\RR)\onslide<8->\supset\mathcal{C}^1(\RR)
\onslide<9->\supset\mathcal{C}^2(\RR) \onslide<10->\supset\cdots
\onslide<11->\supset\mathcal{C}^{\infty}(\RR)
\onslide<12->\supset\mathcal{C}^{\omega}(\RR).\]


\end{frame}
\begin{frame}
{Linear maps}


\begin{Definition}
A map \(T\colon V\to W\) is called {\em linear} if
\[T(v_1+v_2)=T(v_1)+T(v_2)\quad \mbox{and}\quad T(\lambda
v)=\lambda T(v)\] for all \(v,v_1,v_2\in V\), \(\lambda\in\RR\).


\end{Definition}
\onslide<2->
\begin{Theorem}
If \(T\colon\RR^n\to\RR^m\) is linear then there exists an
\(m\)-by-\(n\) matrix \(A\) such that \(T(v)=Av\) for all
\(v\in\RR^n\). Conversely, a matrix defines a linear map
\(v\mapsto Av\).


\end{Theorem}
\onslide<3->
\begin{Example}
Differentiation defines a linear map
\(\frac{d}{dx}\colon\mathcal{C}^1(\RR)\to\mathcal{C}^0(\RR)\).


\end{Example}
\end{frame}
\end{document}
