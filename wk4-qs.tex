\documentclass{article}
\title{Linear Algebra Worksheet 4}
\author{Jonny Evans}
\date{Workshop 4}
\usepackage[utf8]{inputenc}
\usepackage[T1]{fontenc}
\usepackage[a4paper,margin=1in]{geometry}
\usepackage{longtable}
\usepackage{fixltx2e}
\usepackage{graphicx}
\usepackage{longtable}
\usepackage{float}
\usepackage{parallel}
\usepackage{parcolumns}
\usepackage{wrapfig}
\usepackage{rotating}
\usepackage{amsmath}
\usepackage{textcomp}
\usepackage{marvosym}
\usepackage{wasysym}
\usepackage{amssymb}
\usepackage{amsthm,amsmath,amsfonts,amscd,setspace}
\renewcommand{\partname}{Week}
\newcommand{\mks}[1]{\marginpar{\colorbox{white}{\parbox{\marginparwidth}{%
  \setstretch{0.5}{\bf\scriptsize{#1 marks}}}}}}
\usepackage{tikz}
\usetikzlibrary{decorations.markings,decorations.pathmorphing,shapes}
\usepackage{parskip}
\newcommand{\FF}{\mathfrak{f}}
\newcommand{\GG}{\mathfrak{g}}
\newcommand{\CC}{\mathbb{C}}
\newcommand{\QQ}{\mathbb{Q}}
\newcommand{\RR}{\mathbb{R}}
\newcommand{\UU}{\mathbb{U}}
\newcommand{\XX}{\mathbb{X}}
\newcommand{\YY}{\mathbb{Y}}
\newcommand{\ZZ}{\mathbb{Z}}
\newcommand{\Link}{\operatorname{Link}}
\newcommand{\Cone}{\operatorname{Cone}}
\newcommand{\colim}{\operatorname{colim}}
%\newcommand{\amal}{\operatorname{amal}}
\newcommand{\adj}{\operatorname{adj}}
\newcommand{\cp}[1]{\mathbf{CP}^{#1}}
\newcommand{\rp}[1]{\mathbf{RP}^{#1}}
\newcommand{\OP}[1]{\mathrm{#1}}
\newcommand{\ma}{\begin{pmatrix}}
\newcommand{\mz}{\end{pmatrix}}
\newcommand{\tka}{\begin{center}\begin{tikzpicture}}
\newcommand{\tkz}{\end{tikzpicture}\end{center}}
\newcommand{\matr}[4]{\left(\begin{array}{cc}#1 & #2\\ #3 & #4\end{array}\right)}
\newcommand{\mthrthr}[9]{\left(\begin{array}{ccc}#1 & #2 & #3\\ #4 & #5 & #6\\ #7 & #8 & #9\end{array}\right)}
\newcommand{\vect}[2]{\left(\begin{array}{c}#1\\#2\end{array}\right)}
\newcommand{\vthr}[3]{\left(\begin{array}{c}#1\\#2\\#3\end{array}\right)}
\newcommand{\TO}[3]{#1\stackrel{#2}{\longrightarrow}#3}
\newenvironment{Proof}{\begin{proof}}{\end{proof}\ignorespacesafterend}
\makeatletter
\renewcommand*\env@matrix[1][*\c@MaxMatrixCols c]{%
  \hskip -\arraycolsep
  \let\@ifnextchar\new@ifnextchar
  \array{#1}}
\makeatother

\begingroup
\makeatletter
\@for\theoremstyle:=definition,remark,plain\do{%
\expandafter\g@addto@macro\csname th@\theoremstyle\endcsname{%
\addtolength\thm@preskip\parskip
}%
}
\endgroup
\usepackage{graphicx}
\usepackage[capitalise]{cleveref}
\newtheorem{Theorem}{Theorem}[section]
\newtheorem{Lemma}[Theorem]{Lemma}
\newtheorem{Corollary}[Theorem]{Corollary}
\newtheorem{Proposition}[Theorem]{Proposition}
\theoremstyle{remark}
\newtheorem{Remark}[Theorem]{Remark}
\theoremstyle{definition}
\newtheorem{Definition}[Theorem]{Definition}
\newtheorem{Example}[Theorem]{Example}
\newtheorem{Exercise}[Theorem]{Exercise}
\newtheorem{Question}[Theorem]{Question}
\newtheorem{Solution}[Theorem]{Solution}
\newtheorem{Answer}[Theorem]{Answer}
\crefname{Theorem}{Theorem}{Theorems}
\Crefname{Theorem}{Theorem}{Theorems}
\crefname{Lemma}{Lemma}{Lemmas}
\Crefname{Lemma}{Lemma}{Lemmas}
\crefname{Corollary}{Corollary}{Corollaries}
\Crefname{Corollary}{Corollary}{Corollaries}
\crefname{Claim}{Claim}{Claims}
\Crefname{Claim}{Claim}{Claims}
\crefname{Proposition}{Proposition}{Propositions}
\Crefname{Proposition}{Proposition}{Propositions}
\crefname{Remark}{Remark}{Remarks}
\Crefname{Remark}{Remark}{Remarks}
\crefname{Definition}{Definition}{Definitions}
\Crefname{Definition}{Definition}{Definitions}
\crefname{Example}{Example}{Examples}
\Crefname{Example}{Example}{Examples}
\crefname{Exercise}{Exercise}{Exercises}
\Crefname{Exercise}{Exercise}{Exercises}
\crefname{Question}{Question}{Questions}
\Crefname{Question}{Question}{Questions}
\crefname{Solution}{Solution}{Solutions}
\Crefname{Solution}{Solution}{Solutions}
\crefname{Answer}{Answer}{Answers}
\Crefname{Answer}{Answer}{Answers}
\crefformat{footnote}{#2\footnotemark[#1]#3}

\begin{document}
\maketitle
\setcounter{section}{4}


\begin{Exercise}\label{exr:routine1}
Find the determinants of the following matrices by using the
inductive formula. \[A=\ma 3 & 2 & 1 \\ 1 & 0 & 7 \\ 0 & 1 &
0\mz,\quad B=\ma 1 & 2 & 5 & 13 \\ 0 & 1 & 2 & 1 \\ 1 & 0 & 1 & 3
\\ 1 & 1 & 1 & 1\mz,\quad C=\ma 1 & 1 & 1 & 1 & 1 \\ 0 & 0 & 1 & 0 &
0 \\ 0 & 0 & 0 & 1 & 0 \\ 0 & 0 & 0 & 0 & 1 \\ 1 & 0 & 0 & 0 &
0\mz.\]


\end{Exercise}
\begin{Exercise}\label{exr:routine3}
Compute the determinant of the matrix \[A=\ma 1 & t & 1 \\ 2 & 3 & 1
\\ 1 & 0 & 2\mz.\] For which value of \(t\) does this matrix fail to
be invertible? For this value of \(t\), find an element of
\(\ker(A)\).




\end{Exercise}
\begin{Exercise}\label{exr:routineeval}
For each matrix below, find its characteristic polynomial, its
eigenvalues and its eigenvectors.
\begin{gather*}
A=\ma 1 & 5 \\ 2 & 4 \mz,\quad
B=\ma 3 & -2 \\ 1 & 3 \mz,\quad
C=\ma 0 & i \\ i & 0 \mz\\
D=\ma 0 & 1 & 0 \\ 0 & 0 & 1 \\ 1 & 0 & 0 \mz,\quad
E=\ma 1 & 1 & 0 \\ 1 & 0 & 1 \\ 0 & 1 & 1\mz,\quad
F=\ma 0 & 1 & 0 \\ 1 & 0 & 1 \\ 0 & 1 & 0\mz\\
G=\ma -1 & -2 & 2 \\ -1 & 1 & 1 \\ -5 & -4 & 6\mz,\quad
H=\ma -2 & 3 & -3 \\ -6 & 7 & -6 \\ -6 & 6 & -5\mz,\quad
J=\ma 18 & -5 & -6 \\ 81 & -20 & -18 \\ -22 & 6 & 7\mz
\end{gather*}


\end{Exercise}
\begin{Exercise}\label{exr:detsum}
Write down two \(2\)-by-\(2\) matrices \(A,B\) with
\(\det(A)=\det(B)=1\). Find \(\det(A+B)\). Repeat twice more with
different matrices. Can you get any value for \(\det(A+B)\)?


\end{Exercise}
\begin{Exercise}
Suppose that \(A\) has an eigenvector \(v\) with eigenvalue
\(\lambda\). Show that \(\exp(A)\) has \(v\) as an eigenvector and
find the eigenvalue.


\end{Exercise}
\begin{Exercise}\label{exr:jacobian}
The {\em Jacobian} of a differentiable map \(F\colon\RR^n\to\RR^m\)
is the \(m\)-by-\(n\) matrix \[Jac(F)=\ma \frac{\partial
F_1}{\partial x_1} & \frac{\partial F_1}{\partial x_2} & \cdots &
\frac{\partial F_1}{\partial x_{n-1}} & \frac{\partial F_1}{\partial
x_n} \\ \frac{\partial F_2}{\partial x_1} & \frac{\partial
F_2}{\partial x_2} & \cdots & \frac{\partial F_2}{\partial x_{n-1}}
& \frac{\partial F_2}{\partial x_n} \\ \vdots & \vdots & & \vdots &
\vdots \\ \frac{\partial F_m}{\partial x_1} & \frac{\partial
F_m}{\partial x_2} & \cdots &\frac{\partial F_m}{\partial x_{n-1}} &
\frac{\partial F_m}{\partial x_n}\mz,\] where
\(F(x_1,\ldots,x_n)=\ma
F_1(x_1,\ldots,x_n)\\ \vdots\\ F_m(x_1,\ldots,x_n)\mz\).


Find \(\det(Jac(F))\) in the following examples:
\begin{enumerate}
\item \(m=n=2\), \(F(r,\phi)=\ma r\cos\phi \\ r\sin\phi\mz\).
\item \(m=n=3\), \(F(r,\theta,\phi)=\ma r\sin\theta\cos\phi
\\ r\sin\theta\sin\phi \\ r\cos\theta\mz\)


\end{enumerate}
{\em If \(m=n\) and \(F\) is a change of coordinates
\(\RR^n\to\RR^n\) then the determinant of the Jacobian matrix is an
important quantity: if \(y=F(x)\) then the volume element
\(dy_1\cdots dy_n\) is equal to \(\det(Jac(F))dx_1\cdots dx_n\). In
the examples we've just computed the volume element in polar and
spherical coordinates.}


\end{Exercise}
\begin{Exercise}\label{exr:orthaxis}
Suppose that \(A\) is an \(n\)-by-\(n\) orthogonal matrix
(\(AA^T=I\)).
\begin{enumerate}
\item Show that \(\det(A)=\pm 1\).
\item If \(\det(A)=1\), show that \(\det(A-I)=(-1)^n\det(A-I)\). (Hint:
Use the fact that \(A-I=A(I-A^T)\).)
\item Deduce that if \(n\) is odd then any orthogonal matrix with
determinant one has a fixed vector.


\end{enumerate}
\end{Exercise}
\begin{Exercise}\label{exr:detprod}
Let \(A\) be an \(n\)-by-\(n\) matrix with characteristic polynomial
\(\chi_A(t)\); suppose that \(\chi_A(t)\) has \(n\) distinct
roots. By considering \(\chi_A(0)\), prove that \(\det(A)\) is the
product of the eigenvalues of \(A\). One of the coefficients in the
polynomial \(\chi_A(t)\) is equal to minus the sum of the
eigenvalues of \(A\): which coefficient? {\em (Hint: Recall that if
a polynomial \(p(t)\) of degree \(n\) has roots
\(\lambda_1,\ldots,\lambda_n\) and the coefficient of \(t^n\) is
\(1\) then \(p(t)=(t-\lambda_1)(t-\lambda_2)\cdots(t-\lambda_n)\))}.


\end{Exercise}
\setcounter{section}{3}
\section{Assessed problems for Week 4}


Total marks available: 20.


\begin{Question}
Using the inductive formula, find the determinant of \(M_t=\ma 1 & 0
& t \\ 15 & 13 & 2 \\ 8 & 7 & 1\mz\). For which value of \(t\) is
this matrix not invertible? \mks{4} For this value of \(t\), find a
vector \(v\) such that \(M_tv=0\).


\end{Question}
\begin{Question}
Let \(a,b,c\) be numbers. Using the inductive formula, \mks{4} find
the determinant of \(\ma a & b & c \\ b & c & a \\ c & a & b\mz\).


\end{Question}
\begin{Question}
Find the eigenvalues and any eigenvectors of the following matrices
\mks{4+8} \[\ma 2 & 1 \\ -1 & 4\mz,\quad \ma 1 & 2 & 3 \\ 2 & 0 & 2
\\ 3 & 2 & 1\mz.\]


\end{Question}
\end{document}
