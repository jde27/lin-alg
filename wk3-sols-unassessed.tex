\documentclass{article}
\title{Linear Algebra Worksheet 3}
\author{Jonny Evans}
\date{Workshop 3}
\include{head}
\begin{document}
\maketitle
\setcounter{section}{1}


\begin{Exercise}\label{exr:inv2}
For each matrix below, determine if it is invertible and, if it is,
write down the inverse.
\begin{align*}
A&=\ma 2 & 3 \\ 1 & 8\mz & B&=\ma 1 & -1 \\ 0 & 2\mz & C&=\ma 91 & 45 \\ 2 & 1\mz\\
D&=\ma 1 & 2 \\ 1 & -3 \mz & E&=\ma 1 & 2 \\ -3 & -6\mz & F&=\ma 1 & 1 \\ 1 & 1\mz
\end{align*}


\end{Exercise}
\begin{Exercise}
For which values of the parameter \(t\) are the following matrices
invertible? \[\ma t & 1 \\ 1 & t\mz,\quad\ma \cos t & \sin t
\\ \sin t & -\cos t\mz,\quad \ma 2 & 1 \\ t & 7\mz.\]


\end{Exercise}
\begin{Exercise}\label{exr:invech}
For each matrix below, write down the augmented matrix, put the
left-hand side into echelon form using only row operations of type
\(R_i\mapsto R_i+\lambda R_j\) and hence compute the
determinant. Then further, put the left-hand side into reduced
echelon form, and hence compute the inverses.
\begin{align*}
A&=\begin{pmatrix} 1 & 5 & -2 \\ -1 & 0 & -5 \\ 4 & -3 & 3 \end{pmatrix}&
B&=\begin{pmatrix} -5 & 1 & 3 \\ -4 & -1 & 3 \\ 5 & -3 & -4 \end{pmatrix} \\
C&=\ma 1 & 2 & 0 & 3 \\ -3 & -5 & 0 & 0 \\ -1 & -2 & 1 & -3 \\ 4 & 7 & 1 & -2\mz&
D&=\ma 2 & 0 & 2 & 0 \\ 0 & 1 & 1 & 1\\ 2 & 0 & 5 & 0 \\ 1 & 1 & 1 & 2\mz\\
E&=\ma 3 & 0 & 0 & 0 \\ -17 & -1 & 2 & 0 \\ -4 & -1 & 1 & 0 \\ 63 & 21 & -21 & -12\mz&
F&=\ma -20 & -106 & -2 & 0 \\ 0 & -3 & -1 & 0 \\ 20 & 100 & -1 & 0 \\ -100 & -500 & 10 & 1 \mz
\end{align*}


\end{Exercise}
\begin{Exercise}\label{exr:elem}
Write down the inverses of the \(4\)-by-\(4\) elementary matrices
\[E_{42}(4),\quad E_{21}(2),\quad E_{4}(3),\quad E_{34}(1),\quad
E_{13}(2)\] (for example the inverse of \(E_{42}(4)\) is
\(E_{42}(-4)\), rather than as a full matrix) find the product
\[E_{42}(4)E_{21}(2)E_4(3)E_{34}(1)E_{13}(2)\] as a \(4\)-by-\(4\)
matrix. What is the inverse of this product?


\end{Exercise}
\begin{Exercise}
Let \(A\) be an \(m\)-by-\(n\) matrix. Prove the following
statements:
\begin{itemize}
\item The number of leading indices is less than or equal to
\(\min(m,n)\) (where (\(\min(a,b)\) denotes the minimum of \(a\)
and \(b\)).
\item If \(m<n\), the equation \(Av=0\) has a nonzero solution (Hint:
This is equivalent to showing there are some free indices).
\item Suppose \(m=n\). Then \(A\) is invertible if and only if \(Av=0\)
has no solution other than \(v=0\).
\item Suppose that \(m=n\) and that \(B\) is another \(n\)-by-\(n\)
matrix. If \(A\) is not invertible then \(BA\) is not invertible.


\end{itemize}
\end{Exercise}
\newpage


\section{Solutions}


\begin{Solution}
The matrices have determinants:
\begin{align*}
\det(A)&=13& \det(B)&=2& \det(C)&=1\\
\det(D)&=-5& \det(E)&=0& \det(F)&=0.
\end{align*}
Therefore \(A,B,C,D\) are invertible, with inverses:
\begin{align*}
A^{-1}&=\frac{1}{13}\ma 8 & -3 \\ -1 & 2\mz & B^{-1}&=\frac{1}{2}\ma 2 & 1 \\ 0 & 1\mz\\
C^{-1}&=\ma 1 & -45 \\ 2 & 91\mz& D^{-1}&=-\frac{1}{5}\ma -3 & -2 \\ -1 & 1\mz.
\end{align*}


\end{Solution}
\begin{Solution}
The determinants of the three matrices are:
\[t^2-1,\quad -\cos^2 t-\sin^2t=-1,\quad 14-t,\]
so the determinant is nonzero for \(t\neq \pm 1\), for all \(t\) and
for \(t\neq 14\) respectively.


\end{Solution}
\begin{Solution}


\end{Solution}
(a) We start with the matrix \[A=\begin{pmatrix}[ c c c | c c c ] 1 &
5 & -2 & 1 & 0 & 0 \\ -1 & 0 & -5 & 0 & 1 & 0 \\ 4 & -3 & 3 & 0 & 0 &
1 \end{pmatrix}\]


Clear row \(2\) using \(R_2\mapsto R_2+(1)R_1\)\[\begin{pmatrix}[ c c
c | c c c ] 1 & 5 & -2 & 1 & 0 & 0 \\ 0 & 5 & -7 & 1 & 1 & 0 \\ 4 & -3
& 3 & 0 & 0 & 1 \end{pmatrix}\]Clear row \(3\) using \(R_3\mapsto
R_3+(-4)R_1\)\[\begin{pmatrix}[ c c c | c c c ] 1 & 5 & -2 & 1 & 0 & 0
\\ 0 & 5 & -7 & 1 & 1 & 0 \\ 0 & -23 & 11 & -4 & 0 & 1
\end{pmatrix}\]Clear row \(3\) using \(R_3\mapsto
R_3+(23/5)R_2\)\[\begin{pmatrix}[ c c c | c c c ] 1 & 5 & -2 & 1 & 0 &
0 \\ 0 & 5 & -7 & 1 & 1 & 0 \\ 0 & 0 & -106/5 & 3/5 & 23/5 & 1
\end{pmatrix}\]
{\bf At this point, we see that the determinant is \(-106\).}


Make leading entry in row 1 equal to 1 using
\(R_1\mapsto (1/5)R_1\). \[\begin{pmatrix}[ c c c | c c c ] 1 & 5 &
-2 & 1 & 0 & 0 \\ 0 & 1 & -7/5 & 1/5 & 1/5 & 0 \\ 0 & 0 & -106/5 & 3/5
& 23/5 & 1 \end{pmatrix}\]Clear column 2 using \(R_1\mapsto
R_1+(-5)R_2\)\[\begin{pmatrix}[ c c c | c c c ] 1 & 0 & 5 & 0 & -1 & 0
\\ 0 & 1 & -7/5 & 1/5 & 1/5 & 0 \\ 0 & 0 & -106/5 & 3/5 & 23/5 & 1
\end{pmatrix}\]Make leading entry in row 2 equal to 1 using
\(R_2\mapsto (-5/106)R_2\). \[\begin{pmatrix}[ c c c | c c c ] 1 & 0
& 5 & 0 & -1 & 0 \\ 0 & 1 & -7/5 & 1/5 & 1/5 & 0 \\ 0 & 0 & 1 & -3/106
& -23/106 & -5/106 \end{pmatrix}\]Clear column 3 using \(R_1\mapsto
R_1+(-5)R_3\)\[\begin{pmatrix}[ c c c | c c c ] 1 & 0 & 0 & 15/106 &
9/106 & 25/106 \\ 0 & 1 & -7/5 & 1/5 & 1/5 & 0 \\ 0 & 0 & 1 & -3/106 &
-23/106 & -5/106 \end{pmatrix}\]Clear column 3 using \(R_2\mapsto
R_2+(7/5)R_3\)\[\begin{pmatrix}[ c c c | c c c ] 1 & 0 & 0 & 15/106 &
9/106 & 25/106 \\ 0 & 1 & 0 & 17/106 & -11/106 & -7/106 \\ 0 & 0 & 1 &
-3/106 & -23/106 & -5/106 \end{pmatrix}\]Now the right-hand block is
the inverse of the matrix we started with.


\newpage


(b) We start with the matrix \[B=\begin{pmatrix}[ c c c | c c c ] -5 &
1 & 3 & 1 & 0 & 0 \\ -4 & -1 & 3 & 0 & 1 & 0 \\ 5 & -3 & -4 & 0 & 0 &
1 \end{pmatrix}\]


Clear row \(2\) using \(R_2\mapsto R_2+(-4/5)R_1\)\[\begin{pmatrix}[ c
c c | c c c ] -5 & 1 & 3 & 1 & 0 & 0 \\ 0 & -9/5 & 3/5 & -4/5 & 1 & 0
\\ 5 & -3 & -4 & 0 & 0 & 1 \end{pmatrix}\]Clear row \(3\) using
\(R_3\mapsto R_3+(1)R_1\)\[\begin{pmatrix}[ c c c | c c c ] -5 & 1 & 3
& 1 & 0 & 0 \\ 0 & -9/5 & 3/5 & -4/5 & 1 & 0 \\ 0 & -2 & -1 & 1 & 0 &
1 \end{pmatrix}\]Clear row \(3\) using \(R_3\mapsto
R_3+(-10/9)R_2\)\[\begin{pmatrix}[ c c c | c c c ] -5 & 1 & 3 & 1 & 0
& 0 \\ 0 & -9/5 & 3/5 & -4/5 & 1 & 0 \\ 0 & 0 & -5/3 & 17/9 & -10/9 &
1 \end{pmatrix}\] {\bf At this point, we see that the determinant is
\(-3\).}


Make leading entry in row 0 equal to 1 using
\(R_0\mapsto (-1/5)R_0\). \[\begin{pmatrix}[ c c c | c c c ] 1 & -1/5
& -3/5 & -1/5 & 0 & 0 \\ 0 & -9/5 & 3/5 & -4/5 & 1 & 0 \\ 0 & 0 & -5/3
& 17/9 & -10/9 & 1 \end{pmatrix}\]Make leading entry in row 1 equal to
1 using \(R_1\mapsto (-5/9)R_1\). \[\begin{pmatrix}[ c c c | c c c ]
1 & -1/5 & -3/5 & -1/5 & 0 & 0 \\ 0 & 1 & -1/3 & 4/9 & -5/9 & 0 \\ 0 &
0 & -5/3 & 17/9 & -10/9 & 1 \end{pmatrix}\]Clear column 2 using
\(R_1\mapsto R_1+(1/5)R_2\)\[\begin{pmatrix}[ c c c | c c c ] 1 & 0 &
-2/3 & -1/9 & -1/9 & 0 \\ 0 & 1 & -1/3 & 4/9 & -5/9 & 0 \\ 0 & 0 &
-5/3 & 17/9 & -10/9 & 1 \end{pmatrix}\]Make leading entry in row 2
equal to 1 using \(R_2\mapsto (-3/5)R_2\). \[\begin{pmatrix}[ c c c |
c c c ] 1 & 0 & -2/3 & -1/9 & -1/9 & 0 \\ 0 & 1 & -1/3 & 4/9 & -5/9 &
0 \\ 0 & 0 & 1 & -17/15 & 2/3 & -3/5 \end{pmatrix}\]Clear column 3
using \(R_1\mapsto R_1+(2/3)R_3\)\[\begin{pmatrix}[ c c c | c c c ] 1
& 0 & 0 & -13/15 & 1/3 & -2/5 \\ 0 & 1 & -1/3 & 4/9 & -5/9 & 0 \\ 0 &
0 & 1 & -17/15 & 2/3 & -3/5 \end{pmatrix}\]Clear column 3 using
\(R_2\mapsto R_2+(1/3)R_3\)\[\begin{pmatrix}[ c c c | c c c ] 1 & 0 &
0 & -13/15 & 1/3 & -2/5 \\ 0 & 1 & 0 & 1/15 & -1/3 & -1/5 \\ 0 & 0 & 1
& -17/15 & 2/3 & -3/5 \end{pmatrix}\]Now the right-hand block is the
inverse of the matrix we started with.


\newpage


(c) We start with the matrix \[C=\begin{pmatrix}[ c c c c | c c c c ]
1 & 2 & 0 & 3 & 1 & 0 & 0 & 0 \\ -3 & -5 & 0 & 0 & 0 & 1 & 0 & 0 \\ -1
& -2 & 1 & -3 & 0 & 0 & 1 & 0 \\ 4 & 7 & 1 & 2 & 0 & 0 & 0 & 1
\end{pmatrix}\]


Clear row \(2\) using \(R_2\mapsto R_2+(3)R_1\)\[\begin{pmatrix}[ c c
c c | c c c c ] 1 & 2 & 0 & 3 & 1 & 0 & 0 & 0 \\ 0 & 1 & 0 & 9 & 3 & 1
& 0 & 0 \\ -1 & -2 & 1 & -3 & 0 & 0 & 1 & 0 \\ 4 & 7 & 1 & 2 & 0 & 0 &
0 & 1 \end{pmatrix}\]Clear row \(3\) using \(R_3\mapsto
R_3+(1)R_1\)\[\begin{pmatrix}[ c c c c | c c c c ] 1 & 2 & 0 & 3 & 1 &
0 & 0 & 0 \\ 0 & 1 & 0 & 9 & 3 & 1 & 0 & 0 \\ 0 & 0 & 1 & 0 & 1 & 0 &
1 & 0 \\ 4 & 7 & 1 & 2 & 0 & 0 & 0 & 1 \end{pmatrix}\]Clear row \(4\)
using \(R_4\mapsto R_4+(-4)R_1\)\[\begin{pmatrix}[ c c c c | c c c c ]
1 & 2 & 0 & 3 & 1 & 0 & 0 & 0 \\ 0 & 1 & 0 & 9 & 3 & 1 & 0 & 0 \\ 0 &
0 & 1 & 0 & 1 & 0 & 1 & 0 \\ 0 & -1 & 1 & -10 & -4 & 0 & 0 & 1
\end{pmatrix}\]Clear row \(4\) using \(R_4\mapsto
R_4+(1)R_2\)\[\begin{pmatrix}[ c c c c | c c c c ] 1 & 2 & 0 & 3 & 1 &
0 & 0 & 0 \\ 0 & 1 & 0 & 9 & 3 & 1 & 0 & 0 \\ 0 & 0 & 1 & 0 & 1 & 0 &
1 & 0 \\ 0 & 0 & 1 & -1 & -1 & 1 & 0 & 1 \end{pmatrix}\]Clear row
\(4\) using \(R_4\mapsto R_4+(-1)R_3\)\[\begin{pmatrix}[ c c c c | c c
c c ] 1 & 2 & 0 & 3 & 1 & 0 & 0 & 0 \\ 0 & 1 & 0 & 9 & 3 & 1 & 0 & 0
\\ 0 & 0 & 1 & 0 & 1 & 0 & 1 & 0 \\ 0 & 0 & 0 & -1 & -2 & 1 & -1 & 1
\end{pmatrix}\]
{\bf At this point, we see that the determinant is \(-1\).}


Clear column 2 using \(R_1\mapsto
R_1+(-2)R_2\)\[\begin{pmatrix}[ c c c c | c c c c ] 1 & 0 & 0 & -15 &
-5 & -2 & 0 & 0 \\ 0 & 1 & 0 & 9 & 3 & 1 & 0 & 0 \\ 0 & 0 & 1 & 0 & 1
& 0 & 1 & 0 \\ 0 & 0 & 0 & -1 & -2 & 1 & -1 & 1 \end{pmatrix}\]Make
leading entry in row 3 equal to 1 using \(R_3\mapsto (-1)R_3\).
\[\begin{pmatrix}[ c c c c | c c c c ] 1 & 0 & 0 & -15 & -5 & -2 & 0 &
0 \\ 0 & 1 & 0 & 9 & 3 & 1 & 0 & 0 \\ 0 & 0 & 1 & 0 & 1 & 0 & 1 & 0
\\ 0 & 0 & 0 & 1 & 2 & -1 & 1 & -1 \end{pmatrix}\]Clear column 4 using
\(R_1\mapsto R_1+(15)R_4\)\[\begin{pmatrix}[ c c c c | c c c c ] 1 & 0
& 0 & 0 & 25 & -17 & 15 & -15 \\ 0 & 1 & 0 & 9 & 3 & 1 & 0 & 0 \\ 0 &
0 & 1 & 0 & 1 & 0 & 1 & 0 \\ 0 & 0 & 0 & 1 & 2 & -1 & 1 & -1
\end{pmatrix}\]Clear column 4 using \(R_2\mapsto
R_2+(-9)R_4\)\[\begin{pmatrix}[ c c c c | c c c c ] 1 & 0 & 0 & 0 & 25
& -17 & 15 & -15 \\ 0 & 1 & 0 & 0 & -15 & 10 & -9 & 9 \\ 0 & 0 & 1 & 0
& 1 & 0 & 1 & 0 \\ 0 & 0 & 0 & 1 & 2 & -1 & 1 & -1 \end{pmatrix}\]Now
the right-hand block is the inverse of the matrix we started with.


\newpage


(d) We start with the matrix \[D=\begin{pmatrix}[ c c c c | c c c c ]
2 & 0 & 2 & 0 & 1 & 0 & 0 & 0 \\ 0 & 1 & 1 & 1 & 0 & 1 & 0 & 0 \\ 2 &
0 & 5 & 0 & 0 & 0 & 1 & 0 \\ 1 & 1 & 1 & 2 & 0 & 0 & 0 & 1
\end{pmatrix}\]


Clear row \(3\) using \(R_3\mapsto R_3+(-1)R_1\)\[\begin{pmatrix}[ c c
c c | c c c c ] 2 & 0 & 2 & 0 & 1 & 0 & 0 & 0 \\ 0 & 1 & 1 & 1 & 0 & 1
& 0 & 0 \\ 0 & 0 & 3 & 0 & -1 & 0 & 1 & 0 \\ 1 & 1 & 1 & 2 & 0 & 0 & 0
& 1 \end{pmatrix}\]Clear row \(4\) using \(R_4\mapsto
R_4+(-1/2)R_1\)\[\begin{pmatrix}[ c c c c | c c c c ] 2 & 0 & 2 & 0 &
1 & 0 & 0 & 0 \\ 0 & 1 & 1 & 1 & 0 & 1 & 0 & 0 \\ 0 & 0 & 3 & 0 & -1 &
0 & 1 & 0 \\ 0 & 1 & 0 & 2 & -1/2 & 0 & 0 & 1 \end{pmatrix}\]Clear row
\(4\) using \(R_4\mapsto R_4+(-1)R_2\)\[\begin{pmatrix}[ c c c c | c c
c c ] 2 & 0 & 2 & 0 & 1 & 0 & 0 & 0 \\ 0 & 1 & 1 & 1 & 0 & 1 & 0 & 0
\\ 0 & 0 & 3 & 0 & -1 & 0 & 1 & 0 \\ 0 & 0 & -1 & 1 & -1/2 & -1 & 0 &
1 \end{pmatrix}\]Clear row \(4\) using \(R_4\mapsto
R_4+(1/3)R_3\)\[\begin{pmatrix}[ c c c c | c c c c ] 2 & 0 & 2 & 0 & 1
& 0 & 0 & 0 \\ 0 & 1 & 1 & 1 & 0 & 1 & 0 & 0 \\ 0 & 0 & 3 & 0 & -1 & 0
& 1 & 0 \\ 0 & 0 & 0 & 1 & -5/6 & -1 & 1/3 & 1 \end{pmatrix}\] {\bf At
this point, we see that the determinant is \(6\).}


Make leading entry in row 0 equal to 1 using \(R_0\mapsto (1/2)R_0\).
\[\begin{pmatrix}[ c c c c | c c c c ] 1 & 0 & 1 & 0 & 1/2 & 0 & 0 & 0
\\ 0 & 1 & 1 & 1 & 0 & 1 & 0 & 0 \\ 0 & 0 & 3 & 0 & -1 & 0 & 1 & 0
\\ 0 & 0 & 0 & 1 & -5/6 & -1 & 1/3 & 1 \end{pmatrix}\]Make leading
entry in row 2 equal to 1 using \(R_2\mapsto (1/3)R_2\).
\[\begin{pmatrix}[ c c c c | c c c c ] 1 & 0 & 1 & 0 & 1/2 & 0 & 0 & 0
\\ 0 & 1 & 1 & 1 & 0 & 1 & 0 & 0 \\ 0 & 0 & 1 & 0 & -1/3 & 0 & 1/3 & 0
\\ 0 & 0 & 0 & 1 & -5/6 & -1 & 1/3 & 1 \end{pmatrix}\]Clear column 3
using \(R_1\mapsto R_1+(-1)R_3\)\[\begin{pmatrix}[ c c c c | c c c c ]
1 & 0 & 0 & 0 & 5/6 & 0 & -1/3 & 0 \\ 0 & 1 & 1 & 1 & 0 & 1 & 0 & 0
\\ 0 & 0 & 1 & 0 & -1/3 & 0 & 1/3 & 0 \\ 0 & 0 & 0 & 1 & -5/6 & -1 &
1/3 & 1 \end{pmatrix}\]Clear column 3 using \(R_2\mapsto
R_2+(-1)R_3\)\[\begin{pmatrix}[ c c c c | c c c c ] 1 & 0 & 0 & 0 &
5/6 & 0 & -1/3 & 0 \\ 0 & 1 & 0 & 1 & 1/3 & 1 & -1/3 & 0 \\ 0 & 0 & 1
& 0 & -1/3 & 0 & 1/3 & 0 \\ 0 & 0 & 0 & 1 & -5/6 & -1 & 1/3 & 1
\end{pmatrix}\]Clear column 4 using \(R_2\mapsto
R_2+(-1)R_4\)\[\begin{pmatrix}[ c c c c | c c c c ] 1 & 0 & 0 & 0 &
5/6 & 0 & -1/3 & 0 \\ 0 & 1 & 0 & 0 & 7/6 & 2 & -2/3 & -1 \\ 0 & 0 & 1
& 0 & -1/3 & 0 & 1/3 & 0 \\ 0 & 0 & 0 & 1 & -5/6 & -1 & 1/3 & 1
\end{pmatrix}\]Now the right-hand block is the inverse of the matrix
we started with.


\newpage


(e) We start with the matrix \[\begin{pmatrix}[ c c c c | c c c c ] 3
& 0 & 0 & 0 & 1 & 0 & 0 & 0 \\ -17 & -1 & 2 & 0 & 0 & 1 & 0 & 0 \\ -4
& -1 & 1 & 0 & 0 & 0 & 1 & 0 \\ 63 & 21 & -21 & -12 & 0 & 0 & 0 & 1
\end{pmatrix}\]


Clear column \(1\), row \(2\) using \(R_2\mapsto
R_2+(17/3)R_1\)\[\begin{pmatrix}[ c c c c | c c c c ] 3 & 0 & 0 & 0 &
1 & 0 & 0 & 0 \\ 0 & -1 & 2 & 0 & 17/3 & 1 & 0 & 0 \\ -4 & -1 & 1 & 0
& 0 & 0 & 1 & 0 \\ 63 & 21 & -21 & -12 & 0 & 0 & 0 & 1
\end{pmatrix}\]Clear column \(1\), row \(3\) using \(R_3\mapsto
R_3+(4/3)R_1\)\[\begin{pmatrix}[ c c c c | c c c c ] 3 & 0 & 0 & 0 & 1
& 0 & 0 & 0 \\ 0 & -1 & 2 & 0 & 17/3 & 1 & 0 & 0 \\ 0 & -1 & 1 & 0 &
4/3 & 0 & 1 & 0 \\ 63 & 21 & -21 & -12 & 0 & 0 & 0 & 1
\end{pmatrix}\]Clear column \(1\), row \(4\) using \(R_4\mapsto
R_4+(-21)R_1\)\[\begin{pmatrix}[ c c c c | c c c c ] 3 & 0 & 0 & 0 & 1
& 0 & 0 & 0 \\ 0 & -1 & 2 & 0 & 17/3 & 1 & 0 & 0 \\ 0 & -1 & 1 & 0 &
4/3 & 0 & 1 & 0 \\ 0 & 21 & -21 & -12 & -21 & 0 & 0 & 1
\end{pmatrix}\]Clear column \(2\), row \(3\) using \(R_3\mapsto
R_3+(-1)R_2\)\[\begin{pmatrix}[ c c c c | c c c c ] 3 & 0 & 0 & 0 & 1
& 0 & 0 & 0 \\ 0 & -1 & 2 & 0 & 17/3 & 1 & 0 & 0 \\ 0 & 0 & -1 & 0 &
-13/3 & -1 & 1 & 0 \\ 0 & 21 & -21 & -12 & -21 & 0 & 0 & 1
\end{pmatrix}\]Clear column \(2\), row \(4\) using \(R_4\mapsto
R_4+(21)R_2\)\[\begin{pmatrix}[ c c c c | c c c c ] 3 & 0 & 0 & 0 & 1
& 0 & 0 & 0 \\ 0 & -1 & 2 & 0 & 17/3 & 1 & 0 & 0 \\ 0 & 0 & -1 & 0 &
-13/3 & -1 & 1 & 0 \\ 0 & 0 & 21 & -12 & 98 & 21 & 0 & 1
\end{pmatrix}\]Clear column \(3\), row \(4\) using \(R_4\mapsto
R_4+(21)R_3\)\[\begin{pmatrix}[ c c c c | c c c c ] 3 & 0 & 0 & 0 & 1
& 0 & 0 & 0 \\ 0 & -1 & 2 & 0 & 17/3 & 1 & 0 & 0 \\ 0 & 0 & -1 & 0 &
-13/3 & -1 & 1 & 0 \\ 0 & 0 & 0 & -12 & 7 & 0 & 21 & 1 \end{pmatrix}\]
{\bf At this point, we see that the determinant is \(-36\).}


Make leading entry in row 1 equal to 1 using
\(R_1\mapsto (1/3)R_1\). \[\begin{pmatrix}[ c c c c | c c c c ] 1 & 0
& 0 & 0 & 1/3 & 0 & 0 & 0 \\ 0 & -1 & 2 & 0 & 17/3 & 1 & 0 & 0 \\ 0 &
0 & -1 & 0 & -13/3 & -1 & 1 & 0 \\ 0 & 0 & 0 & -12 & 7 & 0 & 21 & 1
\end{pmatrix}\]Make leading entry in row 2 equal to 1 using
\(R_2\mapsto (-1)R_2\). \[\begin{pmatrix}[ c c c c | c c c c ] 1 & 0
& 0 & 0 & 1/3 & 0 & 0 & 0 \\ 0 & 1 & -2 & 0 & -17/3 & -1 & 0 & 0 \\ 0
& 0 & -1 & 0 & -13/3 & -1 & 1 & 0 \\ 0 & 0 & 0 & -12 & 7 & 0 & 21 & 1
\end{pmatrix}\]Make leading entry in row 3 equal to 1 using
\(R_3\mapsto (-1)R_3\). \[\begin{pmatrix}[ c c c c | c c c c ] 1 & 0
& 0 & 0 & 1/3 & 0 & 0 & 0 \\ 0 & 1 & -2 & 0 & -17/3 & -1 & 0 & 0 \\ 0
& 0 & 1 & 0 & 13/3 & 1 & -1 & 0 \\ 0 & 0 & 0 & -12 & 7 & 0 & 21 & 1
\end{pmatrix}\]Clear column 3 using \(R_2\mapsto
R_2+(2)R_3\)\[\begin{pmatrix}[ c c c c | c c c c ] 1 & 0 & 0 & 0 & 1/3
& 0 & 0 & 0 \\ 0 & 1 & 0 & 0 & 3 & 1 & -2 & 0 \\ 0 & 0 & 1 & 0 & 13/3
& 1 & -1 & 0 \\ 0 & 0 & 0 & -12 & 7 & 0 & 21 & 1 \end{pmatrix}\]Make
leading entry in row 4 equal to 1 using \(R_4\mapsto (-1/12)R_4\).
\[\begin{pmatrix}[ c c c c | c c c c ] 1 & 0 & 0 & 0 & 1/3 & 0 & 0 & 0
\\ 0 & 1 & 0 & 0 & 3 & 1 & -2 & 0 \\ 0 & 0 & 1 & 0 & 13/3 & 1 & -1 & 0
\\ 0 & 0 & 0 & 1 & -7/12 & 0 & -7/4 & -1/12 \end{pmatrix}\]Now the
right-hand block is the inverse of the matrix we started with.


\newpage


(f) We start with the matrix \[\begin{pmatrix}[ c c c c | c c c c ]
-20 & -106 & -2 & 0 & 1 & 0 & 0 & 0 \\ 0 & -3 & -1 & 0 & 0 & 1 & 0 & 0
\\ 20 & 100 & -1 & 0 & 0 & 0 & 1 & 0 \\ -100 & -500 & 10 & 1 & 0 & 0 &
0 & 1 \end{pmatrix}\]


Clear column \(1\), row \(3\) using \(R_3\mapsto
R_3+(1)R_1\)\[\begin{pmatrix}[ c c c c | c c c c ] -20 & -106 & -2 & 0
& 1 & 0 & 0 & 0 \\ 0 & -3 & -1 & 0 & 0 & 1 & 0 & 0 \\ 0 & -6 & -3 & 0
& 1 & 0 & 1 & 0 \\ -100 & -500 & 10 & 1 & 0 & 0 & 0 & 1
\end{pmatrix}\]Clear column \(1\), row \(4\) using \(R_4\mapsto
R_4+(-5)R_1\)\[\begin{pmatrix}[ c c c c | c c c c ] -20 & -106 & -2 &
0 & 1 & 0 & 0 & 0 \\ 0 & -3 & -1 & 0 & 0 & 1 & 0 & 0 \\ 0 & -6 & -3 &
0 & 1 & 0 & 1 & 0 \\ 0 & 30 & 20 & 1 & -5 & 0 & 0 & 1
\end{pmatrix}\]Clear column \(2\), row \(3\) using \(R_3\mapsto
R_3+(-2)R_2\)\[\begin{pmatrix}[ c c c c | c c c c ] -20 & -106 & -2 &
0 & 1 & 0 & 0 & 0 \\ 0 & -3 & -1 & 0 & 0 & 1 & 0 & 0 \\ 0 & 0 & -1 & 0
& 1 & -2 & 1 & 0 \\ 0 & 30 & 20 & 1 & -5 & 0 & 0 & 1
\end{pmatrix}\]Clear column \(2\), row \(4\) using \(R_4\mapsto
R_4+(10)R_2\)\[\begin{pmatrix}[ c c c c | c c c c ] -20 & -106 & -2 &
0 & 1 & 0 & 0 & 0 \\ 0 & -3 & -1 & 0 & 0 & 1 & 0 & 0 \\ 0 & 0 & -1 & 0
& 1 & -2 & 1 & 0 \\ 0 & 0 & 10 & 1 & -5 & 10 & 0 & 1
\end{pmatrix}\]Clear column \(3\), row \(4\) using \(R_4\mapsto
R_4+(10)R_3\)\[\begin{pmatrix}[ c c c c | c c c c ] -20 & -106 & -2 &
0 & 1 & 0 & 0 & 0 \\ 0 & -3 & -1 & 0 & 0 & 1 & 0 & 0 \\ 0 & 0 & -1 & 0
& 1 & -2 & 1 & 0 \\ 0 & 0 & 0 & 1 & 5 & -10 & 10 & 1 \end{pmatrix}\]
{\bf At this point, we see that the determinant is \(-60\).}


Make leading entry in row 1 equal to 1 using \(R_1\mapsto
(-1/20)R_1\). \[\begin{pmatrix}[ c c c c | c c c c ] 1 & 53/10 & 1/10
& 0 & -1/20 & 0 & 0 & 0 \\ 0 & -3 & -1 & 0 & 0 & 1 & 0 & 0 \\ 0 & 0 &
-1 & 0 & 1 & -2 & 1 & 0 \\ 0 & 0 & 0 & 1 & 5 & -10 & 10 & 1
\end{pmatrix}\]Make leading entry in row 2 equal to 1 using
\(R_2\mapsto (-1/3)R_2\). \[\begin{pmatrix}[ c c c c | c c c c ] 1 &
53/10 & 1/10 & 0 & -1/20 & 0 & 0 & 0 \\ 0 & 1 & 1/3 & 0 & 0 & -1/3 & 0
& 0 \\ 0 & 0 & -1 & 0 & 1 & -2 & 1 & 0 \\ 0 & 0 & 0 & 1 & 5 & -10 & 10
& 1 \end{pmatrix}\]Clear column 2 using \(R_1\mapsto
R_1+(-53/10)R_2\)\[\begin{pmatrix}[ c c c c | c c c c ] 1 & 0 & -5/3 &
0 & -1/20 & 53/30 & 0 & 0 \\ 0 & 1 & 1/3 & 0 & 0 & -1/3 & 0 & 0 \\ 0 &
0 & -1 & 0 & 1 & -2 & 1 & 0 \\ 0 & 0 & 0 & 1 & 5 & -10 & 10 & 1
\end{pmatrix}\]Make leading entry in row 3 equal to 1 using
\(R_3\mapsto (-1)R_3\). \[\begin{pmatrix}[ c c c c | c c c c ] 1 & 0
& -5/3 & 0 & -1/20 & 53/30 & 0 & 0 \\ 0 & 1 & 1/3 & 0 & 0 & -1/3 & 0 &
0 \\ 0 & 0 & 1 & 0 & -1 & 2 & -1 & 0 \\ 0 & 0 & 0 & 1 & 5 & -10 & 10 &
1 \end{pmatrix}\]Clear column 3 using \(R_1\mapsto
R_1+(5/3)R_3\)\[\begin{pmatrix}[ c c c c | c c c c ] 1 & 0 & 0 & 0 &
-103/60 & 51/10 & -5/3 & 0 \\ 0 & 1 & 1/3 & 0 & 0 & -1/3 & 0 & 0 \\ 0
& 0 & 1 & 0 & -1 & 2 & -1 & 0 \\ 0 & 0 & 0 & 1 & 5 & -10 & 10 & 1
\end{pmatrix}\]Clear column 3 using \(R_2\mapsto
R_2+(-1/3)R_3\)\[\begin{pmatrix}[ c c c c | c c c c ] 1 & 0 & 0 & 0 &
-103/60 & 51/10 & -5/3 & 0 \\ 0 & 1 & 0 & 0 & 1/3 & -1 & 1/3 & 0 \\ 0
& 0 & 1 & 0 & -1 & 2 & -1 & 0 \\ 0 & 0 & 0 & 1 & 5 & -10 & 10 & 1
\end{pmatrix}\]Now the right-hand block is the inverse of the matrix
we started with.


\newpage


\begin{Solution}
We have
\begin{align*}
E_{42}(4)^{-1}&=E_{42}(-4)&E_{21}(2)^{-1}&=E_{21}(-2)&E_4(3)^{-1}&=E_4(1/3)\\
E_{34}(1)^{-1}&=E_{34}(-1)&E_{13}(2)^{-1}&=E_{13}(-2)& &
\end{align*}
We can calculate the product
\[E_{42}(4)E_{21}(2)E_4(3)E_{34}(1)E_{13}(2)\] by performing row
operations to the identity (starting with \(R_1\mapsto R_1+2R_3\),
ending with \(R_4\mapsto R_4+4R_2\)), yielding: \[\ma 1 & 0 & 2 & 0
\\ 2 & 1 & 0 & 0 \\ 0 & 0 & 1 & 1 \\ 8 & 4 & 0 & 3\mz\] The inverse
of this product is the product
\[E_{13}(2)^{-1}E_{34}(1)^{-1}E_4(3)^{-1}E_{21}(2)^{-1}E_{42}(4)^{-1},\]
which is (again, applying row operations to the identity) \[\ma 1 &
-8/3 & -2 & 2/3 \\ -2 & 1 & 0 & 0 \\ 0 & 4/3 & 1 & -1/3 \\ 0 & -4/3
& 0 & 1/3\mz.\]


\end{Solution}
\begin{Solution}
\begin{itemize}
\item Each leading index refers to a row and there are \(m\) rows, so
there are at most \(m\) leading indices.
\item If \(m<n\) then there must be some free indices because there are
\(n\) columns and at most \(m\) leading indices. The space of
solutions to \(Av=0\) has dimension equal to the number of free
indices providing it's nonempty, so if it's nonempty it contains a
nonzero solution. It's nonempty because \(0\) is always a
solution.
\item If \(A\) is invertible then its reduced echelon form is the
identity, so \(Av=0\) has a unique solution, which is
\(v=0\). Conversely, if \(A\) is not invertible then its reduced
echelon form is not the identity. Since \(m=n\), this means there
has to be a free index, so again \(Av=0\) has a
positive-dimensional solution space.
\item If \(A\) is not invertible then there is a nonzero \(v\) with
\(Av=0\) (using the previous part of the question). Then
\(BAv=0\). Therefore \(BA\) is not invertible (using the previous
part of the question).


\end{itemize}
\end{Solution}
\newpage
\end{document}
