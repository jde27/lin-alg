\documentclass{article}
\title{Linear Algebra Worksheet 4}
\author{Jonny Evans}
\date{Workshop 4}
\usepackage[utf8]{inputenc}
\usepackage[T1]{fontenc}
\usepackage[a4paper,margin=1in]{geometry}
\usepackage{longtable}
\usepackage{fixltx2e}
\usepackage{graphicx}
\usepackage{longtable}
\usepackage{float}
\usepackage{parallel}
\usepackage{parcolumns}
\usepackage{wrapfig}
\usepackage{rotating}
\usepackage{amsmath}
\usepackage{textcomp}
\usepackage{marvosym}
\usepackage{wasysym}
\usepackage{amssymb}
\usepackage{amsthm,amsmath,amsfonts,amscd,setspace}
\renewcommand{\partname}{Week}
\newcommand{\mks}[1]{\marginpar{\colorbox{white}{\parbox{\marginparwidth}{%
  \setstretch{0.5}{\bf\scriptsize{#1 marks}}}}}}
\usepackage{tikz}
\usetikzlibrary{decorations.markings,decorations.pathmorphing,shapes}
\usepackage{parskip}
\newcommand{\FF}{\mathfrak{f}}
\newcommand{\GG}{\mathfrak{g}}
\newcommand{\CC}{\mathbb{C}}
\newcommand{\QQ}{\mathbb{Q}}
\newcommand{\RR}{\mathbb{R}}
\newcommand{\UU}{\mathbb{U}}
\newcommand{\XX}{\mathbb{X}}
\newcommand{\YY}{\mathbb{Y}}
\newcommand{\ZZ}{\mathbb{Z}}
\newcommand{\Link}{\operatorname{Link}}
\newcommand{\Cone}{\operatorname{Cone}}
\newcommand{\colim}{\operatorname{colim}}
%\newcommand{\amal}{\operatorname{amal}}
\newcommand{\adj}{\operatorname{adj}}
\newcommand{\cp}[1]{\mathbf{CP}^{#1}}
\newcommand{\rp}[1]{\mathbf{RP}^{#1}}
\newcommand{\OP}[1]{\mathrm{#1}}
\newcommand{\ma}{\begin{pmatrix}}
\newcommand{\mz}{\end{pmatrix}}
\newcommand{\tka}{\begin{center}\begin{tikzpicture}}
\newcommand{\tkz}{\end{tikzpicture}\end{center}}
\newcommand{\matr}[4]{\left(\begin{array}{cc}#1 & #2\\ #3 & #4\end{array}\right)}
\newcommand{\mthrthr}[9]{\left(\begin{array}{ccc}#1 & #2 & #3\\ #4 & #5 & #6\\ #7 & #8 & #9\end{array}\right)}
\newcommand{\vect}[2]{\left(\begin{array}{c}#1\\#2\end{array}\right)}
\newcommand{\vthr}[3]{\left(\begin{array}{c}#1\\#2\\#3\end{array}\right)}
\newcommand{\TO}[3]{#1\stackrel{#2}{\longrightarrow}#3}
\newenvironment{Proof}{\begin{proof}}{\end{proof}\ignorespacesafterend}
\makeatletter
\renewcommand*\env@matrix[1][*\c@MaxMatrixCols c]{%
  \hskip -\arraycolsep
  \let\@ifnextchar\new@ifnextchar
  \array{#1}}
\makeatother

\begingroup
\makeatletter
\@for\theoremstyle:=definition,remark,plain\do{%
\expandafter\g@addto@macro\csname th@\theoremstyle\endcsname{%
\addtolength\thm@preskip\parskip
}%
}
\endgroup
\usepackage{graphicx}
\usepackage[capitalise]{cleveref}
\newtheorem{Theorem}{Theorem}[section]
\newtheorem{Lemma}[Theorem]{Lemma}
\newtheorem{Corollary}[Theorem]{Corollary}
\newtheorem{Proposition}[Theorem]{Proposition}
\theoremstyle{remark}
\newtheorem{Remark}[Theorem]{Remark}
\theoremstyle{definition}
\newtheorem{Definition}[Theorem]{Definition}
\newtheorem{Example}[Theorem]{Example}
\newtheorem{Exercise}[Theorem]{Exercise}
\newtheorem{Question}[Theorem]{Question}
\newtheorem{Solution}[Theorem]{Solution}
\newtheorem{Answer}[Theorem]{Answer}
\crefname{Theorem}{Theorem}{Theorems}
\Crefname{Theorem}{Theorem}{Theorems}
\crefname{Lemma}{Lemma}{Lemmas}
\Crefname{Lemma}{Lemma}{Lemmas}
\crefname{Corollary}{Corollary}{Corollaries}
\Crefname{Corollary}{Corollary}{Corollaries}
\crefname{Claim}{Claim}{Claims}
\Crefname{Claim}{Claim}{Claims}
\crefname{Proposition}{Proposition}{Propositions}
\Crefname{Proposition}{Proposition}{Propositions}
\crefname{Remark}{Remark}{Remarks}
\Crefname{Remark}{Remark}{Remarks}
\crefname{Definition}{Definition}{Definitions}
\Crefname{Definition}{Definition}{Definitions}
\crefname{Example}{Example}{Examples}
\Crefname{Example}{Example}{Examples}
\crefname{Exercise}{Exercise}{Exercises}
\Crefname{Exercise}{Exercise}{Exercises}
\crefname{Question}{Question}{Questions}
\Crefname{Question}{Question}{Questions}
\crefname{Solution}{Solution}{Solutions}
\Crefname{Solution}{Solution}{Solutions}
\crefname{Answer}{Answer}{Answers}
\Crefname{Answer}{Answer}{Answers}
\crefformat{footnote}{#2\footnotemark[#1]#3}

\begin{document}
\maketitle
\setcounter{section}{4}


\begin{Exercise}\label{exr:routine1}
Find the determinants of the following matrices by using the
inductive formula. \[A=\ma 3 & 2 & 1 \\ 1 & 0 & 7 \\ 0 & 1 &
0\mz,\quad B=\ma 1 & 2 & 5 & 13 \\ 0 & 1 & 2 & 1 \\ 1 & 0 & 1 & 3
\\ 1 & 1 & 1 & 1\mz,\quad C=\ma 1 & 1 & 1 & 1 & 1 \\ 0 & 0 & 1 & 0 &
0 \\ 0 & 0 & 0 & 1 & 0 \\ 0 & 0 & 0 & 0 & 1 \\ 1 & 0 & 0 & 0 &
0\mz.\]


\end{Exercise}
\begin{Exercise}\label{exr:routine3}
Compute the determinant of the matrix \[A=\ma 1 & t & 1 \\ 2 & 3 & 1
\\ 1 & 0 & 2\mz.\] For which value of \(t\) does this matrix fail to
be invertible? For this value of \(t\), find an element of
\(\ker(A)\).




\end{Exercise}
\begin{Exercise}\label{exr:routineeval}
For each matrix below, find its characteristic polynomial, its
eigenvalues and its eigenvectors.
\begin{gather*}
A=\ma 1 & 5 \\ 2 & 4 \mz,\quad
B=\ma 3 & -2 \\ 1 & 3 \mz,\quad
C=\ma 0 & i \\ i & 0 \mz\\
D=\ma 0 & 1 & 0 \\ 0 & 0 & 1 \\ 1 & 0 & 0 \mz,\quad
E=\ma 1 & 1 & 0 \\ 1 & 0 & 1 \\ 0 & 1 & 1\mz,\quad
F=\ma 0 & 1 & 0 \\ 1 & 0 & 1 \\ 0 & 1 & 0\mz\\
G=\ma -1 & -2 & 2 \\ -1 & 1 & 1 \\ -5 & -4 & 6\mz,\quad
H=\ma -2 & 3 & -3 \\ -6 & 7 & -6 \\ -6 & 6 & -5\mz,\quad
J=\ma 18 & -5 & -6 \\ 81 & -20 & -18 \\ -22 & 6 & 7\mz
\end{gather*}


\end{Exercise}
\begin{Exercise}\label{exr:detsum}
Write down two \(2\)-by-\(2\) matrices \(A,B\) with
\(\det(A)=\det(B)=1\). Find \(\det(A+B)\). Repeat twice more with
different matrices. Can you get any value for \(\det(A+B)\)?


\end{Exercise}
\begin{Exercise}
Suppose that \(A\) has an eigenvector \(v\) with eigenvalue
\(\lambda\). Show that \(\exp(A)\) has \(v\) as an eigenvector and
find the eigenvalue.


\end{Exercise}
\begin{Exercise}\label{exr:jacobian}
The {\em Jacobian} of a differentiable map \(F\colon\RR^n\to\RR^m\)
is the \(m\)-by-\(n\) matrix \[Jac(F)=\ma \frac{\partial
F_1}{\partial x_1} & \frac{\partial F_1}{\partial x_2} & \cdots &
\frac{\partial F_1}{\partial x_{n-1}} & \frac{\partial F_1}{\partial
x_n} \\ \frac{\partial F_2}{\partial x_1} & \frac{\partial
F_2}{\partial x_2} & \cdots & \frac{\partial F_2}{\partial x_{n-1}}
& \frac{\partial F_2}{\partial x_n} \\ \vdots & \vdots & & \vdots &
\vdots \\ \frac{\partial F_m}{\partial x_1} & \frac{\partial
F_m}{\partial x_2} & \cdots &\frac{\partial F_m}{\partial x_{n-1}} &
\frac{\partial F_m}{\partial x_n}\mz,\] where
\(F(x_1,\ldots,x_n)=\ma
F_1(x_1,\ldots,x_n)\\ \vdots\\ F_m(x_1,\ldots,x_n)\mz\).


Find \(\det(Jac(F))\) in the following examples:
\begin{enumerate}
\item \(m=n=2\), \(F(r,\phi)=\ma r\cos\phi \\ r\sin\phi\mz\).
\item \(m=n=3\), \(F(r,\theta,\phi)=\ma r\sin\theta\cos\phi
\\ r\sin\theta\sin\phi \\ r\cos\theta\mz\)


\end{enumerate}
{\em If \(m=n\) and \(F\) is a change of coordinates
\(\RR^n\to\RR^n\) then the determinant of the Jacobian matrix is an
important quantity: if \(y=F(x)\) then the volume element
\(dy_1\cdots dy_n\) is equal to \(\det(Jac(F))dx_1\cdots dx_n\). In
the examples we've just computed the volume element in polar and
spherical coordinates.}


\end{Exercise}
\begin{Exercise}\label{exr:orthaxis}
Suppose that \(A\) is an \(n\)-by-\(n\) orthogonal matrix
(\(AA^T=I\)).
\begin{enumerate}
\item Show that \(\det(A)=\pm 1\).
\item If \(\det(A)=1\), show that \(\det(A-I)=(-1)^n\det(A-I)\). (Hint:
Use the fact that \(A-I=A(I-A^T)\).)
\item Deduce that if \(n\) is odd then any orthogonal matrix with
determinant one has a fixed vector.


\end{enumerate}
\end{Exercise}
\begin{Exercise}\label{exr:detprod}
Let \(A\) be an \(n\)-by-\(n\) matrix with characteristic polynomial
\(\chi_A(t)\); suppose that \(\chi_A(t)\) has \(n\) distinct
roots. By considering \(\chi_A(0)\), prove that \(\det(A)\) is the
product of the eigenvalues of \(A\). One of the coefficients in the
polynomial \(\chi_A(t)\) is equal to minus the sum of the
eigenvalues of \(A\): which coefficient? {\em (Hint: Recall that if
a polynomial \(p(t)\) of degree \(n\) has roots
\(\lambda_1,\ldots,\lambda_n\) and the coefficient of \(t^n\) is
\(1\) then \(p(t)=(t-\lambda_1)(t-\lambda_2)\cdots(t-\lambda_n)\))}.


\end{Exercise}
\newpage
\setcounter{section}{3}
\section{Solutions}


\begin{Solution}\label{sol:exr:routine1}
(a) We start with the matrix \(A=\ma 3 & 2 & 1 \\ 1 & 0 & 7 \\ 0 & 1
& 0\mz\). Note that the final row only has one nonzero entry, so we
should expand around that. There is only one term that contributes:
\[-\det\ma 3 & 1 \\ 1 & 7\mz=-(21-1)=-20.\]


(b) We start with the matrix \(B=\ma 1 & 2 & 5 & 13 \\ 0 & 1 & 2 & 1
\\ 1 & 0 & 1 & 3 \\ 1 & 1 & 1 & 1\mz\). Expanding down the leftmost
column, we get \[\det(B)=\det\ma 1 & 2 & 1 \\ 0 & 1 & 3 \\ 1 & 1 &
1\mz-0+\det\ma 2 & 5 & 13 \\ 1 & 2 & 1 \\ 1 & 1 & 1\mz-\det\ma 2 & 5
& 13 \\ 1 & 2 & 1 \\ 0 & 1 & 3\mz.\] We have
\begin{align*}
\det\ma 1 & 2 & 1 \\ 0 & 1 & 3 \\ 1 & 1 & 1\mz&=1(1\times 1-3\times 1)-0+1(2\times 3-1\times 1)\\
&=3,\\
\det\ma 2 & 5 & 13 \\ 1 & 2 & 1 \\ 1 & 1 & 1\mz&=2(2\times 1-1\times 1)-5(1\times 1-1\times 1)+13(1\times 1-2\times 1)\\
&=-11,\\
\det\ma 2 & 5 & 13 \\ 1 & 2 & 1 \\ 0 & 1 & 3\mz&=2(2\times 3-1\times 1)-(5\times 3-1\times 13)+0\\
&=8,\\
\end{align*}
so \(\det(B)=-16\).


(c) We start with the matrix \(C=\ma 1 & 1 & 1 & 1 & 1 \\ 0 & 0 & 1
& 0 & 0 \\ 0 & 0 & 0 & 1 & 0 \\ 0 & 0 & 0 & 0 & 1 \\ 1 & 0 & 0 & 0 &
0\mz\). Expanding down the second column gives \[\det(C)=-\det\ma 0
& 1 & 0 & 0 \\ 0 & 0 & 1 & 0 \\ 0 & 0 & 0 & 1 \\ 1 & 0 & 0 & 0\mz.\]
Expanding this subdeterminant down the first column gives
\[\det(C)=-\left(-\det\ma 1 & 0 & 0 \\ 0 & 1 & 0 \\ 0 & 0 &
1\mz\right)=1. \qedhere\]


\end{Solution}
\begin{Solution}\label{sol:exr:routine3}
We have \[\det\ma 1 & t & 1 \\ 2 & 3 & 1 \\ 1 & 0 & 2\mz=\det\ma t &
1 \\ 3 & 1 \mz-0+2\ma 1 & t \\ 2 & 3\mz,\] by expanding along the
bottom row, which gives \[\det\ma 1 & t & 1 \\ 2 & 3 & 1 \\ 1 & 0 &
2\mz=3-3t.\] This is zero (i.e. the matrix is not invertible) if and
only if \(t=1\). If we set \(t=1\), an element of \(\ker(A)\) is
then a solution to \(Av=0\), that is a vector \(v=\ma x
\\ y\\ z\mz\) such that \[\ma 0 \\ 0 \\ 0\mz=Av=\ma x+y+z\\ 2x+3y+z
\\ x+2z\mz.\] These equations imply that \(x=-2z\) and \(y=-x-z=z\),
so the vector \(\ma -2 \\ 1 \\ 1\mz\) is in (in fact spans) the
kernel of \(A\) when \(t=1\). \qedhere






\end{Solution}
\begin{Solution}\label{sol:exr:routineeval}
(a) If \(A=\ma 1 & 5 \\ 2 & 4 \mz\) then the characteristic
polynomial is \[\det\ma 1-t & 5 \\ 2 & 4-t \mz=t^2-5t-6.\] The roots
are \(\frac{5\pm\sqrt{25+24}}{2}=-1,6\). The corresponding
eigenvectors are \(\ma 1 \\ -2/5\mz\) and \(\ma 1 \\ 1\mz\). For
example, you get the first one by solving \[\ma -x \\ -y \mz=\ma 1 &
5 \\ 2 & 4\mz\ma x \\ y\mz=\ma x+5y \\ 2x+4y\mz,\] which gives
\(y=-2x/5\).


(b) If \(B=\ma 3 & -2 \\ 1 & 3\mz\) then the characteristic
polynomial is \[\det\ma 3-t & -2 \\ 1 & 3-t\mz=t^2-6t+11.\] The
roots are \(\frac{6\pm\sqrt{36-44}}{2}=3\pm i\sqrt{2}\). The
corresponding eigenvectors are \(\ma 1 \\ \pm i\sqrt{2}\mz\). To see
this, you need to solve \[\ma (3\pm i\sqrt{2})x \\ (3\pm
i\sqrt{2})y\mz=\ma 3 & -2 \\ 1 & 3\mz\ma x \\ y\mz=\ma 3x-2y
\\ x+3y\mz,\] which gives \(y=\pm ix/\sqrt{2}\).


(c) If \(C=\ma 0 & i \\ i & 0\mz\) then the characteristic
polynomial is \[\det\ma -t & i \\ i & -t\mz=t^2+1,\] so the roots
are \(t=\pm i\). The corresponding eigenvectors are \(\ma \pm 1
\\ 1\mz\).


(d) If \(D=\ma 0 & 1 & 0 \\ 0 & 0 & 1 \\ 1 & 0 & 0 \mz\) then we
have \[\det(D-tI)=\det\ma -t & 1 & 0 \\ 0 & -t & 1 \\ 1 & 0 &
-t\mz=-t^3+1.\] The solutions of \(t^3=1\) are \(1,\mu,\mu^2\),
where \(\mu=e^{2\pi i/3}\). The corresponding eigenvectors are:
\[\ma 1 \\ 1 \\ 1 \mz,\quad\ma 1 \\ \mu \\ \mu^2\mz,\quad\ma 1
\\ \mu^2 \\ \mu\mz.\] For example, you can find the second one by
solving \[\ma y\\ z \\x\mz=D\ma x \\ y \\ z\mz=\ma \mu x\\ \mu y
\\ \mu z\mz,\] which gives \(y=\mu x\) and \(z=\mu^2x\).


(e) If \(E=\ma 1 & 1 & 0 \\ 1 & 0 & 1 \\ 1 & 1 & 0 \mz\) then we
have \[\det(E-tI)=\det\ma 1-t & 1 & 0 \\ 1 & -t & 1 \\ 1 & 1 &
-t\mz=(1-t)(-t(1-t)-1)-(1-t)=-t^3+2t^2+t-2.\] It is easy to check
that \(t=1\) is a solution. By long division of polynomials, we get
that \(-t^3+2t^2+t-2=(t-1)(-t^2+t+2)\), and the roots of
\(-t^2+t+2=0\) are \(\frac{-1\pm\sqrt{1+8}}{2}=-1,2\). Therefore the
eigenvalues of \(E\) are \(-1,1,2\). The corresponding eigenvectors
are \(\ma 1 \\ -2 \\ 1 \mz\), \(\ma -1 \\ 0 \\ 1\mz\) and \(\ma 1
\\ 1 \\ 1\mz\). For example, to find the first one, we need to solve
\[\ma x+y \\ x+z \\ y+z\mz=E\ma x \\ y \\ z \mz=\ma -x \\ -y
\\ -z\mz,\] which means \(y=-2x=-2z\), so \(\ma 1 \\ -2 \\ 1\mz\)
works.


(f) If \(F=\ma 0 & 1 & 0 \\ 1 & 0 & 1 \\ 0 & 1 & 0\mz\) then the
characteristic polynomial is \[\chi_F(t)=\det(F-tI)=\det\ma -t & 1 &
0 \\ 1 & -t & 1 \\ 0 & 1 & -t\mz=-t(t^2-1)-(-t)=-t^3+2t.\] This
factors as \(-t(t^2-2)\) so the eigenvalues are \(0,\pm
\sqrt{2}\). The eigenvectors are \(\ma 1 \\ 0 \\ -1\mz\), \(\ma 1
\\ \sqrt{2}\\ 1\mz\), \(\ma 1 \\ -\sqrt{2}\\ 1\mz\). For example,
the first one comes from solving the equation \[\ma y \\ x+z
\\ y\mz=F\ma x \\ y\\ z\mz=\ma 0 \\ 0 \\ 0\mz,\] which implies
\(y=0\) and \(x=-z\).


\end{Solution}
\begin{Solution}\label{sol:exr:ellipsoid}
If we use the matrix \(A=\ma 2 & -1 & 0 \\ -1 & 2 & -1 \\ 0 & -1 &
2\mz\) and \(v=\ma x \\ y \\ z\mz\) then \(v^TAv=1\) is our
ellipsoid. The characteristic polynomial of \(A\) is
\[\chi_A(t)=\det\ma 2-t & -1 & 0 \\ -1 & 2-t & -1 \\ 0 & -1 &
2-t\mz=(2-t)((2-t)^2-1)-(2-t)=(2-t)(t^2-4t+2).\] The eigenvalues are
therefore \(2\), \(2\pm\sqrt{2}\), which are all positive. The
corresponding eigenvectors are \(u,v,w\) solving \(Au=2u\),
\(Av=(2+\sqrt{2})v\) and \(Aw=(2-\sqrt{2})w\). For example, if
\(u=\ma x \\ y \\ z\mz\) then \[2u=\ma 2x \\ 2y \\ 2z\mz=Au=\ma
2x-y\\ -x+2y-z\\ 2z-y\mz,\] so \(y=0\) and \(x=-z\), so \(u=\ma 1
\\ 0 \\ -1\mz\) up to scale. Similarly, we find \(v=\ma -1/\sqrt{2}
\\ 1 \\ -1/\sqrt{2}\mz\) and \(w=\ma 1/\sqrt{2}\\ 1
\\ 1/\sqrt{2}\mz\). These are then the principal directions and the
principal radii are
\(\frac{1}{\sqrt{2}},\frac{1}{\sqrt{2\pm\sqrt{2}}}\).


(g) If \(G=\ma -1 & -2 & 2 \\ -1 & 1 & 1 \\ -5 & -4 & 6\mz\) then
the characteristic polynomial is \[\det\ma -1-t & -2 & 2 \\ -1 & 1-t
& 1 \\ -5 & -4 & 6-t\mz=-t^3+6t^2-11t+6,\] which has roots
\(1,2,3\). You can guess that \(1\) is a root, and then find the
other factors by polynomial long division. The eigenvectors are
respectively \(\ma 1 \\ 0 \\ 1 \mz\), \(\ma 0 \\ 1 \\ 1\mz\) and
\(\ma 1 \\ 1 \\ 3\mz\).


(h) If \(H=\ma -2 & 3 & -3 \\ -6 & 7 & -6 \\ -6 & 6 & -5\mz\) then
the characteristic polynomial turns out to be \[-t^3+3t-2,\] which
has roots \(t=1,2\) (\(1\) is a repeated root). The eigenvector for
\(2\) is \(\ma 1 \\ 2 \\ 2 \mz\). For the eigenvalue \(1\), we solve
\[\ma x \\ y \\ z\mz=\ma -2 & 3 & -3 \\ -6 & 7 & -6 \\ -6 & 6 &
-5\mz\ma x \\ y \\ z\mz=\ma -2x+3y-3z\\ -6x+7y-6z\\ -6x+6y-5z\mz.\]
All three equations reduce to \(y=x+z\), which cuts out a plane in
\(\RR^3\). In other words, any vector of the form \(\ma x \\ x+z
\\ z\mz\) is an eigenvector. We can get away with specifying two
linearly independent eigenvectors spanning this plane, say \(\ma 1
\\ 1 \\ 0 \mz\) anda \(\ma 0 \\ 1 \\ 1 \mz\).


(j) If \(J=\ma 18 & -5 & -6 \\ 81 & -20 & -18 \\ -22 & 6 & 7 \mz\)
then the characteristic polynomial turns out to be
\[-t^3+5t^2-7t+3,\] which has roots \(1,3\) (where \(1\) is
repeated). The eigenvector for \(3\) is \(\ma -4
\\ -18\\ 5\mz\). The eigenvector for \(1\) is \(\ma -3 \\ -15
\\ 4\mz\) (although the eigenvalue is repeated, there is only one
eigenvector).


\end{Solution}
\begin{Solution}
There are many possible answers, for example \(A=\ma 1 & 0 \\ 0 &
1\mz\), \(B=\det\ma 2 & 0 \\ 0 & 1/2\mz\), or \(A=\det\ma 1 & 4 \\ 1
& 5\mz\) \(B=\det\ma 2 & 1 \\ 5 & 3\mz=1\). These give
\(\det(A+B)=9/2\) and \(\det(A+B)=-6\) respectively. The point is,
there is no way to predict \(\det(A+B)\) just from knowing
\(\det(A)\) and \(\det(B)\). For example, \(\det\ma t & 0 \\ 0 &
1/t\mz=1\) so if you take \(A=\ma 1 & 0 \\ 0 & 1\mz\) and \(B=\ma t
& 0 \\ 0 & 1/t\mz\) then you get \(\det(A+B)=2+t+1/t\) and this
takes on every possible value as \(t\) varies. For example, if you
want to get \(\det(A+B)=D\) then you pick \(t\) satisfying
\(2+t+1/t=D\) (which is equivalent to \(t^2+(2-D)t+1=0\) or
\(t=\frac{2-D\pm\sqrt{D^2-4D}}{2}\)).


\end{Solution}
\begin{Solution}
If \(Av=\lambda v\) then \(\exp(A)v=\sum_{n\geq
0}\frac{1}{n!}A^nv=\sum_{n\geq
0}\frac{1}{n!}\lambda^nv=e^{\lambda}v\) so \(\exp(A)\) has \(v\) as
an eigenvector with eigenvalue \(e^{\lambda}\).


\end{Solution}
\begin{Solution}
\begin{enumerate}
\item The Jacobian matrix is \[\ma \frac{\partial(r\cos\phi)}{\partial
r} & \frac{\partial(r\cos\phi)}{\partial \phi}
\\ \frac{\partial(r\sin\phi)}{\partial r} &
\frac{\partial(r\sin\phi)}{\partial\phi}\mz=\ma \cos\phi &
-r\sin\phi \\ \sin\phi & r\cos\phi\mz,\] whose determinant is
\(r\cos^2\phi+r\sin^2\phi=r\). ! The Jacobian matrix is \[\ma
\sin\theta\cos\phi & r\cos\theta\cos\phi & -r\sin\theta\sin\phi
\\ \sin\theta\sin\phi & r\cos\theta\sin\phi & r\sin\theta\cos\phi
\\ \cos\theta & -r\sin\theta & 0\mz,\] whose determinant is (using
the inductive formula, expanding along the bottom row):
\[\cos\theta(r^2\cos\theta\sin\theta\cos^2\phi +
r^2\cos\theta\sin\theta\sin^2\phi) -
(-r\sin\theta)(r\sin^2\theta\cos^2\phi +
r\sin^2\theta\sin^2\phi),\] which simplifies to \(r^2\sin\theta\).


\end{enumerate}
\end{Solution}
\begin{Solution}\label{sol:exr:orthaxis}
\begin{enumerate}
\item We have \(1=\det(I)=\det(AA^T)=\det(A)\det(A^T)=\det(A)^2\), so
\(\det(A)=\pm 1\).
\item We have
\begin{align*}
\det(A-I)&=\det(A(I-A^T))\\
&=\det(A)\det(I-A^T)\\
&=\det(A)\det(I-A)\\
&=\det(A)\det(-I)\det(A-I)\\
&=(-1)^n\det(A-I),
\end{align*}
so if \(n\) is odd then \(\det(A-I)=-\det(A-I)\), which means
\(\det(A-I)=0\).
\item Since \(\det(A-I)=0\), \(1\) is a root of the characteristic
polynomial, so \(A\) has an eigenvector with eigenvalue
\(1\). Such a vector is a fixed vector. \qedhere


\end{enumerate}
\end{Solution}
\begin{Solution}
The characteristic polynomial is both \(\det(A-tI)\) and
\((t-\lambda_1)\cdots(t-\lambda_n)\) where
\(\lambda_1,\ldots,\lambda_n\) are the eigenvalues. The constant
coefficient is therefore both \(\chi_A(0)\) and
\(\lambda_1\cdots\lambda_n\), so
\(\chi_A(0)=\lambda_1\cdots\lambda_n\), and the coefficient of
\(t^{n-1}\) is \(-\lambda_1-\cdots-\lambda_n\).
\end{Solution}
\end{document}
