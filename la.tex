\documentclass{article}
\title{Linear Algebra}
\author{Jonny Evans}
\usepackage[utf8]{inputenc}
\usepackage[T1]{fontenc}
\usepackage[a4paper,margin=1in]{geometry}
\usepackage{longtable}
\usepackage{fixltx2e}
\usepackage{graphicx}
\usepackage{longtable}
\usepackage{float}
\usepackage{parallel}
\usepackage{parcolumns}
\usepackage{wrapfig}
\usepackage{rotating}
\usepackage{amsmath}
\usepackage{textcomp}
\usepackage{marvosym}
\usepackage{wasysym}
\usepackage{amssymb}
\usepackage{amsthm,amsmath,amsfonts,amscd,setspace}
\renewcommand{\partname}{Week}
\newcommand{\mks}[1]{\marginpar{\colorbox{white}{\parbox{\marginparwidth}{%
  \setstretch{0.5}{\bf\scriptsize{#1 marks}}}}}}
\usepackage{tikz}
\usetikzlibrary{decorations.markings,decorations.pathmorphing,shapes}
\usepackage{parskip}
\newcommand{\FF}{\mathfrak{f}}
\newcommand{\GG}{\mathfrak{g}}
\newcommand{\CC}{\mathbb{C}}
\newcommand{\QQ}{\mathbb{Q}}
\newcommand{\RR}{\mathbb{R}}
\newcommand{\UU}{\mathbb{U}}
\newcommand{\XX}{\mathbb{X}}
\newcommand{\YY}{\mathbb{Y}}
\newcommand{\ZZ}{\mathbb{Z}}
\newcommand{\Link}{\operatorname{Link}}
\newcommand{\Cone}{\operatorname{Cone}}
\newcommand{\colim}{\operatorname{colim}}
%\newcommand{\amal}{\operatorname{amal}}
\newcommand{\adj}{\operatorname{adj}}
\newcommand{\cp}[1]{\mathbf{CP}^{#1}}
\newcommand{\rp}[1]{\mathbf{RP}^{#1}}
\newcommand{\OP}[1]{\mathrm{#1}}
\newcommand{\ma}{\begin{pmatrix}}
\newcommand{\mz}{\end{pmatrix}}
\newcommand{\tka}{\begin{center}\begin{tikzpicture}}
\newcommand{\tkz}{\end{tikzpicture}\end{center}}
\newcommand{\matr}[4]{\left(\begin{array}{cc}#1 & #2\\ #3 & #4\end{array}\right)}
\newcommand{\mthrthr}[9]{\left(\begin{array}{ccc}#1 & #2 & #3\\ #4 & #5 & #6\\ #7 & #8 & #9\end{array}\right)}
\newcommand{\vect}[2]{\left(\begin{array}{c}#1\\#2\end{array}\right)}
\newcommand{\vthr}[3]{\left(\begin{array}{c}#1\\#2\\#3\end{array}\right)}
\newcommand{\TO}[3]{#1\stackrel{#2}{\longrightarrow}#3}
\newenvironment{Proof}{\begin{proof}}{\end{proof}\ignorespacesafterend}
\makeatletter
\renewcommand*\env@matrix[1][*\c@MaxMatrixCols c]{%
  \hskip -\arraycolsep
  \let\@ifnextchar\new@ifnextchar
  \array{#1}}
\makeatother

\begingroup
\makeatletter
\@for\theoremstyle:=definition,remark,plain\do{%
\expandafter\g@addto@macro\csname th@\theoremstyle\endcsname{%
\addtolength\thm@preskip\parskip
}%
}
\endgroup
\usepackage{graphicx}
\usepackage[capitalise]{cleveref}
\newtheorem{Theorem}{Theorem}[section]
\newtheorem{Lemma}[Theorem]{Lemma}
\newtheorem{Corollary}[Theorem]{Corollary}
\newtheorem{Proposition}[Theorem]{Proposition}
\theoremstyle{remark}
\newtheorem{Remark}[Theorem]{Remark}
\theoremstyle{definition}
\newtheorem{Definition}[Theorem]{Definition}
\newtheorem{Example}[Theorem]{Example}
\newtheorem{Exercise}[Theorem]{Exercise}
\newtheorem{Question}[Theorem]{Question}
\newtheorem{Solution}[Theorem]{Solution}
\newtheorem{Answer}[Theorem]{Answer}
\crefname{Theorem}{Theorem}{Theorems}
\Crefname{Theorem}{Theorem}{Theorems}
\crefname{Lemma}{Lemma}{Lemmas}
\Crefname{Lemma}{Lemma}{Lemmas}
\crefname{Corollary}{Corollary}{Corollaries}
\Crefname{Corollary}{Corollary}{Corollaries}
\crefname{Claim}{Claim}{Claims}
\Crefname{Claim}{Claim}{Claims}
\crefname{Proposition}{Proposition}{Propositions}
\Crefname{Proposition}{Proposition}{Propositions}
\crefname{Remark}{Remark}{Remarks}
\Crefname{Remark}{Remark}{Remarks}
\crefname{Definition}{Definition}{Definitions}
\Crefname{Definition}{Definition}{Definitions}
\crefname{Example}{Example}{Examples}
\Crefname{Example}{Example}{Examples}
\crefname{Exercise}{Exercise}{Exercises}
\Crefname{Exercise}{Exercise}{Exercises}
\crefname{Question}{Question}{Questions}
\Crefname{Question}{Question}{Questions}
\crefname{Solution}{Solution}{Solutions}
\Crefname{Solution}{Solution}{Solutions}
\crefname{Answer}{Answer}{Answers}
\Crefname{Answer}{Answer}{Answers}
\crefformat{footnote}{#2\footnotemark[#1]#3}

\begin{document}
\maketitle
\tableofcontents


\vspace{1cm}


The intention is that each section corresponds to a single
lecture. The course comprises 19 lectures plus an in-class test in
lecture 20. I have therefore left two hours leeway, which will get
filled up with more examples if it becomes clear it is not needed.


Workshops are scheduled to happen after lectures 2, 6, 10, 14, 18 and
the first four will have assessed questions.


\clearpage
\section{Matrices and transformations}
\subsection{Angles and rotations}


A vector \(v=\ma x \\ y \mz\) in the plane is an arrow pointing \(x\)
units to the right and \(y\) units up. By Pythagoras's theorem, the
length of \(v\) is \(|v|=\sqrt{x^2+y^2}\). If it makes an angle
\(\theta\) with the horizontal then \(x=|v|\cos\theta\) and
\(y=|v|\sin\theta\).


\tka
\draw[thick,->] (0,0) -- (5,3);
\draw[dotted,thick] (0,0) -- (5,0);
\draw[dotted,thick] (5,0) -- (5,3);
\node at (2.5,0) [below] {\(x=|v|\cos\theta\)};
\node at (5,1.5) [right] {\(y=|v|\sin\theta\)};
\node[rotate=30] at (2.3,1.8) {\(|v|=\sqrt{x^2+y^2}\)};
\draw (0.7,0) arc [radius=0.7,start angle=0,end angle=30];
\node at (0.85,0.25) {\(\theta\)};
\tkz


\begin{Theorem}\label{thm:rotmat2}
Let \(v=\ma x \\ y \mz\) be a vector and let \(w\) be the vector
obtained by rotating \(v\) an angle \(\phi\) around its
basepoint. Then \[w=\ma x\cos\phi - y\sin\phi \\ x\sin\phi +
y\cos\phi \mz.\]
\end{Theorem}
\begin{Proof}
We know that \(v=\ma x \\ y \mz=\ma |v|\cos\theta
\\ |v|\sin\theta\mz\) where \(\theta\) is the angle \(v\) makes with
the horizontal. After rotation, we know the following things about
\(w\):
\begin{itemize}
\item its length agrees with the length of \(v\), i.e. \(|w|=|v|\).
\item the angle \(w\) makes with the horizontal is \(\theta+\phi\).
\end{itemize}
Therefore
\begin{align*}
w&=\ma |w|\cos(\theta+\phi) \\ |w|\sin(\theta+\phi)\mz\\
&=\ma |v|\cos(\theta+\phi) \\ |v|\sin(\theta+\phi)\mz\\
&=\ma |v|\cos\theta\cos\phi-|v|\sin\theta\sin\phi \\
|v|\sin\theta\cos\phi+|v|\cos\theta\sin\phi\mz\\
&=\ma x\cos\phi-y\sin\phi \\ x\cos\phi+y\sin\phi\mz.\qedhere
\end{align*}


\end{Proof}
\subsection{Matrix notation}


Inspired by \cref{thm:rotmat2}, we introduce a new piece of notation
which allows us to separate out the dependence of a rotated vector
\(w\) on the initial vector \(v\) and on the rotation angle \(\phi\).


\begin{Definition}
A {\em 2-by-2 matrix} is a 2-by-2 array of numbers, like \(A=\ma a &
b \\ c& d\mz\). Given a matrix and a vector \(v=\ma x \\ y\mz\), we
define \(Av\) to be the new
vector \begin{equation}\label{eq:mmult2}Av=\ma a & b \\ c & d\mz\ma
x \\ y \mz:=\ma ax+by \\ cx+dy\mz .\end{equation} We say that \(Av\)
is obtained from \(v\) by the action of \(A\), in other words that
matrices {\em act on} vectors.


\end{Definition}
In the context of \cref{thm:rotmat2}, the matrix of the rotation by
angle \(\phi\) is \begin{equation}\label{eq:rotmat2}A=\ma \cos\phi &
-\sin\phi\\ \sin\phi & \cos\phi\mz\end{equation} and the rotated
vector is \(w=Av\).


\begin{Remark}
Vector notation lets us think of arrows as pairs of numbers. Matrix
notation lets us think of transformations (rotations, reflections,
etc) as grids of numbers.


\end{Remark}
\begin{Remark}
How do you remember a formula like \cref{eq:mmult2}? The mnemonic I
like is as follows. To get the first entry of \(Av\), you ``multiply
the top row of \(A\) into \(v\)'', that is you perform the
multiplications \(ax\) and \(by\) (working across the top row of
\(A\) and down the column of \(v\)) and sum them.


\tka
\node at (0,0) {\(\ma a & b \\ c & d\mz\ma x \\ y\mz\)};
\draw[->] (-0.8,0.2) -- (0.1,0.2);
\draw[->] (0.65,0.4) -- (0.65,-0.5);
\node at (2.5,0) {\(=\quad ax+by\)};
\tkz


To get the second
entry, you multiply the second row of \(A\) into \(v\).


\tka
\node at (0,0) {\(\ma a & b \\ c & d\mz\ma x \\ y\mz\)};
\draw[->] (-0.8,-0.22) -- (0.1,-0.22);
\draw[->] (0.65,0.4) -- (0.65,-0.5);
\node at (2.5,0) {\(=\quad cx+dy\)};
\tkz


\end{Remark}
\subsection{Linear maps}


Now, given any 2-by-2 array of numbers, we get a geometric
transformation of the plane \(\RR^2\to\RR^2\), \(v\mapsto Av\). We
call such a transformation arising from a matrix a {\em linear
map}\footnote{We will see an equivalent definition of linear maps
later, which makes no reference to matrices.}.


\begin{Example}
The matrix \(I=\ma 1 & 0 \\ 0 & 1\mz\) represents the {\em identity
transformation}, that is the map which sends the vector \(\ma x \\ y
\mz\) to itself. For this reason, this matrix is usually called the
identity matrix. It plays the same role in the theory of matrices
that the number \(1\) plays in usual arithmetic, so sometimes I may
end up writing \(1\) instead of \(I\) (in my research I always write
\(1\)).


\end{Example}
\begin{Example}
The matrix \(A=\ma -1 & 0 \\ 0 & 1 \mz\) defines a reflection in the
\(y\)-axis: the vector \(\ma 0 \\ 1\mz\), pointing along the
\(y\)-axis, is fixed; the vector \(\ma 1 \\ 0 \mz\) pointing along
the \(x\)-axis goes to \(\ma -1 \\ 0 \mz\).


\end{Example}
\begin{Example}
Rotation by \(\pi/2\) radians (90 degrees) is represented by the
matrix in \cref{eq:rotmat2} with \(\phi=\pi/2\), that is \(\ma 0 &
-1\\ 1 & 0\mz\).


\end{Example}
\begin{Example}
The matrix \(\ma 1 & 1 \\ 0 & 1\mz\) represents a {\em shear} in the
\(x\)-direction. For example, vectors along the \(x\)-axis are
fixed:
\[\ma 1 & 1\\0 & 1\mz\ma x \\ 0\mz =\ma x \\ 0\mz;\]
vectors at height \(y\) shear \(y\) units to the right:
\[\ma 1 & 1\\0 & 1\mz\ma x \\ y\mz =\ma x+y \\ y\mz.\]


\tka
\draw[->,thick,red] (0,0) -- (2,0);
\draw[->,thick,blue] (0,0) -- (0,2);
\draw[->,thick,purple] (0,0) -- (2,2);
\draw[dotted,->,thick] (0,2) -- (1.8,2);
\tkz


\end{Example}
\begin{Example}
Consider the matrix \(A=\ma 0 & 1 \\ 1 & 0 \mz\). This has a fixed
vector \(v=\ma 1 \\ 1\mz\) such that \(Av=v\) (indeed, if \(v=\ma
x\\y\mz\) and \(Av=v\) then \[\ma x\\ y\mz=\ma 0 & 1 \\ 1 & 0 \mz\ma
x \\ y\mz=\ma y \\ z\mz,\] so any vector with \(x=y\) is
fixed. Moreover, \(Aw=-w\) where \(w=\ma 1 \\ -1\mz\), which is
orthogonal to \(v\). Therefore \(A\) represents a reflection in the
line containing \(v\).


\tka
\draw[red,->,thick] (0,0) -- (2,2) node [above right] {\(u=\ma 1\\ 1\mz\)};
\draw[blue,->,thick] (0,0) -- (2,-2) node [below right] {\(v=\ma 1\\ -1\mz\)};
\draw[purple,->,thick] (0,0) -- (-2,2) node [above left] {\(Av=\ma -1\\ 1\mz\)};
\draw[dotted,->,thick] (2,-2) to[bend right] (-1.8,2);
\tkz




\end{Example}
\begin{Example}
If \(A=\ma 1 & 1 \\ -1 & 1 \mz\) then we see \(A\ma 1 \\ 0\mz=\ma 1
\\ -1\mz\) and \(A\ma 0 \\ 1\mz=\ma 1 \\ 1\mz\). Plotting these
vectors, we can see that \(A\) represents a rotation by \(-\pi/4\)
radians followed by a rescaling by \(\sqrt{2}\).
\tka
\draw[->,thick] (0,0) -- (1,0) node [right] {\(\ma 1 \\ 0\mz\)};
\draw[->,thick] (0,0) -- (0,1) node [above] {\(\ma 0 \\ 1\mz\)};
\draw[dotted,->,thick] (0,0) -- (1,-1) node [right] {\(A\ma 1 \\ 0 \mz\)};
\draw[dotted,->,thick] (0,0) -- (1,1) node [right] {\(A\ma 0 \\ 1\mz\)};
\tkz


\end{Example}
\begin{Example}
The matrix \(\ma 1 & 0 \\ 0 & 0\mz\). This gives the map \(\ma x
\\ y \mz\mapsto \ma x \\ 0 \mz\), which projects the plane
vertically down to the \(x\)-axis.


\end{Example}
\begin{Example}
The matrix \(A=\ma 1 & 1 \\ 1 & 1 \mz\) sends both \(\ma 1 \\ 0 \mz\) and
\(\ma 0 \\ 1\mz\) to the vector \(\ma 1 \\ 1\mz\). This means that the
transformation defined by \(A\) is an orthogonal projection to the
\(\ma 1 \\ 1\mz\)-line, followed by a rescaling by a factor of
\(2\sqrt{2}\).
\tka
\draw (-1,0) -- (2,0);
\draw (0,-1) -- (0,2);
\draw[->,thick] (0,0) -- (1,0) node [below] {\(\ma 1 \\ 0\mz\)};
\draw[->,thick] (0,0) -- (0,1) node [left] {\(\ma 0 \\ 1\mz\)};
\draw[dotted,->,thick] (0,0) -- (1,1) node [above right] {\(A\ma 1 \\ 0\mz=A\ma 0 \\ 1\mz\)};
\draw[dotted,->,red,thick] (1,0) -- (1/2+0.1,1/2-0.1);
\draw[dotted,->,red,thick] (0,1) -- (1/2-0.1,1/2+0.1);
\draw[dotted,->,red,thick] (1/2,1/2) -- (1,1);
\tkz


\end{Example}
\subsection{Bigger matrices}


Everything we've said so far generalises to higher dimensions.


\begin{Definition}
An \(n\)-vector is a column of \(n\) numbers. We write \(\RR^n\) for
the set of all \(n\)-vectors\footnote{We could also work with
vectors of complex numbers, in which case we'd write \(\CC^n\), or
vectors of rational numbers, in which case we'd write \(\QQ^n\), or
something else entirely.}. An \(m\)-by-\(n\) matrix is a rectangular
array of numbers with \(m\) rows and \(n\) columns. Given an
\(n\)-vector \(v\) and an \(m\)-by-\(n\) matrix \(A\), we get an
\(m\)-vector \(Av\), whose \(i\)th entry is the result of
multiplying the \(i\)th row of \(A\) into the column vector \(v\).


\end{Definition}
\begin{Example}
A \(3\)-by-\(3\) matrix \(\ma a & b & c \\ d & e & f \\ g & h &
i\mz\) defines a linear map \(\RR^3\to\RR^3\), which takes the
3-vector \(\ma x \\ y \\ z\mz\) to \[\ma a & b & c \\ d & e & f \\ g
& h & i\mz\ma x \\ y\\ z\mz=\ma ax+by+cz\\ dx+ey+fz\\ gx+hy+iz\mz.\]
Again, the action of the matrix on the vector is defined by
multiplying the rows of the matrix into the column vector. For
example, the matrix \(\ma \cos\phi & -\sin\phi & 0 \\ \sin\phi &
\cos\phi & 0 \\ 0 & 0 & 1\mz\) defines a rotation by \(\phi\) around
the \(z\)-axis.


\end{Example}
\begin{Example}
A \(2\)-by-\(3\) matrix \(\ma a & b & c \\ d & e & f\mz\) defines a
linear map \(\RR^3\to\RR^2\): \[\ma x \\ y\\ z\mz\mapsto \ma
ax+by+cz\\ dx+ey+fz\mz.\] For example, the matrix \(\ma 1 & 0 & 0
\\ 0 & 1 & 0\mz\) represents the map \(\ma x \\ y \\ z\mz\mapsto\ma
x \\ y\mz\), which is the projection from 3-dimensional space onto
the \(xy\)-plane.


\end{Example}
\begin{Example}
In practice, matrices can be bigger than this. In special
relativity, the maps which change from one spacetime reference frame
to another are given by \(4\)-by-\(4\) matrices called {\em Lorentz
matrices}; in statistics, in linear regression models, you work with
matrices which have one row for each sample, so that could be very
large.


\end{Example}
\begin{Example}\label{exm:3by2}
The \(3\)-by-\(2\) matrix \(\ma 1 & 1 \\ 2 & 0 \\ 0 & 1 \mz\)
defines a linear map from \(\RR^2\) to \(\RR^3\) (an embedding from
the plane into 3-dimensional space): \[\ma 1 & 1 \\ 2 & 0 \\ 0 & 1
\mz\ma v_1 \\ v_2\mz=\ma v_1+v_2\\ 2v_1 \\ v_2\mz.\] This sends the
vector \(\ma 1 \\ 0\mz\) to \(\ma 1 \\ 2 \\ 0\mz\) and \(\ma 0
\\ 1\mz\) to \(\ma 1 \\ 0 \\ 1\mz\). In the picture below we can see
the image of the plane under this linear map.


\tka
\draw[->] (0,0) -- (0,2) node [above] {\(z\)};
\draw[->] (0,0) -- (-1,-1) node [left] {\(x\)};
\draw[->] (0,0) -- (2,0) node [above] {\(y\)};
\draw[red,thick,->] (0,0) -- (-1,1) node [left] {\(\ma 1 \\ 0 \\ 1\mz\)};
\draw[red,thick,->] (0,0) -- (3,-1) node [below] {\(\ma 1\\ 2 \\ 0\mz\)};
\filldraw[fill=red,opacity=0.5,draw=none] (0,0) -- (-2,2) -- (2,2-4/3) -- (4,-4/3) -- cycle;
\tkz


\end{Example}
\begin{Example}
The matrix \(\ma 1 & 0 & -1\\ 0 & 1 & -1\mz\) defines a linear map
from \(\RR^3\) to \(\RR^2\) (a projection from 3-dimensional space
to the plane) which sends the basis vectors \(\ma 1 \\ 0 \\ 0\mz\),
\(\ma 0 \\ 1 \\ 0\mz\) and \(\ma 0 \\ 0 \\ 1\mz\) to the vectors
\(\ma 1 \\ 0\mz\), \(\ma 0 \\ 1\mz\) and \(\ma -1 \\ -1\mz\)
respectively. Try to represent this projection in the picture
below. The blue vectors point along the coordinate axes in 3-d. The
red vectors are the images of the blue vectors under the projection
(in two cases, the projection does nothing, so the blue and red
vectors coincide; we draw them as purple). The dotted lines are the
lines along which we're projecting. The grey shaded region is the
plane onto which we're projecting.


\tka
\draw[blue,->,thick] (0,0) -- (0,2) node [above] {\(z\)};
\draw[purple,->,thick] (0,0) -- (-0.5,-1) node [left] {\(x\)};
\draw[purple,->,thick] (0,0) -- (2,0) node [above] {\(y\)};
\draw[red,thick,->] (0,0) -- (-1,1);
\filldraw[draw=none,fill=gray,opacity=0.5] (-3,-1.5) -- (-2,1.5) -- (3,1.5) -- (2,-1.5) -- cycle;
\draw[dotted,->,thick] (0.5,2.5) -- (-1,1);
\draw[dotted,<-,thick] (0,0) -- (2,2);
\draw[dotted,<-,thick] (-0.5,-1) -- (1.5,1);
\draw[dotted,<-,thick] (2,0) -- (3,1);
\tkz


\end{Example}
\begin{Example}
The matrix \(A=\ma -1 & 0 & 0\\ 0 & 1 & 0 \\ 0 & 0 & 1\mz\) defines
a linear map \(\RR^3\to\RR^3\). We can see that \[A\ma 0 \\ y
\\ z\mz=\ma 0 \\ y \\ z\mz,\qquad A\ma x \\ 0 \\ 0\mz=\ma -x \\ 0
\\ 0\mz.\] This means that \(A\) can be interpreted as a {\em
reflection} in the \(yz\)-plane.


\end{Example}
\begin{Example}\label{exm:rot3d}
The matrix \(A=\ma 0 & -1 & 0 \\ 1 & 0 & 0 \\ 0 & 0 & 1\mz\) defines
a linear map \(\RR^3\to\RR^3\) which fixes the vector \(\ma 0 \\ 0
\\ 1\mz\) and effects a 90 degree rotation in the
\(xy\)-plane. Similarly, the matrix \(B=\ma 1 & 0 & 0 \\ 0 & 0 & -1
\\ 0 & 1 & 0 \mz\) fixes the vector \(\ma 1 \\ 0 \\ 0\mz\) and
effects a 90 degree rotation in the \(yz\)-plane, and the matrix
\(C=\ma 0 & 0 & 1 \\ 0 & 1 & 0 \\ -1 & 0 & 0\mz\) fixes the vector
\(\ma 0 \\ 1 \\ 0\mz\) and effects a 90 degree rotation in the
\(xz\)-plane.


\end{Example}
It is much harder (though still possible) to write down a general
rotation matrix in three dimensions. We will revisit some
three-dimensonal rotation matrices later in the week.


\clearpage
\section{Matrix algebra}
\subsection{Matrix multiplication}
Suppose we are given two matrices \(A=\ma A_{11} & A_{12} \\ A_{21} &
A_{22}\mz\), and \(B=\ma B_{11} & B_{12} \\ B_{21} & B_{22}\mz\). They
each define a transformation of the plane. What happens if we {\em
first} do the transformation associated to \(B\), and {\em then} do
the transformation associated to \(A\)? We get a new transformation
associated to a new matrix, which we call \(AB\).
\begin{align*}
A(B(v))&=\ma A_{11} & A_{12} \\ A_{21} & A_{22}\mz\ma B_{11} & B_{12} \\ B_{21} & B_{22}\mz\ma x \\ y\mz\\
&=\ma A_{11} & A_{12} \\ A_{21} & A_{22}\mz\ma B_{11}x+B_{12}y \\ B_{21}x+B_{22}y\mz\\
&=\ma A_{11}B_{11}x+A_{11}B_{12}y+A_{12}B_{21}x+A_{12}B_{22}y \\ A_{21}B_{11}x+A_{21}B_{12}y+A_{22}B_{21}x+A_{22}B_{22}y\mz\\
&=\ma A_{11}B_{11}+A_{12}B_{21} & A_{11}B_{12}+A_{12}B_{22}\\ A_{21}B_{11}x+A_{22}B_{21} & A_{21}B_{12}+A_{22}B_{22}\mz\ma x \\ y\mz\\
&=:(AB)v
\end{align*}


\begin{Definition}[Matrix multiplication]\label{dfn:matmult}
Given matrices \(A=\ma A_{11} & A_{12} \\ A_{21} & A_{22}\mz\), and
\(B=\ma B_{11} & B_{12} \\ B_{21} & B_{22}\mz\), we define the
matrix product
\[AB=\ma A_{11}B_{11}+A_{12}B_{21} & A_{11}B_{12}+A_{12}B_{22}\\ A_{21}B_{11}x+A_{22}B_{21} & A_{21}B_{12}+A_{22}B_{22}\mz.\]


\end{Definition}
How on earth can we remember this formula? Here are two mnemonics.
\begin{itemize}
\item Just like when we act on a vector using a matrix, we can think of
the entries of \(AB\) as ``multiplying a row of \(A\) into a column
of \(B\)''. More specifically, to get the \(ij\)th entry of \(AB\)
(i.e. \(i\)th row and \(j\)th column) we multiply the \(i\)th row of
\(A\) into the \(j\)th column of \(B\):


\tka
\draw[thick] (-4.3,-1.3) to[bend left] (-4.3,1.3);
\draw[thick] (4.3,-1.3) to[bend right] (4.3,1.3);
\begin{scope}[shift={(-2.3,1)}]
\node at (0,0) {\(\ma A_{11} & A_{12} \\ A_{21} & A_{22}\mz\ma B_{11} & B_{12} \\ B_{21} & B_{22}\mz\)};
\draw[->] (-1.8,0.2) -- (0,0.2);
\draw[->] (0.65,0.4) -- (0.65,-0.5);
\end{scope}
\begin{scope}[shift={(2.3,1)}]
\node at (0,0) {\(\ma A_{11} & A_{12} \\ A_{21} & A_{22}\mz\ma B_{11} & B_{12} \\ B_{21} & B_{22}\mz\)};
\draw[->] (-1.8,0.2) -- (0,0.2);
\draw[->] (1.4,0.4) -- (1.4,-0.5);
\end{scope}
\begin{scope}[shift={(-2.3,-1)}]
\node at (0,0) {\(\ma A_{11} & A_{12} \\ A_{21} & A_{22}\mz\ma B_{11} & B_{12} \\ B_{21} & B_{22}\mz\)};
\draw[->] (-1.8,-0.25) -- (0,-0.25);
\draw[->] (0.65,0.4) -- (0.65,-0.5);
\end{scope}
\begin{scope}[shift={(2.3,-1)}]
\node at (0,0) {\(\ma A_{11} & A_{12} \\ A_{21} & A_{22}\mz\ma B_{11} & B_{12} \\ B_{21} & B_{22}\mz\)};
\draw[->] (-1.8,-0.25) -- (0,-0.25);
\draw[->] (1.4,0.4) -- (1.4,-0.5);
\end{scope}
\tkz
\item We can also write a formula for the \(ij\)th entry:
\[(AB)_{ij}=\sum_{k=1}^2 A_{ik}B_{kj}.\] For example, when \(i=1\),
\(j=2\), this equation gives the entry of the product \(AB\) in the
first row and second column as
\[(AB)_{12}=A_{11}B_{12}+A_{12}B_{22}.\]


\end{itemize}
\begin{Example}
Consider the 90 degree rotation matrix \(A=\ma 0 & -1 \\ 1 & 0
\mz\). We have
\begin{align*}
A^2&=\ma 0 & -1 \\ 1 & 0 \mz\ma 0 & -1 \\ 1 & 0 \mz\\
&=\ma -1 & 0 \\ 0 & -1 \mz.
\end{align*}
This makes sense: two 90 degree rotations compose to give a 180
degree rotation, which sends every point \(\ma x \\ y\mz\) to its
opposite point \(\ma -x \\ -y\mz\).


\end{Example}
\begin{Example}
More generally, if \[R_{\theta_1}=\ma \cos\theta_1 & -\sin\theta_1
\\ \sin\theta_1 & \cos\theta_1\mz\,\qquad R_{\theta_2}=\ma
\cos\theta_2 & -\sin\theta_2 \\ \sin\theta_2 & \cos\theta_2\mz\] are
two rotations then the composite is
\begin{align*}
R_{\theta_1}R_{\theta_2}&=\ma \cos\theta_1 & -\sin\theta_1 \\ \sin\theta_1 & \cos\theta_1\mz\ma \cos\theta_2 & -\sin\theta_2 \\ \sin\theta_2 & \cos\theta_2\mz\\
&=\ma \cos\theta_1\cos\theta_2-\sin\theta_1\sin\theta_2 & -\cos\theta_1\sin\theta_2-\sin\theta_1\cos\theta_2\\ \sin\theta_1\cos\theta_2+\cos\theta_1\sin\theta_2 & -\sin\theta_1\sin\theta_2+\cos\theta_1\cos\theta_2\mz\\
&=\ma \cos(\theta_1+\theta_2) & -\sin(\theta_1+\theta_2) \\ \sin(\theta_1+\theta_2) & \cos(\theta_1+\theta_2)\mz\\
&=R_{\theta_1+\theta_2}.
\end{align*}
(using trigonometric addition formulas). This is what we expect, of
course: rotating by \(\theta_2\) and then \(\theta_1\) amounts to
rotating by \(\theta_1+\theta_2\).


\end{Example}
\begin{Example}
Let \(I=\ma 1 & 0 \\ 0 & 1\mz\) be the identity matrix and \(A\) be
any matrix. Then
\begin{align*}
IA&=\ma 1 & 0 \\ 0 & 1\mz\ma A_{11} & A_{12} \\ A_{21} & A_{22}\mz\\
&=\ma A_{11} & A_{12} \\ A_{21} & A_{22}\mz\\
&=A.
\end{align*}
Similarly, \(AI=A\). As you can see, the identity matrix really
plays the role of the number \(1\) here.


\end{Example}
\subsection{Noncommutativity}


\begin{Remark}
You might be confused about why we write \(AB\) for the
transformation which {\em first} applies \(B\) and {\em then}
applies \(A\). This actually makes perfect sense if you think of
\(A\) and \(B\) as functions acting on vectors: remember that
\(f(g(x))\) means ``apply \(f\) to the result of first applying
\(g\) to \(x\)''.


\end{Remark}
\begin{Remark}
Order matters: most of the time, \(AB\) is {\em not equal to}
\(BA\). In other words, {\em matrix multiplication is not
commutative}. This makes matrices significantly more interesting
algebraic objects than numbers.


\end{Remark}
\subsection{Bigger matrices}
From these examples, and what we've seen for 2-by-2 matrices,
hopefully you can guess the definition of matrix multiplication.


\begin{Definition}
If \(A\) is an \(m\)-by-\(n\) matrix and \(B\) is a \(n\)-by-\(p\)
matrix then \(AB\) is the \(m\)-by-\(p\) matrix whose \(ij\)th entry
is \[(AB)_{ij}=\sum_{k=1}^nA_{ik}B_{kj}.\] In other words, the entry
in the \(i\)th row and \(j\)th column is obtained by multiplying the
\(i\)th row of \(A\) into the \(j\)th column of \(B\). Because \(A\)
has \(n\) columns and \(B\) has \(n\) rows, this multiplication
makes sense.


\end{Definition}
\begin{Remark}
This kind of notation where you see entries of the matrix written
out with subscripts and sums all over the place is called {\em index
notation}. It is extremely useful for when you would otherwise run
out of letters to write your matrices (for example, if your matrix
was \(n\)-by-\(n\) and you didn't know what \(n\) was). If you want
to see it being used to great effect, open any textbook on general
relativity, and glory in the ``d\'{e}bauches des indices''.


\end{Remark}
\begin{Example}
Here are some examples of matrix multiplications:
\begin{align*}
\ma 0 & 0 & 1 \\ 1 & 0 & 0 \\ 0 & 1 & 0\mz \ma x \\ y \\ z\mz&=\ma z \\ x \\ y\mz\\
\ma 1 & 2 & 3 \\ -1 & 1/2 & 0\mz \ma 2 & -3 \\ -1 & 0 \\ 0 & 1\mz&=\ma 0 & 0 \\ -5/2 & 3\mz\\
\ma 1 & -1 & 1 & -1\mz\ma 1 \\ 2 \\ 3 \\ 4\mz&=\ma -2\mz\\
\ma 1 \\ 2 \\ 3 \\ 4\mz\ma 1 & -1 & 1 & -1\mz&=\ma 1 & -1 & 1 & -1\\ 2 & -2 & 2 & -2\\ 3 & -3 & 3 & -3\\4 & -4 & 4 & -4\mz
\end{align*}


\end{Example}
\subsection{Other operations}


We have now seen that you can define the product of two matrices. In
fact, you can do lots of other operations.


\begin{Definition}
Given two \(m\)-by-\(n\) matrices \(A\) and \(B\), we define their
{\em sum} \(A+B\) to be the \(m\)-by-\(n\) matrix \(A+B\) whose
\(ij\)th entry is \[(A+B)_{ij}=A_{ij}+B_{ij}.\] That is, the entry
in the \(i\)th row and \(j\)th column of \(A+B\) is the sum of the
corresponding entries for \(A\) and \(B\).


\end{Definition}
\begin{Remark}
Matrix addition is kind of boring in comparison to matrix
multiplication. Nonetheless, it plays an important role.


\end{Remark}
\begin{Definition}[Scaling]\label{dfn:scaling}
Given a matrix \(A\) and a number \(\lambda\), we define \(\lambda
A\) to be the matrix \[\lambda A\] whose \(ij\)th entry is
\((\lambda A)_{ij}=\lambda A_{ij}\). If \(A\) corresponds to some
geometric transformation then \(\lambda A\) corresponds to the same
geometric transformation followed by a rescaling by a factor of
\(\lambda\).


\end{Definition}
More interestingly, we can define exponentials of matrices.


\begin{Definition}
Given an \(n\)-by-\(n\) matrix \(A\), we define its {\em
exponential} \(\exp(A)\) to be the infinite sum
\[\exp(A)=I+A+\frac{1}{2}A^2+\frac{1}{3!}A^3+\cdots=\sum_{n\geq
0}\frac{1}{n!}A^n,\] where we define \(A^0=I\).


\end{Definition}
\begin{Example}
Let \(A=\ma 0 & 1 \\ 0 & 0\mz\). Then
\begin{align*}
A^2&=\ma 0 & 1 \\ 0 & 0\mz\ma 0 & 1 \\ 0 & 0\mz\\
&=\ma 0 & 0 \\ 0 & 0\mz,
\end{align*}
so \(0=A^3=A^4=\cdots\) and the infinite sum reduces to:
\begin{align*}
\exp(A)&=I+A+0+0+\cdots\\
&=\ma 1 & 1 \\ 0 & 1\mz.
\end{align*}


\end{Example}
\begin{Example}
Let \(A=\ma 0 & -\theta \\ \theta & 0\mz\). Then
\begin{align*}
A^2&=\ma 0 & -\theta \\ \theta & 0\mz\ma 0 & -\theta \\ \theta & 0\mz\\
&=\ma -\theta^2 & 0 \\ 0 & -\theta^2\mz\\
&=-\theta^2I.
\end{align*}
Therefore
\begin{align*}
A^3&=-\theta^2IA=-\theta^2A\\
A^4&=-\theta^2A^2=(-\theta^2)^2I=\theta^4I\\
A^5&=\theta^4IA=\theta^4A\\
A^6&=\theta^4A^2=-\theta^6I.
\end{align*}
Following this pattern, we get \(A^{2n}=(-1)^{2n}\theta^{2n}\) and
\(A^{2n}=(-1)^{2n}\theta^{2n}A\). This means
\begin{align*}
\exp(A)&=\left(I+\frac{1}{2}A^2+\frac{1}{4!}A^4+\cdots\right)+\left(A+\frac{1}{3!}A^3+\cdots\right)\\
&=\left(1-\frac{\theta^2}{2}+\frac{\theta^4}{4!}+\cdots\right)I+\left(\theta-\frac{\theta^3}{3!}+\cdots\right)A\\
&=\cos\theta I+\sin\theta A\\
&=\ma \cos\theta & -\sin\theta \\ \sin\theta & \cos\theta\mz.
\end{align*}
This is, remarkably, the formula for a rotation matrix by an angle
\(\theta\). Starting from a very simple matrix \(\ma 0 & -\theta
\\ \theta & 0\mz\) and using the exponential function, we have ended
up with the general formula for a rotation matrix in the plane. You
can probably imagine that this becomes even more useful as a way of
encoding rotations in 3-dimensions.


\end{Example}
\clearpage


\section{Dot products and orthogonal matrices}
\subsection{Dot product}


Given two vectors \(v,w\in\RR^n\), how do you figure out the angle
between them?


\begin{Definition}[Dot product]\label{dfn:dotproduct}
Given two vectors \(v=\ma v_1\\ \vdots\\ v_n\mz\) and \(w=\ma
w_1\\ \vdots\\ w_n\mz\), we define the {\em dot product} of \(v\)
and \(w\) to be the number \[v\cdot w:=v_1w_1+\cdots+v_nw_n.\]


\end{Definition}
\begin{Theorem}\label{thm:dot}
If \(v\) and \(w\) are separated by an angle \(\phi\) then \(v\cdot
w=|v||w|\cos\phi\).


\end{Theorem}
We will prove the theorem momentarily. Let us first explore it a
little.


\begin{Example}
The vectors \(v=\ma 1 \\ 0 \mz\) and \(w=\ma 0 \\ 1 \mz\) satisfy
\(v\cdot w=1\times 0+0\times 1=0\). Indeed, they are {\em
orthogonal} to one another (i.e. at right-angles), so are separated
by an angle \(\pi/2\) radians, and \(\cos(\pi/2)=0\).


\end{Example}
\begin{Example}
The vectors \(v=\ma 1 \\ 1 \mz\) and \(w=\ma 1 \\ 0 \mz\) satisfy
\(v\cdot w=1\), \(|v|=\sqrt{2}\), \(|w|=1\), so if \(\phi\) is the
angle separating them then \[1=v\cdot
w=|v||w|\cos\phi=\sqrt{2}\cos\phi,\] so
\(\cos\phi=\frac{1}{\sqrt{2}}\), and \(\phi=\pi/4\) radians.


\end{Example}
\begin{Remark}
You may be worried that \(\cos\phi\) doesn't determine \(\phi\)
completely, for example \(\cos(\pi/2)=\cos(3\pi/2)=0\). However, the
ambiguity is precisely whether you are measuring the angle from
\(v\) to \(w\) clockwise or anticlockwise, so don't worry unless
that distinction is important to you.


\end{Remark}
We now move in the direction of proving \cref{thm:dot}. Notice that
the definition of dot product looks a lot like matrix
multiplication. In fact, \[\ma v_1\\ \vdots\\v_n\mz\cdot\ma
w_1\\\vdots\\ w_n\mz=\ma v_1 & \cdots & v_n\mz\ma
w_1\\\vdots\\ w_n\mz.\] In other words, we have turned one of our
column vectors on its side to make it into a row vector. This
operation is called {\em transposition}.


\begin{Definition}
Given an \(m\)-by-\(n\) matrix \(A\) with entries \(A_{ij}\), its
{\em transpose} \(A^T\) is defined to be the \(n\)-by-\(m\) matrix
with entries \(A_{ji}\). For example
\begin{align*}
\ma 1 & 2\\ 3 & 4\mz^T&=\ma 1 & 3 \\2 & 4\mz\\
\ma 1 \\ 2 \\ 3 \\ 4\mz^T&=\ma 1 & 2 & 3 &4\mz.
\end{align*}


\end{Definition}
In other words, we can write dot product as \(v\cdot w=v^Tw\).


\begin{Lemma}
We have \((AB)^T=B^TA^T\).
\end{Lemma}
\begin{Proof}
Since \(A^T_{kj}=A_{jk}\) and \(B^T_{ik}=B_{ki}\), we have
\begin{align*}
(AB)^T_{ij}&=(AB)_{ji}\\
&=\sum_kA_{jk}B_{ki}\\
&=\sum_k A^T_{kj}B^T_{ik}\\
&=\sum_k B^T_{ik}A^T_{kj}\\
&=(B^TA^T)_{ij}.
\end{align*}
Therefore \((AB)^T=B^TA^T\), because all the entries agree. \qedhere


\end{Proof}
\begin{Remark}
You may complain that matrix multiplication is not commutative, so
the step where we switch \(A^T_{kj}B^T_{ik}=B^T_{ik}A^T_{kj}\) is
not valid. Fortunately your objection is invalid: \(A^T_{kj}\) and
\(B^T_{ik}\) are matrix {\em entries} (i.e. numbers!) not matrices
themselves.


\end{Remark}
\subsection{Orthogonal matrices}


\begin{Definition}
An \(n\)-by-\(n\) matrix \(A\) is called {\em orthogonal} if
\(A^TA=I\).


\end{Definition}
\begin{Example}
The rotation matrix \(R_{\theta}=\ma \cos\theta & -\sin\theta
\\ \sin\theta & \cos\theta\mz\) is orthogonal. To see this, note
that \(R_{\theta}^T=\ma \cos\theta & \sin\theta \\ -\sin\theta &
\cos\theta\mz=R_{-\theta}\), so
\(R_{\theta}^TR_{\theta}=R_{\theta-\theta}=I\). In general, you
should think of an orthogonal matrix as giving a higher-dimensional
version of a rotation or reflection.


\end{Example}
\begin{Lemma}
If \(A\) is an orthogonal matrix then \((Av)\cdot(Aw)=v\cdot w\). In
particular, the action of an orthogonal matrix doesn't change the
lengths of vectors.
\end{Lemma}
\begin{Proof}
We have
\begin{align*}
(Av)\cdot(Aw)&=(Av)^TAw\\
&=v^TA^TAw\\
&=v^TIw\\
&=v^Tw\\
&=v\cdot w.
\end{align*}
The length of a vector \(v\) is \(\sqrt{v\cdot
v}=\sqrt{v_1^2+\cdots+v_n^2}\) by Pythagoras's theorem, so
\(|Av|=\sqrt{(Av)\cdot(Av)}=\sqrt{v\cdot v}=|v|\). \qedhere


\end{Proof}
\begin{Proof}[Proof of \cref{thm:dot}]\label{prf:thm:dot}
Because we're only interested in the two vectors \(v\) and \(w\), we
can look at the plane which contains them, and we reduce to the case
where \(v\) and \(w\) are 2-dimensional. Moreover, we can rotate so
that \(v\) points in the positive \(x\)-direction. Rotation is given
by the action of an orthogonal matrix, so \(v\cdot w\) is unchanged
by this. If \(v\) points in the positive \(x\)-direction then
\(v=\ma |v| & 0\mz\) and \(v\cdot w=|v|w_1\), where \(w=\ma
w_1\\ w_2\mz\). Since \(w\) makes an angle \(\phi\) with \(v\),
\(w=\ma |w|\cos\phi \\ |w|\sin\phi\mz\), so the formula
follows. \qedhere


\end{Proof}
\clearpage
\section{3-dimensional rotations}
\subsection{3-dimensional rotations}


Armed with our newfound understanding of angles, let's take a look at
some 3-by-3 rotation matrices and figure out what rotation is being
represented.


\begin{Example}
The matrix \(A=\ma \cos\phi & -\sin\phi & 0 \\ \sin\phi & \cos\phi &
0 \\ 0 & 0 & 1\mz\) is a rotation matrix; just by looking at it, we
can see that the \(z\)-axis is fixed: \[A\ma 0 \\ 0 \\ 1\mz=\ma 0
\\ 0 \\ 1\mz\] and the \(xy\)-plane gets rotated by \(\phi\); for
example, the unit vector \(v=\ma 1 \\ 0 \\ 0 \mz\) goes to the unit
vector \(w=\ma \cos\phi \\ \sin\phi \\ 0 \mz\), and \(v\cdot
w=\cos\phi\), so \(v\) gets rotated by an angle \(\phi\).


\end{Example}
\begin{Example}
In \cref{exm:rot3d}, I claimed that \(C=\ma 0 & 0 & 1 \\ 0 & 1 & 0
\\ -1 & 0 & 0\mz\) is a rotation matrix for \(\RR^3\). That means
there's a fixed vector (the axis) and the plane orthogonal to the
axis is rotated by some angle. Let's figure out what the axis is and
what the angle is.


If \(u=\ma x \\ y\\z\mz\) is a fixed vector then \(u=Cu\), which in
this case means
\[\ma
x \\y \\ z\mz=\ma 0 & 0 & 1 \\ 0 & 1 & 0 \\ -1 & 0 & 0 \mz\ma x \\ y
\\ z\mz=\ma z \\ y \\ -x\mz.\]
This implies \(-x=z=x\), so \(x=z=0\), and we see that the
\(y\)-axis is fixed.


The \(xz\)-plane is orthogonal to the \(y\)-axis, so the next task
is to find by what angle it is rotated. Let us pick a vector (say
\(v=\ma 1 \\ 0 \\ 0\mz\)) in that plane and act using \(C\) to get a
new vector \(Cv=\ma 0 \\ 0 \\ -1\mz\). We note that \(v\cdot Cv=0\),
so in this case the rotation must be through 90 degrees.


\end{Example}
\begin{Example}
Here is a more involved example. The matrix \(D=\ma 0 & 0 & 1 \\ 1 &
0 & 0 \\ 0 & 1 & 0\mz\) defines a rotation in 3 dimensions. To find
the axis \(u\) we need to solve \(u=Du\): \[\ma x \\ y \\ z \mz=\ma
0 & 0 & 1 \\ 1 & 0 & 0 \\ 0 & 1 & 0\mz\ma x \\ y\ \\ z\mz=\ma z \\ x
\\ y\mz,\] which means \(x=y=z\), so the axis points in the
direction of \(u=\ma 1 \\ 1 \\ 1 \mz\). Now pick \(v=\ma 1 \\ -1
\\ 0\mz\) orthogonal to \(u\) (\(u\cdot v=1-1=0\)). Compute \(Av=\ma
0 \\ 1 \\ -1\mz\), and \[v\cdot Av=\ma 1 \\ -1 \\ 0\mz\cdot \ma 0
\\ 1 \\ -1\mz=-1.\] Now \(|v|=|Av|=\sqrt{2}\), so
\(\cos\phi=\frac{v\cdot Av}{\sqrt{2}\sqrt{2}}=-\frac{1}{2}\), so
\(\cos(\phi)=-1/2\) and \(\phi=2\pi/3\).


\end{Example}
\begin{Remark}
How do I recognise when a matrix is a rotation matrix? It turns out
that the rotations are precisely the orthogonal matrices with
determinant one (we will define the determinant of a matrix later).


\end{Remark}
\subsection{Logarithms of rotations*}


We saw earlier that \(\exp\ma 0 & -\theta \\ \theta & 0 \mz=\ma
\cos\theta & -\sin\theta \\ \sin\theta & \cos\theta\mz\). This is a
special case of a beautiful general fact.


\begin{Definition}
We say that a matrix is symmetric (respectively antisymmetric) if
\(A^T=A\) (respectively \(A^T=-A\)).


\end{Definition}
\begin{Theorem}
If \(A\) is an antisymmetric matrix then \(\exp(tA)\) is
orthogonal for all \(t\). Conversely, if \(\exp(tA)\) is
orthogonal for all \(t\) then \(A\) is antisymmetric.
\end{Theorem}
\begin{Proof}
If \(A\) is antisymmetric then
\(\exp(tA)^T=\exp(tA^T)=\exp(-tA)\). We will see below that
\[\exp(-B)\exp(B)=I\] for any matrix \(B\), so this shows that
\(\exp(tA)\) is orthogonal for all \(t\).


For the converse, you can differentiate the expression \(\exp(tA)\)
with respect to \(t\). This is nothing scary: \(\exp(tA)\) is just a
matrix whose coefficients are functions of \(t\), and
differentiation just means differentiating the entries. Here are
some properties of the matrix exponential which we need:
\begin{itemize}
\item \(\frac{d}{dt}\exp(tA)=A\exp(tA)\)
\item \(\exp(tA)^T=\exp(tA^T)\)
\item \(\frac{d}{dt}(M(t)N(t))=\frac{dM(t)}{dt}N+M(t)\frac{dN(t)}{dt}\)
(Leibniz rule).
\end{itemize}
Assuming these properties, we have
\[0=\left.\frac{d}{dt}\right|_{t=0}I=
\left.\frac{d}{dt}\right|_{t=0}(\exp(tA)\exp(tA)^T)=A+A^T,\]
so \(A\) is antisymmetric. \qedhere


\end{Proof}
Let's prove all the properties we wanted. I'll just assume we don't
have to worry about convergence issues for the power series defining
\(\exp\) (it's one of the nicest power series around and you can
always rely on it behaving the way you want it to).


\begin{Lemma}
If \(B\) is a matrix then \(\exp(B)\exp(-B)=I\).
\end{Lemma}
\begin{Proof}
We have
\begin{align*}
\exp(B)\exp(-B)&=\sum_{m\geq 0}\sum_{n\geq 0}\frac{1}{m!n!}B^m(-B)^n\\
&=\sum_{p\geq 0}\sum_{m=0}^p\frac{1}{m!(p-m)!}B^m(-B)^n\\
&=\sum_{p\geq 0}\frac{1}{p!}\sum_{m=0}^p\frac{p!}{m!(p-m)!}B^m(-B)^n\\
&=\sum_{p\geq 0}\frac{1}{p!}(B-B)^p\\
&=\exp(0)=I
\end{align*}
where we substituted \(p=m+n\) and rearranged the infinite sum on
line 2, multiplied by \(p!/p!\) on line 3, and used the binomial
theorem on line 4. \qedhere


\end{Proof}
\begin{Lemma}
\[\frac{d}{dt}\exp(tA)=A\exp(tA).\]
\end{Lemma}
\begin{Proof}
\begin{align*}
\frac{d}{dt}\exp(tA)&=\frac{d}{dt}\sum_{n\geq 0}\frac{t^n}{n!}A^n\\
&=\sum_{n\geq 1}\frac{t^{n-1}}{(n-1)!}A^n\\
&=A\sum_{m\geq 0}\frac{t^m}{m!}A^m\mbox{ relabelling }m=n-1. \qedhere
\end{align*}


\end{Proof}
\begin{Lemma}
\[\exp(B)^T=\exp(B^T).\]
\end{Lemma}
\begin{Proof}
Clearly we have \((B_1+B_2)^T=B_1^T+B_2^T\), and we also have
\((B^n)^T=(B^T)^n\) (using \((AB)^T=B^TA^T\) and
induction). Therefore
\[\exp(B)^T=\left(\sum_{n\geq 0}\frac{1}{n!}B^n\right)^T=\sum_{n\geq 0}\frac{1}{n!}(B^n)^T=\sum_{n\geq 0}\frac{1}{n!}(B^T)^n=\exp(B^T). \qedhere\]


\end{Proof}
\begin{Lemma}
\[\frac{d}{dt}(M(t)N(t))=\frac{dM(t)}{dt}N+M(t)\frac{dN(t)}{dt}.\]
\end{Lemma}
\begin{Proof}
Let's use index notation. The \(ij\) entry of \(M(t)N(t)\) is
\(\sum_kM_{ik}(t)N_{kj}(t)\), so
\begin{align*}
\frac{d}{dt}(M(t)N(t))_{ij}&=\frac{d}{dt}\left(\sum_kM_{ik}(t)N_{kj}(t)\right)\\
&=\sum_k\frac{dM_{ik}(t)}{dt}N_{kj}(t)+\sum_kM_{ik}(t)\frac{dN_{kj}(t)}{dt}\\
&\qquad\qquad\mbox{ using the usual Leibniz rule}\\
&=\left(\frac{dM(t)}{dt}N+M(t)\frac{dN(t)}{dt}\right)_{ij}. \qedhere
\end{align*}


\end{Proof}
\begin{Example}
The general 3-d rotation matrix is therefore \(\exp\ma 0 & \alpha &
\gamma \\ -\alpha & 0 & \beta \\ -\gamma & -\beta & 0\mz\).


\end{Example}
\clearpage
\section{Simultaneous equations}
\subsection{Simultaneous equations}
A system of simultaneous linear equations, like
\begin{align*}
x-y&=-1\\
x+y&=3
\end{align*}
can be written as a single matrix equation \(Av=b\), like
\[\ma 1 & -1 \\ 1 & 1 \mz\ma x \\ y \mz=\ma -1 \\ 3\mz.\]
In fact, we often omit the \(x\)s and \(y\)s completely, and write
instead the {\em augmented matrix}
\[\begin{pmatrix}[cc|c]
1 & -1 & -1 \\ 1 & 1 & 3
\end{pmatrix}\]


\subsection{Row operations}


When we try to solve a system of equations like this, there are a
bunch of operations we perform, like ``add the second equation to the
first'' or ``multiply the first equation by 5'', and we can interpret
these in terms of matrices. We illustrate this using the above
example.


\begin{longtable}{p{5.6cm}p{5.7cm}}
Start with:
{\begin{align*}x-y&=-1\\ x+y&=3.\end{align*}}
&
Write the {\em augmented matrix}
{\[\begin{pmatrix}[cc|c] 1 & -1 & -1\\ 1 & 1 & 3\end{pmatrix}.\]}
\\
Subtract eq. 1 from eq. 2: {\begin{align*}x-y&=-1\\2y&=4.\end{align*}}
&
Subtract row 1 from row 2: {\[\begin{pmatrix}[cc|c] 1 & -1 & -1\\ 0 & 2 & 4\end{pmatrix}.\]}
\\
Halve eq. 2 {\begin{align*}x-y&=-1\\y&=2\end{align*}}
&
Halve row 2: {\[\begin{pmatrix}[cc|c] 1 & -1 & -1\\ 0 & 1 & 2\end{pmatrix}\]}
\\
Add eq. 2 to eq. 1: {\begin{align*}x&=1\\y&=2,\end{align*}}
&
Add row 2 to row 1: {\[\begin{pmatrix}[cc|c] 1 & 0 & 1\\ 0 & 1 & 2\end{pmatrix}\]}
\\
and we're done.
&
i.e. {\[\ma 1 & 0 \\ 0 & 1 \mz\ma x \\ y \mz=\ma 1 \\ 2\mz,\]} or \(x=1\), \(y=2\).
\end{longtable}


\begin{Definition}[Row operations]\label{dfn:rowops}
Given a matrix (possibly augmented with a vertical bar somewhere),
we define the {\em row operations}:
\begin{itemize}
\item (Type I) \(R_i\mapsto R_i+\lambda R_j\): ``add \(\lambda\) times
the \(j\)th row to the \(i\)th row''.
\item (Type II) \(R_i\mapsto \lambda R_i\): ``multiply the \(i\)th row
by \(\lambda\)''.


\end{itemize}
\end{Definition}
So the sequence of row operations used in the above example was:
\(R_2\mapsto R_2-R_1\), \(R_2\mapsto \frac{1}{2}R_2\), \(R_1\mapsto
R_1+R_2\).


\subsection{Echelon forms}


The dream goal of solving simultaneous equations is to reduce to a
system of the form \[x=\mbox{something},\quad y=\mbox{something
else},\ldots\] If you can achieve this (which is not always possible,
for example if your system has no solutions, or has many solutions)
then, in terms of matrices, you have reduced the left-block of your
augmented matrix to the identity matrix. In general, the best we can
hope for is to reduce our matrix to so-called {\em reduced echelon
form}.


\begin{Definition}[Echelon forms]\label{dfn:rowech}
Given a nonzero row \(R\) of a matrix \(M\), we define its {\em
leading entry} to be the leftmost nonzero entry.
\begin{itemize}
\item We say that \(M\) is in {\em echelon form} if, for every nonzero
row \(R_i\), the row \(R_{i-1}\) immediately above it is nonzero
and the leading entry of \(R_{i-1}\) sits to the left of the
leading entry of \(R_i\). In other words, the bottom-left chunk of
\(M\) consists of zeros sitting in a configuration like a set of
steps\footnote{The word ``echelon'' comes from the French word
``\'{e}chelle'' meaning ``ladder''.}.
\item We say that \(M\) is in {\em reduced echelon form} if it is in
echelon form, every leading entry is a \(1\) and every other entry
in a column containing a leading entry vanishes. If the \(i\)th
row of \(M\) has leading entry \(M_{ij}=1\) then we will call
\(j\) the \(i\)th {\em leading index}. We call the other indices
{\em free} and write \(F\) for the set of free indices.


\end{itemize}
\end{Definition}
\begin{Example}
Consider the following matrices
\begin{align*}
A&=\ma 1 & 2 & 3 \\ 0 & 1 & 2 \\ 0 & 0 & 1\mz,& B&=\ma 1 & 1\mz,& C&=\ma 1 & 0 & 1 & 1\\ 0 & 1 & 2 & -1\mz,\\
D&=\ma 0 & 0 & 1 & 0 \\ 0 & 0 & 1 & 0\mz& E&=\ma 2 & 0 & 1 \\ 0 & 3 & 0 \\ 0 & 0 & 0 \mz& F&=\ma 0 & 1 \\ 1 & 0\mz,\\
G&=\ma 1 & 2 & 0 & 1 \\ 0 & 0 & 1 & 8 \\ 0 & 0 & 0 & 0 \\ 0 & 0 & 0 & 0 \\ 0 & 0 & 0 & 0\mz&H&=\ma 0 & 1 & 0 & 0 & 0\\ 0 & 0 & 1 & 0 & 0 \\ 0 & 0 & 0 & 0 & 1 \mz&J&=\ma 0 & 0 & 1 \\ 0 & 1 & 0 \\ 1 & 0 & 0\mz
\end{align*}
\(A,B,C,E,G,H\) are in echelon form. \(B,C,G,H\) are in reduced
echelon form. \(D,F,J\) are in neither. For the matrices in reduced
echelon form:
\begin{itemize}
\item \(B\) has one leading index, \(1\), and one free index \(2\).
\item the leading indices of \(C\) are \(1,2\); the free indices are
\(3,4\).
\item the leading indices of \(G\) are \(1,3\); the free indices are
\(2,4\).
\item the leading indices of \(H\) are \(2,3,5\); the free indices are
\(1,4\).


\end{itemize}
\end{Example}
\subsection{Echelon form and simultaneous equations}


If \(M\) is in reduced echelon form then it is very easy to understand
the corresponding system of simultaneous equations \(Mv=b\). Here are
some illustrative examples.


\begin{Example}
Suppose that \(M=\ma 1 & 0 & 1 & 1 \\ 0 & 1 & 2 & -1\mz\) and
\(b=\ma b_1 \\ b_2\mz\). The system of simultaneous equations
\(Mv=b\) we get is
\begin{align*}
x_1+x_3+x_4&=b_1\\
x_2+2x_3-x_4&=b_2.
\end{align*}
We can rearrange:
\begin{align*}
x_1&=b_1-x_3-x_4\\
x_2&=b_2-2x_3+x_4.
\end{align*}
In other words, for every value of the variables \(x_3,x_4\), we get
a solution \(v=\ma b_1-x_3-x_4\\ b_2-2x_3+x_4\\ x_3\\ x_4\mz\). The
``free variables'' \(x_3,x_4\) are associated with free indices
\(3,4\) and the ``dependent variables'' \(x_1,x_2\) are associated
with leading indices. Here, {\em dependent} means that the values of
\(x_1,x_2\) are determined by \(x_3,x_4\) via the equations.


\end{Example}
\begin{Example}
Suppose that \(M=\ma 1 & 2 & 0 & 1 \\ 0 & 0 & 1 & 8 \\ 0 & 0 & 0 & 0
\\ 0 & 0 & 0 & 0 \\ 0 & 0 & 0 & 0\mz\) and \(b=\ma
b_1\\ b_2\\ b_3\\ b_4\\ b_5\mz\). The system of simultaneous
equations \(Mv=b\) we get is
\begin{align*}
x_1+2x_2+x_4&=b_1\\
x_3+8x_4&=b_2\\
0&=b_3\\
0&=b_4\\
0&=b_5.
\end{align*}
This has solutions if and only if \(b_3=b_4=b_5=0\). In the case
when this condition holds, there are free variables \(x_2,x_4\) (for
the free indices) and dependent variables \(x_1=b_1-2x_2-x_4\),
\(x_3=b_2-8x_4\). The general solution is then \(v=\ma
b_1-2x_2-x_4\\ x_2\\b_2-8x_4\\x_4\mz\) (provided \(b_3=b_4=b_5=0\)).


\end{Example}
More generally, the same reasoning shows:


\begin{Theorem}
Suppose that:
\begin{itemize}
\item \(M\) is an \(m\)-by-\(n\) matrix in reduced echelon form,
\item the first \(k\leq m\) rows of \(M\) are non-zero and the final
\(m-k\) rows are zero,
\item the leading entry in row \(i\leq k\) is in column \(j_i\) (so the
leading indices are \(j_1,\ldots,j_k\)).
\end{itemize}
Then the general solution \(v=\ma x_1\\ \vdots\\x_n\mz\) exists if
and only if \(b_{k+1}=\cdots=b_m=0\) and has free variables \(x_p\)
(where \(p\) runs over the set \(F\) of free indices), dependent
variables \(x_{j_i}=b_i-\sum_{p\in F} M_{ip}x_p\).


\end{Theorem}
\begin{Remark}
In particular, the space of solutions has dimension equal to the
number of free indices.


\end{Remark}
\begin{Example}
Consider the matrix
\(A=\ma 1 & 0 & 2\\ 0 & 1 & 1\mz\). This is in reduced echelon
form. If it is used to form a system of equations \(Av=b\) then
these equations have the form
\begin{align*}
x_1+2x_3&=b_1\\
x_2+x_3&=b_2
\end{align*}
which can be solved immediately: \[x_1=b_1-2x_3,\qquad
x_2=b_2-x_3.\] In other words, for each \(x_3\), we get a solution
\(v=\ma b_1-2x_3\\ b_2-x_3\\ x_3\mz\).


\end{Example}
\begin{Example}
Consider the matrix
\(A=\ma 1 & 0 & 0 \\ 0 & 1 & 0\\ 0 & 0 & 0\mz\). This is in reduced
echelon form. If it is used to form a system of equations \(Av=b\)
then these equations have the form
\begin{align*}
x_1&=b_1\\
x_2&=b_2\\
0&=b_3
\end{align*}
This system can be solved if and only if \(b_3=0\), in which case it
has a solution \(v=\ma b_1 \\ b_2 \\ x_3\mz\) for every possible
value of \(x_3\).


\end{Example}
\begin{Example}
Consider the \(n\)-by-\(n\) identity matrix \(I\). This is in
reduced echelon form. If it is used to form a system of equations
\(Iv=b\) then these equations have the unique solution \(v=\ma
b_1\\ \vdots\\ b_n\mz\). (Duh\footnote{This is a colloquial form of
the Latin QED.}.)


\end{Example}
\begin{Example}
Consider the \(n\)-by-\(n\) zero matrix. This is in reduced echelon
form. If it is used to form a system of equations \(0v=b\) then
these equations have solutions if and only if \(b=0\); if \(b=0\)
then any \(v\) is a solution.


\end{Example}
In other words, once a matrix is in reduced echelon form, it becomes
very transparent how to solve the corresponding system of simultaneous
equations.


\clearpage
\section{Echelon form theorems}
\subsection{Putting a matrix into echelon form}


We will soon see that any matrix can be put into echelon form by row
operations of type I, and further into reduced echelon form by row
operations of types I and II. Let's see some examples.


\begin{Example}
Consider the matrix \[\begin{pmatrix} 2 & 0 & 2 & 0 \\ 0 & 1 & 1 & 1
\\ 2 & 0 & 5 & 0 \\ 1 & 1 & 1 & 2 \end{pmatrix}\] Clear column
\(1\), row \(3\) using \(R_3\mapsto R_3-R_1\)\[\begin{pmatrix} 2 & 0
& 2 & 0 \\ 0 & 1 & 1 & 1 \\ 0 & 0 & 3 & 0 \\ 1 & 1 & 1 & 2
\end{pmatrix}\] Clear column \(1\), row \(4\) using \(R_4\mapsto
R_4-\frac{1}{2}R_1\)\[\begin{pmatrix} 2 & 0 & 2 & 0 \\ 0 & 1 & 1 & 1
\\ 0 & 0 & 3 & 0 \\ 0 & 1 & 0 & 2 \end{pmatrix}\] Clear column
\(2\), row \(4\) using \(R_4\mapsto R_4-R_2\)\[\begin{pmatrix} 2 & 0
& 2 & 0 \\ 0 & 1 & 1 & 1 \\ 0 & 0 & 3 & 0 \\ 0 & 0 & -1 & 1
\end{pmatrix}\] Clear column \(3\), row \(4\) using \(R_4\mapsto
R_4+\frac{1}{3}R_3\)\[\begin{pmatrix} 2 & 0 & 2 & 0 \\ 0 & 1 & 1 & 1
\\ 0 & 0 & 3 & 0 \\ 0 & 0 & 0 & 1 \end{pmatrix}\] This is now in
echelon form. We can go further to reduced echelon form.


Make leading entries in rows 1 and 3 equal to 1 using \(R_1\mapsto
\frac{1}{2}R_1\) and \(R_3\mapsto\frac{1}{3}R_3\). \[\begin{pmatrix}
1 & 0 & 1 & 0 \\ 0 & 1 & 1 & 1 \\ 0 & 0 & 1 & 0 \\ 0 & 0 & 0 & 1
\end{pmatrix}\] Clear column 3 using \(R_1\mapsto R_1-R_3\) and
\(R_2\mapsto R_2-R_3\)\[\begin{pmatrix} 1 & 0 & 0 & 0 \\ 0 & 1 & 0 &
1 \\ 0 & 0 & 1 & 0 \\ 0 & 0 & 0 & 1 \end{pmatrix}\] Clear column 4
using \(R_2\mapsto R_2-R_4\)\[\begin{pmatrix} 1 & 0 & 0 & 0 \\ 0 & 1
& 0 & 0 \\ 0 & 0 & 1 & 0 \\ 0 & 0 & 0 & 1 \end{pmatrix}\]


\end{Example}
This means that for any \(b\) there is a unique solution to \(Mv=b\)
(no free variables and no constraints on \(b\)).


\begin{Example}
In this example, we'll keep track of the augmented column. We start
with the matrix \[\begin{pmatrix}[ccc|c] 1 & -1 & 0 & b_1 \\ 1 & 1 &
-1 & b_2 \\ 4 & 0 & -2 & b_3 \\ 0 & 2 & -1 & b_4 \end{pmatrix}\]


Clear column \(1\), using \(R_2\mapsto R_2-R_1\) and \(R_3\mapsto
R_3-4R_1\)\[\begin{pmatrix}[ccc|c] 1 & -1 & 0 & b_1\\ 0 & 2 & -1
&b_2-b_1 \\ 0 & 4 & -2 & b_3-4b_1\\ 0 & 2 & -1 &b_4 \end{pmatrix}\]
Clear column \(2\), using \(R_3\mapsto R_3-2R_2\) and \(R_4\mapsto
R_4-R_2\) \[\begin{pmatrix}[ccc|c] 1 & -1 & 0 & b_1 \\ 0 & 2 & -1 &
b_2-b_1 \\ 0 & 0 & 0 & b_3-4b_1-2(b_2-b_1) \\ 0 & 0 & 0 &
b_4-(b_2-b_1) \end{pmatrix}\] Make leading entry in row 2 equal to 1
using \(R_2\mapsto\frac{1}{2}R_2\). \[\begin{pmatrix}[ccc|c] 1 & -1
& 0 & b_1 \\ 0 & 1 & -1/2 & (b_2-b_1)/2 \\ 0 & 0 & 0 & b_3-2b_1-2b_2
\\ 0 & 0 & 0 & b_4+b_1-b_2 \end{pmatrix}\] Clear column 2 using
\(R_1\mapsto R_1+R_2\)\[\begin{pmatrix}[ccc|c] 1 & 0 & -1/2 &
(b_1+b_2)/2 \\ 0 & 1 & -1/2 & (b_2-b_1)/2 \\ 0 & 0 & 0 &
b_3-2b_1-2b_2 \\ 0 & 0 & 0 & b_4+b_1-b_2\end{pmatrix}\]


We see that the general solution exists if \(b_4+b_1-b_2=0\) and
\(b_3-2b_1-2b_2=0\), in which case there is one free variable
\(x_3\) and two dependent variables \[x_1=(b_1+b_2+x_3)/2,\qquad
x_2=(b_2-b_1+x_3)/2.\] For example, if \(b=\ma -3 \\ 0 \\ -6
\\ 3\mz\) then \(b_4+b_1-b_2=3-3-0=0\) and
\(b_3-2b_1-2b_2=-6+6-0=0\), and we get the general solution \(\ma
(x_3-3)/2 \\ (x_3+3)/2 \\ x_3\mz\).


\end{Example}
\begin{Example}
Again, we'll keep track of the augmented column. We start with the
matrix \[\begin{pmatrix}[cccc|c] 3 & -2 & -1 & -5 & b_1 \\ -5 & 3 &
2 & -3 & b_2 \\ 0 & -2 & -1 & 1 & b_3\end{pmatrix}\] Clear column
\(1\), row \(2\) using \(R_2\mapsto
R_2+\frac{5}{3}R_1\)\[\begin{pmatrix}[cccc|c] 3 & -2 & -1 & -5 &
b_1\\ 0 & -1/3 & 1/3 & -34/3 & \frac{5}{3}b_1+b_2 \\ 0 & -2 & -1 & 1
& b_3 \end{pmatrix}\] Clear column \(2\), row \(3\) using
\(R_3\mapsto R_3-6R_2\)\[\begin{pmatrix}[cccc|c] 3 & -2 & -1 & -5 &
b_1 \\ 0 & -1/3 & 1/3 & -34/3 & \frac{5}{3}b_1+b_2 \\ 0 & 0 & -3 &
69 & -10b_1-6b_2+b_3 \end{pmatrix}\] Make leading entries in rows 1
and 2 equal to 1 using \(R_1\mapsto \frac{1}{3}R_1\) and
\(R_2\mapsto -3R_2\). \[\begin{pmatrix}[cccc|c] 1 & -2/3 & -1/3 &
-5/3 & \frac{1}{3}b_1 \\ 0 & 1 & -1 & 34 & -5b_1-3b_2 \\ 0 & 0 & -3
& 69 & -10b_1-6b_2+b_3 \end{pmatrix}\] Clear column 2 using
\(R_1\mapsto R_1+\frac{2}{3}R_2\)\[\begin{pmatrix}[cccc|c] 1 & 0 &
-1 & 21 &-3b_1-2b_2 \\ 0 & 1 & -1 & 34 & -5b_1-3b_2 \\ 0 & 0 & -3 &
69 & -10b_1-6b_2+b_3 \end{pmatrix}\] Make leading entry in row 3
equal to 1 using \(R_3\mapsto -\frac{1}{3}R_3\).
\[\begin{pmatrix}[cccc|c] 1 & 0 & -1 & 21 & -3b_1-2b_2\\ 0 & 1 & -1
& 34 & -5b_1-3b_2 \\ 0 & 0 & 1 & -23 &
\frac{10}{3}b_1+2b_2-\frac{1}{3}b_3 \end{pmatrix}\] Clear column 3
using \(R_1\mapsto R_1+R_3\) and \(R_2\mapsto R_2+R_3\)
\[\begin{pmatrix}[cccc|c] 1 & 0 & 0 & -2 & \frac{1}{3}(b_1-b_3) \\ 0
& 1 & 0 & 11 & -\frac{1}{3}(5b_1+3b_2+b_3) \\ 0 & 0 & 1 & -23 &
\frac{10}{3}b_1+2b_2-\frac{1}{3}b_3 \end{pmatrix}\] We see that this
always has a solution, and the general solution is \[\ma
\frac{1}{3}(b_1-b_3)+2x_4\\ -\frac{1}{3}(5b_1+3b_2+b_3)-11x_4\\ \frac{10}{3}b_1+2b_2-\frac{1}{3}b_3+23x_4\\ x_4\mz\]
with one free variable \(x_4\).


\end{Example}
\subsection{Echelon form theorems}


\begin{Theorem}[Echelon form]\label{thm:echelonform}
Every \(m\)-by-\(n\) matrix \(A\) can be put into echelon form using
only the row operations \(R_i\mapsto R_i+\lambda R_j\).
\end{Theorem}
\begin{Proof}
We will prove the theorem by induction on the size of the
matrix. Suppose we have proved the theorem for all \(m'\)-by-\(n\)
matrices with \(m'<m\). The base case for induction is then \(m=1\)
but if there is only one row then the matrix is automatically in
echelon form, which proves the base case. Now for the induction
step.


If your matrix is zero then it's already in echelon form, so without
loss of generality, assume that there is a nonzero row.
\begin{itemize}
\item Of all the nonzero rows, pick the row \(R_i\) whose leading entry
\(A_{ij}\) is furthest to the left (i.e. \(j\) is minimal); if
there are several such rows, pick the topmost (i.e. with \(i\)
minimal).
\item If \(i\neq 1\) (i.e. if \(R_i\) is not the top row) then apply the
row operation \(R_1\mapsto R_1+R_i\) so that the top row also has
leading entry \(A_{ij}\).
\item For \(k=2,\ldots,m\), apply the row operation \(R_k\mapsto
R_k-\frac{A_{kj}}{A_{ij}}R_1\). This ensures that the leading
entries of all nonzero rows below the top are to the right of the
leading entry of the top row.
\end{itemize}
Now consider the \((m-1)\)-by-\(n\) submatrix \(A'\) you get by
erasing the top row \(R_1\). By induction, we can put this into
echelon form using only row operations \(R'_i\mapsto R'_i+\lambda
R'_j\). Such operations don't introduce any leading entries in
column \(j\) or to the left of it because our submatrix \(A'\) has
zero entries in all these columns. Therefore, if we pop \(R_1\) back
on top of \(A'\), the result in in echelon form. Since the row
operations didn't affect \(R_1\), we can think of them as row
operations on \(A\), so we have put \(A\) into echelon form using
only row operations of the specified type. \qedhere


\end{Proof}
\begin{Theorem}[Reduced echelon form]\label{thm:reducedechelonform}
Every \(m\)-by-\(n\) matrix \(A\) can be put into reduced echelon
form by a sequence of row operations \(R_i\mapsto R_i+\lambda R_j\)
and \(R_k\mapsto \lambda R_k\) (\(\lambda\neq 0\)).
\end{Theorem}
\begin{Proof}
First, use \cref{thm:echelonform} to put \(A\) into echelon
form. Now, for each nonzero row \(R_i\), with leading entry
\(A_{ij}\), perform the row operation
\(R_i\mapsto\frac{1}{A_{ij}}R_i\) to make the leading entry equal to
\(1\). Finally, for every nonzero row \(R_i\) and every row \(R_k\)
with \(k\neq i\), perform the row operation \(R_k\mapsto
R_k-A_{kj}R_i\). This clears out the nonzero entries in columns
above and below the leading entry \(A_{ij}\) of \(R_i\). The result
is in reduced echelon form. \qedhere


\end{Proof}
\clearpage
\section{Inverses}
\subsection{Definition and basic properties}


We've seen how to multiply and even exponentiate matrices. Can we
``divide'' by a matrix?


\begin{Theorem}
If \(A=\ma a & b \\ c & d \mz\) is a \(2\)-by-\(2\) matrix with
\(ad-bc\neq 0\) then the matrix \[A^{-1}:=\frac{1}{ad-bc}\ma d & -b
\\ -c & a\mz\] is an {\em inverse} for \(A\) in the sense that
\(AA^{-1}=A^{-1}A=I\).
\end{Theorem}
\begin{Proof}
We'll just check \(A^{-1}A=I\).
\begin{align*}
A^{-1}A&=\frac{1}{ad-bc}\ma d & -b \\ -c & a\mz\ma a & b \\ c & d\mz\\
&=\frac{1}{ad-bc}\ma da-bc & db-bd \\ -ca+ac & -cb+ad\mz\\
&=\frac{1}{ad-bc}\ma ad-bc & 0 \\ 0 & ad-bc\mz\\
&=I. \qedhere
\end{align*}


\end{Proof}
\begin{Remark}
This is great. However, you should never write \(A^{-1}\) as
\(\frac{1}{A}\). The reason is that \(\frac{B}{A}\) could mean
\(A^{-1}B\) or \(BA^{-1}\) and these are in general different
matrices (because matrix multiplication is not commutative).


\end{Remark}
We want to generalise this idea to \(n\)-by-\(n\) matrices.


\begin{Definition}
Let \(A\) be an \(n\)-by-\(n\) (square!) matrix. We say that \(A\)
is {\em invertible} if there exists a matrix \(A^{-1}\) such that
\(A^{-1}A=AA^{-1}=I\).


\end{Definition}
\begin{Remark}
Note that if an inverse exists, it is unique because if \(B,C\) are
two inverses for \(A\) then \(AB=AC=I\) and so
\(B=BI=BAB=BAC=IC=C\).


\end{Remark}
\begin{Lemma}\label{lma:prodinv}
If \(A\) and \(B\) are invertible with inverses \(A^{-1}\) and
\(B^{-1}\) then \(AB\) is invertible with inverse \(B^{-1}A^{-1}\)
(note the order is reversed!)
\end{Lemma}
\begin{Proof}
We have
\[(AB)(B^{-1}A^{-1})=A(BB^{-1})A^{-1}=AIA^{-1}=AA^{-1}=I.\]
Similarly, one can show \((B^{-1}A^{-1})(AB)=I\). \qedhere


\end{Proof}
In this section, we will see an algorithm to test if a matrix is
invertible and, if it is, compute its inverse. We will later introduce
a quantity called the {\em determinant} of a square matrix which is
the analogue of \(ad-bc\) for \(2\)-by-\(2\) matrices in the sense
that a matrix is invertible if and only if its determinant is nonzero
(and there's a formula for the inverse in terms of determinants).


\subsection{Inverse matrices and reduced echelon form}


Observe that finding \(A^{-1}\) is equivalent to solving the
simultaneous equations associated to \(Av=b\). Indeed, if \(A\) is
invertible then \(v=A^{-1}b\) is a solution to \(Av=b\). Since we know
how to solve simultaneous equations, we also know how to find
inverses! In fact, we were secretly doing this already in the chapter
on simultaneous equations.


The following theorem makes this precise.


\begin{Theorem}\label{thm:invech}
Given an \(n\)-by-\(n\) matrix \(A\), form the {\em augmented
matrix} \((A|I_n)\) (where \(I_n\) is the \(n\)-by-\(n\) identity
matrix). Use row operations on the augmented matrix to put \(A\)
into reduced echelon form. Then \(A\) is invertible if and only if
the reduced echelon form of \(A\) is \(I_n\), and in this case, the
result of putting \((A|I_n)\) into reduced echelon form is
\((I_n|A^{-1})\).


\end{Theorem}
We will first use the theorem to compute some examples of inverse
matrices, then we will develop a little more theory and prove the
theorem.


\subsection{Examples}


\begin{Example}
Let's invert the matrix \(\ma -3 & -2 & -4 \\ 2 & 3 & 3 \\ -1 & 4 &
-4\mz\). We start by writing the augmented matrix \[\begin{pmatrix}[
c c c | c c c ] -3 & -2 & -4 & 1 & 0 & 0 \\ 2 & 3 & 3 & 0 & 1 & 0
\\ -1 & 4 & -4 & 0 & 0 & 1 \end{pmatrix}\] Clear column \(1\), row
\(2\) using \(R_2\mapsto R_2+(2/3)R_1\)\[\begin{pmatrix}[ c c c | c
c c ] -3 & -2 & -4 & 1 & 0 & 0 \\ 0 & 5/3 & 1/3 & 2/3 & 1 & 0 \\ -1
& 4 & -4 & 0 & 0 & 1 \end{pmatrix}\] Clear column \(1\), row \(3\)
using \(R_3\mapsto R_3+(-1/3)R_1\) \[\begin{pmatrix}[ c c c | c c c
] -3 & -2 & -4 & 1 & 0 & 0 \\ 0 & 5/3 & 1/3 & 2/3 & 1 & 0 \\ 0 &
14/3 & -8/3 & -1/3 & 0 & 1 \end{pmatrix}\] Clear column \(2\), row
\(3\) using \(R_3\mapsto R_3+(-14/5)R_2\) \[\begin{pmatrix}[ c c c |
c c c ] -3 & -2 & -4 & 1 & 0 & 0 \\ 0 & 5/3 & 1/3 & 2/3 & 1 & 0 \\ 0
& 0 & -18/5 & -11/5 & -14/5 & 1 \end{pmatrix}\] Make leading entry
in row 1 equal to 1 using \(R_1\mapsto
(-1/3)R_1\). \[\begin{pmatrix}[ c c c | c c c ] 1 & 2/3 & 4/3 & -1/3
& 0 & 0 \\ 0 & 5/3 & 1/3 & 2/3 & 1 & 0 \\ 0 & 0 & -18/5 & -11/5 &
-14/5 & 1 \end{pmatrix}\] Make leading entry in row 2 equal to 1
using \(R_2\mapsto (3/5)R_2\). \[\begin{pmatrix}[ c c c | c c c ] 1
& 2/3 & 4/3 & -1/3 & 0 & 0 \\ 0 & 1 & 1/5 & 2/5 & 3/5 & 0 \\ 0 & 0 &
-18/5 & -11/5 & -14/5 & 1 \end{pmatrix}\] Clear column 2 using
\(R_1\mapsto R_1+(-2/3)R_2\)\[\begin{pmatrix}[ c c c | c c c ] 1 & 0
& 6/5 & -3/5 & -2/5 & 0 \\ 0 & 1 & 1/5 & 2/5 & 3/5 & 0 \\ 0 & 0 &
-18/5 & -11/5 & -14/5 & 1 \end{pmatrix}\]Make leading entry in row 3
equal to 1 using \(R_3\mapsto (-5/18)R_3\). \[\begin{pmatrix}[ c c
c | c c c ] 1 & 0 & 6/5 & -3/5 & -2/5 & 0 \\ 0 & 1 & 1/5 & 2/5 & 3/5
& 0 \\ 0 & 0 & 1 & 11/18 & 7/9 & -5/18 \end{pmatrix}\]Clear column 3
using \(R_1\mapsto R_1+(-6/5)R_3\)\[\begin{pmatrix}[ c c c | c c c ]
1 & 0 & 0 & -4/3 & -4/3 & 1/3 \\ 0 & 1 & 1/5 & 2/5 & 3/5 & 0 \\ 0 &
0 & 1 & 11/18 & 7/9 & -5/18 \end{pmatrix}\] Clear column 3 using
\(R_2\mapsto R_2+(-1/5)R_3\)\[\begin{pmatrix}[ c c c | c c c ] 1 & 0
& 0 & -4/3 & -4/3 & 1/3 \\ 0 & 1 & 0 & 5/18 & 4/9 & 1/18 \\ 0 & 0 &
1 & 11/18 & 7/9 & -5/18 \end{pmatrix}\] Now the right-hand block is
the inverse of the matrix we started with.


\end{Example}
\begin{Example}
We start with the matrix \[\begin{pmatrix}[ c c c c | c c c c ] 1 &
-1 & 0 & 3 & 1 & 0 & 0 & 0 \\ -1 & 2 & 1 & 0 & 0 & 1 & 0 & 0 \\ -1 &
1 & 1 & -3 & 0 & 0 & 1 & 0 \\ 1 & 0 & 1 & 7 & 0 & 0 & 0 & 1
\end{pmatrix}\] Clear column \(1\), row \(2\) using \(R_2\mapsto
R_2+(1)R_1\)\[\begin{pmatrix}[ c c c c | c c c c ] 1 & -1 & 0 & 3 &
1 & 0 & 0 & 0 \\ 0 & 1 & 1 & 3 & 1 & 1 & 0 & 0 \\ -1 & 1 & 1 & -3 &
0 & 0 & 1 & 0 \\ 1 & 0 & 1 & 7 & 0 & 0 & 0 & 1 \end{pmatrix}\]Clear
column \(1\), row \(3\) using \(R_3\mapsto
R_3+(1)R_1\)\[\begin{pmatrix}[ c c c c | c c c c ] 1 & -1 & 0 & 3 &
1 & 0 & 0 & 0 \\ 0 & 1 & 1 & 3 & 1 & 1 & 0 & 0 \\ 0 & 0 & 1 & 0 & 1
& 0 & 1 & 0 \\ 1 & 0 & 1 & 7 & 0 & 0 & 0 & 1 \end{pmatrix}\]Clear
column \(1\), row \(4\) using \(R_4\mapsto
R_4+(-1)R_1\)\[\begin{pmatrix}[ c c c c | c c c c ] 1 & -1 & 0 & 3 &
1 & 0 & 0 & 0 \\ 0 & 1 & 1 & 3 & 1 & 1 & 0 & 0 \\ 0 & 0 & 1 & 0 & 1
& 0 & 1 & 0 \\ 0 & 1 & 1 & 4 & -1 & 0 & 0 & 1 \end{pmatrix}\]Clear
column \(2\), row \(4\) using \(R_4\mapsto
R_4+(-1)R_2\)\[\begin{pmatrix}[ c c c c | c c c c ] 1 & -1 & 0 & 3 &
1 & 0 & 0 & 0 \\ 0 & 1 & 1 & 3 & 1 & 1 & 0 & 0 \\ 0 & 0 & 1 & 0 & 1
& 0 & 1 & 0 \\ 0 & 0 & 0 & 1 & -2 & -1 & 0 & 1 \end{pmatrix}\]Clear
column 2 using \(R_1\mapsto R_1+(1)R_2\)\[\begin{pmatrix}[ c c c c |
c c c c ] 1 & 0 & 1 & 6 & 2 & 1 & 0 & 0 \\ 0 & 1 & 1 & 3 & 1 & 1 & 0
& 0 \\ 0 & 0 & 1 & 0 & 1 & 0 & 1 & 0 \\ 0 & 0 & 0 & 1 & -2 & -1 & 0
& 1 \end{pmatrix}\]Clear column 3 using \(R_1\mapsto
R_1+(-1)R_3\)\[\begin{pmatrix}[ c c c c | c c c c ] 1 & 0 & 0 & 6 &
1 & 1 & -1 & 0 \\ 0 & 1 & 1 & 3 & 1 & 1 & 0 & 0 \\ 0 & 0 & 1 & 0 & 1
& 0 & 1 & 0 \\ 0 & 0 & 0 & 1 & -2 & -1 & 0 & 1 \end{pmatrix}\]Clear
column 3 using \(R_2\mapsto R_2+(-1)R_3\)\[\begin{pmatrix}[ c c c
c | c c c c ] 1 & 0 & 0 & 6 & 1 & 1 & -1 & 0 \\ 0 & 1 & 0 & 3 & 0 &
1 & -1 & 0 \\ 0 & 0 & 1 & 0 & 1 & 0 & 1 & 0 \\ 0 & 0 & 0 & 1 & -2 &
-1 & 0 & 1 \end{pmatrix}\]Clear column 4 using \(R_1\mapsto
R_1+(-6)R_4\)\[\begin{pmatrix}[ c c c c | c c c c ] 1 & 0 & 0 & 0 &
13 & 7 & -1 & -6 \\ 0 & 1 & 0 & 3 & 0 & 1 & -1 & 0 \\ 0 & 0 & 1 & 0
& 1 & 0 & 1 & 0 \\ 0 & 0 & 0 & 1 & -2 & -1 & 0 & 1
\end{pmatrix}\]Clear column 4 using \(R_2\mapsto
R_2+(-3)R_4\)\[\begin{pmatrix}[ c c c c | c c c c ] 1 & 0 & 0 & 0 &
13 & 7 & -1 & -6 \\ 0 & 1 & 0 & 0 & 6 & 4 & -1 & -3 \\ 0 & 0 & 1 & 0
& 1 & 0 & 1 & 0 \\ 0 & 0 & 0 & 1 & -2 & -1 & 0 & 1
\end{pmatrix}\]Now the right-hand block is the inverse of the matrix
we started with.


\end{Example}
\clearpage
\section{Inverses from echelon form}
\subsection{Elementary matrices}


\begin{Definition}[Elementary matrices I]\label{dfn:elematrix}
If \(i\neq j\), we write \(E_{ij}(\lambda)\) for the matrix with
ones on the diagonal and zeros elsewhere, except for a \(\lambda\)
in position \(ij\) (\(i\)th row, \(j\)th column). For example, if
we're working with \(3\)-by-\(3\) matrices then \[E_{12}(2)=\ma 1 &
2 & 0 \\ 0 & 1 & 0 \\ 0 & 0 & 1\mz,\quad E_{32}(7)=\ma 1 & 0 & 0
\\ 0 & 1 & 0 \\ 0 & 7 & 1\mz,\quad E_{13}(t)=\ma 1 & 0 & t\\ 0 & 1 &
0 \\ 0 & 0 & 1\mz.\]


\end{Definition}
\begin{Lemma}
If \(A\) and \(E_{ij}(\lambda)\) are \(n\)-by-\(n\) matrices then
\(E_{ij}(\lambda)A\) is the matrix obtained from \(A\) by the row
operation \(R_i\mapsto R_i+\lambda R_j\).
\end{Lemma}
\begin{Proof}
Let's consider the case \(i<j\) (the other case is similar so we
omit it). Consider the product
\[\ma
1 & & \mbox{col }i & & \mbox{col }j & & \\
& \ddots & \downarrow & & \downarrow & & \\
\mbox{row }i&\rightarrow & 1 & & \lambda & &\\
& & & \ddots & & & \\
& & & & 1 & & \\
& & & & & \ddots &\\
& & & & & & 1
\mz
\ma
A_{11} & \cdots & \cdots & & \cdots & \cdots & A_{1n} \\
\vdots & & & & & & \vdots \\
A_{i1} & & & & & & A_{in} \\
\vdots & & & & & & \vdots \\
A_{j1} & & & & & & A_{jn} \\
\vdots & & & & & & \vdots \\
A_{n1} & \cdots & \cdots & & \cdots & \cdots & A_{nn}
\mz\]
The only difference the \(\lambda\) makes is when we multiply the
\(i\)th row into a column of \(A\) (say the \(k\)th column). Instead
of just picking up \(1\times A_{ik}\), we get \(1\times
A_{ik}+\lambda\times A_{jk}\). In other words, the result
\(E_{ij}(\lambda)A\) is obtained from \(A\) by adding \(\lambda\)
times row \(j\) to row \(i\).\qedhere


\end{Proof}
\begin{Definition}[Elementary matrices II]\label{dfn:elematrix2}
We define the elementary matrix \(E_i(\lambda)\) to be the matrix
with \(1\)s on the diagonal and zeros elsewhere, except that the
\(ii\) entry is \(\lambda\). For example, if we're working with
\(4\)-by-\(4\) matrices then \[E_1(5)=\ma 5 & 0 & 0 & 0 \\ 0 & 1 & 0
& 0 \\ 0 & 0 & 1 & 0 \\ 0 & 0 & 0 & 1\mz, E_3(F)=\ma 1 & 0 & 0 & 0
\\ 0 & 1 & 0 & 0 \\ 0 & 0 & F & 0 \\ 0 & 0 & 0 & 1\mz.\]


\end{Definition}
\begin{Lemma}
If \(A\) and \(E_i(\lambda)\) are \(n\)-by-\(n\) matrices then
\(E_i(\lambda)A\) is the matrix obtained from \(A\) by the row
operation \(R_i\mapsto\lambda R_i\).
\end{Lemma}
\begin{Proof}
The only difference between multiplying by the identity and
multiplying by \(E_i(\lambda)\) is that when you multiply the
\(i\)th row of \(E_i(\lambda)\) into the \(j\)th column of \(A\),
you pick up a factor of \(\lambda\). Therefore \(E_i(\lambda)A\)
differs from \(A\) only in that every element on the \(i\)th row is
multiplied by \(\lambda\). For example, if \(A=\ma a & b \\ c &
d\mz\) then \[E_1(\lambda)A=\ma \lambda & 0 \\ 0 & 1 \mz\ma a & b
\\ c & d\mz=\ma \lambda a & \lambda b \\ c & d\mz.\qedhere\]


\end{Proof}
\begin{Lemma}
An elementary matrix \(E_{ij}(\lambda)\) is invertible with inverse
\(E_{ij}(-\lambda)\). An elementary matrix \(E_i(\lambda)\) is
invertible if \(\lambda\neq 0\), in which case its inverse is
\(E_i(1/\lambda)\).
\end{Lemma}
\begin{Proof}
Consider the product
\[\ma
1 & & & & & & \\
& \ddots & & & & & \\
& & 1 & & \lambda & &\\
& & & \ddots & & & \\
& & & & 1 & & \\
& & & & & \ddots &\\
& & & & & & 1
\mz
\ma
1 & & & & & & \\
& \ddots & & & & & \\
& & 1 & & -\lambda & &\\
& & & \ddots & & & \\
& & & & 1 & & \\
& & & & & \ddots &\\
& & & & & & 1
\mz
\]
The only difference between this and \(II=I\) is when you multiply
row \(i\) into column \(j\), when you get
\(1\times(-\lambda)+\lambda\times 1=0\). Therefore this product
equals \(I\).


Consider the product
\[\ma
1 & & & & \\
& \ddots & & &\\
& & \lambda & &\\
& & & \ddots & \\
& & & & 1
\mz\ma
1 & & & & \\
& \ddots & & &\\
& & 1/\lambda & &\\
& & & \ddots & \\
& & & & 1
\mz\]
The only difference between this product and \(II=I\) is when you
multiply row \(i\) into column \(i\), at which point you get
\(\lambda\times(1/\lambda)=1\), so this product equals \(I\).
\qedhere


\end{Proof}
\subsection{Proof of \cref{thm:invech}}


\begin{Proof}[Proof of \cref{thm:invech}]\label{prf:invech}
Suppose we have put \(A\) into reduced echelon form using a sequence
of row operations \(r_1,\ldots,r_k\). Each row operation is
equivalent to multiplying (on the left) by some elementary matrix
\(M_1,\ldots,M_k\). Therefore the reduced echelon form of \(A\) is
\[C:=M_kM_{k-1}\cdots M_1A.\]


If \(C\) is the identity then \(M_k\cdots M_1A=I\), so \(M_k\cdots
M_1=A^{-1}\). If we perform the same row operations to the identity
matrix (sitting on the right hand side of the augmented matrix
\((A|I_n)\)) then we get \(M_k\cdots M_1I=A^{-1}\).


If \(C\) is not the identity matrix, then, since \(C\) is a square
matrix in reduced echelon form, there must be a row of \(C\) which
vanishes. Say this is the \(i\)th row. If \(v\) is the vector with
zeros everywhere except a \(1\) in the \(i\)th row then
\(Cv=0\). Now \(Cv=M_k\cdots M_1Av=0\), and \(M_1,\ldots,M_k\) are
invertible, so \(Av=M_1^{-1}\cdots M_k^{-1}0=0\). Therefore \(A\)
has nontrivial kernel. If \(A\) were invertible then the only
solution to \(Av=0\) is \(v=A^{-1}0=0\), so the kernel would be
trivial. Therefore \(A\) is only invertible if its reduced echelon
form is the identity matrix. \qedhere


\end{Proof}
\begin{Corollary}\label{cor:invprodelem}
A product of elementary matrices is invertible and, conversely, any
invertible matrix is a product of elementary matrices.
\end{Corollary}
\begin{Proof}
Each elementary matrix is invertible, so in for a product
\(M_k\cdots M_1\) of elementary matrices, the inverse is
\(M_1^{-1}\cdots M_k^{-1}\). Conversely, if \(A\) is invertible then
its reduced echelon form is the identity and its inverse is a
product of elementary matrices \(M_k\cdots M_1\) by
\cref{thm:invech}. The inverse of an elementary matrix is again
elementary, therefore \(A=M_1^{-1}\cdots M_k^{-1}\) is a product of
elementary matrices. \qedhere


\end{Proof}
\clearpage
\section{Determinants}
\subsection{Definition and basic examples}


We have seen that a 2-by-2 matrix \(A=\ma a & b \\ c & d \mz\) is
invertible if and only if \(ad-bc\neq 0\). We would like a similarly
nice characterisation of invertibility for \(n\)-by-\(n\) matrices.
We will see that {\bf a matrix is invertible if and only if its {\em
determinant} is nonzero}.


\begin{Definition}[Determinant]\label{dfn:det}
If \(A\) is an \(n\)-by-\(n\) matrix with entries \(A_{ij}\) then we
define the {\em determinant} \(\det(A)\) to be the number obtained
as follows.
\begin{itemize}
\item Pick \(n\) entries of \(A\) with no two in the same row and no two
in the same column. If we write the entry from the \(i\)th row as
\(A_{i\sigma(i)}\) (i.e. it's in the \(\sigma(i)\)th column) then
this means that the map \(i\mapsto\sigma(i)\) is a permutation of
\(\{1,\ldots,n\}\); there are \(n!\) ways of making such a choice.
\item Multiply these entries together to get the number \(\pm
A_{1\sigma(1)}\cdots A_{n\sigma(n)}\). The sign in this expression
is taken to be \(-1\) if your permutation is ``odd'' (i.e. if it
involves an odd number of swaps) and \(+1\) if your permutation is
``even'' (involves an even number of swaps). We will write
\(sgn(\sigma)\) for this sign.
\item Repeat this for every possible choice \(\sigma\) and sum the
numbers that you get.
\end{itemize}
In brief: \[\det(A)=\sum_{\sigma}sgn(\sigma)A_{1\sigma(1)}\cdots
A_{n\sigma(n)},\] where the sum is taken over all permutations
\(\sigma\).


\end{Definition}
\begin{Example}\label{exm:det2}
If \(n=2\) then there are \(n!=2\) choices:
\begin{itemize}
\item \(\sigma\) could be the identity permutation \(1\mapsto 1\),
\(2\mapsto 2\). This is an even permutation (it involves zero
swaps and zero is even) so we get
\(A_{1\sigma(1)}A_{2\sigma(2)}=A_{11}A_{22}\).
\item \(\sigma\) could be the swap \(1\leftrightarrow 2\). This is an
odd permutation (it involves one swap and one is odd) so we get
\(-A_{1\sigma(1)}A_{2\sigma(2)}=-A_{12}A_{21}\).
\end{itemize}
If \(A=\ma a & b \\ c & d\mz\) then this translates into the two
terms \(ad\) and \(-bc\), which we sum to get \(\det(A)=ad-bc\).


\end{Example}
\begin{Example}\label{exm:det3}
If \(n=3\) and \(A=\ma a & b & c \\ d & e & f \\ g & h & i \mz\)
then we get \(n!=6\) choices:


\begin{tabular}{c|ccccccc}
\hline
\(\sigma\) & identity & \(1\leftrightarrow 2\) & \(1\leftrightarrow 3\) & \(2\leftrightarrow 3\) & cyclic \((123)\) & cyclic \((132)\) \\
\hline
contribution & \(aef\) & \(-bdi\) & \(-ceg\) & \(-afh\) & \(bfg\) & \(cdh\)\\
\hline
\end{tabular}


so
\[\det(A)=aei+bfg+cdh-bdi-ceg-afh.\]


\end{Example}
\begin{Example}[Diagonal matrices]\label{exm:diagonal}
If \(D\) is a diagonal matrix with entries
\(\lambda_1,\ldots,\lambda_n\):
\[\ma \lambda_1 & 0 & \cdots & 0\\ 0 & \lambda_2 & & \vdots
\\ \vdots & & \ddots & 0 \\ 0 & \cdots & 0 &\lambda_n\mz\]
then there is only one way to pick a
nonzero entry from each row, which gives
\(\det(D)=\lambda_1\cdots\lambda_n\).


\end{Example}
\begin{Example}[Upper triangular matrices]\label{exm:uppertriangular}
Suppose that \(T\) is an {\em upper triangular matrix}, in other
words all the entries below the diagonal are zero: \[\ma A_{11} &
A_{12} & \cdots & A_{1n}\\ 0 & A_{22} & & A_{2n} \\ \vdots & &
\ddots & \vdots \\ 0 & \cdots & 0 & A_{nn}\mz\] Then we need to pick
something from the first column, which has to be \(A_{11}\), then
something from the second column but this may not be on the first
row as we already picked something from the first row, so this must
be \(A_{22}\), then something from the third column, but this cannot
be on the first or second rows, so it must be \(A_{33}\), and so on,
so we see that \(\det(A)=A_{11}\cdots A_{nn}\) (i.e. \(\det(A)\) is
the product of the diagonal entries). Similarly if \(A\) is
lower-triangular.


\end{Example}
\begin{Example}\label{exm:detelemI}
If \(E_{ij}(\lambda)\) is an elementary matrix with ones on the
diagonal and zeros elsewhere except for a \(\lambda\) in position
\(ij\) then \(\det(E_{ij}(\lambda))=1\). This is because
\(E_{ij}(\lambda)\) is upper (respectively lower) triangular (when
\(i<j\) or \(i>j\) respectively).


\end{Example}
\begin{Example}\label{exm:detelemII}
If \(E_i(\lambda)\) is the elementary matrix with ones on the
diagonal and zeros elsewhere except for a \(\lambda\) in position
\(ii\), then \(E_i(\lambda)\) is diagonal and its determinant is
\(\lambda\).


\end{Example}
\subsection{Some properties of the determinant}


\begin{Lemma}\label{lma:detantisym}
If two rows of \(A\) coincide (that is, for some \(i\neq j\), we
have \(A_{ik}=A_{jk}\) for all \(k\)) then \(\det(A)=0\).
\end{Lemma}
\begin{Proof}
If two rows coincide then each term
\[sgn(\sigma)(\cdots)A_{i\sigma(i)}(\cdots)A_{j\sigma(j)}(\cdots)\]
cancels with the term
\[sgn(\sigma')(\cdots)A_{i\sigma'(i)}(\cdots)A_{j\sigma'(j)}(\cdots)\]
where \(\sigma'\) is the permutation obtained by performing
\(\sigma\) and then switching \(i\leftrightarrow j\). The point is
that this doesn't change the value of the product (because
\(A_{i\sigma(i)}=A_{j\sigma(i)}\) and
\(A_{i\sigma(j)}=A_{j\sigma(j)}\)) but it does change the sign of
the permutation (it introduces an extra swap). \qedhere


\end{Proof}
\begin{Lemma}\label{lma:detrowI}
If \(A'\) is obtained from \(A\) by a row operation \(R_i\mapsto
R_i+\lambda R_j\) then \(\det(A')=\det(A)\).
\end{Lemma}
\begin{Proof}
We have
\begin{align*}
\det(A')&=\sum_{\sigma}sgn(\sigma)(\cdots)(A_{i\sigma(i)}+\lambda A_{j\sigma(i)})(\cdots)\\
&=\sum_{\sigma}sgn(\sigma)(\cdots)A_{i\sigma(i)}(\cdots)+\lambda\sum_{\sigma}sgn(\sigma)(\cdots)A_{j\sigma(i)}(\cdots)\\
&=\det(A)+\lambda\det(B),
\end{align*}
where \(B\) is the matrix obtained from \(A\) by replacing the
\(i\)th row with the \(j\)th row. Since \(B\) has two rows equal,
its determinant vanishes, so \(\det(A')=\det(A)\). \qedhere


\end{Proof}
\begin{Theorem}\label{thm:detech}
Suppose we put \(A\) into echelon form using only row operations
\(R_i\mapsto R_i+\lambda R_j\). Then \(\det(A)\) is the product of
the diagonal entries in the echelon form.
\end{Theorem}
\begin{Proof}
These row operations do not change the determinant, so if \(C\) is
the echelon form of \(A\) thus obtained, we have
\(\det(A)=\det(C)\). By definition, matrices in echelon form are
upper triangular, so \(\det(C)\) is just the product of its diagonal
entries, by \cref{exm:uppertriangular}. \qedhere


\end{Proof}
\begin{Lemma}
If \(A'\) is obtained by swapping two of the rows of \(A\) then
\(\det(A')=-\det(A)\).
\end{Lemma}
\begin{Proof}
Each term (for a permutation \(\sigma\)) in \(\det(A')\) also
appears in \(\det(A)\) for a permutation \(\sigma\) followed by the
swap, and hence with the opposite sign. \qedhere


\end{Proof}
\begin{Remark}
This means you can also swap rows around to reach echelon form and
compute the determinant, {\em provided you multiply by \(-1\) each
time you swap two rows}. This can be useful, for example: \[\det\ma
0 & 0 & 0 & 1 \\ 0 & 1 & 0 & 0 \\ 0 & 0 & 1 & 0 \\ 1 & 0 & 0 &
0\mz=-1\] immediately without the mess of adding row 4 to row 1 and
subtracting row 1 from row 4.


\end{Remark}
\clearpage
\section{Computing determinants}
\subsection{Examples of computing determinants}




\begin{Example}
Let \(A=\begin{pmatrix} 1 & 4 & -4 \\ -2 & -2 & -4 \\ 3 & -3 & 3
\end{pmatrix}\).


Clear row \(2\) using \(R_2\mapsto R_2+(2)R_1\)\[\begin{pmatrix} 1 &
4 & -4 \\ 0 & 6 & -12 \\ 3 & -3 & 3 \end{pmatrix}\] Clear row \(3\)
using \(R_3\mapsto R_3+(-3)R_1\)\[\begin{pmatrix} 1 & 4 & -4 \\ 0 &
6 & -12 \\ 0 & -15 & 15 \end{pmatrix}\] Clear row \(3\) using
\(R_3\mapsto R_3+(5/2)R_2\)\[\begin{pmatrix} 1 & 4 & -4 \\ 0 & 6 &
-12 \\ 0 & 0 & -15 \end{pmatrix}\] This is now in echelon form and
has the same determinant as the matrix we began with, so the
determinant is the product of the diagonal entries, which is
\(-90\).


\end{Example}
\begin{Example}
Let \(B=\begin{pmatrix} 2 & -3 & -1 & 4 \\ 2 & -3 & 2 & 4 \\ 2 & -1
& -4 & -3 \\ 2 & -3 & 4 & 2 \end{pmatrix}\).


Clear row \(2\) using \(R_2\mapsto R_2+(-1)R_1\)\[\begin{pmatrix} 2
& -3 & -1 & 4 \\ 0 & 0 & 3 & 0 \\ 2 & -1 & -4 & -3 \\ 2 & -3 & 4 & 2
\end{pmatrix}\] Clear row \(3\) using \(R_3\mapsto
R_3+(-1)R_1\)\[\begin{pmatrix} 2 & -3 & -1 & 4 \\ 0 & 0 & 3 & 0 \\ 0
& 2 & -3 & -7 \\ 2 & -3 & 4 & 2 \end{pmatrix}\] Clear row \(4\) using
\(R_4\mapsto R_4+(-1)R_1\)\[\begin{pmatrix} 2 & -3 & -1 & 4 \\ 0 & 0
& 3 & 0 \\ 0 & 2 & -3 & -7 \\ 0 & 0 & 5 & -2 \end{pmatrix}\]Add row
\(3\) to row \(2\)\[\begin{pmatrix} 2 & -3 & -1 & 4 \\ 0 & 2 & 0 &
-7 \\ 0 & 2 & -3 & -7 \\ 0 & 0 & 5 & -2 \end{pmatrix}\] Clear row
\(3\) using \(R_3\mapsto R_3+(-1)R_2\)\[\begin{pmatrix} 2 & -3 & -1
& 4 \\ 0 & 2 & 0 & -7 \\ 0 & 0 & -3 & 0 \\ 0 & 0 & 5 & -2
\end{pmatrix}\] Clear row \(4\) using \(R_4\mapsto
R_4+(5/3)R_3\)\[\begin{pmatrix} 2 & -3 & -1 & 4 \\ 0 & 2 & 0 & -7
\\ 0 & 0 & -3 & 0 \\ 0 & 0 & 0 & -2 \end{pmatrix}\]This is now in
echelon form and has the same determinant as the matrix we began
with, so the determinant is the product of the diagonal entries,
which is 24.


\end{Example}
\begin{Example}
Let \(C=\begin{pmatrix} 3 & -3 & -5 & -4 \\ 2 & -5 & 2 & 0 \\ 2 & 3
& -5 & -2 \\ 0 & 3 & -1 & 0 \end{pmatrix}\).


Clear row \(2\) using \(R_2\mapsto R_2+(-2/3)R_1\)\[\begin{pmatrix}
3 & -3 & -5 & -4 \\ 0 & -3 & 16/3 & 8/3 \\ 2 & 3 & -5 & -2 \\ 0 & 3
& -1 & 0 \end{pmatrix}\] Clear row \(3\) using \(R_3\mapsto
R_3+(-2/3)R_1\)\[\begin{pmatrix} 3 & -3 & -5 & -4 \\ 0 & -3 & 16/3 &
8/3 \\ 0 & 5 & -5/3 & 2/3 \\ 0 & 3 & -1 & 0 \end{pmatrix}\] Clear
row \(3\) using \(R_3\mapsto R_3+(5/3)R_2\)\[\begin{pmatrix} 3 & -3
& -5 & -4 \\ 0 & -3 & 16/3 & 8/3 \\ 0 & 0 & 65/9 & 46/9 \\ 0 & 3 &
-1 & 0 \end{pmatrix}\] Clear row \(4\) using \(R_4\mapsto
R_4+(1)R_2\)\[\begin{pmatrix} 3 & -3 & -5 & -4 \\ 0 & -3 & 16/3 &
8/3 \\ 0 & 0 & 65/9 & 46/9 \\ 0 & 0 & 13/3 & 8/3 \end{pmatrix}\]
Clear row \(4\) using \(R_4\mapsto R_4+(-3/5)R_3\)\[\begin{pmatrix}
3 & -3 & -5 & -4 \\ 0 & -3 & 16/3 & 8/3 \\ 0 & 0 & 65/9 & 46/9 \\ 0
& 0 & 0 & -2/5 \end{pmatrix}\]This is now in echelon form and has
the same determinant as the matrix we began with, so the determinant
is the product of the diagonal entries, which is \(26\).


\end{Example}
\clearpage
\section{Formulas for determinants and for inverses}
\subsection{Inductive formula for determinants}


We can expand the determinant as follows. First make your choice of
entry from the first row, say in the \(j\)th column. Now remove the
first row and the \(j\)th column. You're left with a smaller square
matrix, which we'll call \(C_{1j}\), from which you have to select the
remaining entries. The picture below shows how to extract \(C_{12}\)
from a \(4\)-by-\(4\) matrix.


\tka[decoration=snake]
\node at (0,0) {\(\ma A_{11} & A_{12} & A_{13} & A_{14} \\ A_{21} & A_{22} & A_{23} & A_{24} \\ A_{31} & A_{32} & A_{33} & A_{34} \\ A_{41} & A_{42} & A_{43} & A_{44} \mz\)};
\draw[thick] (-1.8,0.65) -- (1.8,0.65);
\draw[thick] (-0.48,0.8) -- (-0.48,-0.8);
\draw[red] (-1.7,-0.85) -- (-1,-0.85) -- (-1,0.45) -- (-1.7,0.45) -- cycle;
\draw[red] (0.2,-0.85) -- (1.7,-0.85) -- (1.7,0.45) -- (0.2,0.45) -- cycle;
\draw[->,thick,decorate] (2,0) -- (2.9,0);
\node at (5,0) {\(C_{12}=\ma A_{21} & A_{23} & A_{24} \\ A_{31} & A_{33} & A_{34} \\ A_{41} & A_{43} & A_{44}\mz\)};
\draw[red] (4.3,-0.6) -- (6.7,-0.6) -- (6.7,0.6) -- (4.3,0.6) -- cycle;
\tkz


As you run over these choices, you obtain the
determinant of this submatrix \(C_{1j}\). Now allow the choice of
\(j\) to vary, and you obtain the following useful inductive formula
for the determinant:
\[\det(A)=A_{11}\det(C_{11})-A_{12}\det(C_{12})+A_{13}\det(C_{13})+\cdots+(-1)^nA_{1n}\det(C_{1n}).\]
In fact, we could have started from any row (say the \(i\)th) and
obtained a similar expression
\[\det(A)=(-1)^{i+1}\left(A_{i1}\det(C_{i1})-A_{i2}\det(C_{i2})+A_{i3}\det(C_{i3})+\cdots+(-1)^nA_{in}\det(C_{in})\right),\]
where \(C_{ij}\) is the submatrix obtained by deleting the \(i\)th row
and the \(j\)th column.


In fact, we could have expanded by going down the \(j\)th column
instead:
\[\det(A)=(-1)^{j+1}\left(A_{1j}\det(C_{1j})-A_{2j}\det(C_{2j})+\cdots+(-1)^nA_{nj}\det(C_{nj})\right)\]
The only non-obvious thing about these formulas is how to get the
signs. The contribution to \(A_{ij}\det(C_{ij})\) to one of these
formulas is the sign \((-1)^{i+j}\) in the \(ij\) position of the grid
below: \[\ma + & - & + & \cdots \\ - & + & - & \cdots \\+ & - & + &
\cdots \\ \vdots & \vdots & \vdots & \mz.\] You can prove this using
index notation if you start from our formula for the determinant, but
rather than go through this, we will simply use the formula to compute
some determinants.


\begin{Remark}
The determinants of submatrices are called {\em
minors}. Historically, the mathematician Sylvester introduced the
word ``matrix'' (the Latin word for {\em womb}) because...


\begin{quotation}
I have in previous papers defined a ``Matrix'' as a rectangular array of
terms, out of which different systems of determinants may be
engendered as from the womb of a common parent.
\end{quotation}


Let it never be said that mathematicians don't have vivid
imaginations.


\end{Remark}
\begin{Example}
Let's calculate the determinant of \[A=\ma 1 & 2 & 3 \\ 4 & 5 & 6
\\ 7 & 8 & 9\mz\] using this inductive formula. We have
\begin{align*}
\det(A)&=\det\ma 5 & 6 \\ 8 & 9\mz-2\det\ma 4 & 6 \\ 7 &
9\mz+3\det\ma 4 & 5 \\ 7 & 8 \mz\\
&=(5\times 9-6\times 8)-2(4\times 9-6\times 7)+3(4\times 8-5\times 7)\\
&=(45-48)-2(36-42)+3(32-35)\\
&=-3+12-9\\
&=0.
\end{align*}


\end{Example}
\begin{Example}
Let's calculate the determinant of \[B=\ma 1 & 1 & 2 & 3 \\ 0 & 0 &
4 & 5 \\ -1 & 2 & 1 & 1 \\ 0 & 0 & 2 & 3\mz\] using the inductive
formula. Note that every entry on the first row is nonzero, so
expanding along the first row would involve calculating four 3-by-3
minors. If, instead, we expand along the second {\em column} then we
only have two nonzero entries, so only need to compute two 3-by-3
minors (first column, second row or fourth row would also have this
advantage; I picked the second column because it makes the signs
more interesting). This gives
\begin{align*}
\det(B)&=-\det\ma 0 & 4 & 5 \\ -1 & 1 & 1\\ 0 & 2 & 3\mz - 2\det\ma 1 & 2 & 3 \\ 0 & 4 & 5 \\ 0 & 2 & 3 \mz\\
&=-\left(-\left(-\det\ma 4 & 5 \\ 2 & 3\mz\right)\right)-2\det\ma 4 & 5 \\ 2 & 3\mz\\
&=-3(4\times 3-5\times 2)\\
&=-6.
\end{align*}
Let's check we did it right using row operations. Add row 1 to row
3: \[\ma 1 & 1 & 2 & 3 \\ 0 & 0 & 4 & 5 \\ 0 & 3 & 3 & 4 \\ 0 & 0 &
2 & 3\mz\] Switch rows 2 and 3 (picking up a minus sign in the
determinant) \[\ma 1 & 1 & 2 & 3 \\ 0 & 3 & 3 & 4 \\ 0 & 0 & 4 & 5
\\ 0 & 0 & 2 & 3\mz\] Subtract twice row 4 from row 3, then switch
them (another sign, which cancels the previous one). \[\ma 1 & 1 &
2 & 3 \\ 0 & 3 & 3 & 4 \\ 0 & 0 & 2 & 3\\ 0 & 0 & 0 & -1 \mz\] The
determinant is the product of the diagonal entries, which is indeed
\(-6\).




\end{Example}
\subsection{Inverses in terms of determinants}


\begin{Definition}
Define the {\em adjugate matrix} of \(A\) to be the matrix
\[\adj(A):=\ma +\det(C_{11}) & -\det(C_{12}) & +\det(C_{13}) & \cdots
\\ \det(C_{21}) & +\det(C_{22}) & -\det(C_{23}) & \cdots
\\+\det(C_{31}) & -\det(C_{32}) & +\det(C_{33}) & \cdots \\ \vdots &
\vdots & \vdots & \mz^T.\]


\end{Definition}
\begin{Theorem}
If \(\det(A)\neq 0\) then \(A^{-1}=\frac{1}{\det(A)}\adj(A)\).
\end{Theorem}
\begin{Proof}
We can compute \(A \adj(A)\). The \(ij\)th entry is precisely the
expression \[\pm(A_{i1}\det(C_{j1})-A_{i2}\det(C_{j2})+\cdots\pm
A_{in}\det(C_{jn}))\] which equals \(\det(A)\) if \(i=j\). If
\(i\neq j\) then this expression is the determinant of the matrix
obtained from \(A\) by replacing the \(j\)th row with the \(i\)th
row, so two rows coincide and the determinant vanishes. Therefore
\(A \adj(A)=\ma \det(A) & 0 & \cdots & 0 \\ 0 & \det(A) & &
0\\ \vdots & & \ddots & 0\\ 0 & \cdots & 0 & \det(A)\mz\), and so
\(\frac{1}{\det(A)}\adj A\) is an inverse for \(A\). \qedhere
\end{Proof}
\clearpage
\section{More on determinants}
\subsection{Further properties of determinants}


\begin{Lemma}\label{lma:detrowII}
If \(A'\) is obtained from \(A\) by a row operation of the form
\(R_i\mapsto \lambda R_i\) then \(\det(A')=\lambda\det(A)\).
\end{Lemma}
\begin{Proof}
\begin{align*}
\det(A')&=\sum_{\sigma}sgn(\sigma)A_{1\sigma(1)}\cdots(\lambda A_{i\sigma(i)})\cdots A_{n\sigma(n)}\\
&=\lambda\sum_{\sigma}sgn(\sigma)A_{1\sigma(1)}\cdots A_{i\sigma(i)}\cdots A_{n\sigma(n)}\\
&=\lambda\det(A). \qedhere
\end{align*}


\end{Proof}
\begin{Theorem}\label{thm:detinv}
An \(n\)-by-\(n\) matrix \(A\) is invertible if and only if its
determinant is nonzero.
\end{Theorem}
\begin{Proof}
Put \(A\) into echelon form using only row operations of type
\(R_i\mapsto R_i+\lambda R_j\). You don't change the
determinant. Now use row operations of type \(R_i\mapsto\lambda
R_i\) (\(\lambda\neq 0\)) to put \(A\) into reduced echelon
form. You change the determinant by a nonzero factor (the product of
all the \(\lambda\)s that you used in the row operations). By
\cref{thm:invech}, a matrix is invertible if and only if its reduced
echelon form is the identity matrix, which has determinant \(1\),
so:
\begin{itemize}
\item if \(A\) is invertible then its determinant differs from \(1\) by
a nonzero factor, and
\item if \(A\) is not invertible then its reduced echelon form has a
zero row somewhere, so the reduced echelon form has determinant
zero and \(\det(A)\) is a multiple of zero, hence zero. \qedhere


\end{itemize}
\end{Proof}
\begin{Theorem}\label{thm:detmult}
If \(A\) and \(B\) are \(n\)-by-\(n\) matrices then
\[\det(AB)=\det(A)\det(B).\]
\end{Theorem}
\begin{Proof}
First, we show this under the assumption that \(A\) is an elementary
matrix.
\begin{itemize}
\item If \(A=E_{ij}(\lambda)\) then \(AB\) is the result of the row
operation \(R_i\mapsto R_i+\lambda R_j\) on \(B\), so
\(\det(AB)=\det(B)\) by \cref{lma:detrowI}. Moreover,
\(\det(A)=1\) by \cref{exm:detelemI}. Therefore
\(\det(A)\det(B)=\det(B)\) too, so the theorem is proved in this
case.
\item If \(A=E_i(\lambda)\) then \(AB\) is the result of the row
operation \(R_i\mapsto \lambda R_i\) on \(B\), so
\(\det(AB)=\lambda\det(B)\) by \cref{lma:detrowI}. Moreover,
\(\det(A)=\lambda\) by \cref{exm:detelemII}. Therefore
\(\det(A)\det(B)=\lambda\det(B)\) too, so the theorem is proved in
this case.
\end{itemize}
Now, if we assume that \(A\) is a product of elementary matrices
then the theorem follows from these two special cases by induction.


If \(A\) is not a product of elementary matrices then \(A\) is not
invertible, so its determinant is zero by
\cref{thm:detinv}. Moreover, \(AB\) is also noninvertible because
\(A\) is not invertible, so \(\det(AB)=0\) by \cref{thm:detinv}, so
\(\det(AB)=0=\det(A)\det(B)\), and the theorem is proved in this
case too. \qedhere


\end{Proof}
\subsection{Geometric interpretation of determinants}


\begin{Theorem}
Suppose that \(A=\ma a & b \\ c & d \mz\). Let \(S\) be the unit
square sitting in the plane and let \(A(S)\) denote the image of
\(S\) under the linear map defined by \(A\). Then \(|\det(A)|\) is
the area of \(A(S)\).
\end{Theorem}
\begin{Proof}
The shape \(A(S)\) is a parallelogram with sides parallel to the
vectors \(\ma a \\ c\mz\) and \(\ma b \\ d\mz\). This parallelogram
has area \(ad-bc\) as we can see by dissection, using the following
picture:


\tka
\begin{scope}[shift={(-4,0)}]
\filldraw[gray,opacity=0.7] (0,0) -- (1,2) -- (3,3) -- (2,1) -- cycle;
\filldraw[red,opacity=0.5] (0,0) -- (2,0) -- (2,2) -- (0,2) -- cycle;
\end{scope}
\filldraw[red] (0,0) -- (1,2) -- (2,2) -- (2,1) -- cycle;
\filldraw[blue] (0,0) -- (1,2) -- (0,1) -- cycle;
\filldraw[green] (0,0) -- (1,0) -- (2,1) -- cycle;
\filldraw[purple] (0,1) -- (0,2) -- (1,2) -- cycle;
\filldraw[purple] (1,0) -- (2,0) -- (2,1) -- cycle;
\filldraw[green] (1,2) -- (3,3) -- (2,2) -- cycle;
\filldraw[blue] (2,1) -- (3,3) -- (2,2) -- cycle;
\tkz


Take the grey parallelogram \(A(S)\), draw the rectangle with
sidelengths \(a\) (along) and \(d\) (up) over it. Move the green
and blue pieces of the parallelogram inside the rectangle as
shown. Now the red, green and blue areas inside the square have the
same area as \(A(S)\). The remaining (purple) part comprises two
triangles which have height \(c\) and base \(b\), so the area of
\(A(S)\) is \(ad-bc\). \qedhere


\end{Proof}
It is much harder to see the following theorems, but they are true:


\begin{Theorem}
If \(A\) is an \(n\)-by-\(n\) matrix then \(|\det(A)|\) is the
volume of \(A(S)\), where \(S\) is the unit cube in
\(n\)-dimensions.


\end{Theorem}
\begin{Remark}
The shape \(A(S)\) is called a {\em parallelopiped}, the
higher-dimensional analogue of a parallelogram.


\end{Remark}
\begin{Theorem}
Let \(a_1,\ldots,a_n\) be \(n\) vectors in \(\RR^n\). Consider the
{\em simplex} with vertices at the origin and at
\(a_1,\ldots,a_n\). The volume of this simplex is
\(\frac{1}{n!}|\det(A)|\), where \(A\) is the matrix with columns
\(a_1,\ldots,a_n\).


\end{Theorem}
\begin{Example}
If \(A=\ma 1 & 0 & 0 \\ 0 & 1 & 0 \\ 0 & 0 & 1\mz\) then the simplex
we get is the tetrahedron shown below. Its volume is \(1/6\)
(because of the above formula, but also because you can dissect and
rearrange a cube into six such tetrahedra).


\tka
\filldraw[fill=lightgray,draw=black,thick,opacity=0.5] (1.5,-0.5) -- (1.8,0.3) -- (0,0) -- cycle;
\filldraw[fill=lightgray,draw=black,thick,opacity=0.5] (1.8,0.3) -- (0,0) -- (0,2) -- cycle;
\filldraw[fill=lightgray,draw=black,thick,opacity=0.5] (1.5,-0.5) -- (0,0) -- (0,2) -- cycle;
\filldraw[fill=lightgray,draw=black,thick,opacity=0.5] (1.5,-0.5) -- (1.8,0.3) -- (0,2) -- cycle;
\tkz


\end{Example}
\begin{Example}
The regular tetrahedron (or d4, for any Dungeons \& Dragons fans out
there) has vertices \[a_0=\ma 1/2 \\ 0 \\ 1/\sqrt{2}\mz,\quad
a_1=\ma -1/2 \\ 0 \\ -1/\sqrt{2}\mz,\quad a_2\ma 0 \\ 1/2
\\ -1/\sqrt{2}\mz,\quad a_3\ma 0 \\ -1/2 \\ 1/\sqrt{2}\mz.\] By
translating this so that the vertex \(a_0\) is at the origin, we get
the vertices \[a_1-a_0,\quad a_2-a_0,\quad a_3-a_0,\] so the volume
is \[\frac{1}{6}|\det(a_1-a_0,a_2-a_0,a_3-a_0)|,\] or
\[\frac{1}{6}\left|\det\ma -1 & -1/2 & -1/2 \\ 0 & 1/2 & -1/2 \\ 0 &
\sqrt{2} & \sqrt{2} \mz\right|=\frac{1}{6\sqrt{2}}.\]


\end{Example}
\begin{Remark}
From this geometric point of view, the fact that
\(\det(AB)=\det(A)\det(B)\) is obvious: \(\det(M)\) is the scaling
factor for volumes under the linear map \(M\), so under the
composite \(AB\) we first scale by \(\det(B)\) and then by
\(\det(A)\), so \(\det(AB)=\det(A)\det(B)\). Unfortunately, we
haven't proved the theorems above which establish the connection
between determinants and scaling of volumes.


\end{Remark}
\clearpage
\section{Eigenvectors and eigenvalues}
\subsection{Definition and basic ideas}


If someone gives you a complicated matrix \(A\), it can be very
difficult to determine salient information about the underlying linear
map associated to \(A\). For example, \(A\) could be a very simple
transformation like a rotation, but happening around an axis that
points in some random direction, which makes the matrix very
complicated. For this reason, we want to look for geometric features
like fixed vectors of \(A\), just as we did for 2-by-2 matrices. It
turns out that the most fruitful thing to study is the following
notion.


\begin{Definition}[Eigenvectors, eigenvalues]\label{dfn:eigenvector}
Let \(A\) be a matrix. A vector \(v\) is called an {\em eigenvector}
for \(A\) with {\em eigenvalue} \(\lambda\) if \(v\neq 0\) and
\[Av=\lambda v.\]


\end{Definition}
For example, a fixed vector is an eigenvector with eigenvalue
\(1\). ``Eigen'' is a German prefix meaning ``self''. An eigenvector
is mapped by \(A\) back to itself rescaled by its eigenvalue.


\begin{Remark}
It is hard to overemphasise the importance of eigenvectors and
eigenvalues. We will see some fun applications in this course, but
you will encounter them again and again in courses throughout your
time as an undergraduate. They are one of the most important notions
in mathematics and science. For example, in quantum mechanics the
most important equation is the Schr\"{o}dinger equation, which is
the eigenvector equation\footnote{Caveat: \(H\) is more like an
infinite-by-infinite matrix, i.e. an operator on a Hilbert space,
which makes the Schr\"{o}dinger equation into a differential
equation.} \(H\psi=E\psi\). Here \(H\) is a linear map called the
Hamiltonian, \(\psi\) is a vector describing the state of the
quantum system, and \(E\) is the energy of the state \(\psi\). For
example, if \(H\) is the Hamiltonian for the hydrogen atom then the
eigenvalues of \(H\) are the possible energies of light that can be
absorbed/emitted by hydrogen (the {\em spectrum} of the hydrogen
atom). The fact that Schr\"{o}dinger's equation predicts the
hydrogen spectrum so well was an early confirmation that quantum
mechanics was on the right track.


\end{Remark}
\subsection{Finding the eigenvectors}


Suppose someone tells you that some matrix \(A\) (say \(\ma 2 & -1
\\ 1 & 0 \mz\)) has some eigenvectors with eigenvalue \(\lambda\) (say
\(1\)). It's now very easy to find all the eigenvectors with this
eigenvalue: you just need to solve the simultaneous equations
\(Av=\lambda v\), in our case
\[\ma 2 & -1 \\ 1 & 0 \mz\ma x \\ y \mz=\ma x \\ y\mz,\]
or
\begin{align*}
2x-y&=x\\
x&=y.
\end{align*}
These equations both reduce to \(y=x\), so the eigenvectors must be
\(\ma x \\ x \mz\).


\begin{Remark}
Note that if \(v\) is an eigenvector with eigenvalue \(\lambda\)
then any rescaling \(\mu v\) is also an eigenvector with eigenvalue
\(\lambda\) because \[A(\mu v)=\mu Av=\mu\lambda v\Rightarrow A(\mu
v)=\lambda(\mu v).\] Therefore you shouldn't be surprised that we
have found a one-parameter family of eigenvectors instead of just
one!


\end{Remark}
``By George!'' you might say, ``the fellow is right, there is indeed
an eigenvector with eigenvalue \(\lambda\)... but how did he know
which \(\lambda\) to tell me?'' For example, if we tried
\(\lambda=2\), we would fail:
\begin{align*}
2x-y&=2x\\
x&=2y
\end{align*}
have no solution other than \(x=y=0\), because the first implies
\(y=0\) and the second implies \(x=y=0\). Remember than eigenvectors
are required to be nonzero. Therefore there is no eigenvector of \(\ma
2 & -1 \\ 1 & 0\mz\) with eigenvalue \(2\).


\subsection{Finding the eigenvalues}


\begin{Theorem}[Characteristic polynomial]\label{thm:charpoly}
The eigenvalues of a matrix \(A\) are the roots of the {\em
characteristic polynomial} \(\chi_A(t)\) of \(A\). This is the
polynomial defined by \(\chi_A(t)=\det(A-tI)\).
\end{Theorem}
\begin{Proof}[Proof of \cref{thm:charpoly}]\label{prf:thm:charpoly}
If \(v\neq 0\) and \(Av=\lambda v\) then \((A-\lambda I)v=0\), so
\(A-\lambda I\) has nontrivial kernel and fails to be invertible. In
particular, \(\det(A-\lambda I)=0\). Conversely, if \(\det(A-\lambda
I)=0\) then \(A-\lambda I\) has nontrivial kernel, so there exists a
vector \(v\) such that \(Av-\lambda v=0\). \qedhere


\end{Proof}
\begin{Example}
For the matrix \(A=\ma 2 & -1 \\ 1 & 0 \mz\) above, we have
\begin{align*}
\chi_A(t)&=\det\ma 2 & -1 \\ 1 & 0 \mz-\ma t & 0 \\ 0 & t\mz\\
&=\det\ma 2-t & -1 \\ 1 & -t\mz\\
&=-t(2-t)+1\\
&=t^2-2t+1.
\end{align*}
This polynomial has \(1\) as a repeated root, so the only eigenvalue
is \(1\) and, as we saw above, the only eigenvector (up to scaling)
is \(\ma 1 \\ 1\mz\).


\end{Example}
\begin{Example}\label{exm:cat}
The matrix \(A=\ma 2 & 1 \\ 1 & 1 \mz\) has characteristic
polynomial \[\det(A-tI)=\det\ma 2-t & 1 \\ 1 &
1-t\mz=(2-t)(1-t)-1=t^2-3t+1,\] which has roots
\(\lambda_1=\frac{3+\sqrt{5}}{2}\) and
\(\lambda_2=\frac{3-\sqrt{5}}{2}\). As eigenvectors, we can take
\[v_1=\ma 1\\ \frac{1+\sqrt{5}}{2}\mz,\qquad v_2=\ma 1
\\ \frac{1-\sqrt{5}}{2}\mz.\]


\end{Example}
\begin{Example}
The matrix \(A=\ma \frac{3}{2} & \frac{5}{2} & 3 \\ -\frac{1}{2} &
-\frac{3}{2} & -3 \\ 1 & 1 & 2 \mz\) has characteristic polynomial
\begin{align*}
\det(A-tI)&=\det\ma \frac{3}{2}-t & \frac{5}{2} & 3 \\ -\frac{1}{2} &
-\frac{3}{2}-t & -3 \\ 1 & 1 & 2-t\mz\\
&=\left(\frac{3}{2}-t\right)\det\ma -\frac{3}{2}-t & -3 \\ 1 & 2-t\mz -\frac{5}{2}\det\ma -\frac{1}{2} & -3 \\ 1 & 2-t\mz\\
&\qquad\qquad +3\det\ma -\frac{1}{2} & -\frac{3}{2}-t \\ 1 & 1\mz\\
&=\left(\frac{3}{2}-t\right)\left(-\left(\frac{3}{2}+t\right)\left(2-t\right)+3\right)-\frac{5}{2}\left(-\frac{1}{2}(2-t)+3\right)\\
&\qquad\qquad+3\left(-\frac{1}{2}+\frac{3}{2}+t\right)\\
&=\left(\frac{3}{2}-t\right)(t^2-t/2)-\frac{5}{2}(t/2+2)+3t+3\\
&=-t^3+2t^2+t-2.
\end{align*}
What are the roots of this polynomial? With cubics, the easiest
method is to guess one of the roots (say \(\alpha\)), divide the
polynomial by \(t-\alpha\) (using polynomial long division) and then
solve the quadratic equation you get. Here, we can see that \(t=1\)
is a solution\footnote{You'd be surprised how often that happens in
carefully-constructed examples.} and dividing \(-t^3+2t^2+t-2\) by
\(t-1\) gives \(-t^2+t+2\), which has solutions \(\frac{-1\pm
\sqrt{3}}{-2}=-1,2\). Therefore the eigenvalues are \(-1,1,2\). The
corresponding eigenvectors are \(\ma -1 \\ 1 \\ 0\mz\), \(\ma 1
\\ -5 \\ 4 \mz\) and \(\ma 1 \\ -1 \\ 1 \mz\). For example, to get
the \(1\)-eigenvector, we solve \(v=Av\) \[\ma x \\ y \\ z\mz=\ma
\frac{3}{2} & \frac{5}{2} & 3 \\ -\frac{1}{2} & -\frac{3}{2} & -3
\\ 1 & 1 & 2 \mz\ma x \\ y \\ z\mz,\] that is
\begin{align*}
\frac{3x}{2}+\frac{5y}{2}+3z&=x\\
-\frac{1}{2}x-\frac{3}{2}y-3z&=y\\
x+y+2z&=z.
\end{align*}
These equations imply \(x+5y+6z=0\) and \(x+y+z=0\), so \(4y+5z=0\),
therefore if we pick \(y=-5\) we get \(z=4\) and \(x=-y-z=1\).


\end{Example}
\clearpage
\section{Applications of eigenvectors}
\subsection{Application I: Differential equations}


Let \(v(t)=\ma x_1(t)\\ \vdots \\ x_n(t)\mz\) be a vector-valued
function, let \(A\) be an \(n\)-by-\(n\) matrix, and consider the
system of differential equations
\begin{align*}
\dot{x}_1&=A_{11}x_1+\cdots+A_{1n}x_n\\
&\vdots\\
\dot{x}_n&=A_{n1}x_1+\cdots+A_{nn}x_n,
\end{align*}
or, more succinctly,
\[\dot{v}=Av.\]


\begin{Example}\label{exm:odeexm}
Consider the system of differential equations
\begin{align*}
\dot{x}&=2x+y\\
\dot{y}&=x+y.
\end{align*}
We can rewrite this as \[\frac{d}{dt}\ma x \\ y \mz=\ma 2 & 1\\ 1 &
1\mz\ma x \\ y \mz.\]


\end{Example}
Suppose that \(A\) has \(n\) eigenvalues
\(\lambda_1,\ldots,\lambda_n\) with eigenvectors
\(v_1,\ldots,v_n\). We can write \(v\) in terms of the basis of
eigenvectors: \[v=\sum_{i=1}^nf_iv_i\] for some collection of numbers
\(f_1,\ldots,f_n\). We have \[\dot{v}=\sum_{i=1}^n\dot{f}_iv_i\] and
\[Av=A\sum_{i=1}^nf_iv_i=\sum_{i=1}^nf_iAv_i=\sum_{i=1}^nf_i\lambda_iv_i.\]
Since \(\dot{v}=Av\), we can equate the coefficients of the vectors
\(v_1,\ldots,v_n\) in these two expressions. We get the much simpler
equation \[\dot{f}_i=\lambda_i f_i,\] with solution
\(f_i(t)=C_ie^{\lambda_i t}\) for some constant \(C_i\). The general
solution to the differential equation is therefore
\[v=\sum_{i=1}^nC_ie^{\lambda_it}v_i.\] Let's apply this to solve the
differential equations from \cref{exm:odeexm}


\begin{Example}
The matrix \(A=\ma 2 & 1 \\ 1 & 1 \mz\) has eigenvalues
\(\lambda_1=\frac{3+\sqrt{5}}{2}\),
\(\lambda_2=\frac{3-\sqrt{5}}{2}\) and eigenvectors \(v_1=\ma
1\\ \frac{1+\sqrt{5}}{2}\mz\) and \(v_2=\ma 1
\\ \frac{1-\sqrt{5}}{2}\mz\). Therefore, the general solution is
\[C_1e^{(3+\sqrt{5})t/2}\ma
1\\ \frac{1+\sqrt{5}}{2}\mz+C_2e^{(3-\sqrt{5})t/2}\ma 1
\\ \frac{1-\sqrt{5}}{2}\mz,\] or
\begin{align*}
x(t)&=C_1e^{(3+\sqrt{5})t/2}+C_2e^{(3-\sqrt{5})t/2},\\
y(t)&=\frac{1+\sqrt{5}}{2}C_1e^{(3+\sqrt{5})t/2}+\frac{1-\sqrt{5}}{2}C_2e^{(3-\sqrt{5})t/2}.
\end{align*}


\end{Example}
\begin{Example}
Consider the system of differential equations
\begin{align*}
\dot{x}&=2x+y\\
\dot{y}&=2y-x.
\end{align*}
We can rewrite this as \[\frac{d}{dt}\ma x \\ y \mz=\ma 2 & 1\\ -1 &
2\mz\ma x \\ y \mz.\] The matrix \(A=\ma 2 & 1 \\ -1 & 2 \mz\) has
characteristic polynomial \[\det(A-tI)=\det\ma 2-t & 1 \\ -1 &
2-t\mz=(2-t)(2-t)+1=t^2-4t+5,\] which has roots \(2\pm i\). The
eigenvectors for these eigenvalues are the solutions to \[\ma (2+i)x
\\ (2+i)y\mz=\ma 2x+y\\2y-x\mz,\] (i.e. \(\ma 1 \\ i\mz\)) and \[\ma
(2-i)x \\ (2-i)y\mz=\ma 2x+y\\2y-x\mz\] (i.e. \(\ma 1
\\ -i\mz\)). Therefore the general solution to the system of
differential equations in this example is \[C_1e^{(2+i)t}\ma 1 \\ i
\mz+C_2e^{(2-i)t}\ma 1 \\ -i\mz,\] or
\[x(t)=C_1e^{(2+i)t}+C_2e^{(2-i)t},\quad
y(t)=iC_1e^{(2+i)t}-iC_2e^{(2-i)t}.\] You should not worry about the
appearance of imaginary numbers here: if the initial condition you
pick is real then all the imaginary terms will group together to
give trigonometric functions, using the facts that
\[\cos(t)=\frac{e^{it}+e^{-it}}{2},\qquad\sin(t)=\frac{e^{it}-e^{-it}}{2i}.\]
For example, let's try and find the solution for the initial
condition \(x(0)=0\), \(y(0)=1\). This means \[C_1+C_2=0,\qquad
i(C_1-C_2)=1,\] that is, \(C_1=-C_2=-i/2\). Substituting these
values for \(C_1,C_2\) we get
\[x(t)=-\frac{i}{2}(e^{(2+i)t}-e^{(2-i)t})=e^{2t}\frac{e^{it}-e^{-it}}{2i}=e^{2t}\sin(t)\]
and \[y(t)=\frac{1}{2}(e^{(2+i)t}+e^{(2+it)})=e^{2t}\cos(t).\]


\end{Example}
Finally, we should investigate what happens when \(A\) has fewer
than \(n\) eigenvectors.


\begin{Example}
Suppose that \(A=\ma 1 & 1 \\ 0 & 1\mz\). The differential equations
we get out of \(A\) are
\begin{align*}
\dot{x}&=x+y\\
\dot{y}&=y.
\end{align*}
We can solve the second equation immediately and get
\(y=C_1e^{t}\). Substituting back into the first, we get
\[\dot{x}=x+C_1e^t.\] Rearranging gives
\[\dot{x}e^{-t}-xe^{-t}=C_1,\] and we note (using the Leibniz rule
for differentiation) that
\[\frac{d}{dt}(xe^{-t})=\dot{x}e^{-t}-xe^{-t},\] so
\[\frac{d}{dt}(xe^{-t})=C_1,\] which gives \[x=(C_1t+C_2)e^t.\]


\end{Example}
In a later course on linear algebra, you will see the {\em Jordan
normal form} theorem for matrices, which tells you that, as long as
you work over \(\CC\), viewed in suitable coordinates, your matrix
always looks like a bunch of blocks which look like this: \[\ma
\lambda & 1 & 0 & \cdots & 0 \\ 0 & \lambda & 1 & &\vdots \\ 0&
&\ddots &\ddots & 0\\ \vdots & & & \lambda & 1 \\ 0 & \cdots &0 &0 &
\lambda\mz.\] For such matrices, you can do something similar to the
previous example.


\subsection{Application II: Ellipsoids}


\begin{Definition}
We say that \(A\) is a {\em positive definite matrix} if \(v^TAv>0\)
for any vector \(v\neq 0\).


\end{Definition}
\begin{Example}
The identity matrix is positive definite because \(v^TIv=v\cdot
v\geq 0\) with equality if and only if \(v=0\).


\end{Example}
\begin{Example}
The matrix \(A=\ma 1 & 0 \\ 0 & -1\mz\) is not positive definite
because \(\ma 0 & 1 \mz A\ma 0\\ 1\mz=-1\).


\end{Example}
\begin{Definition}
An {\em ellipsoid} is a subset in \(n\)-dimensional space having the
form \[\{v\in\RR^n\ :\ v^TAv=c\},\] where \(A\) is a positive
definite symmetric matrix with real entries and \(c>0\) is a
positive real constant.


\end{Definition}
\begin{Example}
Given two numbers \(a,b\in\RR\), the matrix \[A=\ma\frac{1}{a^2} & 0
\\ 0 & \frac{1}{b^2}\mz\] is positive definite. If \(c=1\) then the
corresponding ellipsoid is the ellipse \[\left\{(x,y)\in\RR^2\ :\
\frac{x^2}{a^2}+\frac{y^2}{b^2}=1\right\}\] having semimajor axis
\(a\) and semiminor axis \(b\).


\tka
\draw (0,0) circle [x radius=2cm,y radius=1cm];
\draw[->] (0,0) -- (2,0) node [right] {\(a\)};
\draw[->] (0,0) -- (0,1) node [above] {\(b\)};
\tkz


\end{Example}
\begin{Theorem}\label{thm:ellipsoid}
An ellipsoid defined by a positive definite symmetric matrix \(A\)
can be rotated to the ellipsoid \[\{(u_1,\ldots,u_n)\in\RR^n\ :\
\sum_{i=1}^n\lambda_iu_i^2=c\}\] where
\(\lambda_1,\ldots,\lambda_n\) are the eigenvalues of \(A\) and
\(c\) is some positive number.


\end{Theorem}
We won't prove this theorem in full, because it relies on the fact
that a positive definite symmetric matrix has a basis of eigenvectors
(which is beyond what we have time for). But we'll at least check that
{\em if} \(A\) has a basis of eigenvectors then the result
holds. First, an important lemma.


\begin{Lemma}
Suppose that \(A\) is a symmetric matrix with real entries. Then the
eigenvalues of \(A\) are real and if \(\lambda,\mu\) are distinct
eigenvalues with eigenvectors \(v,w\) respectively then \(v\cdot
w=0\).
\end{Lemma}
\begin{Proof}
Suppose that \(Av=\lambda v\). Consider the expression
\(\bar{v}^TAv\), where \(\bar{v}\) denotes complex
conjugation. Then, because \(A=A^T=\bar{A}^T\), we have
\[\bar{\lambda}\bar{v}^Tv=(\overline{Av})^{T}v=\bar{v}^TAv=\lambda\bar{v}^Tv.\]
Note that if \(v=\ma x_1\\ \vdots\\ x_n\mz\) then
\(\bar{v}^Tv=\sum |x_1|^2+\cdots+|x_n|^2>0\) if \(v\neq 0\), so
dividing through by \(\bar{v}^Tv\) we get \(\bar{\lambda}=\lambda\)
and deduce that \(\lambda\) is real.


If \(v\) and \(w\) are two eigenvectors for distinct eigenvalues
\(\lambda,\mu\) then
\begin{align*}
\lambda w^Tv&=w^T(Av)\\
&=(Aw)^Tv\\
&=\mu w^Tv\\
\end{align*}
so, since \(\lambda\neq\mu\), we must have \(w^Tv=0\), i.e. \(v\cdot
w=0\). \qedhere


\end{Proof}
Now suppose that \(A\) is a real symmetric matrix which has a basis of
\(n\) eigenvectors \(v_1,\ldots,v_n\) with eigenvalues
\(\lambda_1,\ldots,\lambda_n\). By the lemma above, these eigenvalues
are all real and the eigenvectors are orthogonal. Let's rescale the
eigenvectors so that they each have unit length. If we write a vector
\(v\) as \(\sum_{i=1}^nu_iv_i\) then we have
\[v^TAv=\sum_{i=1}^n\sum_{j=1}^nu_ix_jv_i^TAv_j=\sum_{i,j=1}^n\lambda_i
u_i^2,\] since \(v_i^Tv_j=\delta_{ij}\). These \(u_1,\ldots,u_n\) are
the coordinates referred to in \cref{thm:ellipsoid}. In particular,
the {\em principal axes} (the higher-dimensional analogues of the
semi-major and semi-minor axes) are the eigenvectors of \(A\) and the
principal radii are \(\frac{1}{\sqrt{\lambda_i}}\), \(i=1,\ldots,n\).


\begin{Example}
Let \(A=\ma 3/2 & -1/2 \\ -1/2 & 3/2\mz\). This defines an ellipse
\(v^TAv=1\), in other words \[\frac{3}{2}(x^2+y^2)=1+xy.\] The
characteristic polynomial of \(A\) is \[\det\ma 3/2-t & -1/2 \\ -1/2
& 3/2-t\mz=t^3-3t+2,\] so the eigenvalues are \(1\) and \(2\). The
(unit length) eigenvectors are \(v_1=\ma 1/\sqrt{2}
\\ 1/\sqrt{2}\mz\) and \(v_2\ma 1/\sqrt{2} \\ -1/\sqrt{2}\mz\). If
we work with coordinates \(u_1,u_2\) related to \(x,y\) via \[\ma x
\\ y \mz=u_1\ma 1/\sqrt{2} \\ 1/\sqrt{2} \mz+u_2\ma 1/\sqrt{2}
\\ -1/\sqrt{2}\mz\] (that is, \(x=\frac{u_1+u_2}{\sqrt{2}}\),
\(y=\frac{u_1-u_2}{\sqrt{2}}\)) then the equation of the ellipse
\(v^TAv=1\) becomes \(u_1^2+2u_2^2=1\). We see that the change of
coordinates between \(x,y\) and \(u_1,u_2\) is actually a 45 degree
rotation.


\end{Example}
\subsection{Application III: Dynamics}


Consider the matrix \(\ma 2 & 1 \\ 1 & 1\mz\). We have seen
(\cref{exm:cat}) that its eigenvalues are \(\lambda_{\pm}:=\frac{3\pm
\sqrt{5}}{2}\), with eigenvectors \(v_{\pm}=\ma
1\\ \frac{1\pm\sqrt{5}}{2}\mz\).


Suppose we pick a point \(v\in\RR^2\) and write it as
\(v=av_++bv_-\). Then \(Av=\lambda_+av_++\lambda_-bv_-\). Suppose that
\(a\neq 0\) and \(b\neq 0\). Since \(\lambda_+>1\) and
\(\lambda_-<1\), this means that the point moves inwards along \(v_-\)
and outwards along \(v_+\). If we apply \(A\) again and again, we get
\[A^nv=\lambda_+^nav+\lambda_-^nbv_-.\] As \(n\to\infty\),
\(\lambda_+^n\to\infty\) and \(\lambda_-^n\to 0\), so the point gets
closer and closer to the \(v_+\)-eigenline, but gets pushed outwards
along the eigenline. If we draw a rectangle in \(\RR^2\) and apply
\(A\) many times, this square will get stretched outwards in the
\(v_+\) direction and squished inwards in the \(v_-\)-direction.


\tka
\draw[thick,->] (0,0) -- (2*1.618,2) node [below right] {\(v_+\)};
\draw[thick,->] (0,0) -- (-2*0.618,2) node [left] {\(v_-\)};
\draw[thick,red] (1/3,2/3) -- (0.745,0) -- (-1/3,-2/3) -- (-0.745,0) -- cycle;
\draw[thick,orange] (4/3,1) -- (2*0.745,0.745) -- (-4/3,-1) -- (-2*0.745,-0.745) -- cycle;
\draw[thick,yellow] (11/3,7/3) -- (5*0.745,3*0.745) -- (-11/3,-7/3) -- (-5*0.745,-3*0.745) -- cycle;
\tkz


This is typical behaviour of a ``hyperbolic'' dynamical system. Here
are two fun facts which are not unrelated to this.


\begin{Example}[Fibonacci numbers]\label{exm:fib}
The Fibonacci sequence
\[F_1,F_2,F_3,F_4,F_5,F_6,F_7,\ldots=1,1,2,3,5,8,13,\ldots\]
satisfies the recursion \(F_{n+2}=F_{n+1}+F_n\), which we can write
as a matrix equation: \[\ma F_1 \\ F_2\mz=\ma 1 \\ 1\mz,\qquad\ma
F_{n+1} \\ F_{n+2}\mz=\ma 0 & 1 \\ 1 & 1 \mz\ma F_n \\ F_{n+1}
\mz.\] The eigenvalues of \(\ma 0 & 1 \\ 1 & 1 \mz\) are
\(\lambda_{\pm}=\frac{1\pm\sqrt{5}}{2}\) with eigenvectors
\(v_{\pm}=\ma 1
\\ \frac{1\pm\sqrt{5}}{2}\mz\). Although \(\lambda_-\) is negative,
its magnitude is nonetheless \(<1\), so \(\lambda_-^n\to 0\). Also,
\(\lambda_+^n\to\infty\). Therefore \(\ma F_n \\ F_{n+1}\mz=A^n\ma 1
\\ 1 \mz\) tends in the limit \(n\to\infty\) to a vector pointing
along the \(v_+\)-eigenline, which has slope
\(\frac{1+\sqrt{5}}{2}\). This means
\[\lim_{n\to\infty}\frac{F_{n+1}}{F_n}=\frac{1+\sqrt{5}}{2}.\]
This number is known as the golden ratio.


\end{Example}
\begin{Example}[Arnol'd's cat map]\label{exm:arncat}
Let \(A\) be the example above. If you take a square picture of a
cat and use it to tile the plane, then you apply \(A^n\) to the
plane and let \(n\) increase, the picture will get distorted very
quickly. However, at some point, the picture will reappear
more-or-less exactly as you had it to begin with. In fact, if you
have a digital image, it will reappear exactly how it started
(because there's only a finite number of pixels involved). This is
due to a phenomenon called {\em ergodicity} of the flow, whereby
every point, at some time, comes back close to where it started
(except possibly in a different tile). Eventually, many points come
back close to where they started (except possibly in a different
tile) and you see something resembling the image you started with.


\end{Example}
You can see dramatic realisations of this in videos and applets
online.


\clearpage
\section{Subspaces I}
\subsection{Subspaces}


\begin{Definition}[Subspaces]\label{dfn:subspace}
A subset \(V\subset\RR^n\) is called a {\em linear subspace} (or
just subspace) if it satisfies the following conditions:
\begin{itemize}
\item \(v,w\in W\) implies \(v+w\in V\).
\item \(v\in V\), \(\lambda\in\RR\) implies \(\lambda v\in V\).
\end{itemize}
In other words, \(V\) is closed under addition and
rescaling. Subspaces are the natural higher-dimensional
generalisation of lines and planes through the origin in 3-d.


\end{Definition}
Sometimes you want to consider lines or planes which don't pass
through the origin, in which case the following definition comes in
handy:


\begin{Definition}[Affine subspaces]\label{dfn:affinesubspace}
A subset \(V\subset\RR^n\) is called an {\em affine subspace} if
there exists a vector \(w\in\RR^n\) and a linear subspace
\(V'\subset\RR^n\) such that \(V=\{w+v\ :\ v\in V'\}\). In other
words, \(V\) is obtained by translating \(V'\) by the vector \(w\).


\end{Definition}
\begin{Remark}
A {\em line} is a 1-dimensional subspace. A {\em plane} is a
2-dimensional subspace.


\end{Remark}
\begin{Definition}[Codimension]\label{dfn:codimension}
The {\em codimension} of a subspace \(V\subset\RR^n\) is \(p\) if
\(\dim V=n-p\).


\end{Definition}
\begin{Example}
A line in \(\RR^3\) has codimension 2. A plane in \(\RR^4\) has
codimension 2, while a line in \(\RR^4\) has codimension 3.


\end{Example}
\begin{Definition}
A {\em hyperplane} is a subspace of codimension 1. For example, a
line in \(\RR^2\), or a plane in \(\RR^3\).


\end{Definition}
Suppose someone asks you to give them a subspace of \(\RR^n\). You can
answer them in one of two ways:
\begin{itemize}
\item You can write down equations for the subspace, for example you can
say something like:
\begin{itemize}
\item ``it's the line \(x+y=0\) in \(\RR^2\)'',
\item ``it's the plane in \(\RR^3\) cut out by the equation \(z=0\)''.
\end{itemize}
\item You can give them a collection of vectors which ``span'' the
subspace, for example you can say something like:
\begin{itemize}
\item ``it's the line through the origin pointing in the \(\ma 1 \\ -1\mz\)-direction''.
\item ``it's the plane in \(\RR^3\) spanned by the vectors \(\ma 1 \\ 2 \\ 0\mz\) and \(\ma 1 \\ 0 \\ 1\mz\)''.
\end{itemize}
\end{itemize}
We'll focus on these two methods in order, then talk about how to
relate them.


\subsection{Equations for subspaces; kernel}


\begin{Example}
A linear hyperplane is cut out by a single linear equation. More
precisely, a row vector \(r=\ma r_1 & \cdots & r_n\mz\) defines a
linear hyperplane in \(\RR^n\), namely: \[\left\{x=\ma x_1 \\ \vdots \\
x_n\mz\in\RR^n\ :\ rx=0\right\}.\] Equivalently, this is the
hyperplane orthogonal to the column vector \(r^T\).


\end{Example}
\begin{Definition}
Given a linear subspace \(V\subset\RR^n\) and a vector \(w\), we
define the {\em translate} \(w+V=\{v+w\in\RR^n\ :\ v\in V\}\) of
\(V\) by \(w\) to be the affine subspace obtained by translating the
elements of \(V\) along the vector \(w\).


\end{Definition}
\begin{Example}
A row vector \(r=\ma r_1 & \cdots & r_n\mz\) together with a number
\(b\) defines an {\em affine hyperplane} in \(\RR^n\), namely:
\[\left\{x=\ma x_1\\ \vdots\\ x_n\mz\in\RR^n\ :\ rx=b\right\}.\]
Equivalently, this is the hyperplane orthogonal to \(r^T\)
translated by \(\frac{br^T}{|r|^2}\), i.e. translated a certain
amount in the \(r^T\) direction. Note that this is a linear subspace
if and only if \(b=0\).


\end{Example}
\begin{Example}
An \(m\)-by-\(n\) matrix \(A\) define \(m\) linear hyperplanes, cut
out by the equations
\begin{align*}
A_{11}x_1+\cdots+A_{1n}x_n&=0\\
\vdots\qquad\qquad\vdots\qquad\qquad&\vdots\\
A_{m1}x_1+\cdots+A_{mn}x_n&=0.
\end{align*}
A {\em solution} \(v=\ma x_1 \\ \vdots \\ x_n\mz\) to this system of
equations is then a vector \(v\in\RR^n\) satisfying \(Av=0\); in
other words a vector \(v\) which belongs to all \(m\) of the
hyperplanes; in other words a point where the hyperplanes intersect.


\end{Example}
\begin{Definition}
There is a fancy name for the linear subspace given by
\(\{v\in\RR^n\ :\ Av=0\}\). It is called the kernel of \(A\),
written \(\ker(A)\).


\end{Definition}
\begin{Example}
An \(m\)-by-\(n\) matrix \(A\) and a vector \(b\in\RR^m\) define
\(m\) affine hyperplanes, cut out by the equations
\begin{align*}
A_{11}x_1+\cdots+A_{1n}x_n&=b_1\\
\vdots\qquad\qquad\vdots\qquad\qquad&\vdots\\
A_{m1}x_1+\cdots+A_{mn}x_n&=b_m.
\end{align*}
The set of solutions to \(Av=b\) is the intersection of these affine
hyperplanes.


\end{Example}
\begin{Example}\label{exm:threelines1}
Consider the matrix \(A=\ma 1 & 1 \\ 1 & -1 \\ 0 & 1\mz\) and the
vector \(b=\ma 1 \\ 1 \\ 1\mz\). The equations \(Av=b\) define three
{\em lines} (hyperplanes in \(\RR^2\)):
\[x+y=1,\quad x-y=1,\quad y=1\]
drawn red, purple and blue respectively in the diagram below.


\tka
\draw[->] (-2,0) -- (3,0) node [right] {\(x\)};
\draw[->] (0,-2) -- (0,2) node [above] {\(y\)};
\draw[red,thick] (-1,2) -- (2,-1) node [right] {\(x+y=1\)};
\draw[purple,thick] (-1,-2) -- (3,2) node [above] {\(x-y=1\)};
\draw[blue,thick] (-2,1) -- (3,1) node [right] {\(y=1\)};
\tkz


Since the lines don't have a common intersection, we know the system
of equations has no solutions (the lines intersect in pairs, so any
two of the equations admit a solution, but there is no one point
contained in all three lines).


\end{Example}
We see that the intuition that an overdetermined system (more
hyperplanes than dimensions) has no solutions is justified, because
you need your \(n+1\) hyperplanes in \(\RR^n\) to be in very special
position to make them have a common intersection. Nonetheless, it can
happen.


\begin{Example}\label{exm:threelines2}
Let \(A\) be as before but \(b=\ma 3 \\ 1 \\ 1\mz\). This has the
effect of translating one of the lines from \cref{exm:threelines1}
so that it becomes \(x+y=3\). As we see below, these lines below
have a common intersection at \(\ma 2 \\ 1\mz\), so the
overdetermined system has a solution \(x=2\), \(y=1\) (marked with a
dot below).


\tka
\draw[->] (-2,0) -- (3,0) node [right] {\(x\)};
\draw[->] (0,-2) -- (0,2) node [above] {\(y\)};
\draw[red,thick] (1,2) -- (4,-1) node [right] {\(x+y=3\)};
\draw[purple,thick] (-1,-2) -- (3,2) node [above] {\(x-y=1\)};
\draw[blue,thick] (-2,1) -- (3,1) node [right] {\(y=1\)};
\node at (2,1) {\(\bullet\)};
\tkz


\end{Example}
\begin{Remark}
Given a subspace \(V\subset\RR^n\) of dimension \(n-p\) (codimension
\(p\)) and a subspace \(W\subset\RR^n\) of dimension \(n-q\)
(codimension \(q\)), we ``expect'' the intersection \(V\cap W\) to
have dimension \(n-p-q\) (codimension \(p+q\)). In other words,
codimension is {\em usually} additive under intersection. For
example, in \(\RR^3\), a plane (codimension 1) and a line
(codimension 2) will usually intersect at a point (codimension 3),
unless you're in the exceptional situation that the line is
contained inside the plane. As a corollary of this, we {\em expect}
the space of solutions to a system of \(m\) equations in \(n\)
unknowns to be \(n-m\) (each equation cuts down the set of solutions
by one dimension)...except when it isn't!


\end{Remark}
Having made this remark, let us give a more precise characterisation
of the dimension of the space of solutions.


\begin{Theorem}
Let \(A\) be an \(m\)-by-\(n\) matrix and \(b\in\RR^m\) be a
vector. Suppose that \(\ker(A)\) has dimension \(k\) (this number is
called the {\em nullity} of \(A\)). Then, the dimension of the space
of solutions to \(Av=b\), assuming it is nonempty, is equal to
\(k\). Indeed, the space of solutions is a translate of \(\ker(A)\).
\end{Theorem}
\begin{Proof}
If \(v_1,v_2\) are solutions to \(Av=b\) then \(A(v_1-v_2)=b-b=0\),
so the difference \(v_1-v_2\) is in the kernel of \(A\). Similarly,
if \(Av_1=b\) and \(Av=0\) then \(A(v_1+v)=b+0=b\), so adding
elements of the kernel to a solution gives another
solution. Therefore, if we fix one solution \(v_1\), the space of
solutions is \(v_1+\ker(A)=\{v_1+v\ :\ v\in\ker(A)\}\), i.e. a
translate of \(\ker(A)\). \qedhere


\end{Proof}
\begin{Theorem}\label{thm:nullity}
Given a matrix \(A\), its nullity is equal to the number of free
indices once \(A\) has been put into reduced echelon form.
\end{Theorem}
\begin{Proof}
We saw that the general solution to \(Av=b\) has one parameter for
each free index. Therefore it is a space with dimension equal to the
number of free indices. \qedhere


\end{Proof}
\clearpage
\section{Subspaces II}
\subsection{Spanning sets for subspaces}


\begin{Definition}
Given a collection of vectors \(v_1,\ldots,v_k\), a {\em linear
combination} of these vectors is an expression of the form
\[v=\lambda_1v_1+\cdots+\lambda_k v_k\] for some choice of
coefficients \(\lambda_1,\ldots,\lambda_k\). We define the {\em
linear subspace spanned by} \(v_1,\ldots,v_k\) (or the span of
\(v_1,\ldots,v_k\), written \(\mathrm{span}(v_1,\ldots,v_k)\)) to be
the set of all linear combinations of \(v_1,\ldots,v_k\).


\end{Definition}
\begin{Lemma}
For any collection of vectors \(v_1,\ldots,v_k\in\RR^n\), the set
\(\mathrm{span}(v_1,\ldots,v_k)\) is a linear subspace of \(\RR^n\).
\end{Lemma}
\begin{Proof}
If we rescale a linear combination \(\sum_i\lambda_iv_i\) by \(\mu\)
then we get the linear combination \(\sum_i(\mu\lambda_i)v_i\). If
we add two linear combinations \(\sum_i\lambda_iv_i\) and
\(\sum_i\mu_iv_i\) then we get the linear combination
\(\sum_i(\lambda_i+\mu_i)v_i\). Therefore linear combinations form a
linear subspace. \qedhere


\end{Proof}
\begin{Example}
The set of all linear combinations of \(v_1\) is just the set of all
vectors \(\lambda_1 v_1\), \(\lambda_1\in\RR\). In other words, it's
the set of all rescalings of \(v_1\), otherwise known as the line
that points in the \(v_1\)-direction.


\end{Example}
\begin{Example}
Let \(v_1=\ma 1 \\ 0 \\ 0 \mz\) and \(v_2=\ma 0 \\ 1 \\ 0\mz\). The
subspace spanned by \(v_1,v_2\) is the set of all vectors
\(\lambda_1v_1+\lambda_2v_2=\ma \lambda_1 \\ \lambda_2 \\ 0\mz\), in
other words, it is the \(xy\)-plane.


\end{Example}
\begin{Example}
The plane spanned by \(v_1=\ma 1 \\ 0 \\ 0 \mz\), \(v_2=\ma 0 \\ 1 \\ 0\mz\)
and \(v_3=\ma 1 \\ 1 \\ 0\mz\) is {\em also} the \(xy\)-plane, because
adding multiples of \(v_3\) doesn't take you out of this plane. The
issue here is that \(v_3\) is itself a linear combination of \(v_1\)
and \(v_2\) (\(v_3=v_1+v_2\)) so it doesn't change the spanning set.


\end{Example}
\begin{Definition}
A spanning set is called a {\em basis} if it has minimal size.


\end{Definition}
\begin{Theorem}
All bases for the same subspace have the same size. This size is
called the dimension of the subspace. (I haven't actually given you
a formal definition of dimension until now).
\end{Theorem}
\begin{Proof}
This will be proved in your next course on linear algebra, next
year. \qedhere


\end{Proof}
\subsection{Image of a matrix}


\begin{Definition}
The {\em image} of an \(m\)-by-\(n\) matrix \(A\) is the set of all
\(b\in\RR^m\) such that \(Av=b\) has a solution \(v\in\RR^n\).


\end{Definition}
\begin{Lemma}
The image of \(A\) is spanned by the columns of \(A\).
\end{Lemma}
\begin{Proof}
If the columns of \(A\) are \(a_1,\ldots,a_n\in\RR^m\) then \[A\ma
x_1 \\ \vdots \\ x_n\mz=x_1a_1+\cdots+x_na_n,\] so the image of
\(A\) is the set of linear combinations of the columns, as
required. \qedhere


\end{Proof}
\begin{Example}
If \(A=\ma 1 & 1 \\ 2 & 0 \\ 0 & 1\mz\) then the image of \(A\) is
the plane spanned by \(\ma 1 \\ 2 \\ 0\mz\) and \(\ma 1 \\ 0
\\ 1\mz\) (see \cref{exm:3by2}).


\end{Example}
\begin{Definition}
The {\em rank} of \(A\) is defined to be the dimension of the
image.


\end{Definition}
\begin{Theorem}\label{thm:rank}
The rank of \(A\) is equal to the number of leading indices when
\(A\) is put into reduced echelon form.
\end{Theorem}
\begin{Proof}
First note that row operations do not change the rank: if \(A\) and
\(A'\) are related by a row operation then \(A'=EA\) for some
elementary matrix \(E\), and now the map \(b\mapsto Eb\) gives an
isomorphism between the image of \(A\) and the image of \(A'\)
(isomorphism in the sense that \(E\) is an invertible linear
map). Therefore we may assume that \(A\) is in reduced echelon form
by \cref{thm:reducedechelonform}.


So suppose that \(A\) is in reduced echelon form with the first
\(k\) rows nonzero (so that \(k\) equals the number of leading
indices). The equation \(Av=b\) has a solution if and only if
\(b_{k+1}=\cdots=b_m=0\), so the image of \(A\) is equal to the
subspace spanned by the first \(k\) basis vectors, which has
dimension \(k\). \qedhere


\end{Proof}
Here is a useful theorem relating the rank and the nullity of an
\(m\)-by-\(n\) matrix:


\begin{Theorem}[Rank-nullity theorem]\label{thm:ranknullity}
If \(A\) is an \(m\)-by-\(n\) matrix, the rank and the nullity of
\(A\) sum to \(n\).
\end{Theorem}
\begin{Proof}
In reduced echelon form, the number of leading indices and free
indices sum to \(n\) (number of columns), so this follows from
\cref{thm:nullity} and \cref{thm:rank}. \qedhere


\end{Proof}
\subsection{Kernel, image and simultaneous equations}


To relate this to what we said about simultaneous equations, we can
summarise everything we've said as follows:


\begin{Theorem}
Let \(A\) be an \(m\)-by-\(n\) matrix and \(b\in\RR^m\) be a
vector. Then \(Av=b\) has a solution if and only if
\(b\in\mathrm{im}(A)\). If \(Av=b\) has a solution then the space of
solutions is a translate of \(\ker(A)\).


\end{Theorem}
The following diagram may help you to remember whereabouts the kernel
and image of an \(m\)-by-\(n\) matrix \(A\) live:


\tka
\node (A) at (0,0) {\(\RR^n\)};
\node (B) at (2,0) {\(\RR^m\)};
\node at (0,-1) {\(\ker(A)\)};
\node at (2,-1) {\(\mathrm{im}(A)\)};
\draw[thick,->] (A) -- (B) node [midway,above] {\(A\)};
\node[rotate=90] at (0,-1/2) {\(\subseteq\)};
\node[rotate=90] at (2,-1/2) {\(\subseteq\)};
\tkz


\clearpage
\section{Linear maps}


This lecture is intended as a foretaste of things to come. We
introduce an extra layer of abstraction, which suddenly elevates us
above the clouds and we see how to apply linear algebra in contexts we
had not formerly imagined.


\subsection{Linearity}


We defined a linear map \(\RR^n\to\RR^m\) to be a map of the form
\(v\mapsto Av\) where \(A\) is an \(m\)-by-\(n\) matrix. There is a
different way to characterise linear maps, which we now discuss.


\begin{Definition}
An (\(\RR\)-)vector space\footnote{You can replace \(\RR\) by any
field \(k\) (like \(\QQ\) or \(\CC\)) in this definition and get a
\(k\)-vector space. Usually we just omit the field from the notation
and call it a {\em vector space}.} is a set \(V\) together with:
\begin{itemize}
\item a map \(V\times V\to V\), written \((v,w)\mapsto v+w\),
\item a map \(\RR\times V\to V\), written \((\lambda,v)\mapsto \lambda
v\),
\item an element \(0\in V\),
\end{itemize}
such that:
\begin{align*}
u+(v+w)&=(u+v)+w& v+w&=w+v\\
v&=0+v=v+0,&v+(-v)&=0\\
1v&=v&\lambda(\mu v)&=(\lambda\mu)v\\
(\lambda+\mu)v&=\lambda v+\mu v&\lambda(v+w)&=\lambda v+\lambda w\\
\end{align*}
for all \(u,v,w\in V\) and \(\lambda,\mu\in\RR\).


\end{Definition}
For example, \(\RR^n\) equipped with the usual addition and rescaling
action of \(\RR\) is a vector space.


\begin{Definition}
Let \(V,W\) be vector spaces. A map \(T\colon V\to W\) is called
{\em linear}\footnote{Again, if we're working with \(k\)-vector
spaces (e.g \(k=\QQ,\CC\)) then you need to talk about \(k\)-linear
maps and replace \(\RR\) with \(k\) everywhere in this definition.}
if the following conditions are satisfied:
\begin{itemize}
\item for all \(v,w\in\RR^n\) and we have \(T(v+w)=T(v)+T(w)\).
\item for all \(\lambda\in\RR\) and \(v\in\RR^n\), we have \(T(\lambda
v)=\lambda T(v)\).


\end{itemize}
\end{Definition}
\begin{Theorem}
If \(T\colon\RR^n\to\RR^m\) is linear then there exists an
\(n\)-by-\(m\) matrix \(A\) such that \(T(v)=Av\) for all
\(v\in\RR^n\). Conversely, if \(A\) is an \(m\)-by-\(n\) matrix then
a map \(\RR^n\to\RR^m\) of the form \(v\mapsto Av\) is linear.
\end{Theorem}
\begin{Proof}
If \(T\) is linear then it is determined by its values on the basis
vectors \(e_1,\ldots,e_n\). To see this, observe that if \(v=\ma
v_1\\ \vdots\\ v_n\mz=\sum_{i=1}^nv_ie_i\) then
\(T(v)=T(\sum_{i=1}^nv_ie_i)=\sum_{i=1}^nv_iT(e_i)\) by linearity,
so the vectors \(T(e_1),\ldots,T(e_n)\) determine \(T\)
completely. If we pick \(A\) to be the matrix whose columns are
\(T(e_1),\ldots,T(e_n)\) then \(Av=\sum_{i=1}^nv_iT(e_i)=T(v)\), so
the matrix we were looking for exists (and is uniquely specified by
\(T\)).


Conversely, if \(A\) is a matrix then the identity \(A(v+w)=Av+Aw\)
is just the distributivity of matrix multiplication and \(A(\lambda
v)=\lambda Av\) is easy to check. \qedhere


\end{Proof}
In fact, one can prove that any finite-dimensional vector space \(V\) is
isomorphic to \(\RR^n\) for some \(n\). (Isomorphic here means that
there is an invertible linear map \(V\to\RR^n\); finite-dimensional
means that there is a finite spanning set). However, there's nothing
to stop you using {\em infinite-dimensional} vector spaces, and then
things get interesting.


\begin{Example}
The space of continuous functions \(f\colon\RR\to\RR\) is a vector
space, usually called \(\mathcal{C}^0(\RR)\). You can add two
functions \((f+g)(x)=f(x)+g(x)\) and you can rescale a function
\((\lambda f)(x)=\lambda f(x)\) and these operations satisfy the
conditions required of a vector space (the zero function is
\(f(x)=0\)).


\end{Example}
\begin{Example}
The space of once-continuously-differentiable functions is a
subspace of \(\mathcal{C}^0(\RR)\), usually written
\(\mathcal{C}^1(\RR)\subset\mathcal{C}^0(\RR)\).


\end{Example}
\begin{Example}
Differentiation defines a linear map
\(\frac{d}{dx}\colon\mathcal{C}^1(\RR)\to\mathcal{C}^0(\RR)\). It is
linear because
\[\frac{d}{dx}(f+g)(x)=\frac{df}{dx}(x)+\frac{dg}{dx}(x),\qquad\frac{d}{dx}(\lambda
f)(x)=\lambda\frac{df}{dx}(x).\]


\end{Example}
Can we write a matrix for differentiation? We need to pick a basis for
\(\mathcal{C}^1(\RR)\), which is a highly nontrivial task. Let's be a
little less ambitious and restrict to the subspace of {\em analytic
functions}, i.e. functions \(f\) whose Taylor series converges to
\(f\). This is usually written \(\mathcal{C}^{\omega}(\RR)\). The
functions \(f_n(x)=x^n\), \(n=0,1,2,\ldots\), form a {\em Schauder
basis} for this space, which means that any function
\(f\in\mathcal{C}^{\omega}(\RR)\) can be written as an infinite sum of
these functions (namely its Taylor series!). In other words, we are
thinking of the coefficients of the Taylor expansion as coordinates on
the space \(\mathcal{C}^{\omega}(\RR)\). That is, a function \(f\) can
be thought of as an infinite vector \(\ma f(0) \\ \frac{df}{dx}(0)
\\ \frac{1}{2}\frac{d^2f}{dx^2}(0)
\\ \frac{1}{3!}\frac{d^3f}{dx^3}(0)\\\vdots\mz\).


If \(f(x)=\sum_{n\geq 0}a_nx^n\) then \(\frac{df}{dx}=\sum_{n\geq 1}
na_nx^{n-1}=\sum_{n\geq 0}(n+1)a_{n+1}x^n\), so our ``matrix'' for
differentiation is
\[\ma 0 & 1 & 0 & 0 & \cdots \\ 0 & 0 & 2 & 0 & \cdots \\0 & 0 & 0 & 3 &
\\ \vdots & \vdots & \vdots & & &\mz\ma a_0 \\ a_1 \\ a_2
\\ \vdots\mz=\ma a_1 \\ 2a_2 \\ 3a_3\\\vdots\mz.\]
If one restricts instead to {\em periodic functions}
\(f(x+2\pi)=f(x)\) then there is an alternative basis, coming from the
functions \(\sin(nx),\cos(nx)\). The expansion of a function in terms
of this basis is called its Fourier expansion, and again
differentiation of a function can be thought of as a linear
transformation of its Fourier series. This leads to the powerful
method of {\em Fourier transform}, which allows you to convert
differential equations into much simpler linear equations.


\begin{Example}
What is the kernel of differentiation? It is the set of functions
whose derivative is identically zero, in other words, the constant
functions. What is the inverse of differentiation? Well, because
there is a kernel it has no inverse, strictly speaking, but clearly
integration should define an inverse in some sense. This is why it
doesn't make sense to say ``the integral of \(f\)'' unless you also
say ``plus an unknown constant''.


\end{Example}
\begin{Example}
Consider the linear map
\(\frac{d}{dx}\colon\mathcal{C}^{\omega}(\RR)\to\mathcal{C}^{\omega}(\RR)\). What
are the eigenvalues and eigenvectors of this map? A
\(\lambda\)-eigenvector will be a function \(f\) which solves the
equation \[\frac{df}{dx}=\lambda f.\] We can solve this by dividing
through by \(f\) and integrating: \[\ln
f=\int\frac{df}{f}=\int\lambda dx=\lambda x+c,\]
i.e. \(f=Ce^{\lambda x}\). So the \(\lambda\)-eigenline is spanned
by \(f(x)=e^{\lambda x}\) and every \(\lambda\in\RR\) arises as an
eigenvalue.


\end{Example}
\begin{Example}
Similarly, \(\cos(x\sqrt{-\lambda})\) and \(\sin(x\sqrt{-\lambda})\)
are \(\lambda\)-eigenvectors for \(\frac{d^2}{dx^2}\), that is they
solve the differential equation
\[\frac{d^2f}{dx^2}=\lambda f.\]


\end{Example}
We often say ``eigenfunction'' rather than eigenvector in this
context. Finding eigenfunctions and eigenvalues of differential
operators is an incredibly important problem; essentially all of
quantum mechanics boils down to solving this problem for particular
operators.
\end{document}
